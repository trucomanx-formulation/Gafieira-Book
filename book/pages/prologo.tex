
\cleardoublepage % Forces the first chapter to start on an odd page so it's on the right
\newpage
\thispagestyle{empty}

\chapterimage{chapter_letras.pdf} % Table of contents heading image
\chapter*{Prólogo}
\addcontentsline{toc}{chapter}{Prólogo} %% Sale en la pagina contents
Este livro contem um trabalho de pesquisa e recopilação de informações relativas ao samba de gafieira
e a dança a dois. Entre os assuntos abordados temos: a história do samba como expressão cultural,
os subgêneros musicais do samba e as danças derivadas destas; 
também são presentados alguns conceitos de teoria musical,
musicalidade, teorias sobre o aprendizado, controle e isolamento corporal.  

O livro tem sido criado como uma mensagem do meu eu presente para meu eu futuro,
como uma lembrança permanente sobre conceitos e definições importantes no samba de gafieira e a dança a dois;
porém, ao finalizar esta edição do meu trabalho, 
também tenho a esperança de que esta pesquisa seja útil para as pessoas
que precisem informações básicas ao iniciar seu percorrido no estudo do samba de gafieira.

Fazendo uma analogia com um mapa, 
o livro não representa a marca que indica o final do percorrido 
e sim uma das muitas rotas possíveis para iniciar a avançar na direção desejada;
assim, com o passar dos anos, minha esperança é que seja finalizada a visão iniciada nesta edição do livro.  
