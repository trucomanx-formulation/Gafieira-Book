\section{Percepção do breque na música}
\label{sec:percepcionbreak}
\index{Musicalidade!Breque}

Uma habilidade muito comentada quando uma pessoa inicia o estudo da musicalidade 
é a capacidade de distinguir as interrupções na música, 
também chamadas breques (deformação do termo ``break'' em inglês) ou paradinhas.

É importante aclarar que as pausas acontecem o tempo todo na  música,
porém quando mencionamos aos breques, 
em geral as pessoas se referem às pausas que acontecem ao final de uma frase musical 
e que além de marcar o final desta numa linha melódica,
acontecem numa pausa dos instrumentos de acompanhamento da música.
Muitas vesses poderemos observar um tempo de espera maior no breque que entre duas frases consecutivas regulares.
Outra caraterística dos breques é que comumente está pausa é preenchida por uma melodia auxiliar 
que funciona como resposta para a frase anterior e que da uma ideia ao ouvinte de quando o breque terminará.
Na música \hyperref[subsec:monofonica]{\textbf{monofônica}} e \hyperref[subsec:homofonica]{\textbf{homofônica}} 
o breque geralmente pode ser percebido pela pausa do acompanhamento percussivo principal;
já na música \hyperref[subsec:polifonica]{\textbf{polifônica}} o breque pode ser total,
ou só incluir a uma melodia dando entrada a outra voz que pode ou não levar acompanhamento.
\begin{definition}[Breque]
\label{def:breakingoff}  
\index{Musicalidade!Breque}
Também chamado ``Abruptio'' ou ``a breaking-off''. 
Esta é uma parada repentina numa melodia que acontece antes de chegar ao verdadeiro final da música, 
de modo que a peça continua após a pausa \cite[pp. 5]{baker1895dictionary}

Este recurso é geralmente utilizado para agregar mais emoção,
ou para dar uma sensação de antecipação para a nova seção da música. 

A pausa pode ser total ou pode incluir um pequeno solo melódico ou percussivo, 
também existe a possibilidade de combinar ambas opções.
%Definição de abruptio e outras pausas \cite{villavicencio2011retorica}.

\end{definition}

\begin{definition}[Break]
\index{Musicalidade!Break}
É o ponto de junção na qualidade das vozes de tenor, soprano e alto
ou da voz de cabeça e a de tórax;  
em geral pode entender-se como o ponto no qual um registro de voz ou instrumento passa para outro 
\cite[pp. 63]{stainer2009dictionary} \cite[pp. 31]{baker1895dictionary}.
\end{definition}

Com estas descrições do breque na música, 
podemos inferir que se sabemos detetar o 
\hyperref[pos:detetandofinalfrase]{\textbf{final de uma frase musical}}\footnote{A
detecção do final de uma frase musical foi explicado na Pag. \pageref{pos:detetandofinalfrase}.},
então provavelmente saberemos detetar os breques na música.
Porém existem particularidades do breque
que nós ajudarão a predizê-los mais facilmente, como:
\begin{itemize}
%%%%%
\item \hyperref[sec:tensionrelease]{\textbf{Tensão e relaxação:}} Como vimos na Seção \ref{sec:tensionrelease},
existe na música uma dinâmica continua de tensão e relaxação 
que é intuitivamente perceptível se prestamos um pouco atenção à música.
Esta caraterística nos será muito útil, 
pois os breques geralmente são anunciados por uma acumulação acentuada da tensão,
chegando ao ponto de não ter um possível retorno amável à relaxação, 
pelo que a única saída esperada pelo ouvinte é que a música ``estoure'';
ou seja, que tenha um breque.
Para mais detalhes sobre indicadores que ajudam a perceber tensão na música, ir a Tabela \ref{tab:tensionrelease1}.
Exemplos do uso da tensão e relaxação musical na dança podem ser vistos na Seção \ref{sec:musicalidadetensionrelease}.
%%%%%
\item \textbf{Atenção ao tempo forte:} A maioria de breques na música acontecem no tempo forte,
isto é devido a que as frases com \hyperref[subsubsec:finalmasculino]{\textbf{final masculino}} 
tem uma maior sensação de conclusão e tendem a ser mais empolgantes que 
\hyperref[subsubsec:finalfemenino]{\textbf{finais femininos}} que dão uma sensação mais reflexiva, 
suspensiva ou interrogativa, dependendo da cadência.
Os compositores ao criar um breque na música geralmente procuram ter uma pausa conclusiva, explosiva e contundente,
pelo menos em musicas alegres.
Assim, é importante estar atentos ao tempo forte quando, por exemplo, 
pelo acumulo de tensão sabemos que um breque está próximo.
\begin{example}[Breques em músicas (final masculino e/ou com sincopados):]~
\label{ex:breakmasculinos}
\begin{itemize}
\item ``Baile no Elite'' interpretado por Casuarina.
\item ``Cabide'' interpretado por Ana Carolina e Luiz Melodia.
\item ``Reunião de Bacana (Se gritar pega ladrão)'' interpretado pelo grupo Exporta Samba.
\item ``Tiro ao Álvaro'' interpretado por Elis Regina e Adoniran Barbosa. 
\item ``A voz do morro'' interpretado por Diogo Nogueira.
\end{itemize}
\end{example}

%%%%%
\item \textbf{Predizer na melodia e seguir o acompanhamento:}
Como falamos anteriormente a pausa geralmente acontece no tempo forte,
mas é possível achar pausas que acontecem no tempo fraco. 
Pelo que em algumas ocasiones teremos duvidas em reconhecer ou predizer em que tempo acabará a frase,
como é visto no Exemplo \ref{ex:breakforte1}.
\begin{example}
\label{ex:breakforte1}
Na música ``Reunião de Bacana (Se gritar pega ladrão)'' interpretado pelo grupo Exporta Samba,
existe um breque no final da frase: ``Para ser rou\textbf{ba}do...''
na qual a sílaba tônica ``ba'' se percebe próxima ao tempo forte, 
pelo que poderíamos supor que a frase termina em tempo fraco na sílaba ``do'';
porém o que nos resolve esta duvida é o acompanhamento percussivo, 
que claramente finaliza com uma nota em tempo forte.
Assim podemos especular que o breque aconteceu no tempo forte
no final da frase ``Para ser rou\textbf{ba}do...''.
\end{example}
Assim, para evitar estas duvidas na conjetura e qualificação do tipo de breque,
é melhor predizer este usando a melodia principal 
e classificar-lho usando a posição das notas finais do acompanhamento harmônico ou percussivo.
%%%%%
\item \hyperref[ref:PontoCulminanteSuperior]{\textbf{Ponto culminante superior:}} 
A nota mais aguda da frase musical\footnote{O 
ponto culminante superior foi mencionado na Pag. \pageref{ref:PontoCulminanteSuperior}.} 
é um bom indicador para detetar que um final de frase musical está próximo; 
consequentemente, se por outros fatores como acumulação de tensão,
sabemos que estamos perto de um breque, escutar esta nota mais aguda acentuará nossa certeza ao predizer 
a proximidade deste evento.

%\item Podemos detetar o final de uma frase pois, 
%ao ser interpretada como se fosse uma frase falada, 
%percebemos que uma ideia está chegando a seu final.

%\item Podemos aprender a reconhecer os tipos de \hyperref[sec:Cadencia]{\textbf{Cadência}},
%vistas na Seção \ref{sec:Cadencia}.
\end{itemize}



\begin{example}[Breques em músicas com final feminino]~
\label{ex:breakfeminino}
\begin{itemize}
\item ``Amanhã é sábado'' interpretado por Roberta Sá. 
O primeiro breque acontece com um final de frase feminino.
\end{itemize}
\end{example}

\begin{example}[Breques em músicas com final sincopado]~
\label{ex:breaksincopados}
\begin{itemize}
\item ``Eu sou a marrom'' interpretado por Alcione.
\item ``Logo agora'' interpretado por Emílio Santiago.
\end{itemize}
\end{example}

