\section{Percepção do ``break'' da música}
\label{sec:percepcionbreak}
\index{Musicalidade!Breques}
\index{Musicalidade!Break}

Uma habilidade muito comentada quando uma pessoa inicia seu caminho no estudo da musicalidade,
é a capacidade de distinguir as pausas da música, 
também chamadas breques, paradinhas ou ``breaks'' (no inglês).

É importante aclarar que as pausas acontecem o tempo todo na  música,
porém quando nos referimos aos breques, 
falamos dessas pausas que acontecem ao final de uma frase musical,
e que além de marcar o final dessa ideia,
se caraterizam pela pausa dos instrumentos principais da música ou peça,
alem de que muitas vesses podemos observar um tempo de espera maior entre frases musicais, 
em comparação das pausas da música entre duas frases consecutivas regulares.
Outra carateristica dos breques, 
é que pelo geral está pausa é prenchida por uma melodia auxiliar,
que funciona como resposta para a frase anterior, e que da uma ideia ao ouvinte de quando o breque terminará.
Na música \hyperref[subsec:monofonica]{\textbf{monofônica}} e \hyperref[subsec:homofonica]{\textbf{homofônica}} 
o breque geralmente pode ser percebido pela pausa do acompanhamento percusivo principal;
já na música \hyperref[subsec:polifonica]{\textbf{polifônica}} o breque pode ser total,
ou so incluir a uma melodia dando entreda a outra voz que pode ou nao levar acompanhamento.
\begin{definition}[Breque (Break)]
Este é um descanso breve ou pausa para a melodía principal numa música,
este recurso é geralmente utilizado para agregar mais emoção,
ou para dar uma sensação de antecipação para a nova seção da música. 

A pausa pode ser total ou pode incluir um pequeno solo melódico ou percusivo, 
tambem existe a posibilidade de combinar ambas opções.
\end{definition}

Com estas descrições do breque na música, 
podemos inferir que se sabemos detetar o 
\hyperref[pos:detetandofinalfrase]{\textbf{final de uma frase musical}}\footnote{A
detecção do final de uma frase musical foi explicado na Pag. \pageref{pos:detetandofinalfrase}.},
então facilmente saberemos detetar os breques na música.
Porém existem particularidades do breque,
que nós ajudarão a predizer-lhos melhor, como:
\begin{itemize}
%%%%%
\item \hyperref[sec:tensionrelease]{\textbf{Tensão e relaxação:}} Como vimos na Seção \ref{sec:tensionrelease},
existe na música uma dinâmica continua de tensão e relaxação,
que é intuitivamente perceptível se prestamos um pouco atenção  na música.
Esta caraterística da música nos servirá de muito, 
pois os breques geralmente são anunciados por uma acumulação acentuada da tensão,
chegando ao ponto de não ter um possível retorno amável à relaxação, 
pelo que a única saída esperada pelo ouvinte é que a música ``estoure'';
é dizer, que tenha um breque.
Para mais detalhes sobre indicadores que ajudam a perceber tensão na música, ir a Tabela \ref{tab:tensionrelease1}.
Exemplos do uso da tensão e relaxação da música na dança, podem ser vistos na Seção \ref{sec:musicalidadetensionrelease}.
%%%%%
\item \textbf{Atenção ao tempo forte:} A maioria de breques na música acontecem no tempo forte,
isto é devido a que as frases com \hyperref[subsubsec:finalmasculino]{\textbf{final masculino}},
tem uma maior sensação de conclusão, e tendem a ser mais empolgantes que 
\hyperref[subsubsec:finalfemenino]{\textbf{finais femininos}} que dão uma sensação mais reflexiva, 
suspensiva ou interrogativa, dependendo da cadência; de modo que, 
os compositores ao criar um break na música,
geralmente estão interessados em ter uma pausa conclusiva, explosiva e contundente,
pelo menos em musicas alegres.
Assim, é importante estar atentos ao tempo forte quando, por exemplo, 
pelo acumulo de tensão sabemos que um breque está próximo.
\begin{example}[Breques em músicas (final masculino e/ou com sincopados):]~
\label{ex:breakmasculinos}
\begin{itemize}
\item ``Baile no Elite'' interpretado por Casuarina.
\item ``Cabide'' interpretado por Ana Carolina e Luiz Melodia.
\item ``Reunião de Bacana (Se gritar pega ladrão)'' interpretado pelo grupo Exporta Samba.
\item ``Tiro ao Álvaro'' interpretado por Elis Regina e Adoniran Barbosa. 
\item ``A voz do morro'' interpretado por Diogo Nogueira.
\end{itemize}
\end{example}

%%%%%
\item \textbf{Predizer na melodia e seguir o acompanhamento:}
Como falamos anteriormente a pausa geralmente acontece no tempo forte,
mas é possível ver pausas que acontecem no tempo fraco. 
Pelo que em algumas ocasiones, 
teremos a duvida se a pausa chego apos uma nota em tempo forte ou em tempo fraco,
como no Exemplo \ref{ex:breakforte1}.
\begin{example}
\label{ex:breakforte1}
Na música ``Reunião de Bacana (Se gritar pega ladrão)'' interpretado pelo grupo Exporta Samba,
existe um breque na música ao final da frase: ``Para ser rou\textbf{ba}do...''
onde a sílaba tônica ``ba'' cai no tempo forte (masculino), 
pelo que poderíamos pensar que a frase termina em tempo fraco (feminino) na sílaba ``do'';
porém, o que nos resolve esta duvida é o acompanhamento percussivo, 
que claramente finaliza com uma nota em tempo forte;
pelo que deduzimos que as silabas ``bado'' estão sendo executadas juntas numa nota em tempo forte;
de modo que, podemos afirmar que o breque aconteceu em tempo forte.
\end{example}
Assim, para evitar estas duvidas na qualificação do tipo de breque,
é melhor predizer o breque usando a melodia principal,
e caraterizar-lho usando a posição das notas finais do acompanhamento harmônico ou percussivo.
%%%%%
\item \hyperref[ref:PontoCulminanteSuperior]{\textbf{Ponto culminante superior:}} 
É dizer a nota mais aguda da frase musical\footnote{O 
ponto culminante superior foi mencionado na Pag. \pageref{ref:PontoCulminanteSuperior}.}, 
é um bom indicador, porém não infalível, que um final de frase musical está próximo; 
consequentemente, se por outros fatores como acumulação de tensão,
sabemos que estamos perto de um breque, escutar esta nota mais aguda acentuará nossa certeza ao predizer o breque.

%\item Podemos detetar o final de uma frase pois, 
%ao ser interpretada como se fosse uma frase falada, 
%percebemos que uma ideia está chegando a seu final.

%\item Podemos aprender a reconhecer os tipos de \hyperref[sec:Cadencia]{\textbf{Cadência}},
%vistas na Seção \ref{sec:Cadencia}.
\end{itemize}



\begin{example}[Breques em músicas com final feminino]~
\label{ex:breakfeminino}
\begin{itemize}
\item ``Amanhã é sábado'' interpretado por Roberta Sá. 
O primeiro break é feminino.
\end{itemize}
\end{example}

\begin{example}[Breques em músicas com final sincopado]~
\label{ex:breaksincopados}
\begin{itemize}
\item ``Eu sou a marrom'' interpretado por Alcione.
\item ``Logo agora'' interpretado por Emílio Santiago.
\end{itemize}
\end{example}

