%%%%%%%%%%%%%%%%%%%%%%%%%%%%%%%%%%%%%%%%%%%%%%%%%%%%%%%%%%%%%%%%%%%%%%%%%%%%%%%%
%%%%%%%%%%%%%%%%%%%%%%%%%%%%%%%%%%%%%%%%%%%%%%%%%%%%%%%%%%%%%%%%%%%%%%%%%%%%%%%%
\section{Percepção da métrica no samba}
\index{Musicalidade!Métrica no samba}
\label{sec:percepcaosamba1}

As músicas dos subgêneros do samba, que usamos para dançar, 
geralmente são escritas usando \hyperref[subsec:compassobinario]{\textbf{compassos binários}};
pelo que, a procura da métrica das musicas destes subgêneros,
é mais simples, 
pois iniciamos a busca tendo a quase certeza do tipo do compasso, 
só precisaríamos \hyperref[subsec:perceberTF1]{\textbf{achar o tempo forte}} 
usando as indicações descritas na seção \ref{subsec:perceberTF1}.

Assim, para detetar o tempo forte e as \hyperref[sec:pos:Duracion]{\textbf{durações}} dos tempos nos compassos,
devemos ter em conta que quando escutamos uma música, 
na qual é tipicamente dançado o samba, ou especificamente o samba de gafieira,
podemos distinguir que a soma dos sons produzidos pelos instrumentos de percussão\footnote{Ou o
acompanhamento em geral.}, 
geram um padrão de repetição muito particular, 
geralmente relacionado com as onomatopeias: ``tchic tchic tum'' ou ``tum tum'';
onde em ambos  casos existe um  ``tum'' executado com maior 
\hyperref[sec:pos:Intensidade]{\textbf{intensidade}} (potencia sonora),
que está sendo executado-se no tempo forte.
Se conseguimos detetar alguma das duas onomatopeias,
então temos o problema resolvido, pois ambas ocupam exatamente dois tempos,
e dado que para conseguir encaixar as onomatopeias tivemos que reconhecer o tempo forte no ``tum'',
temos todos os elementos para descrever a métrica no tempo da música.

\PRLsep{Analisando a música no samba}

A Figura \ref{fig:abc-caquarela} representa os compassos 18, 19 e 20 da  
composição musical ``Aquarela do Brasil'' escrita
por Ary Barroso em 1939 \cite{AquarelaDoBrasil}; 
a versão mostrada na figura teve arranjos por Irineu Krüger \cite{Irineu}. 
\begin{figure}[ht]
\centering
%\includegraphics[width=\textwidth]{chapters/cap-fundamentos/aquarela.png}
\begin{abc}[name=abc-caquarela]%,options={-O= -c -s 0.8}]
% abcm2ps aquarela.abc  -O aquarela.ps
% ps2epsi aquarela.ps aquarela.eps
%
X: 1 % start of header
T: Brazil - Aquarela do Brasil
C: Music: Ary Barroso, 1939
C: Arranged by: Irineu Krüger
K: C % scale: C major
M: 2/4 % formula do compasso
%
V:1 clef=treble name="Voice Choir" sname="Voice Choir"
V:2 clef=treble name="Eb" sname="Eb"
V:3 clef=treble name="Bb" sname="Bb"
V:4 clef=treble name="Strings" sname="Strings"
V:5 clef=bass   name="D. Bass" sname=""D. Bass"
%
%
[V:1] "18" C'3/2A/2C2  |"19" A3/2(G/2 G/2)E1E/2  |"20" z/2 C'1A/2 C'1C'1  |
w:    Ó Bras-sil        sam-ba_ que dá       bam-bo-leio_ 
w:    Ó Bras-sil        ver-de que dá_       pa-ra~o mun-do 
%
%
[V:2] G1z/2G1z/2G1  | G1z/2G1z/2G1  | G1z/2G1z/2G1  |
%
%
[V:3] z4  | z4  | z4  |
%
%
[V:4] G1z/2G1z/2G1  | G1z/2G1z/2G1  | G1z/2G1z/2G1  |
%
%
[V:5] C,2 G,,2  | C,1 z1 G,,2  | C,2 G,,2  |
\end{abc}
\caption{3 compassos da partitura da composição ``Aquarela do brasil''}
\label{fig:abc-caquarela}
\end{figure}
Nesta versão, a música está escrita seguindo uma 
\hyperref[subsec:homofonica]{\textbf{textura homofônica}} com:
\begin{itemize}
\item \textbf{melodia principal:} 1 voz ou coro de voces (``Voice Choir'') e  
\item \textbf{acompanhamento:} 4 instrumentos (``Eb'',``Bb'',``Strings'' e ``D. Bass''), 
\end{itemize}
que usam uma 
formula de compasso $2/4$, de modo que se tem compassos
binários com tempos com uma \hyperref[sec:pos:Duracion]{\textbf{duração}} de uma semínima (\quarternote).

\subsection{Percepção do: tchic-tchic tum}
Analisando o fragmento de partitura da Figura \ref{fig:abc-caquarela} e escutando a música produzida, 
podemos perceber que os instrumentos executados em conjunto geram um sonido identificável
com a onomatopeia ``tchic tchic tum'', com uma duração de dois tempos.
Assim, o inicio de cada compasso coincide com o ``tum''; 
sendo que este é o momento em que a maioria dos instrumentos produzem um sonido, 
de modo que a sensação para o ouvinte é de uma potencia sonora maior. 
Cada instrumento prolongará seu sonido de forma diferente; 
porem,  podemos dizer que: o ``tum'' ocupa $1$ tempo (\quarternote), 
e que o sonido de um ``tchic'' ocupa médio tempo (0.5\quarternote),
sendo que o primeiro ``tchic'' é executado no tempo fraco de ``D. Bass'', 
e o segundo ``tchic'' solapa e obscurece ao  primeiro, 
que é executado na parte fraca do tempo fraco de ``Strings'' ou ``Eb'' (fazendo um \hyperref[sec:contratempo]{\textbf{contratempo}});
conseguindo assim criar a ilusão da onomatopeia ``tchic tchic tum'', 
com ``tchic''s de médio tempo; de modo que:
\begin{equation}
tchic + tchic = tum ~~ \Longleftrightarrow ~~ tchic = \frac{tum}{2}.
\end{equation}
 
Por outro lado, se a percepção do ouvinte é mais
aguçada, poderá escutar a onomatopeia: ``a tchic-tchic tum''; 
neste caso, o sonido ``tum'' é solapado por o sonido de ``a'',
quando transcorrido um $75\%$ do primeiro tempo do compasso; 
o sonido ``a''  se prolonga incluindo a parte forte do tempo fraco subsequente, 
este sonido é executado pelos instrumentos ``Eb'' e ``Strings'' e constitui uma 
\hyperref[sec:sincope]{\textbf{sincope}} \cite[pp. 143]{medteoria}.


Pelo exposto anteriormente, 
podemos simplificar o acompanhamento na partitura para gerar um sonido com onomatopeia
``tchic tchic tum'', como o mostrado na Figura \ref{fig:abc-contratempo1}.
\begin{figure}[ht]
\centering
\begin{abc}[name=abc-contratempo1,width=0.75\linewidth]
X: 1 % start of header
K: C % scale: C major
M:2/4
%T: Contratempo num compasso binário
V:1 clef=treble name="Instrumento 1" sname="Inst. 1"
V:2 clef=bass   name="Instrumento 2" sname="Inst. 2"
[V:1] |: " ""T/2"G1 " ""T/2"z1 " ""T/2"z1 " ""T/2"G1 | " ""T/2"G1 " ""T/2"z1 " ""T/2"z1 " ""T/2"G1  :|
w:    tum                tchic                       tum                   tchic           
[V:2] |:  "Tempo"C,2 "Tempo"G,,2  | "Tempo"C,2 "Tempo"G,,2  :|
w:    tum       tchic              tum       tchic            
\end{abc}
\caption{Padrão de repetição para gerar um sonido de onomatopeia ``tchic tchic tum''.}
\label{fig:abc-contratempo1}
\end{figure}
Onde o instrumento 1 executa dois sonidos, de modo que o primeiro contribui ao sonido 
``tum'' e o segundo sonido gera o segundo ``tchic'' do compasso; por outro lado,
o instrumento 2 executa um ritmo com um padrão
de repetição de dois sonidos ``tum'' e ``tchic'', nesse ordem;
sendo que a nota executada no tempo forte produz um sonido mais agudo que a 
executada no tempo fraco, isto é assim para poder diferenciar melhor ambos tempos.

\begin{figure}[!h]
\begin{elaboracion}[title=Samba: Ritmo vs. Fala]
\index{Música!Ritmo vs. Fala}
Quando reconhecemos o ritmo do samba com a onomatopeia ``tchic tchic tum'', 
podemos perceber que existe uma contradição, entre 
o que percebemos ao escutar uma música, e a forma como esta é realmente escrita nos compassos na partitura.
Pois como é visto na Figura \ref{fig:abc-contratempo1}, quando escrevemos
um ritmo com um padrão de repetição na ordem ``tum tchic tchic'', 
para o ouvinte é mais natural associar e pensar
que se está executando um ritmo com um padrão ``tchic tchic tum'', 
devido a que quando um ser humano fala, este usa a pausa
para denotar o final de uma palavra. Assim, ao escutar o padrão ``tum tchic tchic'', 
nossa mente otimizada para detetar palavras,
associa que o sonido que tem um silencio maior apos ser executado,
neste caso o ``tum'', marca o final do ciclo do padrão de repetição (a palavra). 

Pelo que quando um músico vê, ao ler uma partitura, 
um padrão de repetição ``tum tchic tchic''; 
um ouvinte interpretará de forma instintiva que o padrão é ``tchic tchic tum'' iniciando em ``tchic''.
Porem não devemos confundir-nos, o ``tum'' é executado no tempo 1 do compasso.
\end{elaboracion}
\label{fig:RitmoVsFala}
\end{figure}

\subsection{Percepção do: tum~tum}

Analisando o fragmento de partitura, da Figura \ref{fig:abc-caquarela}, 
e tentando escutar os compassos, 
para audivelmente isolar o instrumento ``D. Bass'' (que é o bumbo),
podemos perceber que este gera um sonido, 
com um \hyperref[sec:pos:Altura]{\textbf{tom}} muito grave, 
identificável com a onomatopeia ``tum tum'', com uma duração de dois tempos.
No samba este ritmo é muito identificável na maioria das musicas,
e pode servi-nos de referencia à hora de dançar ou simplesmente tentar achar o pulso.
Este instrumento é descrito isoladamente na Figura \ref{fig:abc-contratempo1tumtum},
onde podemos apreciar que o compositor achou interessante diferenciar o tom,
 do som que gerava o bumbo, em cada tempo;
neste caso, um som mais agudo para o tempo forte e um mais grave para o tempo fraco;
porem esta diferença não é uma regra, pelo que se estamos procurando achar os tempos da música escutando o bumbo,
o melhor é simplesmente tentar perceber qual está sendo tocado com maior \hyperref[sec:pos:Intensidade]{\textbf{intensidade}}.
\begin{figure}[ht]
\centering
\begin{abc}[name=abc-contratempo1tumtum,width=0.75\linewidth]
X: 1 % start of header
K: C % scale: C major
M:2/4
%T: Contratempo num compasso binário
V:1 clef=bass   name="D. Bass" sname="D. Bass"      
[V:1] |: "Tempo"C,2 "Tempo"G,,2  | "Tempo"C,2 "Tempo"G,,2  :|
w:    tum       tum         tum       tum            
\end{abc}
\caption{Padrão de repetição para gerar um sonido de onomatopeia ``tum tum''.}
\label{fig:abc-contratempo1tumtum}
\end{figure}

