\chapterimage{chapter_head_2.pdf} % Chapter heading image

\chapter{\textcolor{blue}{Fundamentos de musicalidade na dança}}

\begin{definition}[Musicalidade:] 
\index{Musicalidade}
\label{def:Musicalidade}
O Dicionário Online de Português define musicalidade como \cite{diciomusicalidade}:
\begin{itemize}
\item Particularidade, característica ou estado do que é musical.
\item Tendência natural, sensibilidade ou talento para criar ou tocar música.
\item Sensibilidade para contemplar música; conhecimento sobre música.
\item A demonstração do talento musical de uma pessoa.
\end{itemize}
\end{definition}

\section{Musicalidade, sentir ou entender a música?}
Seguindo a Definição \ref{def:Musicalidade}, podemos inferir como entender a musicalidade na dança.
Esta acontece, quando o dançarino tem um estado de ``sensibilidade'' ou ``conhecimento'' para contemplar ou entender a música,
e assim quando este dança, ``demostrar'' uma coerência entre a música e o que está dançando.

Porem temos um choque de paradigmas, na definição, para atingir um mesmo objetivo que é ser musical.
Podemos sentir ou conhecer; o que é equivalente a dizer que:
\begin{itemize} 
\item podemos saber, por intuição sem entender porquê, ou
\item podemos saber, entendendo mediante o conhecimento adquirido pelo raciocínio e o estudo.\\
\end{itemize}



Neste sentido, para adquirir esta coerência com a música, 
muitos professores indicam que o dançarino deve ``escutar e sentir a música'';
nesse aspecto, argumento que a frase é metaforicamente correta e coincide em parte com a Definição \ref{def:Musicalidade};
porem, pedagogicamente  é pouco favorável para o estudante.
Assim, eu ressaltaria que a musicalidade a nível de ensino se adquire,
estudando e entendendo a música e não só sentindo-a.
 
Para fundamentar minha argumentação, acredito interessante expor o seguinte exemplo: 
\begin{example}
Imaginemos que conhecemos ao matemático ``Srinivasa Aiyangar Ramanujan'';
um prodígio matemático autodidata \cite[pp. 1]{kanigel2016man}, indiano, que 
realizou muitas contribuições à matemática.
E mostramos a ele um problema matemático, como por exemplo uma equação,
e  lhe pedimos a resposta ou solução. 
Então Srinivasa, muito amavelmente, 
observaria um instante o problema e nos daria a solução imediatamente.
Nós surpreendidos pela velocidade e o mínimo esforço na resposta,
perguntaríamos. Como você obteve a resposta? Então ele responderia \cite[pp. 235]{kanigel2016man}: 
%\begin{citando}
%Immediately I heard the problem 
%it was clear that the solution should obviously be a continued fraction; 
%I then thought, Which continued  fraction? And the answer came to my mind.
%\end{citando}
\begin{citando}
No momento em que escutei o problema, 
foi claro pra mim que a resposta devia ser obviamente uma fração continua; 
E então pensei, ¿Qual fração continua? e a resposta chegou a minha mente. 
\end{citando}
\end{example}

A resposta que ele deu, 
é a mesma  que dão as pessoas, quando  dizem que para ter musicalidade ele simplesmente sentiu a música. 
Assim, esta aproximação ao problema é só válida ou eficiente, para pessoas como Srinivasa, 
que talvez tenham uma inspiração divina, 
ou que já nasceram com esse dom ou que por uma longa experiencia de vida, 
tem implementado por ``hardware'', no cérebro, entender a música; 
pelo que eles usam a palavra sentir, 
pois conhecem a resposta, porem não sabem como sabem. 

Isto também pode acontecer com pessoas que observaram muito tempo um ``problema'' ou escutaram muito uma ``canção'', 
e um dia conseguiram ``sentir'' a resposta. Na minha opinião não todos nascemos, 
com esse componente, implementado em nosso cérebro, para entender  a música; 
e não podemos nos dar o luxo de escutar uma canção indefinidamente ate sentir algo; 
o mais eficiente seria estudar música, 
e entender esta baseando-nos em nossa teoria e cruzando esta informação com o que escutamos, 
os padrões observados, a melodia, o ritmo, etc. 
Assim, nos podemos criar por ``software'' o que não temos implementado por ``hardware'', 
derrubando o mito de ``sentir'' a música e passar a ``entender'' ela.

%%%%%%%%%%%%%%%%%%%%%%%%%%%%%%%%%%%%%%%%%%%%%%%%%%%%%%%%%%%%%%%%%%%%%%%%%%%%%%%%
%%%%%%%%%%%%%%%%%%%%%%%%%%%%%%%%%%%%%%%%%%%%%%%%%%%%%%%%%%%%%%%%%%%%%%%%%%%%%%%%
\section{\textcolor{blue}{Percepção rítmica do ouvinte}}
\label{sec:percepcaoouvinte}
Quando escutamos uma música, na qual é tipicamente dançado samba de gafieira,
podemos distinguir que a soma dos sonidos produzidos pelos instrumentos realizam 
um padrão de repetição muito particular, geralmente ligado à onomatopeia: ``Chic Chic Tum'' ou ``tum Tum''.


A Figura \ref{fig:abc-caquarela} representa os compassos 18, 19 e 20 da  
composição musical ``Aquarela do Brasil'' escrita
por Ary Barroso em 1939 \cite{AquarelaDoBrasil}; 
a versão mostrada na figura teve arranjos por Irineu Krüger \cite{Irineu}. 
Nesta versão, a música está representada com 1 voz ou coro de voces (``Voice Choir'') e 4 
instrumentos (``Eb'',``Bb'',``Strings'' e ``D. Bass''), que usam uma 
formula de compasso $2/4$, de modo que cada compasso
é binário e
pode ser preenchido usando duas semínimas (2\quarternote).
\begin{figure}[ht]
\centering
%\includegraphics[width=\textwidth]{chapters/cap-fundamentos/aquarela.png}
\begin{abc}[name=abc-caquarela]
% abcm2ps aquarela.abc  -O aquarela.ps
% ps2epsi aquarela.ps aquarela.eps
%
X: 1 % start of header
T: Brazil - Aquarela do Brasil
C: Music: Ary Barroso, 1939
C: Arranged by: Irineu Krüger
K: C % scale: C major
M: 2/4 % formula do compasso
%
V:1 clef=treble name="Voice Choir" sname="Voice Choir"
V:2 clef=treble name="Eb" sname="Eb"
V:3 clef=treble name="Bb" sname="Bb"
V:4 clef=treble name="Strings" sname="Strings"
V:5 clef=bass   name="D. Bass" sname=""D. Bass"
%
%
[V:1] "18" C'3/2A/2C2  |"19" A3/2(G/2 G/2)E1E/2  |"20" z/2 C'1A/2 C'1C'1  |
w:    Ó Bras-sil        sam-ba_ que dá       bam-bo-leio_ 
w:    Ó Bras-sil        ver-de que dá_       pa-ra~o mun-do 
%
%
[V:2] G1z/2G1z/2G1  | G1z/2G1z/2G1  | G1z/2G1z/2G1  |
%
%
[V:3] z4  | z4  | z4  |
%
%
[V:4] G1z/2G1z/2G1  | G1z/2G1z/2G1  | G1z/2G1z/2G1  |
%
%
[V:5] C,2 G,,2  | C,1 z1 G,,2  | C,2 G,,2  |
\end{abc}
\caption{3 compassos da partitura da composição ``Aquarela do brasil''}
\label{fig:abc-caquarela}
\end{figure}

\subsection{Percepção do: Chic Chic Tum}
Analisando este fragmento de partitura e escutando a música produzida, 
podemos perceber que os instrumentos executados em conjunto geram um sonido identificável
com a onomatopeia ``Chic Chic Tum''.
Assim, o inicio de cada compasso coincide com o ``Tum''; 
sendo que este é o momento em que a maioria dos instrumentos produzem um sonido, 
de modo que a sensação para o ouvinte é de uma potencia sonora maior. 
Cada instrumento prolongará seu sonido de forma diferente; 
porem,  podemos dizer que: o ``Tum'' ocupa $1$ tempo (\quarternote), 
e que cada ``Chic'' ocupa médio tempo (0.5\quarternote),
sendo que o primeiro ``Chic'' é executado no tempo fraco de ``D. Bass'', 
e o segundo ``Chic'' solapa e obscurece ao  primeiro, 
sendo executado na parte fraca do tempo fraco de ``Strings'' ou ``Eb'' (é dizer, fazem contratempos);
conseguindo assim criar a ilusão do ``Chic Chic Tum'', com ``Chic''s de médio tempo ; de modo que:
\begin{equation}
Chic + Chic = Tum ~~ \Longleftrightarrow ~~ Chic = \frac{Tum}{2}.
\end{equation}
 
Por outro lado, se a percepção do ouvinte é mais
aguçada, poderá escutar ``a Chic Chic Tum''; 
neste caso, o sonido ``Tum'' é solapado por o sonido de ``a'',
quando transcorrido um $75\%$ do primeiro tempo do compasso; 
o sonido ``a''  se prolonga incluindo a parte forte do tempo fraco subsequente, 
este sonido é executado pelos instrumentos ``Eb'' e ``Strings'' e constitui uma sincopa \cite[pp. 143]{medteoria}.


Pelo exposto anteriormente, agora podemos simplificar a partitura para gerar um sonido com onomatopeia
``Chic Chic Tum'', como é mostrado na Figura \ref{fig:abc-contratempo1}.
Assim,
o instrumento 1 executa dois sonidos, de modo que o primeiro contribui ao sonido 
``Tum'' e o segundo sonido gera o segundo ``Chic'' do compasso; por outro lado,
o instrumento 2 executa um ritmo com um padrão
de repetição de dois sonidos ``Tum'' e ``Chic'', nesse ordem;
sendo que a nota executada no tempo forte produz um sonido mais agudo que a 
executada no tempo fraco, isto é assim para poder diferenciar melhor ambos tempos.
\begin{figure}[ht]
\centering
\begin{abc}[name=abc-contratempo1,width=0.6\linewidth]
X: 1 % start of header
K: C % scale: C major
M:2/4
%T: Contratempo num compasso binário
V:1 clef=treble name="Instrumento 1" sname="Inst. 1"
V:2 clef=bass   name="Instrumento 2" sname="Inst. 2"
[V:1] |: " ""T/2"G1 " ""T/2"z1 " ""T/2"z1 " ""T/2"G1 | " ""T/2"G1 " ""T/2"z1 " ""T/2"z1 " ""T/2"G1  :|
w:    Tum                     Chic                  Tum                   Chic           
[V:2] |:  "Tempo"C,2 "Tempo"G,,2  | "Tempo"C,2 "Tempo"G,,2  :|
w:    Tum       Chic         Tum       Chic            
\end{abc}
\caption{Padrão de repetição para gerar um sonido de onomatopeia ``Chic Chic Tum''.}
\label{fig:abc-contratempo1}
\end{figure}

Conhecido tudo isto, é fácil perceber como existe uma diference entre 
o que percebemos ao escutar uma música e a forma como esta é escrita na partitura;
pois como é visto na Figura \ref{fig:abc-contratempo1}, quando escrevemos
um sonido com um padrão de repetição na ordem ``Tum Chic Chic'', para o ouvinte é mais natural associar
este sonido com o padrão ``Chic Chic Tum'', devido a que \textbf{quando um ser humano fala, este usa a pausa
para denotar o final de uma palavra}. Da mesma forma, ao escutar uma música, traduzimos
que o sonido que tem um silencio maior apos ser executado marca o final do ciclo
do padrão de repetição. Assim, o que um músico vê ao ler uma partitura
é um padrão de repetição ``Tum Chic Chic'', sendo que  um
ouvinte interpretará de forma instintiva que o padrão é ``Chic Chic Tum ''.

\subsection{\textcolor{red}{Percepção do: tum Tum}}

Analisando o fragmento de partitura, na Figura \ref{fig:abc-caquarela}, 
e tentando issolar o instrumento ``D. Bass'',
podemos perceber que este gera um sonido identificável com a onomatopeia ``tum Tum''.
Podemos ver na Figura \ref{fig:abc-contratempo1tumtum} estre instrumento isoladamente.
\begin{figure}[ht]
\centering
\begin{abc}[name=abc-contratempo1tumtum,width=0.5\linewidth]
X: 1 % start of header
K: C % scale: C major
M:2/4
%T: Contratempo num compasso binário
V:1 clef=bass   name="D. Bass" sname="D. Bass"      
[V:1] |: "Tempo"C,2 "Tempo"G,,2  | "Tempo"C,2 "Tempo"G,,2  :|
w:    Tum       tum         Tum       tum            
\end{abc}
\caption{Padrão de repetição para gerar um sonido de onomatopeia ``tum Tum''.}
\label{fig:abc-contratempo1tumtum}
\end{figure}

%%%%%%%%%%%%%%%%%%%%%%%%%%%%%%%%%%%%%%%%%%%%%%%%%%%%%%%%%%%%%%%%%%%%%%%%%%%%%%%%
%%%%%%%%%%%%%%%%%%%%%%%%%%%%%%%%%%%%%%%%%%%%%%%%%%%%%%%%%%%%%%%%%%%%%%%%%%%%%%%%
\section{\textcolor{green}{Contagem de tempos correográficos}}

falar da possibilidade de contar 123 567 sim se usa um nome diferente a contamgem de tempo musical,
exemplo, tempos coreograficos



\section{\textcolor{green}{Contagens dos passos para o ensino}}
Antes de iniciar esta seção é importante mencionar uma
problemática que é vista com muita frequência nas escolas de dança; 
esta é gerada devido a que: A forma em que os tempos são contados 
na música, é
diferente à realizada entre profissionais da música e da dança. 
Sendo que a contagem dos profissionais da música segue a \hyperref[def:Metrica]{\textbf{métrica}} indicada na partitura,
e no caso de profissionais da dança segue geralmente um enfoque 
particular a cada escola de dança, visando só em muitos casos o fácil entendimento do aluno da
execução do movimento programado para essa aula, e não uma rigorosidade teórica no uso de termos e 
expressões musicais.




\subsection{Contagem de 3 passos em 2 tempos}
Esta diferença na forma de perceber o inicio e o final do ciclo de repetição,
vista na Seção \ref{sec:percepcaoouvinte}, 
leva a um problema quando se quer ser rigoroso na forma de contar os tempos nos compassos; 
por exemplo, na Tabela \ref{tab:ritmo1} 
podemos ver 4 formas distintas, que podem adotar as pessoas, 
para contar os tempos nos compassos indicando a distribuição de tempos, 
onde ``$T$'' representa um tempo do compasso.
\begin{table}[ht]
  \centering
  \begin{tabular}    {c|ccc|c}
    \hline
    Tipos de contagem       & $T/2$ & $T/2$   & $T$ (Forte) & Recomendável?\\
    \hline
    Contagem 1: & Chic  & Chic  & Tum   & Sim\\
    Contagem 2: & 2     & e     & 1     & Sim\\ \hline
    Contagem 3: & Con   & tra  & Tempo & Não\\
    Contagem 4: & 1     & e     & 2     & Não\\  \hline
    Contagem 5: & 1     & 2     & 3     & Depende\\ \hline
    \hline
  \end{tabular}
  \caption{Tipos de contagem na samba de gafieira.}
\label{tab:ritmo1}
\end{table}

As formas de contagem que recomendo são:
\begin{itemize}
\item \textbf{A contagem 1}, 
devido que a principio, pode ser usada sem aprofundar demasiado 
na notação musical, de modo que só precisa ser explicado que a duração de um 
``Tum'' é o dobro que um ``Chic'', e anexar que tipicamente veremos que o ``Tum''
acontece no tempo 1 do compasso; 
%de modo que outra contagem valida seria ``Tum Chic Chic''; 
porem, a contagem 1 não está restrita ao uso destas silabas (``Chic'' e ``Tum''), 
em geral esta contagem representa a quase qualquer padrão de repetição
que use duas silabas diferentes, como por exemplo os padrões: ``Ta Ta Kum'', ``Tic Tic Pa'', etc. 
Este tipo de contagem já é muito usada na pratica e na literatura, pois 
podemos achar variantes como ``Quick-Quick Slow'' (Rápido-Rápido Lento no idioma inglês)
ou ``Tic Tic Tum'' seguindo a notação usada por Perna no seu livro sobre samba de gafieira \cite[pp. 146]{perna2002samba}.
\item \textbf{A contagem 2}, segue a notação de tempos na partitura, este tipo de
contagem é coerente com a musica, porem precisa de uma major explicação, 
para pessoas não iniciadas na musica e a dança. Porem, isto não quer dizer que seu
entendimento seja complexo, e sim que precisa um investimento em horas de aula
um pouco major que a contagem 1.
Mesmo assim, devemos ter cuidado pois pode-se dar o caso que a partitura não tenha compassos binários 
e sim quaternários, com contagens ``2 e 3'' ``4 e 1'', 
criando este tipo de contagem mais caminhos onde podemos perder coerência com a contagem na partitura.\\
\end{itemize}


Entre as contagens que não recomendo estão:
\begin{itemize}
\item \textbf{A contagem 3} (``Con-tra Tempo''), 
devido a que o uso deste padrão pode confundir às pessoas que desconhecem 
a definição formal do termo contratempo \cite[pp. 16]{mascarenhascurso} \cite[pp. 36]{azevedocompor}, 
e levar a confusão de achar que um contratempo é só uma distribuição de 3 tempos, 
sendo um o dobro dos outros dois, em termos de tempos execução.
\item \textbf{A contagem 4} é não recomendada, devido a que como é visto nas Figuras 
\ref{fig:abc-caquarela} e \ref{fig:abc-contratempo1}, musicalmente a contagem estaria invertida,
dado que tipicamente o ``Tum'' se execute no tempo 1.\\
\end{itemize}

Finalmente, a contagem que precisa um cuidado especial:
\begin{itemize}

\item \textbf{A contagem 5} precisa ser bem explicada, 
devido a que não segue as notações musicais; porem, seu uso pode ser resgatado,
se fazemos em todo instante uma aclaração, que se trata de ``tempos coreográficos''.
Assim, ficara claro para o estudante, que estes não correspondem necessariamente, 
com os tempos musicais que também devem ser ensinados.

%não é recomendada, 
%por motivos similares aos apresentados para a contagem 4. 
%Além do fato que os números atribuídos estão distantes da
%notação verdadeira na partitura, mesmo sim esta houvesse sido escrita num compasso quaternário.
\end{itemize}

\subsection{\textcolor{red}{Contagem de 2 passos em 2 tempos}}


Tabela \ref{tab:ritmoconta2}

\begin{table}[ht]
  \centering
  \begin{tabular}    {c|cc|c}
    \hline
    Tipos de contagem       & $T$ (fraco)  & $T$ (Forte)& Recomendável?\\
    \hline
    Contagem 1: & tum   & Tum   & Sim\\
    Contagem 2: & 2     & 1     & Sim\\
    Contagem 3: & tempo & Tempo & Sim\\ \hline
    Contagem 4: & 1     & 2     & Não\\ \hline
    Contagem 5: & 1     & 3     & Depende\\  \hline
    \hline
  \end{tabular}
  \caption{Tipos de contagem na samba de gafieira.}
\label{tab:ritmoconta2}
\end{table}

%%%%%%%%%%%%%%%%%%%%%%%%%%%%%%%%%%%%%%%%%%%%%%%%%%%%%%%%%%%%%%%%%%%%%%%%%%%%%%%%
\section{\textcolor{red}{Procurando um bom ``Timing''}}
Procurando o momento certo

 sincronização [https://www.infopedia.pt/dicionarios/ingles-portugues/timing]
 sentido de oportunidade [https://www.infopedia.pt/dicionarios/ingles-portugues/timing]
%%%%%%%%%%%%%%%%%%%%%%%%%%%%%%%%%%%%%%%%%%%%%%%%%%%%%%%%%%%%%%%%%%%%%%%%%%%%%%%%
\section{\textcolor{red}{Articulação das notas musicais vs. dança}}
Articulação das notas musicais refletido na nossa dança

Legato vs. sttacato

%%%%%%%%%%%%%%%%%%%%%%%%%%%%%%%%%%%%%%%%%%%%%%%%%%%%%%%%%%%%%%%%%%%%%%%%%%%%%%%%
\section{\textcolor{red}{Procurando o tempo forte da música}}
Tempo forte ou tempo 1

simples
\begin{itemize}
\item O tempo em que estatisticamente percebemos com maior potencia sonora. 
\item Se conseguimos identificar audivelmente um padrão de repetição ``Chick-Chick Tum'', o tum é o tempo forte
\item O tempo em que estatisticamente percebemos que o cantor  coloca o acento da palavra \cite[pp. 149]{medteoria}. 
\end{itemize}

complexos 
\begin{itemize}
\item Se percebemos um ``break'' da música com final de frase musical conclusivo 
(satisfatório, com uma sensação de ponto aparte), então este aconteceu no tempo forte.
\end{itemize}

%%%%%%%%%%%%%%%%%%%%%%%%%%%%%%%%%%%%%%%%%%%%%%%%%%%%%%%%%%%%%%%%%%%%%%%%%%%%%%%%
\section{\textcolor{red}{Dançando no tempo forte da música}}

Que significa dançar no tempo forte? 
pisar o tum  no tempo forte.

Se percebo que estou pisando no fraco como corrigir?
podemos usar
\begin{itemize}
\item Caminhada em contratempo
\item Fazemos balaços um número impar de vezes.
\end{itemize}


%%%%%%%%%%%%%%%%%%%%%%%%%%%%%%%%%%%%%%%%%%%%%%%%%%%%%%%%%%%%%%%%%%%%%%%%%%%%%%%%
\section{\textcolor{red}{Percebendo frases musicais}}

\subsection{\textcolor{red}{percebendo frases com final conclusivo}}

\subsection{\textcolor{red}{percebendo frases com final suspensivo}}
\subsubsection{\textcolor{red}{percebendo frases com final suspensivo a contratempo e sincopado}}

%%%%%%%%%%%%%%%%%%%%%%%%%%%%%%%%%%%%%%%%%%%%%%%%%%%%%%%%%%%%%%%%%%%%%%%%%%%%%%%%
\section{\textcolor{red}{Contando e medindo a frase musical}}
Provavelmente 4
\begin{itemize}
\item 4 compassos
\item 8 compassos
\item 2 compassos
\item 16 compassos
\end{itemize}

%%%%%%%%%%%%%%%%%%%%%%%%%%%%%%%%%%%%%%%%%%%%%%%%%%%%%%%%%%%%%%%%%%%%%%%%%%%%%%%%
\section{\textcolor{red}{Frase musical, Rap e musicalizar poesias}}
% https://scielo.conicyt.cl/scielo.php?script=sci_arttext&pid=S0719-32622018000100030
Figura \ref{rap:emocional-protesto1}

\begin{figure}[H]
\centering
\begin{abc}[name=abc-emocional-protesto1]
X: 1 % start of header
K: C stafflines=1 % scale: C major
M: 2/4 %meter - compasso
Q:1/4=80
V:1 clef=perc stem=up %name="Pauta com clave de fá"   sname="Pauta com clave de fá"
[V:1] |:!>!B3/2 B/2 B1 B1| B3/2 B/2 B1 B1 | B2 B2| B2 B1 B1:|
\end{abc}
\caption{Frase de 8 tempos.}
\label{rap:emocional-protesto1}
\end{figure}

Figura \ref{rap:emocional-protesto2}

\begin{figure}[H]
\centering
\begin{abc}[name=abc-emocional-protesto2]
X: 1 % start of header
K: C stafflines=1 % scale: C major
M: 2/4 %meter - compasso
Q:1/4=80
V:1 clef=perc stem=up %name="Pauta com clave de fá"   sname="Pauta com clave de fá"
[V:1] |:!>!B3/2 B/2 B1 B1| B3/2 B/2 B1 B1 | B2 B2| B2 z2:|
\end{abc}
\caption{Frase de 8 tempos.}
\label{rap:emocional-protesto2}
\end{figure}

\begin{citando}
Vida simples, metáfora de um corpo;\\
luzes cobertas, as lágrimas expostas.\\
vida simples, luta contra o tempo;\\
brilha, existe, e terás respostas.\\
\end{citando}


Tabela \ref{tab:verso1}

\begin{table}[h!]
\begin{center}
\begin{tabular}{|l||l||l||l|} % 
\hline
compasso 1 & compasso 2   & compasso 3   & compasso 4 \\ \hline \hline
vida       & simples, me- & táfora de um & corpo,  as \\ \hline
luzes  co- & bertas, as   & lágrimas  ex & postas     \\ \hline
\end{tabular}
\caption{Verso 1.}
\label{tab:verso1}
\end{center}
\end{table}


%%%%%%%%%%%%%%%%%%%%%%%%%%%%%%%%%%%%%%%%%%%%%%%%%%%%%%%%%%%%%%%%%%%%%%%%%%%%%%%%
\section{\textcolor{red}{Percebendo e usando o ``break'' da música}}




%%%%%%%%%%%%%%%%%%%%%%%%%%%%%%%%%%%%%%%%%%%%%%%%%%%%%%%%%%%%%%%%%%%%%%%%%%%%%%%%
\section{\textcolor{red}{Seguindo isoladamente os instrumentos}}
\begin{itemize}
\item Dançando choro-chorinho na melodia esquecendo o ``Chick-Chick Tum''.
\item Trabalhando com marionetas um em cada mão, ou dedo.
\item Cada aluno simula que tem um instrumento virtual e toca ele.
\end{itemize}

%%%%%%%%%%%%%%%%%%%%%%%%%%%%%%%%%%%%%%%%%%%%%%%%%%%%%%%%%%%%%%%%%%%%%%%%%%%%%%%%
\section{\textcolor{red}{Mickey mousing}}

