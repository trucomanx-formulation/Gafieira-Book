\chapterimage{chapter_head_2.pdf} % Chapter heading image

\chapter{\textcolor{blue}{Fundamentos de musicalidade na dança}}

%%%%%%%%%%%%%%%%%%%
\section{\textcolor{green}{Contagem de tempos }}
\label{sec:contagemtempos}
como contar em \cite[pp. 8]{phillips2002sight}

Figura \ref{fig:contartempos24} 

\begin{figure}[h]
    \centering
 \begin{abc}[name=abc-contartempos24]
% abcm2ps contartempos24.abc  -O contartempos24.ps
% ps2epsi contartempos24.ps contartempos24.eps
%
X: 1 % start of header
K: none stafflines=0 %K: C %% Escala de C mayor %
M:  2/4
%T: Contratempo num compasso binário
V:1 clef=none stem=up name="Ritmo"   sname="Ritmo"
%
[V:1] | "1"B2 z2  |"1"B2 "2"B2  | z2 "2"B2  |"1"B4  |"1"B2 "2"B2  |"1"B2 "2"B2  |
%       
\end{abc}
    \caption{Sequencia rítmica em 2/4.}\label{fig:contartempos24}
\end{figure}


Figura \ref{fig:contartempos34} 

\begin{figure}[h]
    \centering
 \begin{abc}[name=abc-contartempos34]
% abcm2ps contartempos34.abc  -O contartempos34.ps
% ps2epsi contartempos34.ps contartempos34.eps
%
X: 1 % start of header
K: none stafflines=0 %K: C %% Escala de C mayor %
M:  3/4
%T: Contratempo num compasso binário
V:1 clef=none stem=up name="Ritmo"   sname="Ritmo"
%
[V:1] | "1"B4 "3"B2 |"1"B2 "2"B2 "3"B2  | z2 "2"B2 "3"B2  |"1"B2 "2"B4  |
%       
\end{abc}
    \caption{Sequencia rítmica em 3/4.}\label{fig:contartempos34}
\end{figure}


%%%%%%%%%%%%%%%%%%%%%%%%%%%%%%%%%%%%%%%%%%%%%%%%%%%%%%%%%%%%%%%%%%%%%%%%%%%%%%%%
\section{\textcolor{green}{Contagem de tempos correográficos}}

Figura \ref{fig:contarfrase1} 

\begin{figure}[h]
    \centering
 \begin{abc}[name=abc-contarfrase1]
% abcm2ps contarfrase1.abc  -O contarfrase1.ps
% ps2epsi contarfrase1.ps contarfrase1.eps
%
X: 1 % start of header
K: none stafflines=0 %K: C %% Escala de C mayor %
M:  2/4
%T: Contratempo num compasso binário
V:1 clef=none stem=up name="Frase"   sname="Frase"
%
[V:1] | "1"B2 "2"z2  |"3"B2 "4"B2  | "5"z2 "6"B2  |"7-8"B4  |
%       
\end{abc}
    \caption{Sequencia rítmica em 2/4.}\label{fig:contarfrase1}
\end{figure}

contando compassos Figura \ref{fig:contarfrase2} 

\begin{figure}[h]
    \centering
 \begin{abc}[name=abc-contarfrase2]
% abcm2ps contarfrase2.abc  -O contarfrase2.ps
% ps2epsi contarfrase2.ps contarfrase2.eps
%
X: 1 % start of header
K: none stafflines=0 %K: C %% Escala de C mayor %
M:  2/4
%T: Contratempo num compasso binário
V:1 clef=none stem=up name="Frase"   sname="Frase"
%
[V:1] | "1"B2 z2  |"2"B2 B2  | "3"z2 B2  |"4"B4  |
%       
\end{abc}
    \caption{Sequencia rítmica em 2/4.}\label{fig:contarfrase2}
\end{figure}

\section{\textcolor{green}{Contagens dos passos para o ensino}}
Antes de iniciar esta seção é importante mencionar uma
problemática que é vista com muita frequência nas escolas de dança; 
esta é gerada devido a que: A forma em que os tempos são contados 
na música, é
diferente à realizada entre profissionais da música e da dança. 
Sendo que a contagem dos profissionais da música segue as normas
e notações indicadas na partitura, como a contagem vista na Seção \ref{sec:contagemtempos}, 
e no caso de profissionais da dança segue geralmente um enfoque 
particular a cada escola de dança, visando só em muitos casos o fácil entendimento do aluno da
execução do movimento programado para essa aula, e não uma rigorosidade teórica no uso de termos e 
expressões musicais.


\subsection{\textcolor{blue}{Percepção rítmica do ouvinte}}
\label{subsec:percepcaoouvinte}
Quando escutamos uma música, na qual é tipicamente dançado samba de gafieira,
podemos distinguir que a soma dos sonidos produzidos pelos instrumentos realizam 
um padrão de repetição muito particular, geralmente ligado à onomatopeia: ``Chic Chic Tum'' ou ``tum Tum''.


A Figura \ref{fig:abc-caquarela} representa os compassos 18, 19 e 20 da  
composição musical ``Aquarela do Brasil'' escrita
por Ary Barroso em 1939 \cite{AquarelaDoBrasil}; 
a versão mostrada na figura teve arranjos por Irineu Krüger \cite{Irineu}. 
Nesta versão, a música está representada com 1 voz ou coro de voces (``Voice Choir'') e 4 
instrumentos (``Eb'',``Bb'',``Strings'' e ``D. Bass''), que usam uma 
formula de compasso $2/4$, de modo que cada compasso
é binário e
pode ser preenchido usando duas semínimas (2\quarternote).
\begin{figure}[ht]
\centering
%\includegraphics[width=\textwidth]{chapters/cap-fundamentos/aquarela.png}
\begin{abc}[name=abc-caquarela]
% abcm2ps aquarela.abc  -O aquarela.ps
% ps2epsi aquarela.ps aquarela.eps
%
X: 1 % start of header
T: Brazil - Aquarela do Brasil
C: Music: Ary Barroso, 1939
C: Arranged by: Irineu Krüger
K: C % scale: C major
M: 2/4 % formula do compasso
%
V:1 clef=treble name="Voice Choir" sname="Voice Choir"
V:2 clef=treble name="Eb" sname="Eb"
V:3 clef=treble name="Bb" sname="Bb"
V:4 clef=treble name="Strings" sname="Strings"
V:5 clef=bass   name="D. Bass" sname=""D. Bass"
%
%
[V:1] "18" C'3/2A/2C2  |"19" A3/2(G/2 G/2)E1E/2  |"20" z/2 C'1A/2 C'1C'1  |
w:    Ó Bras-sil        sam-ba_ que dá       bam-bo-leio_ 
w:    Ó Bras-sil        ver-de que dá_       pa-ra~o mun-do 
%
%
[V:2] G1z/2G1z/2G1  | G1z/2G1z/2G1  | G1z/2G1z/2G1  |
%
%
[V:3] z4  | z4  | z4  |
%
%
[V:4] G1z/2G1z/2G1  | G1z/2G1z/2G1  | G1z/2G1z/2G1  |
%
%
[V:5] C,2 G,,2  | C,1 z1 G,,2  | C,2 G,,2  |
\end{abc}
\caption{3 compassos da partitura da composição ``Aquarela do brasil''}
\label{fig:abc-caquarela}
\end{figure}

\subsubsection{Percepção do: Chic Chic Tum}
Analisando este fragmento de partitura e escutando a música produzida, 
podemos perceber que os instrumentos executados em conjunto geram um sonido identificável
com a onomatopeia ``Chic Chic Tum''.
Assim, o inicio de cada compasso coincide com o ``Tum''; 
sendo que este é o momento em que a maioria dos instrumentos produzem um sonido, 
de modo que a sensação para o ouvinte é de uma potencia sonora maior. 
Cada instrumento prolongará seu sonido de forma diferente; 
porem,  podemos dizer que: o ``Tum'' ocupa $1$ tempo (\quarternote), 
e que cada ``Chic'' ocupa médio tempo (0.5\quarternote),
sendo que o primeiro ``Chic'' é executado no tempo fraco de ``D. Bass'', 
e o segundo ``Chic'' solapa e obscurece ao  primeiro, 
sendo executado na parte fraca do tempo fraco de ``Strings'' ou ``Eb'' (é dizer, fazem contratempos);
conseguindo assim criar a ilusão do ``Chic Chic Tum'', com ``Chic''s de médio tempo ; de modo que:
\begin{equation}
Chic + Chic = Tum ~~ \Longleftrightarrow ~~ Chic = \frac{Tum}{2}.
\end{equation}
 
Por outro lado, se a percepção do ouvinte é mais
aguçada, poderá escutar ``a Chic Chic Tum''; 
neste caso, o sonido ``Tum'' é solapado por o sonido de ``a'',
quando transcorrido um $75\%$ do primeiro tempo do compasso; 
o sonido ``a''  se prolonga incluindo a parte forte do tempo fraco subsequente, 
este sonido é executado pelos instrumentos ``Eb'' e ``Strings'' e constitui uma sincopa \cite[pp. 143]{medteoria}.


Pelo exposto anteriormente, agora podemos simplificar a partitura para gerar um sonido com onomatopeia
``Chic Chic Tum'', como é mostrado na Figura \ref{fig:abc-contratempo1}.
Assim,
o instrumento 1 executa dois sonidos, de modo que o primeiro contribui ao sonido 
``Tum'' e o segundo sonido gera o segundo ``Chic'' do compasso; por outro lado,
o instrumento 2 executa um ritmo com um padrão
de repetição de dois sonidos ``Tum'' e ``Chic'', nesse ordem;
sendo que a nota executada no tempo forte produz um sonido mais agudo que a 
executada no tempo fraco, isto é assim para poder diferenciar melhor ambos tempos.
\begin{figure}[ht]
\centering
\begin{abc}[name=abc-contratempo1]
X: 1 % start of header
K: C % scale: C major
M:2/4
%T: Contratempo num compasso binário
V:1 clef=treble name="Instrumento 1" sname="Inst. 1"
V:2 clef=bass   name="Instrumento 2" sname="Inst. 2"
[V:1] " ""T/2"G1 " ""T/2"z1 " ""T/2"z1 " ""T/2"G1 | " ""T/2"G1 " ""T/2"z1 " ""T/2"z1 " ""T/2"G1  :|
w:    Tum                     Chic                  Tum                   Chic           
[V:2] "Tempo"C,2 "Tempo"G,,2  | "Tempo"C,2 "Tempo"G,,2  :|
w:    Tum       Chic         Tum       Chic            
\end{abc}
\caption{Padrão de repetição para gerar um sonido de onomatopeia ``Chic Chic Tum''.}
\label{fig:abc-contratempo1}
\end{figure}

Conhecido tudo isto, é fácil perceber como existe uma diference entre 
o que percebemos ao escutar uma música e a forma como esta é escrita na partitura;
pois como é visto na Figura \ref{fig:abc-contratempo1}, quando escrevemos
um sonido com um padrão de repetição na ordem ``Tum Chic Chic'', para o ouvinte é mais natural associar
este sonido com o padrão ``Chic Chic Tum'', devido a que \textbf{quando um ser humano fala, este usa a pausa
para denotar o final de uma palavra}. Da mesma forma, ao escutar uma música, traduzimos
que o sonido que tem um silencio maior apos ser executado marca o final do ciclo
do padrão de repetição. Assim, o que um músico vê ao ler uma partitura
é um padrão de repetição ``Tum Chic Chic'', sendo que  um
ouvinte interpretará de forma instintiva que o padrão é ``Chic Chic Tum ''.

\subsubsection{\textcolor{red}{Percepção do: Tum Tum}}

Analisando o fragmento de partitura, na Figura \ref{fig:abc-caquarela}, 
e tentando issolar o instrumento ``D. Bass'',
podemos perceber que este gera um sonido identificável com a onomatopeia ``tum Tum''.
Podemos ver na Figura \ref{fig:abc-contratempo1tumtum} estre instrumento isoladamente.
\begin{figure}[ht]
\centering
\begin{abc}[name=abc-contratempo1tumtum]
X: 1 % start of header
K: C % scale: C major
M:2/4
%T: Contratempo num compasso binário
V:1 clef=bass   name="D. Bass" sname="D. Bass"      
[V:1] "Tempo"C,2 "Tempo"G,,2  | "Tempo"C,2 "Tempo"G,,2  :|
w:    Tum       tum         Tum       tum            
\end{abc}
\caption{Padrão de repetição para gerar um sonido de onomatopeia ``tum Tum''.}
\label{fig:abc-contratempo1tumtum}
\end{figure}


\subsection{Contagem de 3 passos em 2 tempos}
Esta diferença na forma de perceber o inicio e o final do ciclo de repetição,
vista na Seção \ref{subsec:percepcaoouvinte}, 
leva a um problema quando se quer ser rigoroso na forma de contar os tempos nos compassos; 
por exemplo, na Tabela \ref{tab:ritmo1} 
podemos ver 4 formas distintas, que podem adotar as pessoas, 
para contar os tempos nos compassos indicando a distribuição de tempos, 
onde ``$T$'' representa um tempo do compasso.
\begin{table}[ht]
  \centering
  \begin{tabular}    {c|ccc|c}
    \hline
    Tipos de contagem       & $T/2$ & $T/2$   & $T$ & Recomendável?\\
    \hline
    Contagem 1: & Chic  & Chic  & Tum   & Sim\\
    Contagem 2: & 2     & e     & 1     & Sim\\ \hline
    Contagem 3: & Con   & tra  & Tempo & Não\\
    Contagem 4: & 1     & e     & 2     & Não\\
    Contagem 5: & 1     & 2     & 3     & Não\\
    \hline
  \end{tabular}
  \caption{Tipos de contagem na samba de gafieira.}
\label{tab:ritmo1}
\end{table}

As formas de contagem que recomendo são:
\begin{itemize}
\item \textbf{A contagem 1}, 
devido que a principio, pode ser usada sem aprofundar demasiado 
na notação musical, de modo que só precisa ser explicado que a duração de um 
``Tum'' é o dobro que um ``Chic'', e anexar que tipicamente veremos que o ``Tum''
acontece no tempo 1 do compasso; 
%de modo que outra contagem valida seria ``Tum Chic Chic''; 
porem, a contagem 1 não está restrita ao uso destas silabas (``Chic'' e ``Tum''), 
em geral esta contagem representa a quase qualquer padrão de repetição
que use duas silabas diferentes, como por exemplo os padrões: ``Ta Ta Kum'', ``Tic Tic Pa'', etc. 
Este tipo de contagem já é muito usada na pratica e na literatura, pois 
podemos achar variantes como ``Quick-Quick Slow'' (Rápido-Rápido Lento no idioma inglês)
ou ``Tic Tic Tum'' seguindo a notação usada por Perna no seu livro sobre samba de gafieira \cite[pp. 146]{perna2002samba}.
\item \textbf{A contagem 2}, segue a notação de tempos na partitura, este tipo de
contagem é coerente com a musica, porem precisa de uma major explicação, 
para pessoas não iniciadas na musica e a dança. Porem, isto não quer dizer que seu
entendimento seja complexo, e sim que precisa um investimento em horas de aula
um pouco major que a contagem 1.
Mesmo assim, devemos ter cuidado pois pode-se dar o caso que a partitura não tenha compassos binários 
e sim quaternários, com contagens ``2 e 3'' ``4 e 1'', 
criando este tipo de contagem mais caminhos onde podemos perder coerência com a contagem na partitura.
\end{itemize}

~\\

Por outro lado, entre as contagens que não recomendo estão:
\begin{itemize}
\item \textbf{A contagem 3} (``Con-tra Tempo''), 
devido a que o uso deste padrão pode confundir às pessoas que desconhecem 
a definição formal do termo contratempo \cite[pp. 16]{mascarenhascurso} \cite[pp. 36]{azevedocompor}, 
e levar a confusão de achar que um contratempo é só uma distribuição de 3 tempos, 
sendo um o dobro dos outros dois, em termos de tempos execução.
\item \textbf{A contagem 4} é não recomendada, devido a que como é visto nas Figuras 
\ref{fig:abc-caquarela} e \ref{fig:abc-contratempo1}, musicalmente a contagem estaria invertida,
dado que tipicamente o ``Tum'' se execute no tempo 1.

\item \textbf{A contagem 5} não é recomendada, 
por motivos similares aos apresentados para a contagem 4. 
Além do fato que os números atribuídos estão distantes da
notação verdadeira na partitura, mesmo sim esta houvesse sido escrita num compasso quaternário.
\end{itemize}

\subsection{\textcolor{red}{Contagem de 2 passos em 2 tempos}}


%%%%%%%%%%%%%%%%%%%%%%%%%%%%%%%%%%%%%%%%%%%%%%%%%%%%%%%%%%%%%%%%%%%%%%%%%%%%%%%%
\section{\textcolor{red}{Procurando um bom ``Timing''}}
Procurando o momento certo

 sincronização [https://www.infopedia.pt/dicionarios/ingles-portugues/timing]
 sentido de oportunidade [https://www.infopedia.pt/dicionarios/ingles-portugues/timing]
%%%%%%%%%%%%%%%%%%%%%%%%%%%%%%%%%%%%%%%%%%%%%%%%%%%%%%%%%%%%%%%%%%%%%%%%%%%%%%%%
\section{\textcolor{red}{Articulação das notas musicais vs. dança}}
Articulação das notas musicais refletido na nossa dança

Legato vs. sttacato

%%%%%%%%%%%%%%%%%%%%%%%%%%%%%%%%%%%%%%%%%%%%%%%%%%%%%%%%%%%%%%%%%%%%%%%%%%%%%%%%
\section{\textcolor{red}{Procurando o tempo forte da música}}
Tempo forte ou tempo 1

simples
\begin{itemize}
\item O tempo em que estatisticamente percebemos com maior potencia sonora. 
\item Se conseguimos identificar audivelmente um padrão de repetição ``Chick-Chick Tum'', o tum é o tempo forte
\item O tempo em que estatisticamente percebemos que o cantor  coloca o acento da palavra. 
\end{itemize}

complexos 
\begin{itemize}
\item Se percebemos um ``break'' da música com final de frase musical conclusivo 
(satisfatório, com uma sensação de ponto aparte), então este aconteceu no tempo forte.
\end{itemize}

%%%%%%%%%%%%%%%%%%%%%%%%%%%%%%%%%%%%%%%%%%%%%%%%%%%%%%%%%%%%%%%%%%%%%%%%%%%%%%%%
\section{\textcolor{red}{Dançando no tempo forte da música}}

Que significa dançar no tempo forte? 
pisar o tum  no tempo forte.

Se percebo que estou pisando no fraco como corrigir?
podemos usar
\begin{itemize}
\item Caminhada em contratempo
\item Fazemos balaços um número impar de vezes.
\end{itemize}


%%%%%%%%%%%%%%%%%%%%%%%%%%%%%%%%%%%%%%%%%%%%%%%%%%%%%%%%%%%%%%%%%%%%%%%%%%%%%%%%
\section{\textcolor{red}{Percebendo frases musicais}}

\subsection{\textcolor{red}{percebendo frases com final conclusivo}}
\subsection{\textcolor{red}{percebendo frases com final suspensivo}}
\subsubsection{\textcolor{red}{percebendo frases com final suspensivo a contratempo e sincopado}}

%%%%%%%%%%%%%%%%%%%%%%%%%%%%%%%%%%%%%%%%%%%%%%%%%%%%%%%%%%%%%%%%%%%%%%%%%%%%%%%%
\section{\textcolor{red}{Contando e medindo a frase musical}}
Provavelmente 4
\begin{itemize}
\item 4 compassos
\item 8 compassos
\item 2 compassos
\item 16 compassos
\end{itemize}

%%%%%%%%%%%%%%%%%%%%%%%%%%%%%%%%%%%%%%%%%%%%%%%%%%%%%%%%%%%%%%%%%%%%%%%%%%%%%%%%
\section{\textcolor{red}{Frase musical, Rap e musicalizar poesias}}
% https://scielo.conicyt.cl/scielo.php?script=sci_arttext&pid=S0719-32622018000100030
Figura \ref{rap:emocional-protesto1}

\begin{figure}[H]
\centering
\begin{abc}[name=abc-emocional-protesto1]
X: 1 % start of header
K: C stafflines=1 % scale: C major
M: 2/4 %meter - compasso
V:1 clef=perc stem=up %name="Pauta com clave de fá"   sname="Pauta com clave de fá"
[V:1] |:!>!B3/2 B/2 B1 B1| B3/2 B/2 B1 B1 | B2 B2| B2 B1 B1:|
\end{abc}
\caption{Frase de 8 tempos.}
\label{rap:emocional-protesto1}
\end{figure}

\begin{citando}
Vida simples, metáfora de um corpo;\\
luzes cobertas, as lágrimas expostas.\\
vida simples, luta contra o tempo;\\
brilha, existe, e terás respostas.\\
\end{citando}


Tabela \ref{tab:verso1}

\begin{table}[h!]
\begin{center}
\begin{tabular}{|l||l||l||l|} % 
\hline
compasso 1 & compasso 2   & compasso 3   & compasso 4 \\ \hline \hline
vida       & simples, me- & táfora de um & corpo,  as \\ \hline
luzes  co- & bertas, as   & lágrimas  ex & postas     \\ \hline
\end{tabular}
\caption{Verso 1.}
\label{tab:verso1}
\end{center}
\end{table}


%%%%%%%%%%%%%%%%%%%%%%%%%%%%%%%%%%%%%%%%%%%%%%%%%%%%%%%%%%%%%%%%%%%%%%%%%%%%%%%%
\section{\textcolor{red}{Percebendo e usando o ``break'' da música}}




%%%%%%%%%%%%%%%%%%%%%%%%%%%%%%%%%%%%%%%%%%%%%%%%%%%%%%%%%%%%%%%%%%%%%%%%%%%%%%%%
\section{\textcolor{red}{Seguindo isoladamente os instrumentos}}
\begin{itemize}
\item Dançando choro-chorinho na melodia esquecendo o ``Chick-Chick Tum''.
\item Trabalhando com marionetas um em cada mão, ou dedo.
\item Cada aluno simula que tem um instrumento virtual e toca ele.
\end{itemize}

%%%%%%%%%%%%%%%%%%%%%%%%%%%%%%%%%%%%%%%%%%%%%%%%%%%%%%%%%%%%%%%%%%%%%%%%%%%%%%%%
\section{\textcolor{red}{Mickey mousing}}

