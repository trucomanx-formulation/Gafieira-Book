
%%%%%%%%%%%%%%%%%%%%%%%%%%%%%%%%%%%%%%%%%%%%%%%%%%%%%%%%%%%%%%%%%%%%%%%%%%%%%%%%
%%%%%%%%%%%%%%%%%%%%%%%%%%%%%%%%%%%%%%%%%%%%%%%%%%%%%%%%%%%%%%%%%%%%%%%%%%%%%%%%
\section{Contagem de tempos correográficos}
\label{sec:TemposCoreograficos}
\index{Musicalidade!Tempos coreográficos}
Nesta seção presentamos e definimos o conceito de ``tempo coreográfico''.
Quando criamos uma coreografia (grupo de movimentos), 
definimos para esta uma distribuição de tempos em que os movimentos serão executados.
Para conseguir isto, usamos como unidade de medida um tempo de referencia,
de modo que cada movimento do grupo tem sua duração em função deste tempo.
\begin{definition}[Tempo coreográfico:] 
\label{def:tempocoreografico}
é a unidade de medida básica, com que os movimentos de uma coreografia são ordenados e distribuídos.
Um tempo coreográfico não precisa ter uma representação em segundos,
este dado só será relevante quando a coreografia seja encaixada numa música;
nesse caso: 
\begin{itemize}
\item Deverá ser indicada a equivalência entre os tempos coreográficos e musicais.
\item Também, deverá ser indicado a posição do primeiro tempo coreográfico em relação aos tempos musicais.
\end{itemize}
\end{definition}
Assim, a principio, os tempos coregráficos são bastante livres, 
e seu uso e criação depende exclusivamente da imaginação do coreografo.


Para evitar confusões na leitura, usaremos a abreviatura ``TC'' 
antes do número que indica o tempo coreográfico,
para diferenciar este da contagem do tempo musical,
de modo que quatro tempos coreográficos em sequência seriam escritos como: 
\{TC1, TC2, TC3, TC4\}.

De forma similar, para indicar um conjunto de 4 movimentos, 
sugerimos utilizar a abreviatura ``M''; de modo que quatro movimentos de uma coreografia,
ordenados em sequência seriam escritos como: 
\{M1, M2, M3, M4\}.

\begin{example}
Imaginemos que estamos criando o movimento chamado \hyperref[subsec:passo:romario]{\textbf{Romário}},
e decidimos iniciar ele da postura de X; percebemos então que faremos 6 movimentos,
de modo que o terceiro e sexto tem o tempo de espera dobrado, 
em relação aos demais movimentos que tem o mesmo tempo de espera apos serem executados.

Assim para descrever o Romário, antes de ser incrustado na música, 
poderíamos usar a seguinte sequencia de tempos coreográficos: \{TC1, TC2, TC3, TC5, TC6, TC7\},
que correspondem aos movimentos \{M1, M2, M3, M4, M5, M6\} da coreografia, respetivamente.

Para incrustar esta coreografia na música, o único que precisamos indicar,
é que a coreografia inicia no tempo 2 do compasso (tempo fraco)
e que cada tempo coreográfico dura meio tempo musical.

A contagem de tempos coreográficos pode ser vista na primeira linha de texto na pauta mostrada na Figura \ref{fig:contagemtempocoreografico};
na segunda linha de texto da pauta podemos ver a contagem dos tempos musicais,
onde vemos que o movimento todo da coreografia dura 4 tempos musicais.
\end{example}

\begin{figure}[!h]
    \centering
    %\includegraphics[width=\textwidth]{chapters/cap-musicalidade/contagemtempocoreografico.eps}
\begin{abc}[name=abc-contagemtempocoreografico]
X: 1 % start of header
K: C stafflines=1 % scale: C major
M: 2/4 %meter - compasso
%Q:1/4=80
V:1 clef=perc stem=up %name="Pauta com clave de fá"   sname="Pauta com clave de fá"
[V:1] |: B2 B1 B1| B2 B1 B1 | B2 B1 B1 :|
w: ~ TC1 TC2 TC3 TC5 TC6 TC7 
w: ~ 2_ 1  2_ 1 ~  
\end{abc}
    \caption{Contando tempos coregráficos.}
    \label{fig:contagemtempocoreografico}
\end{figure}

