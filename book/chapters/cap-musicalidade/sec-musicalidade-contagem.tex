%%%%%%%%%%%%%%%%%%%%%%%%%%%%%%%%%%%%%%%%%%%%%%%%%%%%%%%%%%%%%%%%%%%%%%%%%%%%%%%%
%%%%%%%%%%%%%%%%%%%%%%%%%%%%%%%%%%%%%%%%%%%%%%%%%%%%%%%%%%%%%%%%%%%%%%%%%%%%%%%%
\section{Contagens dos passos para o ensino}
Antes de iniciar esta seção é importante mencionar uma
problemática que é vista com muita frequência nas escolas de dança; 
esta é gerada a consequencia de que ``a forma em que os tempos são contados 
na música, é diferente à realizada entre os profissionais da música e os da dança''. 
Sendo que a contagem dos profissionais da música segue a \hyperref[def:Metrica]{\textbf{métrica}},
e no caso dos profissionais da dança segue, em varias ocasiões, 
um enfoque particular a cada escola de dança, visando só em muitos casos, 
o fácil entendimento do aluno na execução do movimento programado para o dia da aula, 
e não uma rigorosidade teórica no uso de termos e expressões musicais.



%%%%%%%%%%%%%%%%%%%%%%%%%%%%%%%%%%%%%%%%%%%%%%%%%%%%%%%%%%%%%%%%%%%%%%%%%%%%%%%%
\subsection{Contagem de 3 passos em 2 tempos}
\label{susec:3passos2tempos}
A diferença\footnote{É vista uma explicação das diferenças na percepção rimica na Pag. \pageref{fig:RitmoVsFala}.} 
entre a subjetiva percepção auditiva que podemos ter do inicio de um padrão rítmico, 
e o verdadeiro tempo musical em que o padrão ritmico inicia dentro  de um \hyperref[def:Compasso]{\textbf{compasso}}, 
leva a um problema quando se quer ser rigoroso na forma de contar os tempos dos movimentos na dança; 
por exemplo, na Tabela \ref{tab:ritmo1} podemos ver 5 formas distintas, que  as pessoas adotam, 
para contar os tempos nos compassos indicando a distribuição de tempos, 
onde ``$T$'' representa um tempo do compasso (\hyperref[subsec:compassobinario]{\textbf{binário}}).
\begin{table}[!h]
  \centering
  \begin{tabular}    {c|ccc|c}
    \hline
    Tipos de contagem       & $T/2$ & $T/2$   & $T$ (Forte) & Recomendável?\\
    \hline
    Contagem 1: & tchic  & tchic  & tum   & Sim\\
    Contagem 2: & 2     & e     & 1     & Sim\\ \hline
    Contagem 3: & Con   & tra  & Tempo & Não\\
    Contagem 4: & 1     & e     & 2     & Não\\  \hline
    Contagem 5: & 1     & 2     & 3     & Depende\\ \hline
    \hline
  \end{tabular}
  \caption{Tipos de contagem na samba de gafieira.}
\label{tab:ritmo1}
\end{table}

As formas de contagem que recomendo são:
\begin{itemize}
\item \textbf{A contagem 1}, 
devido que a principio, pode ser usada sem aprofundar demasiado 
na notação musical, de modo que só precisa ser explicado que a duração de um 
``tum'' é o dobro que um ``tchic'', e anexar que tipicamente veremos que o ``tum''
acontece no tempo 1 do compasso.
%de modo que outra contagem valida seria ``tum tchic tchic''; 

A contagem 1 não está restrita ao uso destas silabas (``tchic'' e ``tum''), 
em geral esta contagem representa a quase qualquer padrão de repetição
que use duas silabas diferentes, como por exemplo os padrões: ``ta-ta kum'', ``tic-tic kum'', etc. 
Este tipo de contagem já é muito usada na prática e na literatura sobre dança, pois 
podemos achar variantes como ``quick-quick slow'' (rápido-rápido lento no idioma inglês)
ou ``tic-tic tum'' seguindo a notação usada por Perna no seu livro sobre samba de gafieira \cite[pp. 146]{perna2002samba}.
\item \textbf{A contagem 2}, segue a notação de tempos na partitura, este tipo de
contagem é coerente com a musica, porem precisa de uma major explicação, 
para pessoas não iniciadas na musica e a dança; isto não quer dizer que seu
entendimento seja complexo, e sim que precisa um investimento em horas de aula
um pouco major que a contagem 1.
Mesmo assim, devemos ter cuidado pois pode-se dar o caso que a partitura não tenha compassos binários 
e sim quaternários, com contagens como ``2 e 3'', ``4 e 1''; 
criando este tipo de contagem, mais caminhos onde podemos perder coerência com a contagem seguindo a métrica.
\end{itemize}~


Entre as contagens que não recomendo estão:
\begin{itemize}
\item \textbf{A contagem 3} (``con-tra tempo''), 
devido a que o uso deste padrão pode confundir às pessoas que desconhecem 
a definição formal do termo \hyperref[sec:contratempo]{\textbf{contratempo}} 
\cite[pp. 16]{mascarenhascurso} \cite[pp. 36]{azevedocompor}, 
e levar a confusão de achar que um contratempo é só uma distribuição de 3 tempos, 
sendo um o dobro dos outros dois, em termos de tempos execução.
\item \textbf{A contagem 4} é não recomendada, devido a que como é visto nas Figuras 
\ref{fig:abc-caquarela} e \ref{fig:abc-contratempo1}, musicalmente a contagem estaria invertida,
dado que o tempo 1 deve corresponde ao tempo forte.
\end{itemize}~

Finalmente, a contagem que precisa um cuidado especial:
\begin{itemize}

\item \textbf{A contagem 5} precisa ser bem explicada, 
devido a que não segue a notação musical; 
porem, seu uso é didático, e pode ser resgatado se fazemos em todo instante a aclaração de 
que se trata de \hyperref[sec:TemposCoreograficos]{\textbf{tempos coreográficos}}.
Assim, ficará claro para o estudante, 
que esta contagem não corresponde necessariamente com os tempos musicais, 
que também devem ser ensinados.
Para mais detalhes sobre os tempos coreográficos ver a Seção \ref{sec:TemposCoreograficos}.

%não é recomendada, 
%por motivos similares aos apresentados para a contagem 4. 
%Além do fato que os números atribuídos estão distantes da
%notação verdadeira na partitura, mesmo sim esta houvesse sido escrita num compasso quaternário.
\end{itemize}


%%%%%%%%%%%%%%%%%%%%%%%%%%%%%%%%%%%%%%%%%%%%%%%%%%%%%%%%%%%%%%%%%%%%%%%%%%%%%%%%
\subsection{Contagem de 2 passos em 2 tempos}

Na Tabela \ref{tab:ritmoconta2}  podemos ver 5 formas distintas, que  as pessoas adotam, 
para contar, dois passos em dois tempos nos \hyperref[subsec:compassobinario]{\textbf{compassos binários}}, 
indicando a distribuição de tempos, 
onde ``$T$'' representa um tempo do compasso.

\begin{table}[ht]
  \centering
  \begin{tabular}    {c|cc|c}
    \hline
    Tipos de contagem       & $T$ (fraco)  & $T$ (Forte)& Recomendável?\\
    \hline
    Contagem 1: & tum  & TUM  & Sim\\
    Contagem 2: & 2     & 1     & Sim\\
    Contagem 3: & tempo & Tempo & Sim\\ \hline
    Contagem 4: & 1     & 2     & Não\\ \hline
    Contagem 5: & 1     & 3     & Depende\\  \hline
    \hline
  \end{tabular}
  \caption{Tipos de contagem na samba de gafieira.}
\label{tab:ritmoconta2}
\end{table}



As formas de contagem que recomendo são:
\begin{itemize}
\item \textbf{A contagem 1},
 pode ser usada sem aprofundar demasiado 
na notação musical, onde só precisa ser explicado que cada onomatopeia ``tum'' tem a mesma duração, 
de modo que  o ``tum'' que cai no tempo 1 está sendo falado com maior acento (``TUM'').
\item \textbf{A contagem 2}, segue a mesma ideia que a contagem 1, 
com diferencia que aqui são usados os números dos tempos nos compassos binários.
\item \textbf{A contagem 3}, 
é outra forma de similar a contagem 1, 
com a diferença que em vez de contar usando a onomatopeia ``tum'',
usamos a palavra ``tempo'' para ressaltar que cada um dura um tempo do compasso. 
\end{itemize}~

A forma de contagem que não recomendo é:
\begin{itemize}
\item \textbf{A contagem 4},
está invertida em relação a contagem seguindo a 
\hyperref[def:Metrica]{\textbf{métrica}} dos \hyperref[subsec:compassobinario]{\textbf{compassos binários}},
pelo que pode trazer conduções.
\end{itemize}~

Finalmente, a contagem que precisa um cuidado especial:
\begin{itemize}
\item \textbf{A contagem 5},
precisa ser bem explicada, 
devido a que não segue a notação musical, 
porem esta é uma consequência da ``1,2,3'' vista na Seção \ref{susec:3passos2tempos}; 
Seu uso é didático, e pode ser usada, se fazemos em todo momento a indicação de 
que se trata de \hyperref[sec:TemposCoreograficos]{\textbf{tempos coreográficos}}.
Assim, ficara claro para o estudante, que esta contagem não usa os tempos da música.
Para mais detalhes sobre os tempos coreográficos ver a Seção \ref{sec:TemposCoreograficos}.
\end{itemize}




