
%%%%%%%%%%%%%%%%%%%%%%%%%%%%%%%%%%%%%%%%%%%%%%%%%%%%%%%%%%%%%%%%%%%%%%%%%%%%%%%%
%%%%%%%%%%%%%%%%%%%%%%%%%%%%%%%%%%%%%%%%%%%%%%%%%%%%%%%%%%%%%%%%%%%%%%%%%%%%%%%%

%%%%%%%%%%%%%%%%%%%%%%%%%%%%%%%%%%%%%%%%%%%%%%%%%%%%%%%%%%%%%%%%%%%%%%%%%%%%%%%%
%%%%%%%%%%%%%%%%%%%%%%%%%%%%%%%%%%%%%%%%%%%%%%%%%%%%%%%%%%%%%%%%%%%%%%%%%%%%%%%%
\clearpage
\section{Usando o ``break'' da música}
\label{sec:UsandoBreak}
\index{Musicalidade!Breques}
\index{Musicalidade!Break}

Na Seção \ref{sec:percepcionbreak} foram apresentadas um conjunto de dicas 
para poder detetar os breques (breaks) na música.
Com esta informação um dançarino pode incorporar este aspecto da música na sua dança;
assim, nesta seção apresentaremos um conjunto de sugestões  
de como podemos interpretar os breques\footnote{Estas 
sugestões entram na categoria de ``mapeamento'' de aspectos da música, 
explicado na Figura \ref{fig:interpretacion-corporal}.}.

Entre a lista de sugestões de mapeamento, 
entre o break e os aspectos do movimento, 
temos que:
\begin{itemize}
\item Quando acontece uma pausa (breque) total na música, 
é interessante acompanhar este aspecto realizando também uma pausa total em nossos movimentos.
\item Se o breque acontece,
e um solo melódico ou percussivo preenche a pausa;
então nós temos duas opções, 
\begin{itemize}
\item ou desprezar o solo e permanecer inativos\footnote{Se por exemplo o solo é curto demais.},
\item ou utilizar o solo musical e realizar um solo de dança\footnote{Mesmo se o solo é curto.};
é dizer, realizar uma dança que incorpore aspectos do solo musical,
de modo que fique evidente que é um intermédio entre duas propostas de movimentos,
a anterior e a posterior ao breque.
\end{itemize}
\item Se decidimos dançar o solo no break, 
então podemos lembrar que 
\begin{itemize}
\item é comum mapear movimentos de pés ou de deslocamento ao ritmo 
(acompanhamento ou ritmo da melodia), e 
\item movimentos corporais quase sem deslocamento a aspectos da melodia (mudança de tons, tensão, articulação, etc.).
\end{itemize}
Também existe a possibilidade de misturar ambas técnicas usando movimentos corporais e deslocamento,
se o solo no breque é o suficientemente longo.
\item Também poderíamos simplesmente dançar o silencio da pausa, colocando um movimento corporal sem deslocamento,
para não perder a inercia, até iniciar a seguinte frase musical.
\item Quando decidimos dançar a pausa no break e temos um par de dança,
podemos estar separados ou usar um abraço de dança (fechado ou aberto).
\begin{itemize}
\item Se o abraço é fechado, os movimentos do condutor estarão mais limitados,
e este deverá conduzir os movimentos ao seguidor.
\item Se o par de dança está separado ou o abraço é aberto, 
o seguidor e condutor tem mais liberdade de movimento,
e se desejam podem executar cada um seus movimentos independentemente,
sem necessidade de condução. 
\end{itemize}
Entre os movimentos que poderíamos usar no break, estão por exemplo, 
um samba no pé, um bamboleio circular de quadril (no plano axial), um balanço de ombros (no plano frontal), etc. 
\item Uma vez finalizado o solo, é mais fácil iniciar a seguinte frase musical
no seu primeiro \hyperref[subsec:perceberTF1]{\textbf{tempo forte}}. 
Assim, 
ao finalizar o solo devemos ter bem definido o peso do corpo num pé,
para ter rapidez a iniciar a seguinte frase.
\item Também é interessante esperar o inicio da frase apos o break, 
com os pés próximos; pois se etão separados, 
dar um passo em tempo forte para iniciar a frase musical será difícil,
pois os pés já estariam separados.
\end{itemize}~

No Exemplo \ref{ex:breakvarios} indicamos um conjunto de músicas 
que são ótimas para o treinamento de forma unipessoal,
pois tem uma grande quantidade de breques, 
e solos nos breques.
\begin{example}[Músicas com muitos breques]~
\label{ex:breakvarios}
Em todos os exemplos listados, os breques acontecem na linha melódica principal e o acompanhamento,
enquanto que o cantor faz solos falando; 
dado que a voz é quase de um tom uniforme e não poderia ser considerada preponderantemente melódica,
se sugere, sim se deseja dançar o solo, que se aproveite só ritmicamente a voz.
\begin{itemize}
\item ``Idade não é documento'' interpretado por Moreira da silva.
\item ``Jogando com o capeta'' interpretado por Moreira da silva.
\item ``Na subida do morro'' interpretado por Moreira da silva. 
O primeiro break é feminino.
\end{itemize}
\end{example}

Outras músicas com breaks podem ser encontradas nos Exemplos \ref{ex:breakmasculinos},
\ref{ex:breakfeminino} e 
\ref{ex:breaksincopados}.
