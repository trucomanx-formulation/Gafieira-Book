%%%%%%%%%%%%%%%%%%%%%%%%%%%%%%%%%%%%%%%%%%%%%%%%%%%%%%%%%%%%%%%%%%%%%%%%%%%%%%%%
%%%%%%%%%%%%%%%%%%%%%%%%%%%%%%%%%%%%%%%%%%%%%%%%%%%%%%%%%%%%%%%%%%%%%%%%%%%%%%%%
\section{\textcolor{red}{Cadencia na dança }}
\index{Musicalidade!Cadência}

\begin{definition}[Cadência:] 
\index{Musicalidade!Cadência}
\label{def:cadencia}
O Dicionário Online de Português define cadência como \cite{diciocadencia}:
\begin{itemize}
\item Ritmo; sequência encadeada e regular de sons e de movimentos.
\item [Literatura] Harmonia na forma de pronunciar as palavras; ritmo das palavras, segundo a acentuação tônica de cada sílaba.
\item [Música] Sucessão de notas e de acordes que definem o tom.
\item [Música] Unidade abstrata que mede o tempo musical, marcando as relações de ritmo.
\end{itemize}
\end{definition}


%% 
%% https://books.google.com.br/books?id=s0yb-6BXp1wC&pg=SA6-PA18&dq=cadence+%2B+tango&hl=es-419&sa=X&ved=0ahUKEwivpbuYmdbjAhW-LLkGHUuEAncQ6AEILDAA#v=onepage&q=cadence&f=false


