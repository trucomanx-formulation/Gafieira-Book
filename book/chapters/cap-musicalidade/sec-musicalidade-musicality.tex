
%%%%%%%%%%%%%%%%%%%%%%%%%%%%%%%%%%%%%%%%%%%%%%%%%%%%%%%%%%%%%%%%%%%%%%%%%%%%%%%%
%%%%%%%%%%%%%%%%%%%%%%%%%%%%%%%%%%%%%%%%%%%%%%%%%%%%%%%%%%%%%%%%%%%%%%%%%%%%%%%%
\section{Musicalidade, sentir ou entender a música?}
Seguindo a Definição \ref{def:Musicalidade}, podemos inferir como entender a musicalidade na dança.
\begin{definition}[Musicalidade na dança:] 
\index{Musicalidade}
\label{def:MusicalidadeNaDanca}
Esta acontece, quando o dançarino tem um estado de ``sensibilidade'' ou ``conhecimento'' para contemplar ou entender a música,
e assim quando este dança, ``demostrar'' uma coerência entre a música e o que está dançando.
\end{definition}

Porem temos um choque de paradigmas, na definição, para atingir um mesmo objetivo que é ser musical.
Podemos \textbf{sentir} ou \textbf{conhecer}; o que é equivalente a dizer que:
\begin{itemize} 
\item podemos saber, por intuição e sensibilidade sem entender porquê, ou
\item podemos saber, entendendo mediante o conhecimento adquirido pelo raciocínio e o estudo da música e sua perpecção.
\end{itemize}~



Neste sentido, para adquirir esta coerência com a música, 
muitos professores indicam que o dançarino deve ``escutar e sentir a música'';
nesse aspecto, argumento que a frase é metaforicamente correta e coincide em parte com 
a Definições \ref{def:Musicalidade} e \ref{def:MusicalidadeNaDanca};
porem, pedagogicamente  é pouco favorável para o estudante.
Assim, eu ressaltaria que a musicalidade a nível de ensino se adquire,
estudando e entendendo a música e como esta é percebida, e não só sentindo-a.
 
Para fundamentar minha argumentação, acredito interessante expor o Exemplo \ref{ex:srinivasa}. 
\begin{example}[O matemático que sentia os números:]
\label{ex:srinivasa}
Imaginemos que conhecemos ao matemático ``Srinivasa Aiyangar Ramanujan'';
um prodígio matemático autodidata \cite[pp. 1]{kanigel2016man}, indiano, que 
realizou muitas contribuições à matemática\footnote{Existe 
um filme chamado `` The Man Who Knew Infinity'' (2015) ou 
``O Homem que Viu o Infinito'' em português, que conta a história de vida de Srinivasa Aiyangar Ramanujan.}.
E mostramos a ele um problema matemático, como por exemplo uma equação,
e  lhe pedimos a resposta ou solução. 
Então Srinivasa, muito amavelmente, 
observaria um instante o problema e nos daria a solução imediatamente.
Nós surpreendidos pela velocidade e o mínimo esforço na resposta,
perguntaríamos. Como você obteve a resposta? Então ele responderia \cite[pp. 235]{kanigel2016man}: 
%\begin{citando}
%Immediately I heard the problem 
%it was clear that the solution should obviously be a continued fraction; 
%I then thought, Which continued  fraction? And the answer came to my mind.
%\end{citando}
\begin{citando}
No momento em que escutei o problema, 
foi claro pra mim que a resposta devia ser obviamente uma fração continua; 
e então pensei, ¿Qual fração continua? e a resposta chegou a minha mente. 
\end{citando}

Exemplo de fração continua simples:
\begin{equation}
a_{0}+{\frac {1}{a_{1}+{\frac {1}{a_{2}+{\frac {1}{a_{3}+{\frac {1}{\ddots }}}}}}}}
\end{equation}
\end{example}

A resposta que ``Srinivasa'' deu no Exemplo  \ref{ex:srinivasa}, 
é a mesma  que dão as pessoas, quando  dizem que para ter musicalidade ele simplesmente sentiu a música. 
Assim, esta aproximação ao problema é só válida ou eficiente, para pessoas como Srinivasa, 
que talvez tenham uma inspiração divina, 
ou que já nasceram com esse dom ou que por uma longa experiencia de vida, 
tem implementado por ``hardware'', no cérebro, entender a matemática ou a música; 
pelo que eles usam a palavra sentir, 
pois conhecem a resposta, porem não sabem como sabem. 

Isto também pode acontecer com pessoas que observaram muito tempo um ``problema'', ou escutaram muito uma ``música'', 
e um dia conseguiram ``sentir'' a resposta. Na minha opinião, 
não todos nascemos com esse componente implementado em nosso cérebro, para perceber e processar por ``hardware''(sentir) a música; 
e não podemos nos dar o luxo de escutar uma canção indefinidamente ate ``sentir'' algo. 
O mais eficiente seria estudar música, 
e entender esta baseando-nos em nossa teoria e cruzando esta informação com o que escutamos,
os padrões observados, a melodia, o ritmo, etc. 
Assim, nos podemos criar por ``software'' o que não temos implementado por ``hardware''\footnote{Como no caso do amigo Srinivasa.}, 
derrubando o mito de ``sentir'' a música e passar a ``entender'' ela.

Mas isto não quer dizer que sentir música seja equivocado, é um caminho muito valido sim;
porem pode chegar a ser pouco favorável se temos estudantes baixo nossa responsabilidade,
 que dependem de nós, no seu percorrido para chegar a serem musicais; 
pois se nosso entendimento da musicalidade se baseia unicamente em nossas sensações pessoais,
só poderemos indicar a eles que fechem os olhos e sintam a música.


\subsection{Musicalidade e teoria da informação}

Dançar com musicalidade não é só dançar com o \hyperref[ref:Pulso]{\textbf{pulso}} da música, 
ou na \hyperref[def:Metrica]{\textbf{métrica}},
existem outros aspectos da música, ou de instrumentos isolados, que podemos seguir como 
\begin{inparaitem} 
\item a melodia, 
\item o ritmo,
\item a \hyperref[sub:Articulação]{\textbf{articulação}} das notas, 
\item as \hyperref[sec:texturasmusica]{\textbf{texturas}}, 
\item as \hyperref[sec:Cadencia]{\textbf{cadências}}, 
\item os \hyperref[sec:Motivo]{\textbf{motivos}}, 
\item o \hyperref[sec:fraseio]{\textbf{fraseio}}, 
\item etc.
\end{inparaitem} 

Com todos esses fatores, uma pessoa com musicalidade, 
pode incorporar as caraterísticas que considere que sejam interessantes para 
ser mostradas na sua dança; por exemplo, poderíamos dançar interpretando a letra da música.
Neste ponto chegamos a uma pregunta; se eu danço usando a letra e faço exatamente o oposto,
estou dançando com musicalidade?
Para poder responder isto confiantes na nossa afirmação,
primeiro teríamos que nos fazer outra pergunta, 
se eu danço fazendo o oposto da caraterística da música que tenho escolhido,
minha dança seria uma função direta dessa caraterística?
Neste caso a resposta seria ``sim'', pois mesmo fazendo o oposto,
nossa dança estaria atrelada à caraterística escolhida na música.
Pelo que podemos afirmar que se dançamos fazendo o oposto a uma caraterística da música, 
estamos sim dançando com musicalidade.
\begin{example}[A criança não-independente:]
Imaginemos que temos baixo nossa responsabilidade a uma criança,
e esta precisa ir a dormir cedo, pelo que nos indicamos a ela que já deve ir a dormir;
se a criança é obediente vai a dormir e seus atos estariam em função de nossos comandos;
porem se a criança é ``falsamente'' rebelde, quando indiquemos que deve ir a dormir,
ela ficará acordada, mesmo sentindo sono, por ter um falso senso de independência.
Devido a que nesse caso a criança é dependente de nossas ordens,
pois seus atos estão em função direta do que nos indiquemos.
Um individuo realmente independente, tomaria suas decisões unicamente em função,
do seu próprio analises e conclusões.
Pelo que poderíamos afirmar que nesse caso a criança é dependente da autoridade.
\end{example}

É neste ponto que podemos invocar à ``teoria da informação'' para nos explicar,
ou modelar, esta dependência entre nossa dança e a música.
Nesta área do conhecimento existem duas definições, que são
\begin{itemize} 
\item a informação mútua (binária), que mede a informação que tem em comum dois eventos ou variaveis,
sendo 1.0 quando ambas tem a mesma informação,
zero quando ambas variáveis não tem informação em comum, 
e valores intermediários entre zero e 1.0 quando existe entre eles uma porção da informação \cite[pp. ]{reza2012introduction}. 
\item o coeficiente de correlação (Pearson), 
que indica o grau de dependência a favor ou em contra entre dois eventos ou variáveis,
sendo +1.0 quando os eventos crescem juntos, 
-1.0 quando um cresce enquanto u outro diminui na mesma proporção,
 zero quando os dois eventos não tem nada em comum, 
e valores intermediários entre -1.0 e +1.0 para graus intermediários de correlação \cite[pp. ]{reza2012introduction}.
\end{itemize} 

Note-se que a informação mútua não precisa sinal, pois por exemplo,
se uma variável cresce na mesma proporção que outra diminui, 
a informação mutua será 1.0 pois uma variável é completamente dependente da outra
\begin{example}
Dadas as variáveis  $X$ e $Y$, se $Y=-X+3$; então,
a informação mutua entre X e Y é igual a $1.0$, e
a correlação entre $X$ e $Y$ é igual a $-1.0$.
\end{example}

Das explicações anteriores podemos deduzir a seguinte definição.
\begin{definition}[Musicalidade na dança seguindo a teoria da informação:] 
\index{Musicalidade}
\label{def:MusicalidadeNaDancaIT}
A musicalidade é um termo que indica o grau de informação mutua que existe entre dois eventos: 
\begin{itemize}
\item A música que é percebida.
\item A dança que é executada.
\end{itemize} 
\end{definition}
% https://translate.google.com.br/translate?sl=en&tl=pt&u=https%3A%2F%2Fthisdancinglife.com%2Fmusicality-in-dance%2F
