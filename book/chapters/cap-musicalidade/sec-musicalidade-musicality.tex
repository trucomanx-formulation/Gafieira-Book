
%%%%%%%%%%%%%%%%%%%%%%%%%%%%%%%%%%%%%%%%%%%%%%%%%%%%%%%%%%%%%%%%%%%%%%%%%%%%%%%%
%%%%%%%%%%%%%%%%%%%%%%%%%%%%%%%%%%%%%%%%%%%%%%%%%%%%%%%%%%%%%%%%%%%%%%%%%%%%%%%%
\section{\textcolor{blue}{Musicalidade, sentir ou entender a música?}}
Seguindo a Definição \ref{def:Musicalidade}, podemos inferir como entender a musicalidade na dança.
\begin{definition}[Musicalidade na dança:] 
\index{Musicalidade}
\label{def:MusicalidadeNaDanca}
Esta acontece, quando o dançarino tem um estado de ``sensibilidade'' ou ``conhecimento'' para contemplar ou entender a música,
e assim quando este dança, ``demostrar'' uma coerência entre a música e o que está dançando.
\end{definition}

Porem temos um choque de paradigmas, na definição, para atingir um mesmo objetivo que é ser musical.
Podemos \textbf{sentir} ou \textbf{conhecer}; o que é equivalente a dizer que:
\begin{itemize} 
\item podemos saber, por intuição sem entender porquê, ou
\item podemos saber, entendendo mediante o conhecimento adquirido pelo raciocínio e o estudo.\\
\end{itemize}



Neste sentido, para adquirir esta coerência com a música, 
muitos professores indicam que o dançarino deve ``escutar e sentir a música'';
nesse aspecto, argumento que a frase é metaforicamente correta e coincide em parte com 
a Definição \ref{def:Musicalidade} e a Definição \ref{def:MusicalidadeNaDanca};
porem, pedagogicamente  é pouco favorável para o estudante.
Assim, eu ressaltaria que a musicalidade a nível de ensino se adquire,
estudando e entendendo a música e não só sentindo-a.
 
Para fundamentar minha argumentação, acredito interessante expor o seguinte exemplo: 
\begin{example}
Imaginemos que conhecemos ao matemático ``Srinivasa Aiyangar Ramanujan'';
um prodígio matemático autodidata \cite[pp. 1]{kanigel2016man}, indiano, que 
realizou muitas contribuições à matemática.
E mostramos a ele um problema matemático, como por exemplo uma equação,
e  lhe pedimos a resposta ou solução. 
Então Srinivasa, muito amavelmente, 
observaria um instante o problema e nos daria a solução imediatamente.
Nós surpreendidos pela velocidade e o mínimo esforço na resposta,
perguntaríamos. Como você obteve a resposta? Então ele responderia \cite[pp. 235]{kanigel2016man}: 
%\begin{citando}
%Immediately I heard the problem 
%it was clear that the solution should obviously be a continued fraction; 
%I then thought, Which continued  fraction? And the answer came to my mind.
%\end{citando}
\begin{citando}
No momento em que escutei o problema, 
foi claro pra mim que a resposta devia ser obviamente uma fração continua; 
E então pensei, ¿Qual fração continua? e a resposta chegou a minha mente. 
\end{citando}
\end{example}

A resposta que ele deu, 
é a mesma  que dão as pessoas, quando  dizem que para ter musicalidade ele simplesmente sentiu a música. 
Assim, esta aproximação ao problema é só válida ou eficiente, para pessoas como Srinivasa, 
que talvez tenham uma inspiração divina, 
ou que já nasceram com esse dom ou que por uma longa experiencia de vida, 
tem implementado por ``hardware'', no cérebro, entender a música; 
pelo que eles usam a palavra sentir, 
pois conhecem a resposta, porem não sabem como sabem. 

Isto também pode acontecer com pessoas que observaram muito tempo um ``problema'' ou escutaram muito uma ``música'', 
e um dia conseguiram ``sentir'' a resposta. Na minha opinião não todos nascemos, 
com esse componente, implementado em nosso cérebro, para perceber e processar (sentir) a música; 
e não podemos nos dar o luxo de escutar uma canção indefinidamente ate sentir algo. 
O mais eficiente seria estudar música, 
e entender esta baseando-nos em nossa teoria e cruzando esta informação com o que escutamos, 
os padrões observados, a melodia, o ritmo, a textura, etc. 
Assim, nos podemos criar por ``software'' o que não temos implementado por ``hardware'', 
derrubando o mito de ``sentir'' a música e passar a ``entender'' ela.

\begin{figure}[t]
\begin{elaboracion}[title=Que é o ``flow''?]
\label{page:flow}
\index{Musicalidade!Flow}
\textcolor{red}{Nós nos perdemos no movimento e na música,
esquecendo as restrições do tempo e da normalidade, este fenomeno é conhecido como ``Flow''} \cite{czikszentmihalyi1990flow} \cite{trehub2003developmental}.
\end{elaboracion}
\label{fig:flow}
\end{figure}

\subsection{\textcolor{red}{Musicalidade e teoria da informação}}
% https://translate.google.com.br/translate?sl=en&tl=pt&u=https%3A%2F%2Fthisdancinglife.com%2Fmusicality-in-dance%2F
