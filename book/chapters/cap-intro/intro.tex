\chapterimage{chapter_head_2.pdf} % Chapter heading image

\chapter{Introdução}

%%%%%%%%%%%%%%%%%%%%%%%%%%%%%%%%%%%%%%%%%%%%%%%%%%%%%%%%%%%%%%%%%%%%%%%%%%%%%%%%
%% SECTION
%%%%%%%%%%%%%%%%%%%%%%%%%%%%%%%%%%%%%%%%%%%%%%%%%%%%%%%%%%%%%%%%%%%%%%%%%%%%%%%%
\section{Historia das gafieiras}\index{historia das gafieiras}

\textcolor{red}{O termo gafieira surgiu [A e B]}

\textcolor{red}{cunhado pelo cronista carnavalesco Picareta (Romeu Arede),[D e E]}


~\\


\begin{description}
\item [A] Raça Brasil. [S.l.]: Editora Simbolo Ltda. 1999

\item [B] "A Gafieira na Paisagem Socio-Cultural do Rio de Janeiro". Revista do Instituto Histórico e Geográfico do Rio de Janeiro. [S.l.]: Instituto Histórico e Geográfico do Rio de Janeiro. 1987.



\item [D] Coutinho, Eduardo Granja (2006). Os cronistas de momo: imprensa e carnaval na primeira república. [S.l.]: Editora UFRJ. ISBN 978-85-7108-297-7

\item [E] Diniz, André (2003). Almanaque do choro: a história do chorinho, o que ouvir, o que ler, onde curtir. [S.l.]: Jorge Zahar Editor Ltda.

\end{description}


No livro "O cabrocha" podemos ver uma descrição; escrita  por Jota Efegê em 1931; 
da "Sociedade Recreativa Familiar Bohemios de Botafogo" \cite{jotaefege},
a continuação é mostrado um extracto desse texto:

\begin{tcolorbox}[colback=lowgray,colframe=lowgray]%%
O salão, comquanto não fosse de grandes dimensões, era
de um tamanho regular, confinando com uma pequena saleta
onde tambem se dansava; estava bem affluido. Numa
heterogeneidade foliã, via-se desde a crioulinha blasée, sem
elegancia, desalinhada, á mulatinha pernostica de faces
avermelhadas por um carmin berrante, cabello engommado e
subjugado por travessas e grampos, num á la garçonne
forçado, mas exigido pela moda. Em meio dessas "cabrochas"
e "roxinhas", viam-se algumas moças brancas de apparencia
sobria. São as meninas que não podem fazer um vestido de
seda ou calçar sapatos de setim, para se apresentarem no
Fluminense ou no Flamengo e que nestes clubes se divertem,
ficando em evidencia por serem brancas.
~\\
(Jota Efegê)
\end{tcolorbox}

%\newpage 
%%%%%%%%%%%%%%%%%%%%%%%%%%%%%%%%%%%%%%%%%%%%%%%%%%%%%%%%%%%%%%%%%%%%%%%%%%%%%%%%
%% SECTION
%%%%%%%%%%%%%%%%%%%%%%%%%%%%%%%%%%%%%%%%%%%%%%%%%%%%%%%%%%%%%%%%%%%%%%%%%%%%%%%%
\section{Estatutos da gafieira}\index{Estatuto da Gafieira}
O estatuto da gafieira foi bem retratado, por Billy Blanco,
na letra de sua composição musical intitulada "Estatutos da Gafieira"; 
esta foi interpretada por primeira vez na voz de Inezita Barroso, 
em gravação da "RCA Victor" em janeiro de 1954 \cite{musicaestatuto};
O seguinte texto mostra a letra da música "Estatutos da gafieira".
\begin{tcolorbox}[colback=lowgray,colframe=lowgray]%%
\center{Moço, olha o vexame,}\\
O ambiente exige respeito,\\
Pelos estatutos da nossa gafieira,\\
Dance a noite inteira, mas dance direito!\\
Aliás, pelo artigo 120,\\
O cavalheiro que fizer o seguinte:\\
Subir na parede, dançar de pé pro ar,\\
Morar na bebida sem querer pagar,\\
Abusar da umbigada de maneira folgazã,\\
Prejudicando hoje o bom crioulo de amanhã,\\
Será distintamente censurado,\\
Se balançar o corpo, vai pra mão do delegado.\\
Tá bem, moço?\\
\end{tcolorbox}
O texto é uma tentativa bem-humorada do autor de descrever o que acontecia 
nas gafieiras, porem na época da escrita desta popular samba, não
existiam tais regras, esto é confirmado por um depoimento realizado por 
Billy Blanco no 8 de julho de 2011 \cite{depoimentobilly}; o texto a seguir
mostra um fragmento dessa entrevista.

\begin{tcolorbox}[colback=lowgray,colframe=lowgray]%%
"Observando os acontecimentos de uma gafieira, então, eu imaginei
coisas, porque o compositor vive muito da imaginação. E eu criava situações 
possíveis de serem acontecidas na gafieira, ou então narrava o que
acontecia realmente. Por exemplo, no [samba] Pistom de Gafieira, tinha
um cidadão que era pistonista da orquestra que sempre tocava forte para
disfarçar quando a polícia vinha chegando. Doutra feita, eu tive a ideia
de fazer o estatuto para a gafieira. Então eu humorizei, porque ninguém
dança de pé pro ar, nem sobe em parede, não é? Mas a gente cria uma
extravagância dessas para dar uma certa graça, um certo sentido à música.
Na época, não havia código nenhum, eu apenas criei aquilo e muitas gafieiras 
depois tinham esse estatuto na parede para quem quisesse cantar.
Você vê que as regras do estatuto são umas regras brincalhonas, não é?" 
~\\
(Billy Blanco)
\end{tcolorbox}

Mesmo observando que as regras propostas pelo autor tem um caráter humorístico e sarcástico,
pelo texto é fácil perceber como existia um chamado de atenção para umas
normas básicas a serem cumpridas na gafieira. Por exemplo: 
A linha 7
comenta sobre não fazer movimentos aéreos na pista de dança,
ou também faz referencia a evitar movimentos capoerísticos,
que são mais prováveis a serem vistos na época; tudo isto
pelo evidente espacio reduzido e compartilhado que existe na pista de dança, 
além de que os movimentos aéreos estão pensados para ser
executados em apresentações e não em danças sociais. 
A linha 8, fala
sobre o respeito ao parceiro, pois a pessoa que dança precisa
estar ao 100$\%$ de suas faculdades físicas e mentais, no caso de homens para estar atentos
ao salão e cuidar a sua dama enquanto os movimentos são executados, 
e no caso das damas para evitar problemas
nos giros e outros movimentos que precisem  controle do eixo do corpo; 
além do fato de que algumas pessoas mudam drasticamente seu 
caráter e comportamento, quando estas sofrem os efeitos do álcool.
As linhas de 9 e 10 falam sobre uma dança ou conjunto de expressões artísticas 
afro-brasileiras emolduradas como o nome "samba umbigada" no século XIX, onde
posteriormente nesse mesmo seculo foi chamado de "batuque" e finalmente no inícios do seculo XX foi chamado só 
de "samba" \cite[pp. 47]{diniz2008almanaque} \cite[pp. 85]{sandroni2001feitico}; nestas danças existe
um movimento chamado "umbigada" \cite{da2015historia} que da nome à dança, onde o ventre do homem e da mulher batem geralmente para indicar
a troca de dançarino; assim a letra da música se refere a
 evitar "abusar" de movimentos como a umbigada provem de danças consideradas profanas e não são bem vistas na época.
Finalmente,
a linha 12 fala sobre balançar o corpo, isto deve ser entendido seguindo o contexto cultural da 
época, onde qualquer ato que lembra-se a cultura africana, como umbigada ou capoeira, 
não eram bem vistos, não sendo considerados apropriados para um ambiente mais civilizado; 
assim, esta linha faz referencia a ter um comportamento dentro das normas da ética,
caso contrario seria levado à delegacia.


%\newpage 
%%%%%%%%%%%%%%%%%%%%%%%%%%%%%%%%%%%%%%%%%%%%%%%%%%%%%%%%%%%%%%%%%%%%%%%%%%%%%%%%
%% SECTION
%%%%%%%%%%%%%%%%%%%%%%%%%%%%%%%%%%%%%%%%%%%%%%%%%%%%%%%%%%%%%%%%%%%%%%%%%%%%%%%%
\section{Historia da samba}\index{historia da samba}
O samba é uns dos gêneros musicais mais conhecidos no Brasil \cite[pp. 46-47]{diniz2008almanaque},
entre estes gêneros mais populares temos por exemplo o ``forró'' e o ``Sertanejo'';
sendo que o samba se distingue entre eles, como a principal expressão popular da música brasileira. 

Na literatura latino-americana a palavra ``samba'' é conhecida desde no seculo XIX, 
e na literatura brasileira especificamente desde o ano de 1838; porem sendo a palavra ``samba''
quase desconhecida (em Rio de Janeiro) ate o último quartel do século XIX  \cite[pp. 47]{diniz2008almanaque}\cite[pp. 86]{sandroni2001feitico},
entre as explicações
do origem da palavra ``samba'' a mais conhecida é a que promove que esta vem do 
do idioma quimbundo, sendo derivado da palavra ``semba''  que significa umbigada \cite[pp. 47]{diniz2008almanaque} \cite{da2015historia}.
Uma referencia muito conhecida deste vinculo é a descrita no livro ``O negro e o garimpo em Minas Gerais''
de Mata Machado Filho, onde ele comenta que ``os negros corrigem para semba se 
alguém lhes fala em samba'' \cite[pp. 85]{sandroni2001feitico}. Assim se vê que existe
desde antanho uma relação entre as palavras, 
samba, semba e umbigada.

O gênero samba atual é uma mistura de muitos gêneros musicais de África, América e Europa.


No seculo XIX os viajantes portugueses designavam as danças africanas com a palavra ``batuque'',
e no Brasil existem registros desta palavra desde o século XVIII, sendo que
esta não se usava para referenciar uma dança em particular e sim aos festejos dos negros em geral.
Assim, esta designação foi muito popular ate inícios do seculo XX onde a palavra ``samba''
 virou mais popular para descrever estes festejos. 
\cite[pp. 85]{sandroni2001feitico}.

Entre as danças profanas afro-brasileiras o gesto da umbigada é um elemento muito caraterístico,
de modo que em 1961 (seculo XX) Edson Carneiro definiu englobou as danças que realizam este 
gesto como ``samba-de-umbigada'' . Assim tradições 
musicais como o samba de roda, o jongo, o lundu, o coco, o calango e o cateretê, 
seguindo Edson são englobadas com  ``samba-de-umbigada'' \cite[pp. 85]{sandroni2001feitico}.

 ``batuque''
e posteriormente este seria chamado simplesmente de ``samba'' no seculo XX \cite[pp. 47]{diniz2008almanaque}.


%%%%%%%%%%%%%%%%%%%%%%%%%%%%%%%%%%%%%%%%%%%%%%%%%%%%%%%%%%%%%%%%%%%%%%%%%%%%%%%%
%% SECTION
%%%%%%%%%%%%%%%%%%%%%%%%%%%%%%%%%%%%%%%%%%%%%%%%%%%%%%%%%%%%%%%%%%%%%%%%%%%%%%%%
\section{Historia da samba de gafieira}\index{historia da samba de gafieira}


\textcolor{red}{Historia}
\begin{description}

\item [A] Perna, Marco Antonio (2001). Samba de Gafieira - a história da dança de salão brasileira. ISBN 85-901965-5-0

\end{description}




