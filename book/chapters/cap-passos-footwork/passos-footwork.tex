
%----------------------------------------------------------------------------------------
%	CHAPTER 2
%----------------------------------------------------------------------------------------

\chapter{\textcolor{red}{Trabalho de pés (Footwork) em samba de gafieira}}


\section{\textcolor{red}{Notação para trabalho unipessoal}}


\section{\textcolor{red}{Passos para trabalhar de forma individual}}

\subsection{\textcolor{red}{Trança}}\index{Passo!Trança}

 Figura \ref{fig:abc-pessoaltranca}.

\begin{figure}[h]
  \centering
\begin{abc}[name=abc-pessoaltrancatc,width=0.95\linewidth]
X: 1 % start of header
K: C stafflines=1 % scale: C major
M: 2/4 %meter - compasso
%Q:1/4=80
V:1 clef=perc stem=up name="Ritmo da música" sname="Ritmo"
V:2 clef=perc stem=up name="T. coreográficos"    sname="TC"
[V:1] |: B2  B1  B1 | B2  B1  B1 | B2  B1  B1 | B2  B1  B1:| 
w:       ~   tchic tchic tum tchic tchic tum ~     ~     ~   ~     ~
w:       tum ~     ~     ~   ~     ~     ~   tchic tchic tum tchic tchic tum tchic tchic
w: ~
w: ~
[V:2] |: B1  B1  B1  B1 | B1 B1  B1  B1 | B1 B1  B1  B1 | B1 B1  B1  B1  :| 
w:       ~   ~   TC1 TC2  TC3_ TC5 TC6   TC7 TC8
w:       TC7 TC8 ~   ~    ~  ~  ~  ~     ~   ~   TC1 TC2  TC3_ TC5 TC6 
\end{abc}
\caption{Diagrama de tempos para a trnaça.}
\label{fig:abc-pessoaltranca}
\end{figure}

\subsection{\textcolor{red}{Samba no pe}}\index{Passo!Samba no pé}

\subsection{\textcolor{red}{Samba no pe (para adiante)}}\index{Passo!Samba no pé para adiante}

\subsection{\textcolor{red}{Escovinha pra adiante}}\index{Passo!Escovinha pra adiante}

\subsection{\textcolor{red}{Escovinha pra trás}}\index{Passo!Escovinha pra trás}

\subsection{\textcolor{red}{Escovinha trocando de lados}}\index{Passo!Escovinha trocando de lados}

\subsection{\textcolor{red}{Pica-pau}}\index{Passo!Pica-pau}

\subsection{\textcolor{red}{Pescaria}}\index{Passo!Pescaria}


