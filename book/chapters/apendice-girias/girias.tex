\chapterimage{chapter_head_2.pdf} % Chapter heading image

\chapter{\textcolor{blue}{Neologismos e gírias}}

Dado que no Brasil, no existe um dicionario oficial, 
que regule ou marque o rumbo do significado das palavras;
acho necessário, antes de continuar com as explicações, 
para evitar incorrer, não intencionadamente, em mal entendidos ou ofensas;
emoldurar o uso de alguns termos, nas acepções das palavras propostas por alguns autores.
\begin{definition}[Gíria:] 
\index{Gíria}
\label{def:Giria}
Podem ser achadas as seguintes acepções no Dicionário Priberam da Língua Portuguesa \cite{priberamgiria}:
\begin{itemize}
\item Linguagem característica de um grupo profissional ou sociocultural, é equivalente ao termo JARGÃO.
\item Linguagem usada por determinado grupo, 
geralmente incompreensível para quem não pertence ao grupo e que serve também como meio de realçar a sua especificidade.
\end{itemize}
\end{definition}

\begin{definition}[Neologismo:] 
\index{Neologismo}
\label{def:Neologismo}
Podem ser achadas as seguintes acepções no Dicionário Priberam da Língua Portuguesa \cite{priberamneologismo}:
\begin{itemize}
\item Palavra nova, ou acepção nova de uma palavra já existente na língua.
\item Emprego de palavras novas ou de novas acepções.
\end{itemize}
\end{definition}

\begin{definition}[Neologismo semântico:] 
\index{Neologismo!Neologismo semântico}
\label{def:NeologismoSemantico}
Um neologismo semântico ou neologismo de significado é caraterizado pela modificação do significado de uma palavra já existente na língua;
as gírias em muitas ocasiões constituem exemplos de neologismos semânticos, 
pois algumas gírias dão novos sentidos a palavras já usadas no vocabulário formal \cite[pp. 82-83]{correalingua}.
\end{definition}

\section{Não recomendadas na dança de salão}
%%%%%%%%%%%%%%%%%%%%%%%%%%%%%%%%%%%%%%%%%%%%%%%%%%%%%%%%%%%%%%%%%%%%%%%%%%%%%%%%
\input{chapters/apendice-girias/giria-contratempo} %[OK]

\subsection{Neologismos+Gíria: Ritmo}
