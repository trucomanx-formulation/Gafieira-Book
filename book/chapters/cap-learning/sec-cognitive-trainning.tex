
\section{Treinamento cognitivo}
\label{sec:sec-cognitive-trainning}
\index{Treinamento cognitivo}
\index{Treino cognitivo}

Entre as técnicas de ``intervenção cognitiva''\footnote{Procedimento (intervenção) 
realizado sobre a cognição de um individuo.} se 
destacam o treinamento cognitivo (treino cognitivo) e a estimulação cognitiva
\cite[pp. 10]{mastersthesisLima2017}.
\begin{itemize}
\item \textbf{A estimulação cognitiva} procura uma melhora funcional mediante o exercício diário da mente e 
a repetição de determinadas atividades promotoras da estimulação mental
\cite[pp. 14]{chariglione2013contribuiccoes} \cite[pp. 10]{mastersthesisLima2017}.
\item \textbf{O treinamento cognitivo} procura melhorar o desempenho de uma tarefa 
e/ou funções cognitivas por meio do ensino de estratégias e a adquisição de novas habilidades motoras
\cite[pp. 14]{chariglione2013contribuiccoes} \cite[pp. 10]{mastersthesisLima2017}.
\end{itemize}


%%%%%%%%%%%%%%%%%%%%%%%%%%%%%%%%%%%%%%%%%%%%%%%%%%%%%%%%%%%%%%%%%%%%%%%%%%%%%%%%
\subsection{Treinamento cognitivo-motor}
\label{subsec:sec-motor-cognitive}
\index{Treino cognitivo-motor}


Estudos sobre a cognição estão apontando que o declino cognitivo através do tempo no ser humano 
é evitável e que o processo de envelhecimento do cérebro humano é rico em reorganizações e mudanças;
as evidencias tem sugerido que o cérebro, mesmo entre as pessoas mais velhas, conserva varias vias de plasticidade
\cite{Gutchess579} \cite{wollesen2020effects} \cite[pp. 22]{variseefeitos}.

\begin{definition}[Cognição motora]
\label{def:cognitivo-motor}
\index{Cognição motora}
Chama-se cognição motora (do inglês ``motor cognition'') à habilidade de planejar e executar um movimento;
a cognição motora abrange o como entendemos nosso próprio movimento e como mediante este percebemos o mundo \cite{fuentes2007motor}  \cite[pp. 6]{mastersthesisLima2017}.
Assim a cognição motora estuda como realizamos uma ação para obter um fim, 
que tanto pode ser para um objetivo mecânico (pular ou caminhar) como em reação a uma interação social (dançar).
Os temas que estão agrupados sob o termo de cognição motora são: 
a intenção de agir, planejamento motor, iniciação motora e controle de ação, entre outros \cite{herbet2019awake}.
\begin{comment}
``A chave para a compreensão da natureza da cognição motora é o ciclo percepção-ação, 
ou seja, a transformação da percepção de um padrão perceptivo em padrões de 
movimentos coordenados'' \cite[pp. 6]{mastersthesisLima2017}.
\end{comment}
\end{definition}


O treinamento cognitivo-motor (ou treinamento da \hyperref[def:cognitivo-motor]{\textbf{cognição motora}}) 
tem se demostrado eficiente para aumentar a resistência aos processos degenerativos do cérebro humano que acontecem quando envelhecemos \cite[pp. 22]{variseefeitos}.
Atualmente, especula-se que o treinamento cognitivo-motor seja mais eficiente para melhorar a cognição, 
quanto comparado com a prática isolada de exercícios cognitivos ou físicos 
\cite{wollesen2020effects} \cite[pp. 22]{variseefeitos}
Também é interessante ressaltar que os exercícios de treinamento cognitivo-motor com 
uma forte associação da atividade cognitiva e motora
tem se mostrado mais eficiente que os exercícios de treinamento cognitivo-motor de tarefas em sequencia 
que se mostraram inconclusivos
\cite{wollesen2020effects} \cite[pp. 22]{variseefeitos}. 
Esto tem sentido pois para conseguir atingir a multitarefa o cérebro leva
ao modo inconsciente a maior quantidade possível de tarefas. 
Por exemplo: quando andamos automaticamente na rua enquanto fazemos a soma das compras do mercado,
e prestamos atenção ao transito.

%%%%%%%%%%%%%%%%%%%%%%%%%%%%%%%%%%%%%%%%%%%%%%%%%%%%%%%%%%%%%%%%%%%%%%%%%%%%%%%%
\subsection{Treinamento cognitivo-motor multicomponente}
\label{subsec:sec-motor-cognitive-multicomponente}
\index{Treino cognitivo-motor multicomponente}
Um treinamento cognitivo com múltiplos componentes apresenta requerimentos mais compatíveis 
com nosso dia a dia (andar, vigiar e processar o entorno), 
quando o comparamos com um treino cognitivo-motor com dupla tarefa
\cite[pp. 23]{variseefeitos}. 
Os estudos na área apontam que um aumento da demanda cognitiva num treino cognitivo-motor multicomponente 
(coordenação, equilíbrio, fortalecimento, agilidade, etc.), 
parece promover um aumento superior na melhora cognitiva
\cite[pp. 24]{variseefeitos} \cite[pp. 44]{biehl2016pratica} \cite{medeiros2018impacto}. 

