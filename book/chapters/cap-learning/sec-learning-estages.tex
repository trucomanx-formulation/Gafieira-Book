%%%%%%%%%%%%%%%%%%%%%%%%%%%%%%%%%%%%%%%%%%%%%%%%%%%%%%%%%%%%%%%%%%%%%%%%%%%%%%%%
%%%%%%%%%%%%%%%%%%%%%%%%%%%%%%%%%%%%%%%%%%%%%%%%%%%%%%%%%%%%%%%%%%%%%%%%%%%%%%%%
\section{Estágios da aprendizagem}
\label{sec:aprendizagem}
\index{Aprendizagem!Estágios}
Quando desejamos apreender uma nova habilidade ou conhecimento, já seja tocar o piano,
resolver problemas matemáticos,
jogar futebol, ou dançar; devemos atravessar por um processo de aprendizagem dividido em estágios;
o conhecimento destes estágios nos permitirá modelar melhores estrategias e exercícios para o aprendizado,
focando-nos no estagio em que se encontre cada pessoa.

De acordo com a programação neuro linguística (PNL), 
entre os estágios do processo (ciclo) de aprendizagem podemos achar:
a incompetência inconsciente, a incompetência consciente, 
a competência consciente e finalmente a competência inconsciente 
\cite[pp. 249]{seymourtreinando} \cite[pp. 10]{passadori7} \cite{de2013treinamentos};
ver Tabela \ref{tab:learning1}.

\begin{table}[!h]
  \centering
  \begin{tabular}{|l||l|l|}
   \hline
    ~             & incompetência & competência \\ \hline \hline
    inconsciente  & Estágio 1     & Estágio 4   \\ \hline
    consciente    & Estágio 2     & Estágio 3   \\ \hline
  \end{tabular}
  \caption{Estágios da aprendizagem.}
  \label{tab:learning1}
\end{table}

\begin{description}
%%%%%
\item[incompetência inconsciente:] Neste estágio da aprendizagem não temos consciência do que devemos fazer ou aprender,
\label{ref:IncompetenciaInconsciente}
e consequentemente temos incompetência na área que desejamos desenvolve-nos e/ou vivemos numa ``alegre ignorância'';
``você não sabe o que não sabe'' \cite[pp. 29, 252]{seymourtreinando} \cite{carnegie2014lideranca} \cite{de2013treinamentos}.
Existe a possibilidade que ao desconhecer nossas próprias limitações,
tendamos achar que o problema da falta de sucesso é externo a nós \cite[pp. 10]{passadori7}.
%%%%%
\item[incompetência consciente:] Neste estágio você sabe o que deve fazer ou aprender,
\label{ref:IncompetenciaConsciente}
porem não consegue realizar a tarefa de forma competente;
por este motivo a tarefa exige toda nossa atenção, 
este estágio da aprendizagem pode trazer algum desconforto,
mas a taxa de aprendizagem é maior neste estágio;
``você descobre o que não sabe, consequentemente, descobre uma incompetência nesse ponto''
\cite[pp. 29]{seymourtreinando} \cite{carnegie2014lideranca} \cite[pp. 10, 11]{passadori7} \cite{de2013treinamentos}.
%%%%%
\item[competência consciente:] 
\label{ref:CompetenciaConsciente}
Neste estagio somos competentes, 
porem precisamos de toda nossa atenção consciente 
para realizar uma determinada tarefa; a tarefa ainda não tem-se tornado um hábito em nós 
\cite[pp. 30, 249]{seymourtreinando} \cite{carnegie2014lideranca} \cite{de2013treinamentos}.
Esta é a fase do condicionamento positivo,
onde a pessoa deve estar em exercício continuo com reforços positivos para fixar o aprendizado \cite[pp. 11]{passadori7}.
%%%%%
\item[competência inconsciente:] 
\label{ref:CompetenciaInconsciente}
Neste estágio a habilidade em estudo tem sido plenamente integrada em nós, 
e podemos realizá-la sem esforço, de forma inconsciente, com confiança e naturalidade, 
de modo que podemos realizar outras atividades em paralelo;
``maestria intuitiva e instintiva'' \cite[pp. 30, 249]{seymourtreinando} \cite[pp. 11]{passadori7} \cite{de2013treinamentos}.
\end{description}


\begin{elaboracion}[title=O cerebelo e o comportamento inconsciente, width= 1.00\linewidth]
%   https://books.google.com.br/books?id=cf3AAdIH1UQC&pg=PA35&dq=compet%C3%AAncia+inconsciente+cerebelo&hl=pt-BR&sa=X&ved=0ahUKEwir6qfn4OzkAhVzKLkGHTvZD1wQ6AEIUzAF#v=onepage&q=compet%C3%AAncia%20inconsciente%20cerebelo&f=false
O cerebelo (``cérebro pequeno'' em latim) é uma protuberância ampla e convoluta,
localizada na fossa posterior do crânio, 
e está conectada à parte traseira do tronco encefálico \cite[pp. 93]{gazzanigaciencia} \cite[pp. 87]{carneiro2004atlas}
\cite[pp. 516]{bearneurociencias}.

A função mais evidente do cerebelo é a aprendizagem motora e a memoria motora,
sendo o grande coordenador da ação muscular ``inconsciente'';
ao contrario do cérebro que atua de forma ``consciente'' 
\cite[pp. 93]{gazzanigaciencia} \cite[pp. 87]{carneiro2004atlas} \cite[pp. 516]{bearneurociencias}.

O cerebelo é quem nos permite estar caminhado de forma inconsciente,
e ao mesmo tempo poder raciocinar uma fala com alguém ou planejar nosso dia de trabalho.
\end{elaboracion}
