\newpage
%%%%%%%%%%%%%%%%%%%%%%%%%%%%%%%%%%%%%%%%%%%%%%%%%%%%%%%%%%%%%%%%%%%%%%%%%%%%%%%%
%%%%%%%%%%%%%%%%%%%%%%%%%%%%%%%%%%%%%%%%%%%%%%%%%%%%%%%%%%%%%%%%%%%%%%%%%%%%%%%%
%%%%%%%%%%%%%%%%%%%%%%%%%%%%%%%%%%%%%%%%%%%%%%%%%%%%%%%%%%%%%%%%%%%%%%%%%%%%%%%%

\section{Emoções e sentimentos}
\label{ref:emotionsentimental}
Mesmo que sejam tratados como similares, 
existem diferencias entre as emoções e os sentimentos.
Por exemplo é sabido que:
\begin{itemize}
%
\item Os sentimentos são pessoais e biográficos, enquanto que as emoções são sociais \cite[pp. 42]{hofman2015affective}.
%
\item Todas as emoções provocam sentimentos, porem não todos os sentimentos provem de emoções
\cite[pp. 288]{zanelli2014psicologia} \cite{freitas2015codigo}.
\begin{example}
Você pode emocionar-se positivamente por algo que viu e logo sentir-se num estado duradouro de satisfação pessoal;
porem você pode sentir um estado duradouro de satisfação pessoal sem que alguma emoção tenha desencadeado esse estado.
\end{example}
%
\item As emoções a diferença dos sentimentos podem ser fingidos ou genuínos \cite[pp. 32]{nicolas2018musicas}.
\begin{example}[Ri palhaço\, ri:]
Imaginemos a um palhaço rindo para o público, expressando a emoção da felicidade,
mesmo que ele no seu coração sinta\footnote{Um pode enganar aos demais mais não pode auto enganar-se.} 
um vazio existencial pela perda de um ser querido,
acontecido já faz alguns meses.
\end{example}
\item As emoções são a reação a um estimulo, enquanto que os sentimentos vem de um processo cognitivo \cite{freitas2013psicologia}.
\begin{example}[Transitório vs. Estacionário:] ~

\begin{itemize}
\item Você pode ter raiva (emoção) de forma transitória porque alguém pisou no seu pé.
\item Você pode ter todos os dias, e de forma continua, um ``sentimento'' antagônico (raiva) sobre sua vida, 
pois não esta conforme com suas decisões.
\end{itemize}
\end{example}
\end{itemize}

\index{Emoções}
\subsection{As emoções} 
\label{subsec:emotion}
São fenômenos complexos e com múltiplas dimensões,
sendo a emoção uma resposta automática, rápida, de curta vida e intensa, 
que pode chegar a nos de forma consciente ou inconsciente;
sendo este um impulso neuronal que provoca no organismo a execução de uma ação,
como comportamentos de aproximação ou afastamento
\cite[pp. 288]{zanelli2014psicologia}  \cite{freitas2015codigo}.
As funções da emoção estão ligadas à adaptação e à expressão, 
sendo este um catalisador entre nossa conduta e o meio que nos embrulha;
as emoções também cumprem um papel importante no desenvolvimento da aprendizagem 
(reforços positivos ou negativos)
  \cite{freitas2015codigo},
e permite a articulação social, política e cultural dos afetos\footnote{O 
afeto é visto como uma ``matéria-prima'' da emoção \cite[pp. 43]{hofman2015affective}.} 
\cite[pp. 43]{hofman2015affective}.

As emoções básicas no ser humano são \cite{freitas2015codigo} \cite[pp. 291]{zanelli2014psicologia}:
\begin{inparaitem}
\item felicidade (alegria)  % \cite{freitas2015codigo} \cite[pp. 291]{zanelli2014psicologia}
\item tristeza              % \cite{freitas2015codigo} \cite[pp. 291]{zanelli2014psicologia}
\item repugnância (aversão) % \cite{freitas2015codigo} \cite[pp. 291]{zanelli2014psicologia}
\item surpresa              % \cite{freitas2015codigo} \cite[pp. 291]{zanelli2014psicologia}
\item medo                  % \cite{freitas2015codigo} \cite[pp. 291]{zanelli2014psicologia}
\item raiva (cólera)        % \cite{freitas2015codigo}
\item desprezo              % \cite{freitas2015codigo}
\end{inparaitem}.

Os componentes da emoção são \cite[pp. 26]{redorta2006emocion} \cite{freitas2015codigo} \cite{freitas2013psicologia} :
\begin{description}
\item[Vivencia consciente:] (Ou componente cognitiva) Sensações que as pessoas vivenciamos subjetivamente.
Podemos sentir medo, angustia, raiva, entre outros.
Por exemplo, uma criança ao ver um palhaço pode provocar-lhe alegria, ou medo,
sendo este resultado subjetivo a cada pessoa. 
\item[Reações fisiológicas:] (Ou componente neurofisiológica) Órgão e sistemas emergentes da atividade emocional.
este se manifesta com respostas ``involuntárias'' como: 
taquicardia, 
sudoração, 
vasoconstrição, 
hipertensão, 
mudanças no tom muscular,
rubor, 
sequidade na boca, 
secreções corporais, 
etc.
\item[Comportamento expressivo:] (Ou componente comportamental) 
Como as pessoas expressam essas emoções de forma verbal e não verbal.
No caso da forma verbal, podem ser percebidas, mudanças do tom na voz, no volumem, no ritmo, entre outros;
e no caso da forma não verbal, podemos observar mudanças na forma como o corpo se movimenta,
como um andar erguido ou encurvado, lento ou rápido, etc.

\end{description}


%% \cite{freitas2015codigo}
%% \begin{itemize}
%% \item A cognição
%% \item a expressão facial
%% \item atividades do sistema nervoso autônomo (SNA)
%% \end{itemize}

\index{Sentimentos}
\subsection{Os sentimentos} 
\label{subsec:filling}
São um ``processo cognitivo'' 
de acompanhamento continuo de uma experiencia subjetiva, 
que pode ser provocada, ou não, por emoções \cite[pp. 288]{zanelli2014psicologia} \cite{freitas2013psicologia}.
sentimento refere-se à experiência subjetiva de emoções e humores (mood em inglês) \cite[pp. 42]{hofman2015affective}.
Assim, podemos separar os sentimentos em dois categorias \cite[pp. 288]{zanelli2014psicologia}:
\begin{itemize}
\item Os sentimentos provocados por emoções, proveniente de alterações corporais.
Sendo estes sentimentos um juízo, ou processionamento, que fazemos sobre as emoções. 
\item Os sentimentos de fundo, este é originado pela existência humana.
\end{itemize}

