\section{Compasso e métrica}
\index{Música!Compasso}
\label{sec:compaso}



\begin{description}
\item[Compasso:] \label{def:Compasso} Se define compasso (``measure'' em inglês)
como uma agrupação de \hyperref[sec:Tempo]{\textbf{tempos}} regulares na sua duração,
onde o primeiro destes tempos normalmente é o mais acentuado,
porém a regularidade nesta acentuação pode mudar pela aparição de elementos como \hyperref[sec:contratempo]{\textbf{contratempos}}, 
\hyperref[sec:sincope]{\textbf{sincopas}}, ou enfases em algumas notas  \cite[pp. 513]{apel1969harvard}. 
O número de \hyperref[sec:Tempo]{\textbf{tempos}} no compasso pode ser, dois, trés, quatro, ou ocasionalmente 5 ou mais;
assim, estes tempos são usados como a unidade de medida (temporal) do compasso.
Para diferenciar o inicio e o final dos compassos, 
estes são separados por barras verticais. 
O esquema básico dos valores das notas dentro de um compasso é chamado de \hyperref[def:Metrica]{\textbf{métrica}}  ou \hyperref[def:Metrica]{\textbf{metro}}.
\begin{example}
Na Figura \ref{fig:abc-exemplocompasso1} podemos ver 4 compassos separados por barras verticais.
A quantidade de figuras musicais dentro do compasso varia, 
porém todos todos os compassos tem uma duração de  1 \Halb~(uma mínima).
\end{example}
 
\begin{figure}[h]
\centering
\begin{abc}[name=abc-exemplocompasso1]
% abcm2ps exemplocompasso1.abc  -O exemplocompasso1.ps
% ps2epsi exemplocompasso1.ps exemplocompasso1.eps
%
X: 1 % start of header
K: none stafflines=0 %K: C %% Escala de C mayor %
M: 2/4
%T: Contratempo num compasso binário
V:1 clef=perc stem=up %name="Ritmo 1"   sname="Ritmo 1"
%
[V:1] | B2 B1 B1| B2 B1 B1 | B2 B1 B1 | B2 z2  |
%       
\end{abc}
\caption{Ritmo usando 4 compassos.}
\label{fig:abc-exemplocompasso1}
\end{figure}

\item[Métrica:] \label{def:Metrica} \index{Música!Métrica}
A métrica ou o metro  (``meter'' em inglês) é
um padrão de unidades temporais regulares (também chamados de \hyperref[sec:Tempo]{\textbf{tempos}}), 
sobre o qual uma peça musical ou uma seção dela é medida ou organizada;
uma agrupação métrica completa se denomina \hyperref[def:Compasso]{\textbf{compasso}} \cite[pp. 947]{latham2008diccionario} \cite[pp. 523]{apel1969harvard}.
A métrica é indicada geralmente por uma fração, por exemplo:
${2}/{2}$ , ${3}/{4}$ , ${4}/{4}$, etc; 
em português esta fração é chamada de \hyperref[def:FormulaCompasso]{\textbf{fórmula do compasso}}.
Assim, a métrica de uma porção de peça musical nos indica como os 
\hyperref[sec:Tempo]{\textbf{tempos}} são distribuídos nos compassos,
e consequentemente como será o acento métrico de cada um destes tempos. 
Isto será evidente, mais adiante, 
quando sejam explicadas as caraterísticas dos compassos binários, ternários, quaternários, etc.
\begin{example}
A Figura \ref{fig:abc-exemplocompasso1} mostra 4 compassos com a mesma métrica;
é dizer, igual acentuação e igual número de tempos por compasso.
Neste caso a acentuação é regular dado que todos os compassos usam a mesma formula de compasso, (2/4), 
e não existe nenhuma grafia especial que tire a regularidade das acentuações;
de modo que todos os compassos estão acentuados no primeiro \hyperref[sec:Tempo]{\textbf{tempo}}.
\end{example}

\begin{example}
A Figura \ref{fig:abc-exemplometrica1} mostra 4 compassos com duas distintas métricas;
os dois primeiros usam uma métrica (2/4) e os dois últimos uma métrica (3/4).
\end{example}

\begin{figure}[h]
\centering
\begin{abc}[name=abc-exemplometrica1]
% abcm2ps exemplometrica1.abc  -O exemplometrica1.ps
% ps2epsi exemplometrica1.ps exemplometrica1.eps
%
X: 1 % start of header
K: none stafflines=0 %K: C %% Escala de C mayor %
M: 2/4
%T: Contratempo num compasso binário
V:1 clef=perc stem=up %name="Ritmo 2"   sname="Ritmo 2"
%
[V:1] | B2 B1 B1| B2 B1 B1 |\
M: 3/4 
B2 B1 B2 B1 | B2 B1 B1 z2  |
%       
\end{abc}
\caption{Ritmo com mudança de métrica.}
\label{fig:abc-exemplometrica1}
\end{figure}

\item[Formula do compasso:] \label{def:FormulaCompasso} \index{Música!Fórmula de compasso}
A fórmula do compasso  (``time signature'' em inglês), também chamado indicador do compasso ou indicador do tempo,
é um símbolo que normalmente se representa por uma fração (com ou sem traço), 
e se escreve ao inicio da peça musical, depois da clave; 
ou de uma forma geral, 
ao inicio de algum compasso para indicar a mudança da \hyperref[def:Metrica]{\textbf{métrica}} na música \cite[pp. 760]{latham2008diccionario}  \cite[pp. 852]{apel1969harvard}.
O numerador, da fórmula do compasso, indica o número de \hyperref[sec:Tempo]{\textbf{tempos}} que compõem cada \hyperref[def:Compasso]{\textbf{compasso}} nessa \hyperref[def:Metrica]{\textbf{métrica}}.
Por outro lado, 
o denominador da formula do compasso nos informa a duração de cada um dos \hyperref[sec:Tempo]{\textbf{tempos}}, 
dos compassos pertencentes a essa \hyperref[def:Metrica]{\textbf{métrica}}.
A Tabela \ref{tab:abc-noteslength} em conjunto com a Tabela \ref{tab:abc-noteslengthbasic}, 
indicam o significado do denominador da fórmula do compasso; 
\begin{table}[H]
\centering
\begin{tabular}{|c|c|c|}
\hline
denominador & Figura  & Duração\\ \hline
\hline
$1$   & \fullnote    & $S$ \\ \hline
$2$ & \halfnote    & $S/2$  \\ \hline
$4$ & \quarternote & $S/4$  \\ \hline
$8$ & \eighthnote  & $S/8$  \\ \hline
\end{tabular}
\caption{Duração e símbolos de algumas figuras musicais.}
\label{tab:abc-noteslength}
\end{table}
onde a primeira coluna mostra o denominador da fórmula,
a segunda coluna mostra as figuras musicais que representam cada um dos \hyperref[sec:Tempo]{\textbf{tempos}} do compasso, e 
a terceira indica a duração da figura musical.
É importante
ressaltar que a duração das figuras musicais é relativa, como pode ser visto
na terceira coluna da Tabela \ref{tab:abc-noteslength}, onde estas estão em função
da duração $S$ da semibreve. 
\begin{example}
A Figura \ref{fig:abc-exemplocompasso1} mostra 4 \hyperref[def:Compasso]{\textbf{compassos}} com uma formula de compasso (2/4).
Isto implica que cada compasso tera 2 tempos, 
representados por figuras musicais de 1/4 de semibreve;
é dizer, cada tempo equivale a uma semínima (\Vier). 
\end{example}
\begin{example}
A Figura \ref{fig:abc-exemplometrica1} mostra 4 \hyperref[def:Compasso]{\textbf{compassos}} usando dois diferentes formulas de compasso.
Sendo que os dois primeiros compassos tem dois tempos cada um, 
e os dois últimos compassos tem 3 tempos cada um.
Em todos os casos, os tempos tem durações de figuras musicais de 1/4 de semibreve;
é dizer, cada tempo equivale a uma semínima (\Vier). 
\end{example}
\begin{example}
Outros exemplos de formula de compasso comumente usadas são:
\begin{itemize}
\item A fórmula $\mathbf{2}/2$ com compassos com uma duração de $\mathbf{2}$\halfnote ~(duas mínimas) e
\item a fórmula $\mathbf{4}/4$ com compassos com uma duração de $\mathbf{4}$\quarternote ~(quatro semínimas). 
\end{itemize}
\end{example}
\end{description}~\\


Se classificamos os compassos por sua \hyperref[def:Metrica]{\textbf{métrica}}, 
os três tipos de compassos mais conhecidos são: 
os \hyperref[subsec:compassobinario]{\textbf{compassos binários}}, 
\hyperref[subsec:compassoternario]{\textbf{ternários}} e 
\hyperref[subsec:compassoquaternario]{\textbf{quaternários}} \cite[pp. 27]{adolfo2002musica}.



%%%%%%%%%%%%%%%%%%%%%%%%%%%%%%%%%%%%%%%%%%%%%%%%%%%%%%%%%%%%%%%%%%%%%%%%%%%%%%%%
\subsection{Acentuação dos tempos (parte 1)}
\label{subsec:acentuacion1}
\index{Música!Acentuação}
\index{Música!Tempo!Tempo forte}
\index{Música!Tempo!Tempo fraco}



\begin{tcbinformation} 
\textbf{Pulso:}
\index{Música!Pulso}
\label{ref:Pulso}
O pulso é uma occorrencia regular, porém não organizada por acentos;
o pulso não tem significado até que seja organizado em batidas acentuadas (ou fortes) 
e não acentuadas(ou fracas) \cite[pp. 22]{holland2013music}. 
Exemplo: Os batimentos do coração, uma torneira pingando.

O termo é usado comunmente como equivalente a ``tempo'',
porém é possivel fazer uma difereça.
Exemplo: Num compás 6/8 temos 6 pulsos porém só 2 tempos \cite[pp. 1228]{latham2008diccionario}.
\end{tcbinformation} 

\begin{notation}[]
Nas seguintes seções,
serão usadas as seguintes notações de símbolos para designar aos tempos fortes, fracos e semifortes:
\begin{itemize}
\item ``F''  indica que é um tempo forte, 
\item ``sF'' indica que é um tempo semiforte, 
\item ``f''  indica que é um tempo fraco,
\end{itemize}
\end{notation}


\begin{description}
\item[Acentos principais:] \label{def:acentoprincipal} 
São os acentos dos tempos do compasso  \cite[pp. 142]{medteoria}.
Todos os compassos tem um acento primario no primeiro tempo \cite[pp. 24]{crowther2003usted}.
\item[Acentos secundários:] \label{def:acentosecundario} 
São os acentos das subdivisões dos tempos do compasso \cite[pp. 142]{medteoria}.
Todos os compassos com mais de 3 tempos tem além do acento primário um ou mais acentos secundários \cite[pp. 25]{crowther2003usted},
\begin{itemize}
\item Os compassos quaternários tem um acento secundário no terceiro tempo,
\item os compassos com formula do compasso com numerador 6  tem um acento secundário na quarta subdivisão do compasso, e
\item os compassos com formula do compasso com numerador 9  tem    acentos secundários na quarta e sétima subdivisão do compasso.
\end{itemize} 
\end{description}

Mais detalhes sobre a acentuação dos tempos podem ser vistos na Seção \ref{subsec:acentuacion2}.

\begin{tcbattention}
Os termos ``acentuado'' e ``não acentuado'', 
indicam uma medida relativa das \hyperref[sec:pos:Intensidade]{\textbf{intensidades}} dos sons, 
e em nenhum caso indicam uma intensidade nula para as notas não acentuadas. 
\end{tcbattention}


%%%%%%%%%%%%%%%%%%%%%%%%%%%%%%%%%%%%%%%%%%%%%%%%%%%%%%%%%%%%%%%%%%%%%%%%%%%%%%%%
\subsection{Compasso binário}
\label{subsec:compassobinario}
\index{Música!Compasso!Compasso Binário} O compasso binário ou compasso binário simples,
é uma estrutura que se carateriza por ter compassos com uma  duração de dois \hyperref[sec:Tempo]{\textbf{tempos}},
sendo o primeiro tempo forte (acentuado), e o segundo de tempo fraco (não acentuado)
\cite[pp. 41]{grabner2001teoria} \cite[pp. 66]{adolfo2002musica}\cite[pp. 28]{alves2004teoria}.
Os compassos binários (simples) tem uma fórmula de compasso na forma $2/B$,
onde $B$ pode ser $2$, $4$, $8$, etc. 
\begin{example}
Por exemplo temos, as fórmulas de compassos binários simples: $2/2$, $2/4$, $2/8$,  etc.
\end{example}
\begin{example}
A Figura \ref{compasso:binario}, representa um exemplo de compasso binário simples, 
com fórmula de compasso $2/2$, estes compassos podem ser preenchidos com $2$\halfnote, 
e tempos com uma duração de $S/2$ (uma \halfnote).
Neste caso, podemos ver 3 compassos que são preenchidos com $2$ notas representadas por \halfnote.
Assim, a primeira nota de cada compasso representa um tempo forte (acentuado), e a segunda um tempo fraco (não acentuado).
\end{example}
\begin{figure}[H]
\centering
\begin{abc}[name=abc-compasso1,width=0.70\linewidth]
X: 1 % start of header
K: C % scale: C major
M: 2/2 %meter - compasso
"1ro compasso" G4 F4 |"2do compasso" G4 D4 |"3ro compasso" F4 D4  |
w: F f F f  F f
\end{abc}
\caption{Exemplo de compasso binário (simples).}
\label{compasso:binario}
\end{figure}

Se falamos de forma mais geral, 
podemos ter dois tipos de compassos binários: os simples e os compostos.
Assim, 
para achar a fórmula de um compasso composto, correspondente a um compasso simples (usando quiálteras de três)
usamos a seguinte operação \cite[pp. 74]{alves2004teoria}, 
\begin{equation}\label{eq:comcomposto}
Compasso~simples\times\frac{3}{2}=Compasso~composto.
\end{equation}
De modo que obtemos compassos binários compostos com as seguintes fórmulas de compasso: 
$6/4$, $6/8$, $6/16$, etc.
Os compassos binários compostos, tem a mesma acentuação métrica que os compassos simples,
porém os acentos estão sujeitos a subdivisões ao igual que os tempos. Assim, um compasso binário composto 
tem 2 tempos, um tempo forte (acentuado) no tempo 1 e um tempo fraco (acentuação menor) no tempo 2. 
As subdivisões destes dois tempos se repartirão o acento de forma proporcional,
sendo a primeira parte de cada tempo a mais acentuada \cite[pp. 142]{medteoria} \cite[pp. 7-11]{mascarenhascurso} \cite[pp. 41]{grabner2001teoria}.

\begin{example}
A Figura \ref{compasso:binariocomposto}, representa um exemplo de compasso binário composto, 
com fórmula de compasso $6/4$, 
onde cada um dos dois tempos tem uma duração de $3S/4$ (uma  \halfnote, que é equivalente a \halfnote + \quarternote). 
No exemplo temos duas linhas, na parte inferior da pauta,  preenchidas com os símbolos ``F'' e ``f'',
que representam uma acentuação forte e fraca respetivamente;
a segunda linha representa aos \hyperref[def:acentoprincipal]{\textbf{acentos principais}} dos tempos do compasso, e
a primeira linha representa aos \hyperref[def:acentosecundario]{\textbf{acentos secundários}},
ou seja como estará subdividada a acentuação de cada tempo.

O primeiro compasso é preenchido com duas figuras \halfnote~pontuadas, 
representando estas figuras na pauta, um tempo forte (acentuado) e um tempo fraco (menos acentuado),
respetivamente.
O segundo  compasso é preenchido com $6$ figuras \quarternote, 
que representam as subdivisões dos tempos do compasso, de modo que cada tempo é subdividido em 3 partes iguais.
Podemos ver na primeira linha, inferior à pauta, a relação das acentuações nas subdivisões de cada tempo, isto é F-f-f;
e  na segunda linha está representado a relação das acentuações nos tempos, isto é F-f. 
Assim, a primeira subdivisão terá uma acentuação forte (igual a F), 
a quarta subdivisão uma acentuação menor (igual a f) e  resto serão subdivisões sem acentuação (ou menor a f) \cite[pp. 41]{grabner2001teoria} \cite[pp. 19]{phillips2002sight}.
\end{example}
\begin{figure}[H]
\centering
\begin{abc}[name=abc-compasso1c]
X: 1 % start of header
K: C % scale: C major
M: 6/4 %meter - compasso
"1er compasso" G6 F6 |"2do compasso" G2 D2 D2 F2 D2 D2 |
w: ~ ~ F f f F f f 
w: F f F _ _ f _ _ 
\end{abc}
\caption{Exemplo de compasso binário composto.}
\label{compasso:binariocomposto}
\end{figure}
Alguns autores consideram aos compassos quaternários (ex: 4/4, 4/8) como um caso de compasso binário,
chamando eles de compasso binário duplo \cite[pp. 41]{grabner2001teoria}.



%%%%%%%%%%%%%%%%%%%%%%%%%%%%%%%%%%%%%%%%%%%%%%%%%%%%%%%%%%%%%%%%%%%%%%%%%%%%%%%%
\subsection{Compasso ternário}
\label{subsec:compassoternario}
\index{Música!Compasso!Compasso Ternário} O compasso ternário ou compasso ternário simples,
é uma estrutura que se carateriza por ter compassos com trés tempos,
sendo o primeiro tempo forte (acentuado) e os outros dois fracos (não acentuados) 
\cite[pp. 67]{adolfo2002musica}\cite[pp. 30]{alves2004teoria}. 
Os compassos ternários (simples) tem uma fórmula de compasso da forma $3/B$, 
onde $B$ pode ser $2$, $4$, $8$, etc.
\begin{example}
Por exemplo temos, as fórmulas de compassos ternários simples: $3/2$, $3/4$, $3/8$,  etc.
\end{example}
\begin{example}
A Figura \ref{compasso:ternario}, representa um exemplo com 3 compassos ternários (simples), com 
fórmula de compasso $3/4$, onde cada tempo tem uma duração de $S/4$, é dizer uma \quarternote.
Os compassos são preenchidos com $3$ figuras \quarternote, de modo que cada figura musical representa um tempo.
Na Figura \ref{compasso:ternario}  é fácil perceber
que em todos os compassos, só a nota que é executada no tempo 1 é acentuada.
\end{example}
\begin{figure}[H]
\centering
\begin{abc}[name=abc-compasso2,width=1.0\linewidth]
X: 1 % start of header
K: C % scale: C major
M: 3/4 %meter - compasso
"1ro compasso" G2 F2 F2 |"2do compasso" G2 F2 E2 | "3ro compasso" D2 D2  D2  |
w: F f f  F f f   F f f 
\end{abc}
\caption{Exemplo de compasso ternário.}
\label{compasso:ternario}
\end{figure}

São chamados de compassos ternários compostos,  
quando estes tem uma fórmula de compasso como: $9/4$, $9/8$ e $9/16$.
Para gerar estas formulas de compassos compostos a partir de suas versões simples,
se segue a mesma operação descrita na Equação \ref{eq:comcomposto}.

%%%%%%%%%%%%%%%%%%%%%%%%%%%%%%%%%%%%%%%%%%%%%%%%%%%%%%%%%%%%%%%%%%%%%%%%%%%%%%%%
\subsection{Compasso quaternário}
\label{subsec:compassoquaternario}
\index{Música!Compasso!Compasso Quaternário} O compasso quaternário ou compasso quaternário simples,
é uma estrutura que se carateriza por ter compassos com quatro tempos,
sendo o primeiro tempo, forte (acentuado), o segundo fraco (não acentuado), 
o terceiro semiforte (acentuado porém menor) e o último fraco (não acentuado) 
\cite[pp. 67]{adolfo2002musica}\cite[pp. 32]{alves2004teoria}. 
Os compassos quaternários (simples) tem uma fórmula de compasso da forma $4/B$, 
onde $B$ pode ser $2$, $4$, $8$, etc.
\begin{example}
Por exemplo temos, as fórmulas de compassos ternários simples: $4/2$, $4/4$, $4/8$,  etc.
\end{example}
\begin{example}
A Figura \ref{compasso:quaternario}, representa um exemplo de 3 compassos quaternários, com 
fórmula de compasso $4/4$, onde cada tempo tem uma duração de $S/4$, é dizer uma \quarternote.
Os compassos são preenchidos com $4$ figuras \quarternote, de modo que cada figura representa um tempo.
Na Figura \ref{compasso:quaternario}  é fácil perceber
que em todos os compassos, só as notas que são executadas no tempo 1 e 3 são acentuadas.
\end{example}
\begin{figure}[H]
\centering
\begin{abc}[name=abc-compasso3]
X: 1 % start of header
K: C % scale: C major
M: 4/4 %meter - compasso
"1ro compasso" G2 D2 F2 D2|"2do compasso" F2 D2 C2 F2 |"3ro compasso"  E2 D2 C2 C2|
w: F f sF f   F f sF f   F f sF f
\end{abc}
\caption{Exemplo de compasso quaternário.}
\label{compasso:quaternario}
\end{figure}

São chamados de compassos quaternários compostos,  
quando estes tem uma fórmula de compasso como: $12/4$, $12/8$ e $12/16$.
Para gerar estes compassos compostos a partir de suas versões simples,
se segue a mesma operação descrita na Equação \ref{eq:comcomposto}.


%%% Tipos de acentos
% \cite[pp. 217]{medteoria}.
 
