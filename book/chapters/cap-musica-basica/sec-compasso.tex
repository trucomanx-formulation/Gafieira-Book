\section{\textcolor{green}{Compasso}}\index{Compasso}
\label{sec:compaso}

\begin{description}
\item[Compasso:] O dicionário de Harvard de música \cite[pp. 513]{apel1969harvard} define compasso (``measure'' em inglês)
como um grupo de tempos, batimentos ou pulsos (unidade do tempo musical),
onde o primeiro destes normalmente é acentuado. 
Este número de tempos no compasso pode ser, dois, trés, quatro, ou ocasionalmente 5 ou mais. 
Sendo estos compassos separados por barras verticais e as notas do compasso esquematizados baixo uma métrica.
\begin{example}
Figura \ref{fig:abc-exemplocompasso1}
\end{example}
 
\item[Métrica:] Sobre a métrica  (``meter'' em inglês) o dicionario \cite[pp. 523]{apel1969harvard} explica que é
um padrão de unidades temporais fixas, chamados batimentos, 
pelo qual um período de tempo de uma peça musical ou uma seção dela é medida. 
Agrega tambem que a métrica é indicado geralmente por uma fração, como por exemplo:
${2}/{2}$ , ${3}/{4}$ , ${4}/{4}$, etc. Em português esta fração é chamada de formula do compasso. 
\begin{example}
Figura \ref{fig:abc-exemplocompasso1}
\end{example}
\end{description}

O numerador, da formula do compasso, indica o número de pulsações (tempos) que compõem cada compasso.
Por outro lado o denominador nos informa al longitude temporal de cada um dos tempos do compasso.

\begin{figure}[h]
\centering
\begin{abc}[name=abc-exemplocompasso1]
% abcm2ps exemplocompasso1.abc  -O exemplocompasso1.ps
% ps2epsi exemplocompasso1.ps exemplocompasso1.eps
%
X: 1 % start of header
K: none stafflines=0 %K: C %% Escala de C mayor %
M: 2/4
%T: Contratempo num compasso binário
V:1 clef=perc stem=up name="Ritmo 1"   sname="Ritmo 1"
%
[V:1] | B2 B1 B1| B2 B1 B1 | B2 B1 B1 | B2 z2  |
%       
\end{abc}
\caption{Figuras e pausas}
\label{fig:abc-exemplocompasso1}
\end{figure}



A Tabela \ref{tab:abc-noteslength} exemplifica o significado do denominador da formula do compasso; 
\begin{table}[h]
\centering
\begin{tabular}{|c|c|c|c|}
\hline
denominador & Figura  & Duração & Nome\\ \hline
\hline
$1$   & \fullnote    & $S$   & Semibreve \\ \hline
$2$ & \halfnote    & $S/2$ & Mínima \\ \hline
$4$ & \quarternote & $S/4$ & Semínima \\ \hline
$8$ & \eighthnote  & $S/8$ & Colcheia \\ \hline
\end{tabular}
\caption{Duração e símbolos de algumas figuras musicais}
\label{tab:abc-noteslength}
\end{table}
onde a primeira coluna mostra o denominador da formula,
a segunda coluna mostra as figuras musicais que representam cada um dos tempos do compasso, e 
a terceira e quarta coluna, indicam a duração em segundos e o nome da figura musical.

podemos achar equivalências aos exemplos da formula do compasso dados
anteriormente; onde os compassos com formula $\mathbf{2}/2$ tem cada um, uma duração de $\mathbf{2}$\halfnote ~(duas mínimas),  
compassos com formula $\mathbf{3}/4$ tem uma duração de $\mathbf{3}$\quarternote ~(trés semínimas) 
e $\mathbf{4}/4$ uma duração de $\mathbf{4}$\quarternote ~(quatro semínimas). É importante
ressaltar que a duração em tempo das figuras musicais é relativa, como pode ser visto
na terceira coluna da Tabela \ref{tab:abc-noteslength}, onde as durações estão em função
da duração $S$ da semibreve. 


Se classificamos os compassos por sua métrica, os três tipos mais conhecidos 
são os compassos binários, ternários, quaternários \cite[pp. 27]{adolfo2002musica}.

\subsection{\textcolor{green}{Compasso binário}}\index{Compasso!Compasso Binário} Ou compasso binário simples,
é uma estrutura rítmica que se carateriza por ter compassos com uma  duração de dois tempos,
sendo o primeiro tempo forte (acentuado), e o segundo de tempo fraco (não acentuado)
\cite[pp. 41]{grabner2001teoria} \cite[pp. 66]{adolfo2002musica}\cite[pp. 28]{alves2004teoria}. 
Os compassos binários (simples) tem uma formula de compasso na forma $2/B$,
onde $B$ pode ser $2$, $4$, $8$, etc. 
A Figura \ref{compasso:binario}, representa um exemplo de compasso binário simples, 
com formula de compasso $2/2 \equiv 2$\halfnote, 
e tempos com uma duração de $S/2$ (uma \halfnote), 
sendo que o primeiro compasso contem $2$\halfnote~e o segundo contem $4$\quarternote.
Na Figura \ref{compasso:binario} a sigla ``N.A.'' significa ``Não acentuado'', pelo que é fácil perceber
que em qualquer caso, só a nota que é executada no tempo 1 é acentuada.
\begin{figure}[H]
\centering
\begin{abc}[name=abc-compasso1]
X: 1 % start of header
K: C % scale: C major
M: 2/2 %meter - compasso
"Primeiro compasso" G4 F4 |"Segundo compasso" G2 D2 F2 D2  |
w: Acentuado N.A. Acentuado N.A. N.A. N.A.
\end{abc}
\caption{Exemplo de compasso binário (simples)}
\label{compasso:binario}
\end{figure}

Se falamos de forma mais geral, 
podemos ter dois tipos de compassos binários: os simples e os compostos.
Assim, 
para achar a formula de um compasso composto, correspondente a um compasso simples (usando quialteras de três)
usamos a seguinte operação \cite[pp. 74]{alves2004teoria}, 
\begin{equation}\label{eq:comcomposto}
Compasso~simples\times\frac{3}{2}=Compasso~composto.
\end{equation}
De modo que obtemos compassos binários compostos com as seguintes formulas de compasso: 
$6/4$, $6/8$, $6/16$, etc.
A diferencia do visto nos compassos binários simples, os compassos binários compostos tem 
um pulso forte (Acentuado) no tempo 1 e um pulso semiforte (Acentuado porem menor) no tempo 4, 
de modo que os tempos 2,3,5 e 6,
são classificados como tempos fracos (Não Acentuados)\cite[pp. 41]{grabner2001teoria}.
A Figura \ref{compasso:binariocomposto}, representa um exemplo de compasso binário composto, 
com formula de compasso $6/4 \equiv 6$\quarternote, 
e tempos com uma duração de $S/4$ (uma \quarternote), 
sendo que o primeiro compasso contem $6$\quarternote~e o segundo contem dois $2$\quarternote~e dois $2$\halfnote.
Da Figura \ref{compasso:binariocomposto} é fácil perceber
que em qualquer caso, só são acentuados as nota que são executadas no tempo 1 e 4; 
aclarando que as notas executadas no tempo 4 tem uma acentuação menor que as executadas no tempo 1.
\begin{figure}[H]
\centering
\begin{abc}[name=abc-compasso1c]
X: 1 % start of header
K: C % scale: C major
M: 6/4 %meter - compasso
"Primeiro compasso" G2 D2 D2 F2 D2 D2 |"Segundo compasso" G2 D4 F2 D4  |
w: Acentuado N.A. N.A. Acentuado N.A N.A. Acentuado N.A. Acentuado N.A. 
\end{abc}
\caption{Exemplo de compasso binário composto}
\label{compasso:binariocomposto}
\end{figure}

Alguns autores consideram aos compassos quaternários (ex: 4/4, 4/8) como um caso de compasso binário,
chamando eles de compasso binário duplo \cite[pp. 41]{grabner2001teoria}.




\subsection{\textcolor{green}{Compasso ternário}}\index{Compasso!Compasso Ternário} Ou compasso ternário simples,
é uma estrutura rítmica que se carateriza por ter compassos com trés tempos,
sendo o primeiro pulso forte (acentuado) e os outros dois fracos (não acentuados) 
\cite[pp. 67]{adolfo2002musica}\cite[pp. 30]{alves2004teoria}. 
Os compassos ternários (simples) tem uma formula de compasso da forma $3/B$, 
onde $B$ pode ser $2$, $4$, $8$, etc.
Por exemplo temos, as formulas de compassos ternários simples: $3/2$, $3/4$, $3/8$,  etc.

A Figura \ref{compasso:ternario}, representa um exemplo de compasso ternário (simples), com 
formula de compasso $3/4 \equiv 3$\quarternote, 
onde os tempos tem uma duração de $S/4$, o primeiro compasso contem $3$\quarternote~e
o segundo contem $6$\eighthnote.
Na Figura \ref{compasso:ternario}  é fácil perceber
que em ambos compassos, só a nota que é executada no tempo 1 é acentuada.
\begin{figure}[H]
\centering
\begin{abc}[name=abc-compasso2]
X: 1 % start of header
K: C % scale: C major
M: 3/4 %meter - compasso
"Primeiro compasso" G2 F2 F2 |"Segundo compasso" G1 F1 E1 D1 D1  D1  |
w: Acentuado N.A. N.A. Acentuado N.A N.A.  N.A. N.A. N.A. 
\end{abc}
\caption{Exemplo de compasso ternário}
\label{compasso:ternario}
\end{figure}


São chamados de compassos ternários compostos,  
quando estes tem uma formula de compasso como: $9/4$, $9/8$ e $9/16$.
Para gerar estes compassos compostos a partir de suas versões simples,
se segue a mesma operação descrita na Equação \ref{eq:comcomposto}.


\subsection{\textcolor{green}{Compasso quaternário}}\index{Compasso!Compasso Quaternário} Ou compasso quaternário simples,
é uma estrutura rítmica que se carateriza por ter compassos com quatro tempos,
sendo o primeiro pulso forte (acentuado), o segundo fraco (não acentuado), 
o terceiro semiforte (acentuado porem menor) e o último fraco (não acentuado) 
\cite[pp. 67]{adolfo2002musica}\cite[pp. 32]{alves2004teoria}. 
Os compassos quaternários (simples) tem uma formula de compasso da forma $4/B$, 
onde $B$ pode ser $2$, $4$, $8$, etc.
Por exemplo temos, as formulas de compassos ternários simples: $4/2$, $4/4$, $4/8$,  etc.

A Figura \ref{compasso:quaternario}, representa um exemplo de compassos quaternário, com 
formula de compasso $4/4 \equiv 4$\quarternote, 
onde cada tempo tem uma duração de $S/4$, o primeiro compasso contem $4$\quarternote~e
o segundo contem $8$\eighthnote.
Na Figura \ref{compasso:ternario}  é fácil perceber
que em ambos compassos, só as notas que são executadas no tempo 1 e 3 são acentuadas.
\begin{figure}[H]
\centering
\begin{abc}[name=abc-compasso3]
X: 1 % start of header
K: C % scale: C major
M: 4/4 %meter - compasso
"Primeiro compasso" G2 D2 F2 D2|"Segundo compasso" G1 F1 D1 C1 F1 E1 D1 C1 |
w: Acentuado N.A. Acentuado N.A. Acentuado N.A N.A. N.A. Acentuado N.A. N.A. N.A. 
\end{abc}
\caption{Exemplo de compasso quaternário}
\label{compasso:quaternario}
\end{figure}

São chamados de compassos quaternários compostos,  
quando estes tem uma formula de compasso como: $12/4$, $12/8$ e $12/16$.
Para gerar estes compassos compostos a partir de suas versões simples,
se segue a mesma operação descrita na Equação \ref{eq:comcomposto}.
 
