\section{\textcolor{blue}{Tempo}}\index{Tempo}

Como já foi sugerido na Seção \ref{sec:compaso}, é chamado de "tempo" 
à pulsação básica e unidade de medida dos compassos nas composições musicais;
assim, temos que compassos binários, ternários e quaternários; que tem uma duração de 2 tempos, 
3 tempos e 4 tempos, respetivamente. Por comodidade designaremos com a variável $T$ à duração em segundos de cada tempo,
sendo que o valor de $T$ variará dependendo da formula de compasso usada.

\subsection{\textcolor{red}{Tempo forte}}\index{Tempo!Tempo Forte}

Sobre acentos  \cite[pp. 9]{phillips2002sight}

\subsection{\textcolor{red}{Tempo fraco}}\index{Tempo!Tempo Fraco}

Sobre acentos  \cite[pp. 9]{phillips2002sight}



\subsection{\textcolor{green}{O tempo em diferentes formulas de compasso}}
É importante ressaltar que os compassos que usem a mesma formula de compasso terão sempre a mesma duração em segundos;
por outro lado, podem ser achados casos onde compassos com diferente formula podem ter a mesma duração;
por exemplo: compassos binários com formula $\mathbf{2}/2$, comparados com compassos quaternários com 
formula $\mathbf{4}/4$; ambos compassos tem a mesma duração uma semibreve (\fullnote).

Na Figura \ref{fig:abc-tempo1} podemos ver compassos binários com formula $\mathbf{2}/2$, 
\begin{figure}[H]
\centering
\begin{abc}[name=abc-tempo1]
X: 1 % start of header
K: C % scale: C major
M: 2/2 %meter - compasso
G2 D2 F2 D2 | G4 F4 |
w: T/2 T/2 T/2 T/2  T T
\end{abc}
\caption{Exemplo de dois compassos com 2 tempos de duração T}
\label{fig:abc-tempo1}
\end{figure}
onde estos terão uma duração de dois tempos ($2T$) \cite[pp. 25]{azevedocompor} onde cada tempo ($T$) tem uma duração 
de uma mínima (\halfnote), ver Tabela \ref{tab:abc-noteslength};
por outro lado,
compassos quaternários com formula $\mathbf{4}/4$, como na Figura \ref{fig:abc-tempo2}, 
\begin{figure}[H]
\centering
\begin{abc}[name=abc-tempo2]
X: 1 % start of header
K: C % scale: C major
M: 4/4 %meter - compasso
G2 D2 F2 D2| G4 F4|
w: T T T T 2T 2T
\end{abc}
\caption{Exemplo de dois compassos com 4 tempos de duração T}
\label{fig:abc-tempo2}
\end{figure} 
terão uma duração de 4 tempos ($4T$) \cite[pp. 25]{azevedocompor} onde 
cada tempo ($T$) tem uma duração de uma semínima (\quarternote), ver Tabela \ref{tab:abc-noteslength}.
Assim, estas duas formulas ($2/2$ e $4/4$) representam compassos 
com a mesma duração em segundos, uma semibreve (\fullnote),
porem tem valores diferentes para a variável $T$ em segundos.

\begin{lattention}
Se interpretamos a música mostrada nas Figuras \ref{fig:abc-tempo1} e \ref{fig:abc-tempo2},
podemos perceber que ambas descrevem um sonido ligeiramente diferente; assim, não é fácil
distinguir se o sonido provem de um compasso com formula $2/2$ ou $4/4$.
Em estes casos a dica é perceber se existem dois tipos de tempos fortes, acentuados
com diferentes intensidades; se é assim, então estamos ante um compasso quaternário;
se não, então é um compasso binário.
\end{lattention}


\subsection{\textcolor{red}{Contagem dos tempos no compasso}}

