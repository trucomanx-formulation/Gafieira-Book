\section{Tempo, acentuação e contagem}
\index{Música!Tempo}
\label{sec:Tempo}

Como já foi sugerido na Seção \ref{sec:compaso}, é chamado de ``tempo''
à pulsação básica que é usada como unidade de medida das composições musicais.
Os tempos ao ser agrupados em compassos podem formar diferentes estruturas como por exemplo: 
\hyperref[subsec:compassobinario]{\textbf{compassos binários}}, 
\hyperref[subsec:compassoternario]{\textbf{ternários}} e 
\hyperref[subsec:compassoquaternario]{\textbf{quaternários}}; que tem uma duração de 2 tempos, 
3 tempos e 4 tempos, respetivamente. 



\subsection{Acentuação dos tempos (parte 2)}
\label{subsec:acentuacion2}
\index{Música!Acentuação}
\index{Música!Tempo!Tempo forte}
\index{Música!Tempo!Tempo fraco}


Mais conceitos sobre a acentuação dos tempos podem ser vistos na Seção \ref{subsec:acentuacion1}.

\begin{notation}[] Para o melhor entendimento das seguintes seções usaremos algumas notações.
\begin{itemize}
\item A variável $T$ será usada para designar à duração em segundos de cada tempo,
sendo que o valor de $T$ variará dependendo da formula do compasso usada.

\item As subdivisões dos tempos serão designadas com as seguintes variáveis:
\begin{itemize}
\item ``FF'' indica que é a parte forte de um tempo forte,
\item ``Ff'' indica que é a parte fraca de um tempo forte,
\item ``fF'' indica que é a parte forte de um tempo fraco,
\item ``ff'' indica que é a parte fraca de um tempo fraco.
\end{itemize}
\end{itemize}

\end{notation}

A formula do compasso de uma peça musical ou uma porção dela, 
nos indica quantos tempos e que duração terão estes tempos no compasso; 
porem, além destas informações, 
a formula do compasso também nos indica o \hyperref[def:acentometrico]{\textbf{acento métrico}} \cite[pp. 70]{cardoso1973curso}; 
é dizer quias serão os tempos fortes, semifortes e fracos no compasso.

\begin{tcbinformation} 
\textbf{Acento métrico:}
\index{Música!Acento métrico}
\label{def:acentometrico} 
é constituído pelas acentuações fortes e fracas dos tempos dos compassos, 
a acentuação métrica não precisa ser grafado na pauta;
é dizer, estão sobreintendidos \cite[pp. 141,217]{medteoria}.
\end{tcbinformation} 

\begin{tcbinformation} 
\textbf{Acento dinâmico:}
\index{Música!Acento dinâmico}
\label{def:acentodinamico}
 é o acento indicado e grafado pelo compositor na pauta, como meio de expressão;
consequentemente, pelo geral não coincidem com o acento métrico \cite[pp. 217]{medteoria}.
\end{tcbinformation} 


%\item[O acento rítmico] 
%\index{Música!Acento rítmico}
%\label{def:acentoritmico}
%\cite[pp. 217]{medteoria}.

\begin{description}

\item[O tempo forte (F)] 
\index{Música!Tempo forte}
\label{def:tempoforte} 
sempre será o primeiro tempo de cada compasso. 
Um tempo forte não necessita uma grafia especial que indique que este deve ser articulado
com maior \hyperref[sec:pos:Intensidade]{\textbf{intensidade}}; é dizer, com acentuação.

\item[O tempo semiforte (sF)] 
\index{Música!Tempo semiforte}
\label{def:temposemiforte} 
acontece em alguns tipos de compasso, 
como nos compassos quaternários. 
Um tempo semiforte terá uma acentuação menor à do tempo forte, porem maior à de um tempo fraco. 
Este tempo não necessita uma grafia especial que indique que deve ser articulado
com \hyperref[sec:pos:Intensidade]{\textbf{intensidade}}.

\item[O tempo fraco (f)] 
\index{Música!Tempo fraco}
\label{def:tempofraco} 
corresponde a todos os tempos que não sejam né fortes, né semifortes,
de modo que estes tempos não tem acentuação. 
\end{description}~

Uma versão mais simplificada das informações dadas anteriormente, 
pode ser vista na seguinte equação que mostra a relação de \hyperref[sec:pos:Intensidade]{\textbf{intensidades}}:
\begin{equation}
f ~<~ sF ~<~ F.
\end{equation}

Porem, 
se utilizamos as subdivisões de tempos, 
a relação de \hyperref[sec:pos:Intensidade]{\textbf{intensidades}} pode ser expressada mediante a seguinte equação:
\begin{equation}\label{eq:acentosubdividio}
\{fF=ff\} ~<~  \{f = fF\} ~<~ sF ~<~ \{F = FF\} 
\end{equation}

\begin{tcbattention}
A regularidade na distribuição das acentuações nos tempos, 
só mudará quando se especifiquem  explicitamente dinâmicas sobre as figuras musicais,
de modo que estas dinâmicas modificam as acentuações,
porem não trocam o nome dos tempos, sendo estes sempre chamados de tempos fortes, semifortes e fracos. 
\end{tcbattention}


\begin{example}
Na Figura \ref{fig:abc-tempo1} podemos ver 2 compassos com a mesma métrica, 
tendo eles uma formula de compasso 2/2; é dizer, 
cada compasso tem dois tempos, como uma duração de uma mínima (\halfnote) para cada tempo.
\begin{itemize}
\item O primeiro compasso é preenchido com duas notas que usam figuras musicais que duram um tempo cada um;
de modo que 
a primeira nota será executada no tempo forte, 
possuindo esta nota um \hyperref[def:acentoprincipal]{\textbf{acento  principal}}, e 
a segunda nota é executada no tempo fraco e não leva acento; 
é dizer tem um menor \hyperref[sec:pos:Intensidade]{\textbf{intensidade}} que a nota do primeiro tempo.
\item No segundo compasso, este é preenchido com 4 notas que usam figuras musicais que duram 1/2 tempo cada uma;
Assim, 
\begin{itemize}
\item a primeira nota será executada em FF (acentuada como F),
\item a segunda  nota será executada em Ff (menos acentuada que f),
\item a terceira nota será executada em fF (acentuada como f), e 
\item a quarta   nota será executada em ff (menos acentuada que f).
\end{itemize}
\end{itemize} 
\end{example}
\begin{figure}[H]
\centering
\begin{abc}[name=abc-tempo1,width=0.75\linewidth]
X: 1 % start of header
K: none stafflines=0 %K: C %% Escala de C mayor %
M: 2/2 %meter - compasso
V:1 clef=perc stem=up %name="Ritmo"   sname="Ritmo"
[V:1] | B4 B4 |  B2 B2 B2 B2 |  
w:  T T    T/2 T/2 T/2 T/2 
w:  F f FF Ff fF ff
\end{abc}
\caption{Dois compassos com 2 tempos cada um.}
\label{fig:abc-tempo1}
\end{figure}


\begin{example}
Na Figura \ref{fig:abc-tempo2} podemos ver 2 compassos com a mesma métrica, 
tendo eles uma formula de compasso 4/4; é dizer, 
cada compasso tem quatro tempos, como uma duração de uma semínima (\quarternote) para cada tempo.
\begin{itemize}
\item O primeiro compasso é preenchido com duas notas que usam figuras musicais que duram dois tempos cada um;
de modo que a primeira nota será executada no tempo forte,
possuindo esta nota um \hyperref[def:acentoprincipal]{\textbf{acento  principal}}, 
e o som será sustenido ate completar o seguinte tempo fraco, 
a segunda nota será executada no tempo semiforte,
possuindo esta nota um \hyperref[def:acentosecundario]{\textbf{acento  secundario}},
 e o som será sustenido ate completar o seguinte tempo fraco.
Em ambas figuras não são colocados os símbolos, F e sF, 
pois as notas ocupam mais de um tempo, porem as acentuações são respeitadas.
\item O segundo compasso é preenchido com 4 notas que usam figuras musicais que duram 1 tempo cada uma;
Assim, 
a primeira nota será executada no tempo forte (F),
a segunda  nota será executada no tempo fraco (f),
a terceira nota será executada no tempo semiforte (sF), e 
a quarta   nota será executada no tempo fraco (f).
\end{itemize} 
\end{example}
\begin{figure}[H]
\centering
\begin{abc}[name=abc-tempo2,width=0.75\linewidth]
X: 1 % start of header
K: none stafflines=0 %K: C %% Escala de C mayor %
M: 4/4 %meter - compasso
V:1 clef=perc stem=up %name="Ritmo"   sname="Ritmo"
[V:1] | B4  B4 | B2 B2 B2 B2 | 
w:  2T 2T      T T T T 
w:  Acento Acento      F f sF f 
\end{abc}
\caption{Dois compassos com 4 tempos cada um.}
\label{fig:abc-tempo2}
\end{figure} 

\begin{tcbattention}
As pautas mostradas na Figura \ref{fig:abc-tempo1} e na Figura \ref{fig:abc-tempo2},
tem a mesma distribuição de figuras musicais, porem usam formulas do compasso distintas;
Por este pequeno detalhe o som de ambos ritmos será diferente,
pois as acentuações serão diferentes.
Por exemplo, 
na segunda nota do primeiro compasso, na Figura \ref{fig:abc-tempo1}, 
a nota é acentuada como uma f. Porem, 
na  Figura \ref{fig:abc-tempo1}, a nota é acentuada como uma sF, tendo esta ultima maior intensidade.
Existe um caso similar na terça nota do segundo compasso de ambas figuras.

\end{tcbattention}

\subsection{Contagem dos tempos nos compassos}
A contagens dos tempos no compasso segue a \hyperref[def:Metrica]{\textbf{métrica}}
do compasso, de modo que estes tempos são contados independentemente da quantidade de notas no compasso.
De modo que num compasso binário contaremos 1-2, num ternário contaremos 1-2-3, e
num quaternário 1-2-3-4.
\begin{example}
A Figura \ref{fig:contartempocomp1} representa a contagem de um ritmo,
escrito num compasso binário.
\end{example}
\begin{figure}[h]
    \centering
 \begin{abc}[name=abc-contartempocomp1,width=\linewidth]
% abcm2ps contartempocomp1.abc  -O contartempocomp1.ps
% ps2epsi contartempocomp1.ps contartempocomp1.eps
%
X: 1 % start of header
K: none stafflines=0 %K: C %% Escala de C mayor %
M:  2/4
%T: Contratempo num compasso binário
V:1 clef=perc stem=up %name="Ritmo"   sname="Ritmo"
%
[V:1] | B2 B1 B1  |B1 B1 B2  | B1/2 B1/2 B1 B1 B1 | B2 B2|
w:      1  2  _    1  _  2     1  _ _    2  _       1  2
%       
\end{abc}
    \caption{Sequencia rítmica com formula de compasso 2/4.}\label{fig:contartempocomp1}
\end{figure}


\subsection{Contagem dos compassos}
A contagem dos compassos descreve a quantidade de compassos que tem uma musica ou uma porção dela,
esta contagem se incrementa ao inicio de cada compasso; é dizer no tempo 1 (tempo forte).
\begin{example}
A Figura \ref{fig:contarcompassos2} representa a contagem de compassos num ritmo,
com compassos binários. 
\end{example}
\begin{figure}[h]
    \centering
 \begin{abc}[name=abc-contarcompassos2,width=\linewidth]
% abcm2ps contarcompassos2.abc  -O contarcompassos2.ps
% ps2epsi contarcompassos2.ps contarcompassos2.eps
%
X: 1 % start of header
K: none stafflines=0 %K: C %% Escala de C mayor %
M:  2/4
%T: Contratempo num compasso binário
V:1 clef=perc stem=up %name="Ritmo"   sname="Ritmo"
%
[V:1] | B2 B1 B1  |B1 B1 B2  | B1/2 B1/2 B1 B1 B1 | B2 B2|
w:      1  _  _    2  _  _     3  _ _    _  _       4  _
%       
\end{abc}
    \caption{Sequencia rítmica com formula de compasso 2/4.}\label{fig:contarcompassos2}
\end{figure}

\subsection{Contagem dos tempos das figuras musicais}
A contagem dos tempos das figuras musicais dentro do compasso, 
acompanha a posição das figuras musicais ou notas dentro do compasso. 
É importante ter em conta a duração das figuras musicais pois a contagem só
continua no inicio da seguinte figura musical;
assim, dependendo da duração da figura musical, 
alguns dos tempos do compasso serão contados e outras não \cite[pp. 8]{phillips2002sight}

\begin{example}
A Figura \ref{fig:abc-contagemtempo44} mostra um ritmo que usa 4 compassos quaternários,
com tempos com uma duração de um \quarternote;
assim a contagem dos tempos dentro do compasso vão de 1 ate 4.
Pela irregularidade na duração das figuras musicais alguns tempos não são contados explicitamente.
A contagem pode ser visto na parte inferior de cada figura musical.
\end{example}
\begin{figure}[H]
\centering
\begin{abc}[name=abc-contagemtempo1,width=\linewidth]
X: 1 % start of header
K: none stafflines=0 %K: C %% Escala de C mayor %
M: 4/4 %meter - compasso
V:1 clef=perc stem=up %name="Ritmo"   sname="Ritmo"
[V:1] | "2T"B4  "2T"B4 | "4T"B8 |  "T"B2 "2T"B4 "T"B2 |  "T/2"B2 "T/2"z2 "T/2"B2  "T/2"z2| 
w:       1 3             1          1 2  4               1 3
%w:      2T 2T            4T         T 2T T               T/2 T/2
\end{abc}
\caption{Sequencia rítmica usando um compasso quaternário.}
\label{fig:abc-contagemtempo44}
\end{figure} 



\begin{example}
A Figura \ref{fig:contartempos24}  mostra um ritmo que usa 6 compassos binários,
com tempos com uma duração de um \quarternote;
assim a contagem dos tempos dentro do compasso só pode ser 1 ou 2.
Na parte inferior de cada figura musical pode verse a contagem que deve ser feita no ritmo.
\end{example}
\begin{figure}[H]
    \centering
 \begin{abc}[name=abc-contartempos24,width=\linewidth]
% abcm2ps contartempos24.abc  -O contartempos24.ps
% ps2epsi contartempos24.ps contartempos24.eps
%
X: 1 % start of header
K: none stafflines=0 %K: C %% Escala de C mayor %
M:  2/4
%T: Contratempo num compasso binário
V:1 clef=perc stem=up %name="Ritmo"   sname="Ritmo"
%
[V:1] | "T"B2 "T"z2  |"T"B2 "T"B2  | "T"z2 "T"B2  |"2T"B4  |"T"B2 "T"B2  |"T"B2 "T"B2  |
w:       1             1     2           2       1        1     2       1     2
%w:         T          T     T           T       2T       T T           T T
%       
\end{abc}
\caption{Sequencia rítmica usando um compasso binário.}
\label{fig:contartempos24}
\end{figure}

\begin{comment} 
\begin{example}
A Figura \ref{fig:contartempos34}   mostra um ritmo que usa 4 compassos ternários,
com tempos com uma duração de um \quarternote;
assim a contagem dos tempos dentro do compasso só pode ser 1, 2 ou 3.
Na parte superior de cada figura musical pode verse a contagem que deve ser feita no ritmo.
\end{example}
\begin{figure}[h]
    \centering
 \begin{abc}[name=abc-contartempos34]
% abcm2ps contartempos34.abc  -O contartempos34.ps
% ps2epsi contartempos34.ps contartempos34.eps
%
X: 1 % start of header
K: none stafflines=0 %K: C %% Escala de C mayor %
M:  3/4
%T: Contratempo num compasso binário
V:1 clef=perc stem=up name="Ritmo"   sname="Ritmo"
%
[V:1] | "1"B4 "3"B2 |"1"B2 "2"B2 "3"B2  | z2 "2"B2 "3"B2  |"1"B2 "2"B4  |
%       
\end{abc}
\caption{Sequencia rítmica usando um compasso ternário.}
\label{fig:contartempos34}
\end{figure}
\end{comment} 
