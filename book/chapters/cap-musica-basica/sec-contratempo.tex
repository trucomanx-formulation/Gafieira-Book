\section{Contratempo}
\label{sec:contratempo}
\index{Música!Contratempo}
Um contratempo acontece quando \cite[pp. 16]{mascarenhascurso} 
\cite[pp. 36]{azevedocompor}: 
\begin{itemize}
\item As notas são executadas em tempos fracos do compasso ou nas partes fracas dos tempos, e 
\item estas notas estão intercaladas por silencios no seu correspondente tempo forte ou parte forte do tempo.
\end{itemize}

Neste sentido, 
o contratempo pode ser visto como a omissão de notas nos tempos fortes ou nas partes fortes dos tempos \cite[pp. 146]{medteoria}.

\begin{example}
A Figura \ref{fig:abc-contratempoa} mostra quatro compassos binários com formula $2/4$, e
com tempos de uma duração de uma semínima (\quarternote). 
\begin{itemize}
\item No primeiro compasso existe um contratempo na nota a executada no tempo fraco.
\item No segundo  compasso existem contratempos nas notas executadas em Ff e ff.
\item No terceiro compasso existe um contratempo na nota a executada em Ff.
\item No quarto   compasso existe um contratempo na nota a executada em ff.
\end{itemize}
\end{example}
\begin{figure}[H]
\centering
\begin{abc}[name=abc-contratempoa]
X: 1 % start of header
K: none stafflines=0 %K: C %% Escala de C mayor %
M:2/4
%T: Contratempo num compasso binário
V:1 clef=perc stem=up %name="A" sname="A"
[V:1] |"F"z2 "f"B2 | "FF"z1 "Ff"B1  "fF"z1 "ff"B1 | "FF"z1 "Ff"B1   "f"z2 |  "F"z2 "fF"z1 "ff"B1  |
w:          T          T/2            T/2             T/2                      T/2
\end{abc}
\caption{Contratempos no tempos fracos ou nas partes fracas dos tempos}
\label{fig:abc-contratempoa}
\end{figure}



Num sentido mais amplo, o contratempo é a acentuação de um tempo fraco em vez de um tempo forte \cite[pp. 147]{medteoria}. 
Assim, a palavra ``contratempo'', referencia a como estão configuradas ou acentuadas as notas no compasso.


\begin{example} 
A Figura \ref{fig:abc-contratempob} mostra um caso similar ao da Figura \ref{fig:abc-contratempoa};
porem, os contratempos são  expressados como a acentuação (grafada na pauta) de um tempo fraco, 
em vez de um silencio no tempo forte \cite[pp. 147]{medteoria}. 
É usado o símbolo $>$ para indicar esta acentuação na partitura.
\end{example}
\begin{figure}[H]
\centering
\begin{abc}[name=abc-contratempob]
X: 1 % start of header
K: none stafflines=0 %K: C %% Escala de C mayor %
M:2/4
%T: Contratempo num compasso binário
V:1 clef=perc stem=up %name="A" sname="A"
[V:1] |"F"B2 "f"+accent+B2 | "FF"B1 "Ff"+accent+B1  "fF"B1 "ff"+accent+B1 | "FF"B1 "Ff"+accent+B1  "f"B2  | "F"B2 "fF"B1  "ff"+accent+B1  | 
w:    T     T                T/2    T/2             T/2    T/2              T/2    T/2             T       T      T/2             T/2  
\end{abc}
\caption{Contratempos pela acentuação dos tempos fracos ou nas partes fracas dos tempos}
\label{fig:abc-contratempob}
\end{figure}

