\section{\textcolor{green}{Contratempo}}\index{Contratempo}
Um contratempo acontece quando as notas (representadas por figuras musicais na partitura) 
são executadas em tempos fracos do compasso
ou nas partes fracas dos tempos, sendo que estas estão intercaladas por pausas nos tempos
fortes ou partes fortes dos tempos \cite[pp. 16]{mascarenhascurso} 
\cite[pp. 36]{azevedocompor}, neste sentido o contratempo pode ser visto como a 
omissão de notas nos tempos fortes ou nas partes fortes dos tempos \cite[pp. 146]{medteoria}.
Ou ``num sentido mais amplo, o contratempo é a acentuação de um tempo fraco em vez de um tempo forte'' \cite[pp. 147]{medteoria}. 

Assim, a palavra ``contratempo'', referencia a como estão configuradas ou acentuadas 
as notas no compasso. Por exemplo:
A Figura \ref{fig:contratempoa} mostra 
quatro compassos (binários) com formula $2/4$, em cada compasso existem 
contratempos nos tempos fracos ou nas partes fracas dos tempos, sendo que cada tempo
tem uma duração de uma semínima (\quarternote) e cada compasso uma duração 
de uma mínima (\halfnote), ou seja duas semínimas (2\quarternote). 
\begin{itemize}
\item ``F''  indica que é o tempo é forte, 
\item ``f''  indica que é o tempo é fraco,
\item ``FF'' indica que é a parte forte de um tempo forte,
\item ``Ff'' indica que é a parte fraca de um tempo forte,
\item ``fF'' indica que é a parte forte de um tempo fraco,
\item ``ff'' indica que é a parte fraca de um tempo fraco, 
\end{itemize} 

finalmente
a figura musical \ViPa~ indica um silencio da mesma duração que uma semínima (\quarternote)
e a figura musical \AcPa~ indica um silencio da mesma duração que uma colcheia (\eighthnote).
\begin{figure}[H]
\centering
\begin{abc}[name=abc-contratempoa]
X: 1 % start of header
K: C % scale: C major
M:2/4
%T: Contratempo num compasso binário
V:1 clef=treble name="A" sname="A"
[V:1] "F"z2 "f"G2 | "FF"z1 "Ff"G1  "fF"z1 "ff"G1 | "FF"z1 "Ff"G1  "f"G2 |  "F"z2 "fF"z1 "ff"G1  |
w:          T          T/2            T/2             T/2     Tempo                 T/2
\end{abc}
\caption{Contratempos no tempos fracos ou nas partes fracas dos tempos}
\label{fig:abc-contratempoa}
\end{figure}
Na Figura \ref{fig:abc-contratempoa}, existem contratempos em todos os compassos porem estes estão
configurados de distintas formas;
no primeiro compasso acontece um contratempo dado que a única nota é executada 
no tempo fraco do compasso, no segundo compasso acontecem contratempos pois as 
notas são executadas nas partes fracas de cada tempo,
no terceiro compasso acontece um contratempo pela execução de uma nota na parte 
fraca do tempo forte, sendo o resto do tempo preenchido com um silencio, e 
finalmente no quarto compasso acontece um contratempo pela execução de uma nota
na parte fraca do tempo fraco, sendo o resto do compasso preenchido com silêncios.


Por outro lado, 
a Figura \ref{fig:abc-contratempob} mostra um caso similar ao da Figura \ref{fig:abc-contratempoa},
com contratempos expressados como a acentuação de um tempo fraco em vez de um silencio no tempo forte \cite[pp. 147]{medteoria}. 
É usado o símbolo $>$ para indicar esta acentuação na partitura.
\begin{figure}[H]
\centering
\begin{abc}[name=abc-contratempob]
X: 1 % start of header
K: C % scale: C major
M:2/4
%T: Contratempo num compasso binário
V:1 clef=treble name="A" sname="A"
[V:1] "F"G2 "f"+accent+G2 | "FF"G1 "Ff"+accent+G1  "fF"G1 "ff"+accent+G1 | "FF"G1 "Ff"+accent+G1  "f"G2  | "F"G2 "fF"G1  "ff"+accent+G1  | 
w:    T     T                T/2    T/2             T/2    T/2              T/2    T/2             T       T      T/2             T/2  
\end{abc}
\caption{Contratempos pela acentuação dos tempos fracos ou nas partes fracas dos tempos}
\label{fig:abc-contratempob}
\end{figure}
