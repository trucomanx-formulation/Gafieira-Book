\section{Andamento}
\index{Música!Andamento}
\label{sec:Andamento}

Como temos visto nos capítulos anteriores, 
as figuras musicais podem ser colocadas em alturas fixas (ex: Um lá a 440Hz);
porém ate agora as durações das figuras musicais numa melodia só foram descritas de forma relativa,
como foi visto na Tabela \ref{tab:abc-noteslengthbasic}.

Assim, é preciso indicar qualitativa ou quantitativamente aos interpretes,
qual será a duração de cada figura musical da melodia a executar;
pelo que, ao inicio da pauta, é comum ver indicado de forma qualitativa o ``andamento''  
com palavras como: ``rápido'', ``moderado'' ou ``lento'' \cite[pp. 29]{holst1998abc} \cite[pp. 115]{mascarenhascurso};
assim, o andamento se define como o grau de lentidão ou rapidez ao executar uma peça musical \cite[pp. 115]{mascarenhascurso}.

Se o compositor quer ser mais exato, 
também pode indicar o andamento de uma peça musical de forma quantitativa,
descrevendo quantas figuras musicais podem ser executadas por minuto;
por exemplo a indicação, \Vier=80, 
significa que levaria um minuto para que 80 semínimas fossem executadas \cite[pp. 29]{holst1998abc};
um exemplo disto pode ser visto na Figura \ref{fig:andamento1}. 

\begin{figure}[!h]
\centering
\begin{abc}[name=abc-andamento1]
X: 1 % start of header
K: C stafflines=1 % scale: C major
M: 2/4 %meter - compasso
Q:1/4=80
V:1 clef=perc stem=up %name="Pauta com clave de fá"   sname="Pauta com clave de fá"
[V:1] | !>!B3/2 B/2 B1 B1| B3/2 B/2 B1 B1 | B2 B2| B2 z2 |
\end{abc}
\caption{Frase rítmica com um andamento de \Vier=80.}
\vspace{-20pt}
\label{fig:andamento1}
\end{figure}
 

