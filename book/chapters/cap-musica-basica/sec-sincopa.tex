\section{Síncope}
\label{sec:sincope}
\index{Música!Síncope}
Uma síncope é um som articulado sobre um tempo fraco, 
ou na parte fraca do tempo, que se prolonga até ocupar a parte forte do tempo seguinte \cite[pp. 143]{medteoria}.
\cite[pp. 44]{alves2004teoria}
\cite[pp. 15]{mascarenhascurso}
\begin{example}
A Figura \ref{fig:abc-sincopea} representa um ritmo com 3 síncopes.
\begin{itemize}
\item A primeira síncope acontece pela prolongação da nota articulada no tempo fraco,
do primeiro compasso, que se prolonga até o tempo forte do segundo compasso.
\item A segunda síncope acontece pela prolongação da nota articulada na parte fraca do tempo fraco,
do segundo compasso, que se prolonga até a parte forte do tempo forte do terceiro compasso.
\item A terceira síncope acontece pela prolongação da nota articulada no tempo fraco,
do terceiro compasso, que se prolonga até a parte forte do tempo forte do quarto compasso.
\end{itemize}
\end{example}
\begin{figure}[H]
\centering
\begin{abc}[name=abc-sincopea]
X: 1 % start of header
K: none stafflines=0 %K: C %% Escala de C mayor %
M:2/4
%T: síncope num compasso binário
V:1 clef=perc stem=up %name="A" sname="A"
[V:1] |B2 (B2 | B2) B1 (B1  |B1) B1 (B2 | B1)  B1 B2 |
w:     F  f     F   fF ff    FF   Ff f    FF   Ff f 
\end{abc}
\caption{Síncopes articuladas nos tempos fracos ou nas partes fracas dos tempos.}
\label{fig:abc-sincopea}
\end{figure}

\begin{tcbattention}
Uma consequência da síncope é a suspensão do acento normal de um tempo forte do compasso,
pela prolongação do tempo fraco anterior, provocando assim o deslocamento do acento métrico.
\cite[pp. 143]{medteoria}
\end{tcbattention}


\begin{example}
A Figura \ref{fig:abc-sincopeb} representa um ritmo com 4 síncopes.
\begin{itemize}
\item A primeira síncope acontece no primeiro compasso, quando é articulada  uma nota em Ff e esta se prolonga até fF.
\item A segunda  síncope acontece no segundo  compasso, esta é similar à primeira sincope; 
porém, a diferença da anterior, a prolongação não se especifica por uma ligadura, 
e sim por uma figura musical de 1 tempo de duração. 
\item A terceira síncope acontece no terceiro compasso, quando é articulada  uma nota em Ff e esta se prolonga até f.
\item A quarta  síncope acontece no quarto    compasso, esta é similar à terceira sincope; 
porém, a diferença da anterior, a prolongação não se especifica por uma ligadura, 
e sim por uma figura musical de 1 tempo e médio de duração. 
\end{itemize}
\end{example}
\begin{figure}[H]
\centering
\begin{abc}[name=abc-sincopeb]
X: 1 % start of header
K: none stafflines=0 %K: C %% Escala de C mayor %
M:2/4
%T: síncope num compasso binário
V:1 clef=perc stem=up %name="A" sname="A"
[V:1] | B1 (B1 B1) B1  | B1 B2 B1 | B1 (B1 B2) | B1 B3 | 
w:      FF  Ff fF  ff    FF ~  ff       FF  Ff f     FF  ~     
\end{abc}
\caption{Síncopes articuladas nas partes fracas dos tempos fracos que se prolongam até a parte forte do tempo seguinte.}
\label{fig:abc-sincopeb}
\end{figure}

\begin{description}
\item[Nota sincopada:] É aquela nota que ocupa o lugar onde deveria cair o acento normal \cite[pp. 144]{medteoria}.
\begin{example}
Na Figura \ref{fig:abc-sincopeb} temos notas sincopadas, 
no primeiro compasso em fF, e no terceiro compasso em f.
\end{example}

\item[Ritmo sincopado:] São ritmos que contem deslocamentos dos acentos métricos normais do compasso \cite[pp. 144]{medteoria}.
\begin{example}
A Figura \ref{fig:abc-sincopea} e a Figura \ref{fig:abc-sincopeb} representam ritmos sincopados.
\end{example}
\end{description}
