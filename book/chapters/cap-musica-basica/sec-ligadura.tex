
\section{Ligadura}
\index{Música!Ligadura}
\label{sec:ligadura}

A ligadura é uma linha curva que se coloca sobre duas ou mais notas da mesma altura, 
indicando que somente a primeira é articulada, 
e esta tem uma duração equivalente a soma de todas as notas ligadas \cite[pp. 35]{cardoso1973curso}.

\begin{example}
A Figura \ref{fig:total-ligadura} descreve 3 casos de uso de ligadura.
\begin{itemize}
\item Na Figura \ref{fig:abc-ligadura1} pode-se ver que, no primeiro compasso, 
tem-se uma ligadura entre a primeira e a segunda figura musical; 
no segundo compasso tem-se uma representação equivalente ao primeiro compasso; 
porém, usando ponto de aumento.
\item Na Figura \ref{fig:abc-ligadura2} pode-se ver que, no primeiro compasso, 
tem-se ligaduras entre as três primeiras figuras musicais; 
no segundo compasso tem-se uma representação equivalente ao primeiro compasso; 
porém, usando dois pontos de aumento.
\item Na Figura \ref{fig:abc-ligadura3} tem-se um exemplo de uso de ligadura entre figuras musicais de diferentes compassos,
de modo que a duração da última nota, do primeiro compasso, se prolonga até o segundo compasso.
\end{itemize}
\end{example}


\begin{figure}[!ht]
    \centering
    \begin{subfigure}[b]{0.6\textwidth}
\begin{abc}[name=abc-ligadura1]
X: 1 % start of header
K: C %% Escala de C mayor %
M: 2/4
%T: Contratempo num compasso binário
V:1 clef=G  %name="Ritmo 1"   sname="Ritmo 1"
%
[V:1] | (B2 B1) C1 | B3 C1   |    
\end{abc}
\vspace{-10pt}
\caption{Uso de ligaduras entre duas notas}
\label{fig:abc-ligadura1}
    \end{subfigure}
    ~%add desired spacing between images, e. g. ~, \quad, \qquad, \hfill etc. 
      %(or a blank line to force the subfigure onto a new line)
    \begin{subfigure}[b]{0.6\textwidth}
\begin{abc}[name=abc-ligadura2]
X: 1 % start of header
K: C %% Escala de C mayor %
M: 2/4
%T: Contratempo num compasso binário
V:1 clef=G  %name="Ritmo 1"   sname="Ritmo 1"
%
[V:1] | (A2 (A1) A1/2) C1/2 | A7/2 C1/2  |
\end{abc}
\vspace{-10pt}
\caption{Uso de ligaduras entre três notas.}
\label{fig:abc-ligadura2}
    \end{subfigure}
    ~%add desired spacing between images, e. g. ~, \quad, \qquad, \hfill etc. 
      %(or a blank line to force the subfigure onto a new line)
    \begin{subfigure}[b]{0.6\textwidth}
\begin{abc}[name=abc-ligadura3]
X: 1 % start of header
K: C %% Escala de C mayor %
M: 2/4
%T: Contratempo num compasso binário
V:1 clef=G  %name="Ritmo 1"   sname="Ritmo 1"
%
[V:1] | B3  (G1 | G2) B2  |
\end{abc}
\vspace{-10pt}
\caption{Uso de ligaduras em notas de compassos distintos.}
\label{fig:abc-ligadura3}
    \end{subfigure}
    \caption{Distintos usos da ligadura.}\label{fig:total-ligadura}
\end{figure}

