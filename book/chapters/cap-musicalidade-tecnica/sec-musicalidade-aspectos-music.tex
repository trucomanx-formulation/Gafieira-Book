

%%%%%%%%%%%%%%%%%%%%%%%%%%%%%%%%%%%%%%%%%%%%%%%%%%%%%%%%%%%%%%%%%%%%%%%%%%%%%%%%
%%%%%%%%%%%%%%%%%%%%%%%%%%%%%%%%%%%%%%%%%%%%%%%%%%%%%%%%%%%%%%%%%%%%%%%%%%%%%%%%
\section{\textcolor{red}{Dançar na música}}
\label{subsec:dancamusica}
\index{Musicalidade!Dançar na música}

nesse sentido na Seção \ref{sec:elementosmusica} foram listados,
os elementos da música, sendo estes:
\begin{inparaitem}
\item \hyperref[sec:pos:Ritmo]{\textbf{Ritmo}}
\item \hyperref[sec:pos:Melodia]{\textbf{Melodia}}
\item \hyperref[sec:pos:Harmonia]{\textbf{Harmonia}}
\item \hyperref[sec:pos:Contraponto]{\textbf{Contraponto}}
\end{inparaitem}



\subsection{\textcolor{red}{Seguindo isoladamente os instrumentos}}
\label{sec:seguindoinstrumentos}
Usa o tema de monofonia polifonia e homofonia
Seria uma forma de mickey mousing
\begin{itemize}
\item Dançando choro no ritmo, esquecendo o ``Tchic-Tchic Tum''.
\item Trabalhando com marionetas um em cada mão, ou dedo.
\item Cada aluno simula que tem um instrumento virtual e toca ele.
\end{itemize}

%% https://translate.google.com/translate?sl=en&tl=es&u=http%3A%2F%2Fjoymotiondance.com%2Ftexture%2F
tem a ver com texturas na musica, cada camada é um instruemnto


%%%%%%%%%%%%%%%%%%%%%%%%%%%%%%%%%%%%%%%%%%%%%%%%%%%%%%%%%%%%%%%%%%%%%%%%%%%%%%%%
%%%%%%%%%%%%%%%%%%%%%%%%%%%%%%%%%%%%%%%%%%%%%%%%%%%%%%%%%%%%%%%%%%%%%%%%%%%%%%%%
%%%%%%%%%%%%%%%%%%%%%%%%%%%%%%%%%%%%%%%%%%%%%%%%%%%%%%%%%%%%%%%%%%%%%%%%%%%%%%%%
\begin{comment}

\subsection{Emoções vs. sentimentos}
\label{ref:emotionsentimental}
Mesmo que sejam tratados como similares, 
existem diferencias entre as emoções e os sentimentos.
Por exemplo é sabido que:
\begin{itemize}
\item Alguns sentimentos estão relacionados com as emoções especificas \cite{freitas2015codigo} \cite[pp. 32]{nicolas2018musicas}.
\item Todas as emoções provocam sentimentos, porem não todos os sentimentos provem de emoções
\cite[pp. 288]{zanelli2014psicologia} \cite{freitas2015codigo}.
\item Os sentimentos podem provocar emoções \cite{freitas2013psicologia}.
\item As emoções a diferença dos sentimentos podem ser fingidos ou genuínos \cite[pp. 32]{nicolas2018musicas}.
\item As emoções são a reação a um estimulo, enquanto que os sentimentos vem de um processo cognitivo \cite{freitas2013psicologia}.
\end{itemize}

\subsubsection{As emoções} 
São fenômenos complexos e com múltiplas dimensões,
sendo a emoção uma resposta automática, rápida, de curta vida e intensa, 
que pode chegar a nos de forma consciente ou inconsciente;
sendo este um impulso neuronal que provoca no organismo a execução de uma ação,
como comportamentos de aproximação ou afastamento
\cite[pp. 288]{zanelli2014psicologia}  \cite{freitas2015codigo}.
As funções da emoção estão ligadas à adaptação e à expressão, 
sendo este um catalisador entre nossa conduta e o meio que nos embrulha;
as emoções também cumprem um papel importante no desenvolvimento da aprendizagem 
(reforços positivos ou negativos)
  \cite{freitas2015codigo},
e permite a articulação social, política e cultural dos afetos \cite[pp. 32]{nicolas2018musicas}.

As emoções básicas no ser humano são \cite{freitas2015codigo} \cite[pp. 291]{zanelli2014psicologia}:
\begin{inparaitem}
\item felicidade (alegria)  % \cite{freitas2015codigo} \cite[pp. 291]{zanelli2014psicologia}
\item tristeza              % \cite{freitas2015codigo} \cite[pp. 291]{zanelli2014psicologia}
\item repugnância (aversão) % \cite{freitas2015codigo} \cite[pp. 291]{zanelli2014psicologia}
\item surpresa              % \cite{freitas2015codigo} \cite[pp. 291]{zanelli2014psicologia}
\item medo                  % \cite{freitas2015codigo} \cite[pp. 291]{zanelli2014psicologia}
\item raiva (cólera)        % \cite{freitas2015codigo}
\item desprezo              % \cite{freitas2015codigo}
\end{inparaitem}.

Os componentes da emoção são \cite[pp. 26]{redorta2006emocion} \cite{freitas2015codigo} \cite{freitas2013psicologia} :
\begin{description}
\item[Vivencia consciente:] (Ou componente cognitiva) Sensações que as pessoas vivenciamos subjetivamente.
Podemos sentir medo, angustia, raiva, entre outros.
Por exemplo, uma criança ao ver um palhaço pode provocar-lhe alegria, ou medo,
sendo este resultado subjetivo a cada pessoa. 
\item[Reações fisiológicas:] (Ou componente neurofisiológica) Órgão e sistemas emergentes da atividade emocional.
este se manifesta com respostas ``involuntárias'' como: 
taquicardia, 
sudoração, 
vasoconstrição, 
hipertensão, 
mudanças no tom muscular,
rubor, 
sequidade na boca, 
secreções corporais, 
etc.
\item[Comportamento expressivo:] (Ou componente conductual) 
Como as pessoas expressam essas emoções de forma verbal e não verbal.
No caso da forma verbal, podem ser percebidas, mudanças do tom na voz, no volumem, no ritmo, entre outros;
e no caso da forma não verbal, podemos observar mudanças na forma como o corpo se movimenta,
como um andar erguido ou encovado, lento ou rápido, etc.

\end{description}


%% \cite{freitas2015codigo}
%% \begin{itemize}
%% \item A cognição
%% \item a expressão facial
%% \item atividades do sistema nervoso autônomo (SNA)
%% \end{itemize}


\subsubsection{Os sentimentos} 
São um ``processo cognitivo'' 
de acompanhamento continuo de uma experiencia subjetiva, 
que pode ser provocada, ou não, por emoções \cite[pp. 288]{zanelli2014psicologia} \cite{freitas2013psicologia}.
Assim, podemos separar os sentimentos em dois categorias \cite[pp. 288]{zanelli2014psicologia}:
\begin{itemize}
\item Os sentimentos provocados por emoções, proveniente de alterações corporais.
Sendo estes sentimentos um juízo, ou procesamento, que fazemos sobre as emoções. 
\item Os sentimentos de fundo, este é originado pela existência humana.
\end{itemize}


\end{comment}
