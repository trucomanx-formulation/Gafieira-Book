
%%%%%%%%%%%%%%%%%%%%%%%%%%%%%%%%%%%%%%%%%%%%%%%%%%%%%%%%%%%%%%%%%%%%%%%%%%%%%%%%
%%%%%%%%%%%%%%%%%%%%%%%%%%%%%%%%%%%%%%%%%%%%%%%%%%%%%%%%%%%%%%%%%%%%%%%%%%%%%%%%
\newpage
\section{Leitmotiv ou motivo condutor}
\label{sec:leitmotivdanca}
\index{Musicalidade!Leitmotif}
\index{Musicalidade!Leitmotiv}

\begin{tcbinformation} 
\textbf{Motivo:}
Do termo \textbf{motif} em inglês ou  \textbf{motiv}  em alemão.\\

\textbf{Nas artes cênicas:}
\index{Artes cênicas!Motivo}
\index{Artes cênicas!Motif}
É uma unidade da trama que não pode ser subdividida em unidades menores;
o motivo é uma unidade autônoma de ação e funcional na narrativa \cite[pp. 221]{pavis1998dictionary}.

\textbf{Na música:}
\index{Música!Motivo}
\index{Música!Motif}
É geralmente definido como um gesto musical identificável de longitude indeterminada \cite[pp. 123]{powrie2006changing}.

\end{tcbinformation}


\begin{notation}[Motivo condutor]
Do termo \textbf{leitmotif} em inglês ou  \textbf{leitmotiv}  em alemão \cite[pp. 7]{bribitzer2015understanding}.
\end{notation}



\subsection{Leitmotiv de Wagner}
\label{subsec:LeitmotivDeWagner}

Richard Wagner foi um compositor alemão de século XIX; 
que elaborou uma teoria própria da opera como ``drama musical''.
Onde as palavras cantadas estavam em relação com a ``ação externa'' dos eventos
enquanto a orquestra proporcionava o marco emocional, ou ``ação interna'' do drama.
a forma da música era mantida pelo \textbf{leitmotiv}, 
também conhecido como o ``\textbf{motivo} condutor'' \cite[pp. 229-230]{holst1998abc};
este era um tema básico e breve, 
que se repetia sempre que se encenava algo relacionado a um personagem, 
situação ou ideia simbólica na opera \cite[pp. 229-230]{holst1998abc} \cite[pp. 465-466]{apel1969harvard}.
Porém o leitmotiv não é rígido na sua estrutura, são feitas pequenas modificações no ritmo, intervalos, etc.
Para ajustar aos requerimentos de cada situação em cada ato da opera \cite[pp. 466]{apel1969harvard}.

O termo leitmotiv não foi cunhado por Wagner, 
e sim pelo seu amigo H. von Wolzogem em 1978, 
para denotar o método de composição de Wagner.
Pois o próprio Wagner usava o termo ``Grundthema'', tema básico,
para designar o mesmo assunto \cite[pp. 465]{apel1969harvard}.



\subsection{Leitmotiv no teatro e na  literatura}

No teatro, o leitmotiv é uma técnica muito usada  \cite[pp. 197]{pavis1998dictionary}.
\begin{itemize}
\item Na comedia o leitmotiv sucede como uma repetição cômica \cite[pp. 197]{pavis1998dictionary}.
\begin{example}
Para nós seria muito fácil identificar a situação, emocional e espacialmente,
ao pensar em frases como: 
``Foi sem querer querendo'',
``Pipipipipipipi'' 
``Palma, palma! Não priemos cânico!'' e 
``Suspeitei desde o principio''.
Todas pertencentes ao seriado cômico ``Chaves''.
\end{example} 
\item No teatro poético, ou na poesia, o leitmotiv se manifesta como a repetição de uma palavra, 
linha (verso) ou de uma figura literária ( símbolo ou metáfora) \cite[pp. 197]{pavis1998dictionary}.
\begin{example}
Na ``Rima LIII'', de Gustavo Adolfo Bécquer, temos leitmotiv na palavra ``Volverán''
e na frase ``ésas... ¡no volverán!''.
\end{example}
\begin{citando}
\textbf{Volverán} las oscuras golondrinas\\
en tu balcón sus nidos a colgar,\\
y otra vez con el ala a sus cristales\\
jugando llamarán.\\
\\
Pero aquéllas que el vuelo refrenaban\\
tu hermosura y mi dicha a contemplar,\\
aquéllas que aprendieron nuestros nombres...\\
\textbf{ésas... ¡no volverán!}\\
\\
\textbf{Volverán} las tupidas madreselvas\\
de tu jardín las tapias a escalar\\
y otra vez a la tarde aún más hermosas\\
sus flores se abrirán.\\
\\
Pero aquellas cuajadas de rocío\\
cuyas gotas mirábamos temblar\\
y caer como lágrimas del día...\\
\textbf{ésas... ¡no volverán!}\\
\end{citando}
\end{itemize}

\subsection{Leitmotiv no cinema}
\label{subsec:LeitmotivCine}

Quando escutamos a música pertencente a um filme, 
estamos propensos a ficar inundados de imagens e emoções que a música nos evoca,
isto é porque existe uma forte relação entre o auditivo e o visual.
No âmbito cinematográfico, a componente visual é a que generalmente supera em protagonismo à parte auditiva;
inclusive nos inícios do cinema (mudo) era o visual a única componente.
Mesmo que atualmente a parte auditiva tenha ganhado maior importância,
ainda coloquialmente pensamos e falamos em ir a ``ver'' um filme e não em ir a ``escutar'' um.
Inclusive muitas das pessoas que se consideram cinéfilas prestam pouca atenção 
á música cinematográfica (do inglês ``film scoring'') 
e em especial à música usada empregando a técnica do ``underscoring''\footnote{Para 
mais detalhes do ``underscoring'' ir a Página \pageref{page:underscoring}. } \cite[pp. 255]{bribitzer2015understanding}.

Compositores como Max Steiner (1888-1971) \cite[pp. 194]{nasta2004perspective} 
ou  Howard Shore (1946 - $\sim$), tem entendido o peso dramático e emocional,
que dão os temas (passagens melódicos completos \cite[pp. 1496]{latham2008diccionario}) associados a ideias na trama cinematográfica;
reforçando, anunciando ou destacando a carga emocional destas ideias.
Em tempos modernos compositores como John Williams (1932 - $\sim$) tem
incursionado na criação temas orquestrais para serem usados em filmes \cite[pp. 255-256]{bribitzer2015understanding};
por exemplo, ninguém pode esquecer os temas principais dos filmes de ``superman'' protagonizado pelo já falecido ator ``Christopher Reeve'',
ou mais recentemente o trabalho de Williams nos filmes de ``Harry Potter''.

Assim, desde que as composições de tipo orquestral cobraram força no cinema moderno,
era quase inevitável que  técnicas provenientes da música clássica ou erudita entraram entraram à música cinematográfica,
neste caso o uso do leitmotiv de Wagner \cite[pp. 256]{bribitzer2015understanding}, visto na Seção \ref{subsec:LeitmotivDeWagner}.



\begin{figure}[t]
\begin{elaboracion}[title=Que é o ``underscoring''?]
\label{ref:Underscoring}
\index{Musicalidade!Underscoring}
O ``underscoring'' é a reprodução de música em segundo plano, abaixo do diálogo ou a ação principal numa cena,
que da o marco emocional ou subjetivo à mesma;
por exemplo, colocar uma música triste numa cena triste, 
una música que nos deixe tensos segundos antes de que o grupo em cena entre a assaltar o banco, 
una música meditativa e estoica quando o protagonista se perde no horizonte caminhado a um futuro desconhecido, etc.
\end{elaboracion}
\label{page:underscoring}
\end{figure}

Claudia Gorbman indica que em muitos casos, 
existe uma diferença difusa entre um tema (num filme) e o leitmotiv de Wagner;
mesmo que a maioria de escritores indiquem a importância do leitmotiv na música cinematográfica,
eles   expressam suas dúvidas sobre o uso correto deste conceito em vários contextos específicos \cite[pp. 190]{nasta2004perspective}.
O uso do leitmotiv em Hollywood, tende a reduzir estruturas musicais complexas, 
a rótulos semânticos simples; entendendo a música como se fosse construída principalmente por meio da ancoragem textual simples;
obviamente, temas musicais curtos, ajudam ao espectador meio a manter a atenção nos personagens ou  anunciar alguma mudança dramática,
mas o proposito da música no cine é claro, este é ser linguagem e  gesto, mas que um simples rótulo  \cite[pp. 195]{nasta2004perspective}.

% mesmo quando está só está implícita ou referida \cite[pp. 7-8]{bribitzer2015understanding}.

Finalmente, existe sim uma diferença entre tema e leitmotiv, 
um tema é uma passagem melódica principal, usada na música tonal; 
que a diferença dos termos ``ideia'' ou ``motivo'', 
o termo ``tema'' indica uma frase completa ou período, 
que é usado nas partes mais importantes de uma obra \cite[pp. 1496]{latham2008diccionario}. 
Por outro, lado o termo leitmotiv é usado para designar a uma melodia curta ou motivo, 
o qual o compositor tem impregnado com um significado na obra. 

\subsection{Leitmotiv na música}

Na música, de forma geral,
significa motivo condutor \cite[pp. 230]{holst1998abc} \cite[pp. 465]{apel1969harvard}, 
e é uma passagem melodia curta (motivo) que é constantemente recorrente, e
que está associada a um personagem, situação ou ideia abstrata;
é dizer, que etiqueta um motivo outorgando-lhe  um conteúdo semântico \cite[pp. 230]{holst1998abc} \cite[pp. 7-8]{bribitzer2015understanding}.
Os leitmotiv não são estruturas rígidas, e pequenas modificações podem e são feitas,
no ritmo, intervalos, harmonia, orquestração ou acompanhamento  \cite[pp. 466]{apel1969harvard}
\cite[pp. 465]{apel1969harvard} \cite[pp. 188]{nasta2004perspective} etc. 
Para poder ser adaptadas a cada contexto do mensagem musical;
também pode ser combinado com outros leitmotiv, a fim criar uma nova situação dramática \cite[pp. 188]{nasta2004perspective}.







\begin{example}
Um exemplo interessante é ``Le carnaval des animaux'' (em inglês ``The Carnival of the Animals'')
que é uma ``musical suite'' de quatorze movimentos do compositor compositor romântico, Camille Saint-Saëns. 
\begin{description}
\item[VII. Aquarium:] Neste movimento, podemos exercitar nossa imaginação observando os peses nadando,
delicadamente na água, em particular poderemos perceber ou associar o leitmotiv da aleta caudal dos peses, 
precedendo o movimento dos mesmos.

\item[IV. Tortoises:] Aqui poderemos entender a melodia como o contexto emocional do movimento das tartarugas no mar,
e  o piano esquerda marca um ritmo, constante, cujo motivo, pode ser entendido como o leitmotiv do avanço das tartarugas na água,
num continuo abrupto, leve, leve.

%\item [V. The Elephant] Neste movimento, underscoring dos elefantes caminhando
%\item[XIII. The Swan] Neste movimento, underscoring de cisne limpando as penas
\end{description}

\end{example}

\begin{example}
O conto de fadas sinfônico para crianças, `` Peter and the Wolf '', 
é uma composição musical escrita por Sergei Prokofiev em 1936.
Nesta composição cada instrumento musical está associado com um personagem na trama;
pelo que se conseguimos isolar mentalmente cada instrumento, 
será fácil reconhecer na melodia executada o leitmotiv que representa a cada personagem.
Assim temos:
\begin{description}
\item[O pássaro:] Representado pela flauta.
\item[O pato:] Representado pelo Oboé.
\item[O gato:] Representado pelo clarinete.
\item[O avô:] Representado pelo Fagote (bassoon).
\item[O lobo:] Representado pelo ``French horns''.
\item[Os caçadores:] Representado pelos instrumentos de sopro (madeiras)
e trombeta, com tiros nos timbales e bumbo.
\item[Peter:] Representado pelos instrumentos de corda
\end{description}
\end{example}

\subsection{Motivo vs. leitmotiv}
A diferença entre o motivo (motif) e o leitmotiv, 
é que  enquanto o motivo cumpre o papel de ser um elemento de repetição (mutável)
para a construção de estruturas mais complexas (frases, temas ou melodias por exemplo),
o leitmotiv tem uma camada a mais; pois além de ser um motivo, 
este tem o proposito de conduzir, indicar ou anunciar diretamente
a uma personagem, situação ou ideia simbólica.

Assim, e usando uma metáfora, se o motivo é um tijolo e uma melodia é a parede, 
os leitmotiv são os tijolos que  você marcou com giz, quando criança, para souber quanto cresceu.
Nesse momento cada tijolo semelhante, e com uma marca, expressa uma personagem, situação ou ideia,
e sua recorrência ou regularidade ajuda a contar uma historia.
É dizer, o tijolo marcado indica o leitmotiv do crescimento, na obra expressada na parede.
Existirão claro outros motivos e alguns serão leitmotiv também; por exemplo, os tijolos de outras cores,
salpicados ao longo da parede, indicam um leitmotiv da ideia: vamos  a ``conseguir'' finalizar de algum jeito.



\subsection{Leitmotiv aplicado à dança}

Pelo exposto nas subseções anteriores sobre o leitmotiv em outras artes,
sabemos que um motivo na música é um som ou conjunto de notas musicais,
que se carateriza por ser um padrão de repetição (mutável) usado ao longo de uma obra.
Conhecemos também que um motivo pode virar leitmotiv se ele está vinculado a algum elemento físico ou figurativo da trama.
Assim, para aplicar esta técnica na dança, podemos usar as seguintes indicações:
\begin{description}
\item[1)] Primeiro devemos observar ou perceber se existe, na música algum pequeno, e indivisível, padrão de repetição (motivo), 
seja este regular ou irregular na forma ou no tempo ao longo da peça.
\item[2)] Identificado este motivo, existem duas possibilidades, 
\begin{description}
\item[a)] A primeira é que pela própria estrutura da peça musical, este motivo esteja sendo usado de fato como um leitmotiv, 
na mensagem que o compositor quer transmitir.
Se este é o caso, deveremos incorporar o motivo na nossa dança, 
tentando expressar com nossos movimentos a ideia que o compositor quer transmitir.
Já seja: um bater de assas, a força da vontade, o amor, o medo, a tristeza, etc.
\item[b)] Outra possibilidade é que o motivo (ou sonido) só seja uma peça estrutural na música, 
não vinculada a nenhuma ideia na narrativa.
Se este é o caso, então seremos nós que elevaremos este motivo à categoria de leitmotiv,
atrelando a ele de forma biunívoca alguma dinâmica de nossa dança, que consideremos em consonância com o som, 
dando-lhe assim um vinculo figurativo com nosso movimento.
Por exemplo podemos atrelar o movimento a alguma palavra ou frase curta do cantor ou a 
algum motivo rítmico ou melódico produzido pelo cavaquinho, 
pandeiro, cuíca, agogô, etc.
\end{description}
\item[3)] Finalmente na execução do leitmotiv,
\begin{description} 
\item[a)] podemos usar este como um enfeite que não altere a informação intercambiada durante a condução;
por exemplo: um movimento de ombros, cabeça, quadril, ou simplesmente um estalar de dedos. 
\item[b)] Ou podemos atrelar o motivo a algum movimento ou passo como: cruzado, tesoura, romario, cadeirinha, etc.  
\end{description}
\end{description}

\begin{example}
Na música ``laranja madura'' de Ataulfo Alves interpretado por Casuarina,
\begin{itemize}
\item  o \hyperref[def:Condutor]{\textbf{condutor}} pode, indicar a execução de um ``Romário'',
cada vez que o cantor inicie a frase ``o tem marimbondo no pé''; 
e 
\item qualquer pessoa do \hyperref[def:Par]{\textbf{par}} de dança pode sorrir 
mexendo os ombros (3 vezes no plano frontal) imediatamente depois que o cantor fale ``laranja madura'',
quando o cavaquinho ou cuica inicia a execução de um motivo em resposta a essa frase.
\end{itemize}

O ``Romário'' representa a escolha do movimento correspondente a um leitmotiv existente na trama,
como foi visto em \textbf{2.a}, sendo executado como parte da comunicação na condução, como explicado em \textbf{3.b}.

O mexer de ombros seria um motivo estrutural na música, que tem sido elevado por nós à categoria de leitmotiv,
como foi visto em \textbf{2.b}; porém, esta vez o movimento não está sendo parte da comunicação na condução,
e ficaria mais como um enfeite pessoal como explicado em \textbf{3.a}.

\end{example}

\begin{example}
\label{ex:leitmotiftiroalvaro}
Na música ``Tiro ao Álvaro'' de Adoniran Barbosa e interpretado  por Elis Regina,
\begin{itemize}
\item qualquer pessoa do \hyperref[def:Par]{\textbf{par}} de dança pode fazer um enfeite (ombros, quadril, cabeça ou sorriso),
sem afetar a informação transmitida na condução, apos cada breque da voz;
especificamente, quando o instrumento de corda executa uma única nota (nota $X$), em resposta a esse breque.
Estos momentos são, quando Elis Regina canta: 
\begin{itemize}
\item ``De tanto levá'' (nota $X$).
\item ``Meu peito até'' (nota $X$).
\end{itemize}
\end{itemize}
A nota $X$ seria um som que tem sido elevado por nós à categoria de leitmotiv,
pelo movimento e expressão que atrelamos a ele;
como foi visto em \textbf{2.b}. 
Como o tempo da nota X é pequeno, neste caso a sugestão é fazer um enfeite como explicado em \textbf{3.a},
que dá a ideia: ``se dar, dá! E se não, não atrapalho a meu par na condução''.

\end{example}


\subsection{Mickey mousing vs. leitmotiv}
Entre estas duas técnicas não existe exclusão, pois ambas abordagens são transversais entre sim,
pelo que podem ser usadas ao mesmo tempo.
Pois, como é visto na seção \ref{sec:mikeymousing}, a técnica do ``mikey mousing'',
indica uma relação de sincronia temporal entre os sons percebidos e os elementos na cena;
como por exemplo, cada pisada, piscar de olhos, golpes e sustos,  acontecem em sincronia com alguns sonidos.
Porém, tudo isto é feito sem a necessidade de conservar o uso dos mesmos sons ou motivos para cada um destes elementos.
No caso do uso da técnica do leitmotiv temos um motivo (um som)
que se repete, porém este é atrelado a um sujeito físico ou figurativo.

Assim, se por exemplo na nossa dança realizamos a técnica do mickey mousing, 
sincronizando cada movimento com um som, 
e agregamos uma camada a mais a nossa dança, de modo que estes sons tenham uma relação biunívoca com esses movimentos; 
então, poderíamos afirmar que temos escolhido os leitmotiv de cada um de nossos movimentos (ver Exemplo \ref{ex:leitmotiftiroalvaro}). 

