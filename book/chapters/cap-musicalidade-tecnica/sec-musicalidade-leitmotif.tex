\newpage
%%%%%%%%%%%%%%%%%%%%%%%%%%%%%%%%%%%%%%%%%%%%%%%%%%%%%%%%%%%%%%%%%%%%%%%%%%%%%%%%
%%%%%%%%%%%%%%%%%%%%%%%%%%%%%%%%%%%%%%%%%%%%%%%%%%%%%%%%%%%%%%%%%%%%%%%%%%%%%%%%

\section{\textcolor{blue}{Leitmotiv ou motivo condutor}}
\index{Musicalidade!Leitmotif}
\index{Musicalidade!Leitmotiv}

\begin{tcbinformation} 
\textbf{Motivo:}
Do termo \textbf{motif} em inglês ou  \textbf{motiv}  em alemão.\\

\textbf{Nas artes cênicas:}
\index{Artes cênicas!Motivo}
\index{Artes cênicas!Motif}
É uma unidade da trama que não pode ser subdividida em unidades menores,
o motivo é uma unidade autônoma de ação, uma unidade funcional da narrativa \cite[pp. 221]{pavis1998dictionary}.

\textbf{Na música:}
\index{Música!Motivo}
\index{Música!Motif}
É geralmente definido como um gesto musical identificável de longitude indeterminada \cite[pp. 123]{powrie2006changing}.

\end{tcbinformation}


\begin{notation}[Motivo condutor]
Do termo \textbf{leitmotif} em inglês ou  \textbf{leitmotiv}  em alemão.
\end{notation}



\subsection{Leitmotiv de Wagner}

Richard Wagner foi um compositor alemão de século XIX; 
que elaborou uma teoria própria da opera como ``drama musical''.
Onde as palavras cantadas estavam em relação com a ``ação externa'' dos eventos
enquanto a orquestra proporcionava o marco emocional, ou ``ação interna'' do drama.
a forma da música era mantida pelo \textbf{leitmotiv}, 
também conhecido como o ``\textbf{motivo} condutor'' \cite[pp. 229-230]{holst1998abc};
este era um tema básico e breve, 
que se repetia sempre que se encenava algo relacionado a um personagem, 
situação ou ideia simbólica na opera \cite[pp. 229-230]{holst1998abc} \cite[pp. 465-466]{apel1969harvard}.
Porem o leitmotiv não é rígido na sua estrutura, são feitas pequenas modificações no ritmo, intervalos, etc.
Para ajustar aos requerimentos de cada situação em cada ato da opera \cite[pp. 466]{apel1969harvard}.

O termo leitmotiv não foi cunhado por Wagner, 
e sim pelo seu amigo H. von Wolzogem em 1978, 
para denotar o método de composição de Wagner.
Pois o próprio Wagner usava o termo ``Grundthema'', tema básico,
para designar o mesmo assunto \cite[pp. 465]{apel1969harvard}.



\subsection{Leitmotiv no teatro e na  literatura}

No teatro, o leitmotiv é uma técnica muito usada  \cite[pp. 197]{pavis1998dictionary}.
\begin{itemize}
\item Na comedia o leitmotiv sucede como uma repetição cômica \cite[pp. 197]{pavis1998dictionary}.
\begin{example}
Para nós seria muito fácil identificar a situação, emocional e espacialmente,
ao pensar em frases como: 
``Foi sem querer querendo'',
``Pipipipipipipi'' 
``Palma, palma! Não priemos cânico!'' e 
``Suspeitei desde o principio''.
Todas pertencentes ao seriado cômico ``Chaves''.
\end{example} 
\item No teatro poético, ou na poesia, o leitmotiv se manifesta como a repetição de uma palavra, 
linha (verso) ou de uma figura literária ( símbolo ou metáfora) \cite[pp. 197]{pavis1998dictionary}.
\begin{example}
Na ``Rima LIII'', de Gustavo Adolfo Bécquer, temos leitmotiv na palavra ``Volverán''
e na frase ``ésas... ¡no volverán!''.
\end{example}
\begin{citando}
\textbf{Volverán} las oscuras golondrinas\\
en tu balcón sus nidos a colgar,\\
y otra vez con el ala a sus cristales\\
jugando llamarán.\\
\\
Pero aquéllas que el vuelo refrenaban\\
tu hermosura y mi dicha a contemplar,\\
aquéllas que aprendieron nuestros nombres...\\
\textbf{ésas... ¡no volverán!}\\
\\
\textbf{Volverán} las tupidas madreselvas\\
de tu jardín las tapias a escalar\\
y otra vez a la tarde aún más hermosas\\
sus flores se abrirán.\\
\\
Pero aquellas cuajadas de rocío\\
cuyas gotas mirábamos temblar\\
y caer como lágrimas del día...\\
\textbf{ésas... ¡no volverán!}\\
\end{citando}
\end{itemize}

\subsection{\textcolor{red}{Leitmotiv no cinema}}

No âmbito da música cinematográfica (do inglês ``film scoring''),
existem regras e convenções, para a música que é composta para 
ser usada especificamente em filmes.
\begin{comment}
Assim, o sistema cinematográfico clássico é generalizado pelos 
estudiosos do cinema em duas concepções \cite[pp. 121]{powrie2006changing}:
\begin{itemize}
\item A visão de que o espectador é um sujeito autônomo que percebe 
e procura ver detalhes dramaticamente importantes,
que deem lugar a uma postura anti-idealista em que o espectador está 
posicionado dentro de uma ideologia psíquica \cite[pp. 121]{powrie2006changing}.
\item A visão que os filmes de Hollywood trabalham com o objetivo 
de articular a história dentro de um discurso invisível \cite[pp. 121]{powrie2006changing}.
\end{itemize}
\end{comment}

motivo principal \cite[pp. 7]{bribitzer2015understanding} 

\subsection{Leitmotiv na música}



Na música, de forma geral,
significa motivo condutor \cite[pp. 230]{holst1998abc} \cite[pp. 465]{apel1969harvard}, 
e é uma frase musical curta e constantemente recorrente, 
que está associada a um personagem, situação ou ideia simbólica \cite[pp. 230]{holst1998abc}.
Os leitmotiv não são estruturas rígidas, e pequenas modificações podem e são feitas,
no ritmo, intervalos,  \cite[pp. 466]{apel1969harvard}
\cite[pp. 465]{apel1969harvard} etc. Para poder ser adaptadas a cada contexto do mensagem musical.

\begin{example}
Um exemplo interessante é ``Le carnaval des animaux'' (em inglês ``The Carnival of the Animals'')
que é uma ``musical suite'' de quatorze movimentos do compositor compositor romântico, Camille Saint-Saëns. 
\begin{description}
\item[VII. Aquarium:] Neste movimento, podemos exercitar nossa imaginação observando os peses nadando,
delicadamente na água, em particular poderemos perceber ou associar o leitmotiv da aleta caudal dos peses, 
precedendo o movimento dos mesmos.

\item[IV. Tortoises:] Aqui poderemos entender a melodia como o contexto emocional do movimento das tartarugas no mar,
e  o piano esquerda marca um ritmo, constante, cujo motivo, pode ser entendido como o leitmotiv do avanço das tartarugas na água,
num continuo abrupto, leve, leve.

%\item [V. The Elephant] Neste movimento, underscoring dos elefantes caminhando
%\item[XIII. The Swan] Neste movimento, underscoring de cisne limpando as penas
\end{description}

\end{example}

\begin{example}
O conto de fadas sinfônico para crianças, `` Peter and the Wolf '', 
é uma composição musical escrita por Sergei Prokofiev em 1936.
Nesta composição cada instrumento musical está associado com um personagem na trama;
pelo que se conseguimos isolar mentalmente cada instrumento, 
será fácil reconhecer na melodia executada o leitmotiv que representa a cada personagem.
Assim temos:
\begin{description}
\item[O pássaro:] Representado pela flauta.
\item[O pato:] Representado pelo Oboé.
\item[O gato:] Representado pelo clarinete.
\item[O avô:] Representado pelo Fagote (bassoon).
\item[O lobo:] Representado pelo ``French horns''.
\item[Os caçadores:] Representado pelos instrumentos de sopro (madeiras)
e trombeta, com tiros nos timbales e bumbo.
\item[Peter:] Representado pelos instrumentos de corda
\end{description}
\end{example}

\subsection{Motivo vs. leitmotiv}
A diferença entre o motivo (motif) e o leitmotiv, 
é que  enquanto o motivo cumpre o papel de ser um elemento de repetição (mutável)
para a construção de estruturas mais complexas (frases, temas ou melodias por exemplo),
o leitmotiv tem uma camada a mais; pois além de ser um motivo, 
este tem o proposito de conduzir, indicar ou anunciar diretamente
a uma personagem, situação ou ideia simbólica.

Assim, e usando uma metáfora, se o motivo é um tijolo e uma melodia é a parede, 
os leitmotiv são os tijolos que  você marcou com giz, quando criança, para souber quanto cresceu.
Nesse momento cada tijolo semelhante, e com uma marca, expressa uma personagem, situação ou ideia,
e sua recorrência ou regularidade ajuda a contar uma historia.
É dizer, o tijolo marcado indica o leitmotiv do crescimento, na obra expressada na parede.
Existirão claro outros motivos e alguns serão leitmotiv também; por exemplo, os tijolos de outras cores,
salpicados ao longo da parede, indicam um leitmotiv da ideia: vamos  a ``conseguir'' finalizar de algum jeito.



\subsection{\textcolor{blue}{Leitmotiv aplicado à dança}}

Pelo exposto nas subseções anteriores sobre o leitmotiv em outras artes,
sabemos que um motivo é um som ou conjunto de notas musicais,
que se carateriza por ser um padrão de repetição (mutável) usado ao longo de uma obra;
sabemos também que um motivo pode virar um leitmotiv se ele está vinculado a algum elemento físico ou figurativo na trama.
Assim, para aplicar estos conceitos  na dança, podemos usar as seguintes indicações:
\begin{itemize}
\item Primeiro devemos observar ou perceber se existe, na música algum pequeno, e indivisível, padrão de repetição (motivo), 
seja este regular ou irregular ao longo da peça.
\item Identificado este motivo, existem duas possibilidades, 
\begin{itemize}
\item Que pela própria estrutura da peça musical, este motivo esteja sendo usado de fato como um leitmotiv, 
na mensagem que quer transmitir o compositor.
Se este é o caso, deveremos incorporar o motivo na nossa dança, 
e tentar expressar com nossos movimentos a ideia que quer transmitir o compositor.
já seja: a leveza de um pássaro, a força da vontade, o amor, o medo, a tristeza, etc.
\item Outra possibilidade é que o motivo seja só uma peça estrutural da música, não vinculada a nenhuma ideia na narrativa.
Se este é o caso, então seremos nos que elevaremos este motivo à categoria de leitmotiv,
atrelando a ele alguma dinâmica de nossa dança, que consideremos em consonância com o som.
 Por exemplo podemos atrelar nossos movimentos a algum motivo rítmico ou melódico produzido pelo cavaquinho, 
o pandeiro, a cuíca, o agogô, etc.
\end{itemize}
\item Finalmente na execução do leitmotiv,
\begin{itemize} 
\item podemos usar este como um enfeite que não altere a informação intercambiada durante a condução;
por exemplo: um movimento de ombros, cabeça, quadril, ou simplesmente um estalar de dedos. 
\item Ou podemos atrelar o motivo a algum movimento ou passo como: cruzado, tesoura, romario, cadeirinha, etc.  
\end{itemize}
\end{itemize}
%https://translate.google.com.br/translate?sl=en&tl=pt&u=https%3A%2F%2Fthisdancinglife.com%2Fmusicality-in-dance%2F

\subsection{Mickey mousing vs. leitmotiv}
Entre estas duas técnicas não existe exclusão, pois ambas abordagens são transversais entre sim,
pelo que podem ser usadas ao mesmo tempo.
Pois, como é visto na seção \ref{sec:mikeymousing}, a técnica do ``mikey mousing'',
indica uma relação de sincronia temporal entre os sons percebidos e os elementos na cena;
como por exemplo, cada pisada, piscar de olhos, golpes, sustos, etc. acontecem em sincronia com alguns sonidos.
Porem, tudo isto é feito sem a necessidade de conservar o uso dos mesmos sons ou motivos para cada um destes elementos.
No caso do uso da técnica do leitmotiv temos um motivo (um som)
que se repete, porem este é atrelado a um sujeito físico ou figurativo.

Assim, se por exemplo na nossa dança realizamos a técnica do mickey mousing, 
sincronizando cada som com um movimento, 
e agregamos uma camada a mais a nossa dança, de modo que cada motivo na música, 
tenha exclusividade num tipo de movimento; então estaremos também usado a técnica do leitmotiv. 

