\chapterimage{chapter_head_2.pdf} % Chapter heading image

\chapter{Fundamentos de notação musical}
Nas seguintes sub seções abordaremos os termos compasso, tempo e contratempo,
num sentido rítmico (distribuição de tempos) e não explicaremos o significado 
melódico (distribuição de frequências) das 
figuras musicais na partitura, devido a que as explicações mostradas aqui estão
orientadas para um público interessado na dança, que precisa numa primeira 
aproximação à música, conhecer rapidamente a parte rítmica dela. Para aprofundar mais na parte 
melódica recomendamos acudir a materiais, livros ou revistas especializadas \cite{medteoria}
\cite{azevedocompor} \cite{alves2004teoria} \cite{mascarenhascurso} \cite{adolfo2002musica} \cite{grabner2001teoria}.
%%%%%%%%%%%%%%%%%%%%%%%%%%%%%%%%%%%%%%%%%%%%%%%%%%%%%%%%%%%%%%%%%%%%%%%%%%%%%%%%
\section{Compasso}
\label{sec:compaso}

O dicionário de Harvard de música \cite{randel2003harvard} define compasso (``meter'' em inglês)
como: ``O padrão em que uma sucessão constante de pulsos rítmicos é organizada'', também comenta que
``o compasso de uma obra ou uma porção dela é indicado geralmente por uma fração'', como por exemplo:
$2/2$ , $3/4$ , $4/4$, etc; em português é chamada a esta fração como a formula do compasso. 

 
O denominador da formula do compasso indica o valor da pulsação básica (tempo) do padrão (compasso), 
e o numerador indica o número dessas pulsações que compõem o compasso. Assim, com a ajuda da
Tabela \ref{tab:noteslength}, onde a primeira coluna mostra a formula usando só o denominador do compasso,
a segunda coluna mostra as figuras musicais que o denominador evoca e a terceira e
quarta coluna representam a duração em segundos e o nome da figura musical; assim,
podemos achar equivalências aos exemplos de formula do compasso dados
anteriormente; onde os compassos com formula $\mathbf{2}/2$ tem cada um duração de $2$\halfnote ~(duas mínimas),  
compassos com formula $\mathbf{3}/4$ uma duração de $3$\quarternote ~(trés semínimas) 
e $\mathbf{4}/4$ uma duração de $4$\quarternote ~(quatro semínimas). É importante
ressaltar que a duração em tempo das figuras musicais é relativa, como pode ser visto
na terceira coluna da Tabela \ref{tab:noteslength}, onde as durações estão em função
da duração $S$ da semibreve. 
\begin{table}[h]
\centering
\begin{tabular}{|c|c|c|c|}
\hline
formula & Figura  & Duração & Nome\\ \hline
\hline
$1/1$   & \fullnote    & $S$   & Semibreve \\ \hline
$1/2$ & \halfnote    & $S/2$ & Mínima \\ \hline
$1/4$ & \quarternote & $S/4$ & Semínima \\ \hline
$1/8$ & \eighthnote  & $S/8$ & Colcheia \\ \hline
\end{tabular}
\caption{Duração e símbolos de algumas figuras musicais}
\label{tab:noteslength}
\end{table}

Se classificamos aos compassos por sua métrica, os três tipos mais conhecidos 
são os compassos binários, ternários, quaternários \cite[pp. 27]{adolfo2002musica}.

\textbf{Um compasso binário} é uma estrutura rítmica que se carateriza por ter compassos com dois tempos,
sendo o primeiro pulso forte e o segundo fraco \cite[pp. 66]{adolfo2002musica}\cite[pp. 28]{alves2004teoria}. Em geral podem ser chamados
de compassos binários quando a formula de compasso tem a forma $2A/B$, 
onde $A$ pode ser $1$ ou $3$ e $B$ pode ser $2$, $4$, $8$, etc.
Por exemplo as formulas: $2/8$, $2/4$, $2/2$,  $6/16$, $6/8$ e  $6/4$;
representam compassos binários.
Alguns autores consideram aos compassos quaternários (ex: 4/4, 12/4) como um caso de compasso binário,
chamando eles de compasso binário duplo \cite[pp. 41]{grabner2001teoria}.
A Figura \ref{compasso:binario}, representa um exemplo de compasso binário, com 
formula de compasso $2/2 \equiv 2$\halfnote, onde o primeiro e segundo
compasso tem uma duração de $S$ (uma semibreve), o primeiro compasso contem $2$\halfnote~e
o segundo contem $4$\quarternote.
\begin{figure}[H]
\centering
\begin{abc}[name=compasso1]
X: 1 % start of header
K: C % scale: C major
M: 2/2 %meter - compasso
"Primeiro compasso" G4 F4 |"Segundo compasso" G2 D2 F2 D2  |
\end{abc}
\caption{Exemplo de compasso binário}
\label{compasso:binario}
\end{figure}

\textbf{Um compasso ternário} 
é uma estrutura rítmica que se carateriza por ter compassos com trés tempos,
sendo o primeiro pulso forte e os outros dois fracos 
\cite[pp. 67]{adolfo2002musica}\cite[pp. 30]{alves2004teoria}. Em geral podem ser chamados
de compassos ternários quando a formula de compasso tem a forma $3^A/B$, 
onde $A$ pode ser $1$ ou $2$ e $B$ pode ser $2$, $4$, $8$, etc.
Por exemplo as formulas: $3/8$, $3/4$, $3/2$,  $9/16$, $9/8$ e $9/4$;
representam compassos ternários (com denominador múltiplo de 3 exclusivamente) 
A Figura \ref{compasso:ternario}, representa um exemplo de compasso ternário, com 
formula de compasso $3/4 \equiv 3$\quarternote, onde o primeiro e segundo
compasso tem uma duração de $0.75S$, o primeiro compasso contem $3$\quarternote~e
o segundo contem $6$\eighthnote.
\begin{figure}[H]
\centering
\begin{abc}[name=compasso2]
X: 1 % start of header
K: C % scale: C major
M: 3/4 %meter - compasso
"Primeiro compasso" G2 F2 F2 |"Segundo compasso" G1 F1 E1 D1 D1  D1  |
\end{abc}
\caption{Exemplo de compasso ternário}
\label{compasso:ternario}
\end{figure}


\textbf{Um compasso quaternário}  
é uma estrutura rítmica que se carateriza por ter compassos com quatro tempos,
sendo o primeiro pulso forte o segundo fraco o terceiro semiforte e o ultimo fraco 
\cite[pp. 67]{adolfo2002musica}\cite[pp. 32]{alves2004teoria}. 
Em geral podem ser chamados
de compassos quaternários quando a formula de compasso tem a forma $4A/B$, 
onde $A$ pode ser $1$ ou $3$ e $B$ pode ser $2$, $4$, $8$, etc.
Por exemplo as formulas: $4/8$, $4/4$, $4/2$,  $12/16$, $12/8$ e $12/4$;
representam compassos quaternários 
A Figura \ref{compasso:quaternario}, representa um exemplo de compassos quaternário, com 
formula de compasso $4/4 \equiv 4$\quarternote, onde o primeiro e segundo
compasso tem uma duração de $S$, o primeiro compasso contem $4$\quarternote~e
o segundo contem $8$\eighthnote.
\begin{figure}[H]
\centering
\begin{abc}[name=compasso3]
X: 1 % start of header
K: C % scale: C major
M: 4/4 %meter - compasso
"Primeiro compasso" G2 D2 F2 D2|"Segundo compasso" G1 F1 D1 C1 F1 E1 D1 C1 |
\end{abc}
\caption{Exemplo de compasso quaternário}
\label{compasso:quaternario}
\end{figure}


%%%%%%%%%%%%%%%%%%%%%%%%%%%%%%%%%%%%%%%%%%%%%%%%%%%%%%%%%%%%%%%%%%%%%%%%%%%%%%%%
\section{Tempo}

Como já foi sugerido na Seção \ref{sec:compaso}, é chamado de "tempo" 
à pulsação básica e unidade de medida dos compassos nas composições musicais;
assim, temos que compassos binários, ternários e quaternários tem uma duração 2 tempos, 
3 tempos e 4 tempos, respetivamente. Por comodidade chamaremos $T$ à duração em segundos de cada tempo,
sendo que o valor de $T$ variará dependendo da formula de compasso usada.


É importante
ressaltar que os compassos que usem a mesma formula de compasso terão sempre a mesma duração em segundos;
por outro lado, podem ser achados casos onde compassos com diferente formula podem ter a mesma duração;
por exemplo: compassos binários com formula $\mathbf{2}/2$, como na Figura \ref{fig:tempo1}, 
\begin{figure}[H]
\centering
\begin{abc}[name=tempo1]
X: 1 % start of header
K: C % scale: C major
M: 2/2 %meter - compasso
G2 D2 F2 D2 | G4 F4 |
w: T/2 T/2 T/2 T/2  Tempo Tempo
\end{abc}
\caption{Exemplo de dois compassos com 2 tempos de duração T}
\label{fig:tempo1}
\end{figure}
terão uma duração de dois tempos ($2T$) \cite[pp. 25]{azevedocompor} onde cada tempo ($T$) tem uma duração 
de uma mínima (\halfnote), ver Tabela \ref{tab:noteslength}.

Por outro lado,
compassos quaternários com formula $\mathbf{4}/4$, como na Figura \ref{fig:tempo2}, 
\begin{figure}[H]
\centering
\begin{abc}[name=tempo2]
X: 1 % start of header
K: C % scale: C major
M: 4/4 %meter - compasso
G2 D2 F2 D2| G4 F4|
w: Tempo Tempo Tempo Tempo 2T 2T
\end{abc}
\caption{Exemplo de dois compassos com 4 tempos de duração T}
\label{fig:tempo2}
\end{figure} 
terão uma duração de 4 tempos ($4T$) \cite[pp. 25]{azevedocompor} onde 
cada tempo ($T$) tem uma duração de uma semínima (\quarternote), ver Tabela \ref{tab:noteslength}.
Assim, estas duas formulas ($2/2$ e $4/4$) representam compassos 
com a mesma duração em segundos, uma semibreve (\fullnote),
porem usam tempos com diferente valor $T$ em segundos.

\begin{lattention}
Se interpretamos a música mostrada nas Figuras \ref{fig:tempo1} e \ref{fig:tempo2},
podemos perceber que ambas descrevem o mesmo sonido; assim, não é factível
distinguir só escutando se o sonido provem de um compasso com formula $2/2$ ou $4/4$.
Em estos casos e similares o mais factível é afirmar que pertencem a alguma sub categoria da família dos
compassos binários, seguindo o critério de alguns autores \cite[pp. 41]{grabner2001teoria} que afirmam 
 que os compassos
quaternários são uma sub categoria de compassos binários.
\end{lattention}

%%%%%%%%%%%%%%%%%%%%%%%%%%%%%%%%%%%%%%%%%%%%%%%%%%%%%%%%%%%%%%%%%%%%%%%%%%%%%%%%
\section{Contratempo}
Um contratempo acontece quando as notas (representadas por figuras musicais na partitura) 
são executadas em tempos fracos do compasso
ou nas partes fracas dos tempos, e estas são intercaladas por pausas nos tempos
fortes ou partes fortes dos tempos \cite[pp. 16]{mascarenhascurso} 
\cite[pp. 36]{azevedocompor}, neste sentido o contratempo pode ser visto como a 
omissão de notas nos tempos fortes ou nas partes fortes dos tempos \cite[pp. 146]{medteoria}.
Ou ``num sentido mais amplo, o contratempo é a acentuação de um tempo fraco em vez de um tempo forte'' \cite[pp. 147]{medteoria}. 

Assim é fácil de perceber que a palavra ``contratempo'' na música, 
é um adjetivo que faz referencia a como estão configuradas ou acentuadas 
as notas no compasso. Por exemplo:
A Figura \ref{fig:contratempoa} mostra 
quatro compassos (binários) com formula $2/4$, em cada compasso existem 
contratempos nos tempos fracos ou nas partes fracas dos tempos, sendo que cada tempo
tem uma duração de uma semínima (\quarternote) e cada compasso uma duração 
de uma mínima (\halfnote), ou seja duas semínimas (2\quarternote). 
\begin{itemize}
\item ``F''  indica que é o tempo é forte, 
\item ``f''  indica que é o tempo é fraco,
\item ``FF'' indica que é a parte forte de um tempo forte,
\item ``Ff'' indica que é a parte fraca de um tempo forte,
\item ``fF'' indica que é a parte forte de um tempo fraco,
\item ``ff'' indica que é a parte fraca de um tempo fraco, 
\end{itemize} 

finalmente
a figura musical \ViPa~ indica um silencio da mesma duração que uma semínima (\quarternote)
e a figura musical \AcPa~ indica um silencio da mesma duração que uma colcheia (\eighthnote).
\begin{figure}[H]
\centering
\begin{abc}[name=contratempoa]
X: 1 % start of header
K: C % scale: C major
M:2/4
%T: Contratempo num compasso binário
V:1 clef=treble name="A" sname="A"
[V:1] "F"z2 "f"G2 | "FF"z1 "Ff"G1  "fF"z1 "ff"G1 | "FF"z1 "Ff"G1  "f"G2 |  "F"z2 "fF"z1 "ff"G1  |
w:          Tempo          T/2            T/2             T/2     Tempo                 T/2
\end{abc}
\caption{Contratempos no tempos fracos ou nas partes fracas dos tempos}
\label{fig:contratempoa}
\end{figure}
Na Figura \ref{fig:contratempoa}, existem contratempos em todos os compassos porem estes estão
configurados de distintas formas;
no primeiro compasso acontece um contratempo dado que a única nota é executada 
no tempo fraco do compasso, no segundo compasso acontecem contratempos pois as 
notas são executadas nas partes fracas de cada tempo,
no terceiro compasso acontece um contratempo pela execução de uma nota na parte 
fraca do tempo forte, sendo o resto do tempo preenchido com um silencio, e 
finalmente no quarto compasso acontece um contratempo pela execução de uma nota
na parte fraca do tempo fraco, sendo o resto do compasso preenchido com silêncios.


Por outro, A Figura \ref{fig:contratempob} mostra como o contratempo pode ser 
expressado como a acentuação de um tempo fraco em vez de um tempo forte \cite[pp. 147]{medteoria}. 
\begin{figure}[H]
\centering
\begin{abc}[name=contratempob]
X: 1 % start of header
K: C % scale: C major
M:2/4
%T: Contratempo num compasso binário
V:1 clef=treble name="A" sname="A"
[V:1] "F"G2 "f"+accent+G2 | "FF"G1 "Ff"+accent+G1  "fF"G1 "ff"+accent+G1 | "FF"G1 "Ff"+accent+G1  "f"G2  | 
w:    Tempo Tempo           T/2    T/2             T/2    T/2              T/2    T/2             Tempo   
\end{abc}
\caption{Contratempos pela acentuação dos tempos fracos ou nas partes fracas dos tempos}
\label{fig:contratempob}
\end{figure}

%%%%%%%%%%%%%%%%%%%%%%%%%%%%%%%%%%%%%%%%%%%%%%%%%%%%%%%%%%%%%%%%%%%%%%%%%%%%%%%%
\section{Contagem dos tempos musicais desde o ponto de vista do ouvinte}
Antes de iniciar esta seção é importante mencionar uma
problemática que é vista com muita frequência nas escolas de dança; esta é gerada devido a que: A forma em que os tempos são contados 
nos compassos, é
diferente à realizada entre profissionais da música e da dança. 
Sendo que a contagem dos profissionais da música segue as normas
e notações indicadas na partitura, e no caso de profissionais da dança segue geralmente um enfoque 
particular a cada escola de dança, visando só em muitos casos o fácil entendimento do aluno da
execução dos movimentos, e não uma rigorosidade teórica no uso de termos e 
expressões musicais.

Conhecido Tudo isto,
quando escutamos uma música na qual é tipicamente dançado samba de gafieira,
podemos distinguir que a soma dos sonidos produzidos pelos instrumentos realizam 
um padrão de repetição muito particular, geralmente ligado à onomatopeia ``Chic Chic Tum''.
Este padrão está constituído de dois sons ``Chic'' da mesma duração em tempo, 
e um sonido ``Tum'' que ocupa o mesmo tempo que a duração de dois ``Chic''.


A Figura \ref{fig:caquarela} representa os compassos 18, 19 e 20 da  
composição musical ``Aquarela do Brasil'' escrita
por Ary Barroso em 1939 \cite{AquarelaDoBrasil}; 
a versão mostrada na figura teve arranjos por Irineu Krüger \cite{Irineu}. 
Nesta versão, a música está representada com 1 voz ou coro de voces (``Voice Choir'') e 4 
instrumentos (``Eb'',``Bb'',``Strings'' e ``D. Bass''), que usam uma 
formula de compasso $2/4$, de modo que cada compasso
é binário e
pode ser preenchido usando duas semínimas (2\quarternote).

\begin{figure}[H]
\centering
%\includegraphics[width=\textwidth]{chapters/cap-fundamentos/aquarela.png}
\begin{abc}[name=caquarela]
% abcm2ps aquarela.abc  -O aquarela.ps
% ps2epsi aquarela.ps aquarela.eps
%
X: 1 % start of header
T: Brazil - Aquarela do Brasil
C: Music: Ary Barroso, 1939
C: Arranged by: Irineu Krüger
K: C % scale: C major
M: 2/4 % formula do compasso
%
V:1 clef=treble name="Voice Choir" sname="Voice Choir"
V:2 clef=treble name="Eb" sname="Eb"
V:3 clef=treble name="Bb" sname="Bb"
V:4 clef=treble name="Strings" sname="Strings"
V:5 clef=bass   name="D. Bass" sname=""D. Bass"
%
%
[V:1] "18" C'3/2A/2C2  |"19" A3/2(G/2 G/2)E1E/2  |"20" z/2 C'1A/2 C'1C'1  |
w:    Ó Bras-sil        sam-ba_ que dá       bam-bo-leio_ 
w:    Ó Bras-sil        ver-de que dá_       pa-ra~o mun-do 
%
%
[V:2] G1z/2G1z/2G1  | G1z/2G1z/2G1  | G1z/2G1z/2G1  |
%
%
[V:3] z4  | z4  | z4  |
%
%
[V:4] G1z/2G1z/2G1  | G1z/2G1z/2G1  | G1z/2G1z/2G1  |
%
%
[V:5] C,2 G,,2  | C,1 z1 G,,2  | C,2 G,,2  |
\end{abc}
\caption{3 compassos da partitura da composição ``Aquarela do brasil''}
\label{fig:caquarela}
\end{figure}
Analisando esta partitura \cite{Irineu} e escutando a música produzida, podemos perceber facilmente
que os instrumentos executados geram um sonido
com a onomatopeia ``e Chic Chic Tum'' ou ``Chic Chic Tum'' dependendo da percepção do ouvinte. 
Assim, como pode ser escutado, o inicio de cada compasso coincide com o ``Tum''; 
sendo que este é o momento em que a maioria dos instrumentos produzem um sonido, de modo que a sensação para o 
ouvinte é de uma potencia de som maior. Cada instrumento prolongará seu sonido de
forma diferente, porem na percepção final podemos dizer que o ``Tum'' ocupa 1 
tempo (\quarternote) se percebemos ``Chic Chic Tum'' (ver ``D. Bass'') ou 3/4 de tempo se percebemos
``e Chic Chic Tum''. Imediatamente depois do ``Tum'', a 3/4 do primeiro tempo do compasso,
na parte fraca do tempo fraco, 
acontece um sonido de onomatopeia ``e'' que se prolonga ate a parte forte do tempo fraco, 
este sonido é executado pelos instrumentos ``Eb'' e ``Strings'' constituindo assim uma sincopada\cite[pp. 143]{medteoria}.
%%Esta é uma sincopa
O segundo tempo (tempo fraco) inicía com o instrumento ``D. Bass'' que produz o primeiro ``Chic''
e finalmente os instrumentos ``Eb'' e ``Strings'' executam o segundo ``Chic''
na parte fraca do tempo fraco, preenchendo o resto do tempo com silêncios, é dizer
fazem contratempos.

Pelo exposto anteriormente, agora podemos simplificar a partitura para gerar um sonido com onomatopeia
``Chic Chic Tum'', como é mostrado na Figura \ref{fig:contratempo1}.
Assim,
o instrumento 1 executa dois sonidos, de modo que o primeiro contribui ao sonido 
``Tum'' e o segundo sonido gera o segundo ``Chic'' do compasso; por outro lado,
o instrumento 2 executa um ritmo com um padrão
de repetição de dois sonidos ``Tum'' e ``Chic'', nesse ordem;
sendo que a nota executada no tempo forte produz um sonido mais agudo que a 
executada no tempo fraco.
\begin{figure}[H]
\centering
\begin{abc}[name=contratempo1]
X: 1 % start of header
K: C % scale: C major
M:2/4
%T: Contratempo num compasso binário
V:1 clef=treble name="Instrumento 1" sname="Inst. 1"
V:2 clef=bass   name="Instrumento 2" sname="Inst. 2"
[V:1] " ""T/2"G1 " ""T/2"z1 " ""T/2"z1 " ""T/2"G1 | " ""T/2"G1 " ""T/2"z1 " ""T/2"z1 " ""T/2"G1  :|
w:    Tum                     Chic                  Tum                   Chic           
[V:2] "Tempo"C,2 "Tempo"G,,2  | "Tempo"C,2 "Tempo"G,,2  :|
w:    Tum       Chic         Tum       Chic            
\end{abc}
\caption{Padrão de repetição para gerar um sonido de onomatopeia ``Chic Chic Tum''.}
\label{fig:contratempo1}
\end{figure}

Pelo exposto anteriormente é fácil perceber, como existe uma diference entre 
o que percebemos ao escutar uma música e a forma como esta é escrita na partitura;
pois como é visto na Figura \ref{fig:contratempo1}, quando escrevemos
um sonido com o padrão de repetição ``Tum Chic Chic'', para o ouvinte é mais natural associar
este sonido com o padrão ``Chic Chic Tum'', devido a que \textbf{ao falar o ser humano usa a pausa
para denotar o final de uma palavra}, da mesma forma, ao escutar uma música, traduzimos
que o sonido que tem um silencio maior apos ser executado marca o final do ciclo
do padrão de repetição. Assim, o que um músico vê ao ler uma partitura
é um padrão de repetição ``Tum Chic Chic'', sendo que  um
ouvinte interpretará de forma instintiva que o padrão é ``Chic Chic Tum ''.

Esta diferença na contagem leva a um problema, sim se quer ser rigoroso na 
forma de contar os tempos nos compassos; por exemplo, na Tabela \ref{tab:ritmo1} 
podemos ver 4 formas distintas de contar os tempos nos compassos indicando a 
distribuição de tempos, onde ``A'' representa uma porção de tempo.
\begin{table}[ht]
  \centering
  \begin{tabular}    {c|ccc|c}
    \hline
    Tipos de contagem       & A/2 & A/2   & A & Recomendável?\\
    \hline
    Contagem 1: & Chic  & Chic  & Tum   & Sim\\
    Contagem 2: & Con   & -tra  & Tempo & Não\\
    Contagem 3: & 1     & e     & 2     & Não\\
    Contagem 4: & 1     & 2     & 3     & Não\\
    \hline
  \end{tabular}
  \caption{Tipos de contagem na samba de gafieira.}
\label{tab:ritmo1}
\end{table}
A contagem 1 é a que recomendo, devido a que pode ser usada sem aprofundar demasiado 
na notação musical, de modo que só precisa ser explicado que a duração de um 
``Tum'' é o dobro que um ``Chic'', e anexado que geralmente veremos que o ``Tum''
acontece no tempo 1 da música, de modo que outra contagem valida seria ``Tum Chic Chic''; 
porem a contagem 1 não está restrita ao uso destas silabas (``Chic'' e ``Tum''), em geral esta
contagem representa a quase qualquer padrão de repetição
que use duas silabas diferentes como os padrões: ``Ta Ta Kum'', ``Tic Tic Pa'', etc. Sendo
que o padrão que não recomendo é o visto na contagem 2 (``Con-tra Tempo''), devido
a que o uso deste padrão pode confundir às pessoas que desconhecem a definição formal
do termo contratempo, e levar a confusão de achar que um contratempo é só uma 
distribuição de 3 sonidos sendo um o dobro do outros dois, em termos de tempos execução.
Outra forma que não recomendo é a contagem 3, devido a como é visto nas Figuras 
\ref{fig:caquarela} e \ref{fig:contratempo1}, é muito comum que o ``Tum'' 
inicie no tempo 1, de modo que uma correta contagem neste formato seria ``2 e 1'',
mesmo assim, esta distribuição de tempos dependerá da forma em que a partitura for
escrita, inclusive pode-se dar o caso que a partitura não tenha compassos binários 
e sim quaternários, criando assim este tipo de contagem mais caminhos onde podemos errar.
Consequentemente, e por motivos similares aos presentados para a contagem 3, 
não recomendo a contagem 4, e derivados. Em geral as contagens 3 e 4, só deveriam 
ser usadas se a pessoa está segura da forma em que a música está escrita na partitura.




