%%%%%%%%%%%%%%%%%%%%%%%%%%%%%%%%%%%%%%%%%%%%%%%%%%%%%%%%%%%%%%%%%%%%%%%%%%%%%%%%
\section{Frase}
\label{sec:Frase}
\index{Música!Frase}
Uma frase é uma unidade musical com sentido e conclusão;
é formada por um grupo de notas que nos dão impreseão de que todas pertencem a um mesmo conjunto,
e se caraterizada pela relação entre melodia, ritmo e armonia,
que termina numa \hyperref[sec:Cadencia]{\textbf{cadência}} \cite[pp. 624]{latham2008diccionario} \cite[pp. 335]{medteoria} \cite[pp. 34]{bennett1993elementos};
é dizer o final da frase pode ser percebido porque a ideia completou o sentido, 
e finaliza num repouso ou uma cadência, alem de que a frase seguinte presenta um contraste por estar expresando outra ideia.

As frases musicais se combinam para formar outras unidades mais longas e completas, 
denominadas \hyperref[sec:Periodo]{\textbf{periodos}} \cite[pp. 624]{latham2008diccionario}.
De forma geral as melodias são construidas usando frases musicais sujeitas a determinadas
regras, de forma similar a como articulamos frases na linguagem falada \cite[pp. 334]{medteoria}.


A longitude de uma frase pode variar, 
porem é comum ver que as frases tem 4 
\hyperref[sec:compaso]{\textbf{compassos}} \cite[pp. 624]{latham2008diccionario} \cite[pp. 34]{bennett1993elementos}, 
ou tambem 8 compassos  \cite[pp. 335]{medteoria} \cite[pp. 34]{bennett1993elementos},
e tende a continuar com outra frase de resposta da mesma longitude \cite[pp. 624]{latham2008diccionario}.

Na notação musical, 
o compositor pode indicar uma frase, agrupando todas as figuras musicais pertencentes a esta, 
com um símbolo de ligadura escrito acima ou abaixo das notas musicais, 
este simbolo é chamado de ligadura fraseológica \cite[pp. 49]{medteoria} 
\cite[pp. 624]{latham2008diccionario} \cite[pp. 34]{bennett1993elementos}.

A Figura \ref{ritmo:ex2frasesmusicais1} mostra um exemplo de duas frases musicais indicadas pelo uso do simbolo de ligadura.
\begin{figure}[H]
\centering
\begin{abc}[name=abc-ex2frasesmusicais1,options={-O= -c -s 1.5}]
X: 1 % start of header
K: C % scale: C major
M: 4/4 %meter - compasso
V:1 %name="Pauta com clave de fá"   sname="Pauta com clave de fá"
[V:1] (B2 A2 G2 F2| G3 A1 B2 F2) |(C'2 B2 A2 G2| F8)|
\end{abc}
\caption{Duas frases msuicais de 2 compassos cada um.}
\label{ritmo:ex2frasesmusicais1}
\end{figure}

As frases musicais tambem podem ser formadas por 2 ou 3 semifrases \cite[pp. 335]{medteoria}.
Com diferencia das frases, que tendem a ser constrastantes; 
as semifrases são identificaveis pois tendem a ser semelhantes entre sim.


%En la música contrapuntística, las frases de las diferentes voces se sobreponen,
%excepto en las cadencias más importantes  \cite[pp. 624]{latham2008diccionario}.

\subsection{Inicio da frase musical}
O inicio de um ritmo pode ter três formas \cite[pp. 147]{medteoria}:
\begin{itemize}
\item Tético
\item Anacrúsico ou protético
\item Acéfalo ou decapitado
\end{itemize}

\subsubsection{Ritmo tético}
\index{Música!Tético}
É chamado do ritmo tético se este inicia no primeiro tempo do compasso, 
é dizer no tempo forte \cite[pp. 147]{medteoria}.
A Figura \ref{ritmo:iniciotetico1} mostra um exemplo de ritmo tético.
\begin{figure}[H]
\centering
\begin{abc}[name=abc-iniciotetico1]
X: 1 % start of header
K: C % scale: C major
M: 2/4 %meter - compasso
V:1 %name="Pauta com clave de fá"   sname="Pauta com clave de fá"
[V:1] G1/2 A B1/2 B1 A1| B1 B/2 A/2 G2 |
\end{abc}
\caption{Ritmo tético.}
\label{ritmo:iniciotetico1}
\end{figure}

\subsubsection{Ritmo anacrústico ou protético}
\index{Música!Anacrústico}
\index{Música!Protético}
É chamado do ritmo anacrústico ou protético se este inicia antes 
do  tempo forte do compasso \cite[pp. 147-148]{medteoria}.
Na contagem de compassos do ritmo, se diz que este inicia no compasso 1 (primeiro compasso) \cite[pp. 147]{medteoria}.
A Figura \ref{ritmo:anacrustico1} mostra um exemplo de ritmo anacrústico.
\begin{figure}[H]
\centering
\begin{abc}[name=abc-anacrustico1,width=0.8\linewidth]
X: 1 % start of header
K: C % scale: C major
M: 2/4 %meter - compasso
V:1 %name="Pauta com clave de fá"   sname="Pauta com clave de fá"
[V:1]   G1/2 A1/2| B1 B/2 A/2 G2 |
\end{abc}
\caption{Ritmo anacrústico.}
\label{ritmo:anacrustico1}
\end{figure}
Não são grafadas as pausas antes da anacruse.
É chamado \textbf{anacruse} às figuras musicais que que estão 
antes do primeiro tempo forte do ritmo, \cite[pp. 148]{medteoria}.
No caso do exemplo da Figura \ref{ritmo:anacrustico1},
a anacruse está formada pelas duas primeiras notas, estas são ``sol~lá''.
Na contagem de compassos do ritmo, se diz que este inicia no compasso 0, 
antes do primeiro compasso \cite[pp. 148]{medteoria}.

Se dice que o ritmo é anacrústico, quando as notas no compasso inicial, 
ocupam menos da metade dele para compassos binarios e quaternarios,
e menos de dois terços para compassos ternarios \cite[pp. 149]{medteoria}.

\subsubsection{Ritmo acéfalo ou decapitado}
\index{Música!Acéfalo}
É chamado do ritmo acéfalo ou decapitado quando este inicia numa pausa;
assim, este inicia num tempo fraco \cite[pp. 149]{medteoria}.

Se dice que o ritmo e acefalo, quando as notas no compasso inicial, 
ocupam mais da metade dele para compassos binarios e quaternarios,
e mais de dois terços para compassos ternarios \cite[pp. 149]{medteoria}.

A Figura \ref{ritmo:acefalo1} mostra um exemplo de ritmo acéfalo.
\begin{figure}[H]
\centering
\begin{abc}[name=abc-acefalo1]
X: 1 % start of header
K: C % scale: C major
M: 2/4 %meter - compasso
V:1 %name="Pauta com clave de fá"   sname="Pauta com clave de fá"
[V:1] z F A G1/2 A1/2| B1 B/2 A/2 G2 |
\end{abc}
\caption{Ritmo acéfalo.}
\label{ritmo:acefalo1}
\end{figure}

\subsection{Final da frase musical}
Um ritmo pode ter dois tipos de final, 
conclusivo tambem chamado masculino e 
suspensivo tambem chamado feminino \cite[pp. 150]{medteoria}.

\subsubsection{Frases com final conclusivo}

Se diz que um ritmo tem final masculino, ou final conclusivo, 
quando este termina no tempo forte do compasso \cite[pp. 150]{medteoria}.

A Figura \ref{ritmo:conclusivo1} mostra um exemplo de frase com final conclusivo.
\begin{figure}[H]
\centering
\begin{abc}[name=abc-conclusivo1]
X: 1 % start of header
K: C % scale: C major
M: 2/4 %meter - compasso
V:1 %name="Pauta com clave de fá"   sname="Pauta com clave de fá"
[V:1] z F A G1/2 A1/2| B1 A/2 G1 A1 |F1 z1 z2|
\end{abc}
\caption{Frase com final conclusivo.}
\label{ritmo:conclusivo1}
\end{figure}


\subsubsection{Frases com final suspensivo}

Se diz que um ritmo tem final feminino, ou final suspensivo, 
quando este termina em algum tempo fraco do compasso \cite[pp. 150]{medteoria}.

A Figura \ref{ritmo:suspensivo1} mostra um exemplo de frase com final suspensivo.
\begin{figure}[H]
\centering
\begin{abc}[name=abc-suspensivo1]
X: 1 % start of header
K: C % scale: C major
M: 2/4 %meter - compasso
V:1 %name="Pauta com clave de fá"   sname="Pauta com clave de fá"
[V:1] z F A G1/2 A1/2| B1 A/2 G1 A1 |F1 A1 z2|
\end{abc}
\caption{Frase com final suspensivo.}
\label{ritmo:suspensivo1}
\end{figure}


