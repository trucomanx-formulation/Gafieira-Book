%%%%%%%%%%%%%%%%%%%%%%%%%%%%%%%%%%%%%%%%%%%%%%%%%%%%%%%%%%%%%%%%%%%%%%%%%%%%%%%%
\section{Frase}
\label{sec:Frase}
\index{Música!Frase}
Uma frase é uma unidade musical com sentido e conclusão;
é formada por um grupo de notas que nos dão impressão de que todas pertencem a um mesmo conjunto,
e se caraterizada pela relação entre melodia, ritmo e harmonia,
que termina numa \hyperref[sec:Cadencia]{\textbf{cadência}} \cite[pp. 624]{latham2008diccionario} \cite[pp. 335]{medteoria} \cite[pp. 34]{bennett1993elementos};
é dizer o final da frase pode ser percebido porque a ideia completou o sentido, 
e finaliza num repouso ou uma cadência, além de que a frase seguinte presenta um contraste por estar expressando outra ideia.

\PRLsep{Longitude e usos}

As frases musicais se combinam para formar outras unidades mais longas e completas, 
denominadas \hyperref[sec:Periodo]{\textbf{periodos}} \cite[pp. 624]{latham2008diccionario}.
De forma geral as melodias são construídas usando frases musicais sujeitas a determinadas
regras, de forma similar a como articulamos frases na linguagem falada \cite[pp. 334]{medteoria}.


A longitude de uma frase pode variar, 
porem é comum ver que as frases tem
\begin{itemize}
\item 4 \hyperref[sec:compaso]{\textbf{compassos}} \cite[pp. 624]{latham2008diccionario} \cite[pp. 34]{bennett1993elementos}, 
ou também 
\item 8 compassos  \cite[pp. 335]{medteoria} \cite[pp. 34]{bennett1993elementos};
\end{itemize}
também é comum perceber que as frases que terminam deixando uma sensação de ambiguidade, 
tendem a continuar com outra frase de resposta da mesma longitude \cite[pp. 624]{latham2008diccionario}.


\begin{tcbinformation}
\label{ref:PontoCulminanteSuperior} 
\index{Música!Ponto culminante superior}
\index{Música!PCS}
\index{Música!Clímax}
\textbf{Ponto culminante superior vs. clímax:}
Toda melodia tem uma nota com maior tom (mais aguda), que geralmente está próxima ao final da frase musical;
esta nota é denominada como ponto culminante superior \cite[pp. 336]{medteoria}.
Por outro lado alguns autores fazem uma diferencia entre,
o ponto \textbf{culminante superior} (PCS) e o clímax (Ponto culminante máximo);
onde em cada fragmento da melodia pode ser identificado um PCS,
porem o \textbf{clímax} é a nota mais alta da melodia, como um todo, 
com a característica que a nota é única e está próxima ao final \cite[pp. 12]{melos2012} \cite{HARTMANN2013} \cite[pp. 50]{holland2013music}.
\label{ref:climax}
\end{tcbinformation} 

\PRLsep{Notação}

Na notação musical, 
o compositor pode indicar uma frase, agrupando todas as figuras musicais pertencentes a esta, 
com um símbolo de ligadura escrito acima ou abaixo das notas musicais, 
este simbolo é chamado de ligadura fraseológica \cite[pp. 49]{medteoria} 
\cite[pp. 624]{latham2008diccionario} \cite[pp. 34]{bennett1993elementos}.

A Figura \ref{ritmo:ex2frasesmusicais1} mostra um exemplo de duas frases musicais indicadas pelo uso do simbolo de ligadura.
\begin{figure}[H]
\centering
\begin{abc}[name=abc-ex2frasesmusicais1,options={-O= -c -s 1.5}]
X: 1 % start of header
K: C % scale: C major
M: 4/4 %meter - compasso
V:1 %name="Pauta com clave de fá"   sname="Pauta com clave de fá"
[V:1] (B2 A2 G2 F2| G3 A1 B2 F2) |(C'2 B2 A2 G2| F8)|
\end{abc}
\caption{Duas frases msuicais de 2 compassos cada um.}
\label{ritmo:ex2frasesmusicais1}
\end{figure}

\PRLsep{Estrutura interna}

As frases musicais também podem ser formadas por 2 ou 3 semifrases \cite[pp. 335]{medteoria}.
Com diferencia das frases, que tendem a ser contrastantes; 
as semifrases são identificáveis pois tendem a ser semelhantes entre sim.

\PRLsep{Tipos de frases}

\begin{description}
\item[Frase melódica:]  é uma frase musical formada por um conjunto de tons,
este tipo de frase forma parte de uma linha melódica. 
\item[Frase rítmica:] é uma frase musical formada unicamente por uma distribuição de tempos (ritmo),
este tipo de frase pode ser extraída de um frase melódica, ou servir de base;
também encontraremos frases rítmicas na descrição do ritmo de instrumento de percussão.
\end{description}~

\PRLsep{Frases e texturas musicais}

Na música com \hyperref[subsec:polifonica]{\textbf{textura polifônica}}, 
as frases das diversas linhas melódicas, 
geralmente não finalizam (\hyperref[sec:Cadencia]{\textbf{cadenciam}}) simultaneamente;
por outro lado nas músicas com \hyperref[subsec:homofonica]{\textbf{textura homofônica}},
onde existe uma única linha melódica,
a harmonia de acompanhamento cadência simultaneamente com a melodia \cite{AFraseMelodicaDeterminantes}.

\begin{tcbattention}
Na música contrapontística, é dizer com \hyperref[subsec:polifonica]{\textbf{textura polifônica}}, 
as frases musicais das diferentes vozes se sobrepõem,
com exceção das cadências más importantes onde convergem \cite[pp. 624]{latham2008diccionario}.
\end{tcbattention}

%Na música polifônica há uma distinção maior entre Frase Melódica e Frase estrutural


%%%%%%%%%%%%%%%%%%%%%%%%%%%%%%%%%%%%%%%%%%%%%%%%%%%%%%%%%%%%%%%%%%%%%%%%%%%%%%%%
\subsection{Inicio da frase musical}
O inicio de um ritmo pode ter três formas \cite[pp. 147]{medteoria}:
\begin{itemize}
\item Tético
\item Anacrúsico ou protético
\item Acéfalo ou decapitado
\end{itemize}

\subsubsection{Ritmo tético}
\label{subsub:Tetico}
\index{Música!Tético}
É chamado do ritmo tético se este inicia no primeiro tempo do compasso, 
é dizer no tempo forte \cite[pp. 147]{medteoria}.
A Figura \ref{ritmo:iniciotetico1} mostra um exemplo de ritmo tético.
\begin{figure}[H]
\centering
\begin{abc}[name=abc-iniciotetico1]
X: 1 % start of header
K: C % scale: C major
M: 2/4 %meter - compasso
V:1 %name="Pauta com clave de fá"   sname="Pauta com clave de fá"
[V:1] G1/2 A B1/2 B1 A1| B1 B/2 A/2 G2 |
\end{abc}
\caption{Ritmo tético.}
\label{ritmo:iniciotetico1}
\end{figure}

\subsubsection{Ritmo anacrústico ou protético}
\label{subsub:anacrustica}
\index{Música!Anacrústico}
\index{Música!Protético}
É chamado do ritmo anacrústico ou protético se este inicia antes 
do  tempo forte do compasso \cite[pp. 147-148]{medteoria}.
Na contagem de compassos do ritmo, se diz que este inicia no compasso 1 (primeiro compasso) \cite[pp. 147]{medteoria}.
A Figura \ref{ritmo:anacrustico1} mostra um exemplo de ritmo anacrústico.
\begin{figure}[H]
\centering
\begin{abc}[name=abc-anacrustico1,width=0.8\linewidth]
X: 1 % start of header
K: C % scale: C major
M: 2/4 %meter - compasso
V:1 %name="Pauta com clave de fá"   sname="Pauta com clave de fá"
[V:1]   G1/2 A1/2| B1 B/2 A/2 G2 |
\end{abc}
\caption{Ritmo anacrústico.}
\label{ritmo:anacrustico1}
\end{figure}
Não são grafadas as pausas antes da anacruse.
É chamado \textbf{anacruse} às figuras musicais que que estão 
antes do primeiro tempo forte do ritmo, \cite[pp. 148]{medteoria}.
No caso do exemplo da Figura \ref{ritmo:anacrustico1},
a anacruse está formada pelas duas primeiras notas, estas são ``sol~lá''.
Na contagem de compassos do ritmo, se diz que este inicia no compasso 0, 
antes do primeiro compasso \cite[pp. 148]{medteoria}.

Se disse que o ritmo é anacrústico, quando as notas no compasso inicial, 
ocupam menos da metade dele para compassos binários e quaternários,
e menos de dois terços para compassos ternários \cite[pp. 149]{medteoria}.

\subsubsection{Ritmo acéfalo ou decapitado}
\label{subsub:Acefalo}
\index{Música!Acéfalo}
É chamado do ritmo acéfalo ou decapitado quando este inicia numa pausa;
assim, este inicia num tempo fraco \cite[pp. 149]{medteoria}.

Se disse que o ritmo e acéfalo, quando as notas no compasso inicial, 
ocupam mais da metade dele para compassos binários e quaternários,
e mais de dois terços para compassos ternários \cite[pp. 149]{medteoria}.

A Figura \ref{ritmo:acefalo1} mostra um exemplo de ritmo acéfalo.
\begin{figure}[H]
\centering
\begin{abc}[name=abc-acefalo1]
X: 1 % start of header
K: C % scale: C major
M: 2/4 %meter - compasso
V:1 %name="Pauta com clave de fá"   sname="Pauta com clave de fá"
[V:1] z F A G1/2 A1/2| B1 B/2 A/2 G2 |
\end{abc}
\caption{Ritmo acéfalo.}
\label{ritmo:acefalo1}
\end{figure}

%%%%%%%%%%%%%%%%%%%%%%%%%%%%%%%%%%%%%%%%%%%%%%%%%%%%%%%%%%%%%%%%%%%%%%%%%%%%%%%%
\subsection{Final da frase melódica seguindo o ritmo}
Um ritmo pode ter dois tipos de final, 
masculino e feminino \cite[pp. 150]{medteoria}.

\subsubsection{Frases com final masculino}
\label{subsubsec:finalmasculino}

Se diz que um ritmo tem final masculino, 
quando este termina no tempo forte do compasso \cite[pp. 150]{medteoria}.

A Figura \ref{ritmo:masculino1} mostra um exemplo de frase com final masculino.
\begin{figure}[H]
\centering
\begin{abc}[name=abc-masculino1]
X: 1 % start of header
K: C % scale: C major
M: 2/4 %meter - compasso
V:1 %name="Pauta com clave de fá"   sname="Pauta com clave de fá"
[V:1] z F A G1/2 A1/2| B1 A/2 G1 A1 |F1 z1 z2|
\end{abc}
\caption{Frase com final masculino.}
\label{ritmo:masculino1}
\end{figure}


\subsubsection{Frases com final feminino}
\label{subsubsec:finalfemenino}
Se diz que um ritmo tem final feminino, 
quando este termina em algum tempo fraco do compasso \cite[pp. 150]{medteoria}.

A Figura \ref{ritmo:femenino1} mostra um exemplo de frase com final feminino.
\begin{figure}[H]
\centering
\begin{abc}[name=abc-femenino1]
X: 1 % start of header
K: C % scale: C major
M: 2/4 %meter - compasso
V:1 %name="Pauta com clave de fá"   sname="Pauta com clave de fá"
[V:1] z F A G1/2 A1/2| B1 A/2 G1 A1 |F1 A1 z2|
\end{abc}
\caption{Frase com final femenino.}
\label{ritmo:femenino1}
\end{figure}

%%%%%%%%%%%%%%%%%%%%%%%%%%%%%%%%%%%%%%%%%%%%%%%%%%%%%%%%%%%%%%%%%%%%%%%%%%%%%%%%
\subsection{Final da frase melódica seguindo o acorde de tônica}
\label{subsec:FinalAbertoFechado}
A Tabela \ref{tab:tablefinaltipo} mostra as descrições dos resultados obtidos ao combinar,
o tipo de final rítmico da melodia com o tipo de acorde na cadencia da mesma \cite[pp. 43]{autores2017cuerpo}.

\begin{table}[!h]
  \centering
  \begin{tabular}{|l|p{3cm}|p{2.5cm}|p{3.5cm}|}
  \hline
  ~                      & \textbf{Tônica} & \textbf{Outras do acorde de tônica} & \textbf{Fora do acorde de tônica} \\ \hline \hline
  \textbf{F. masculino}  & conclusiva ou afirmativa  & inconclusa & suspensiva ou interrogativa  \\ \cline{1-2}
  \textbf{F. feminino}   & inconclusa                & ~ & ~   \\ \hline
  \end{tabular}  
  \caption{Tipos de final da frase musical seguindo a cadência}
  \label{tab:tablefinaltipo}
\end{table}

\subsubsection{Formula melódica conclusiva ou afirmativa}
É uma formula que dá a sensação de repouso absoluto,
e que tem final masculino  usando a tônica.

\subsubsection{Formula melódica suspensiva ou interrogativa}
É uma formula que dá a sensação de um descanso provisional,
finaliza numa nota que não pertence as notas do ``acorde'' de tônica. 

\subsubsection{Formula melódica inconclusa}
É uma formula que pode finalizar na tônica com final feminino, ou
em outra nota do ``acorde'' de tônica e com final masculino ou feminino.
