
\section{Modos de entender a música}
\label{sec:ModosEntenderMusica}
Podem existir vários modos de entender a música, como por exemplo o ``tonal'' e o ``modal''.
A diferencia entre estes dois paradigmas da música, 
radica na forma em que a música desenvolve sua existência no tempo \cite[pp. 155]{arbones2012armonia}.


\subsection{Música tonal}
\label{sec:MusicaTonal}
\index{Música!Música tonal}
A música tonal inicia no Barroco, aproximadamente entre os anos 1600 e 1750,
e tem a  caraterística de ser um tipo de  música que se projeta para adiante;
em cada momento da musica tonal, um acorde pelo geral conduz a outro,
pois existe um jogo de tensão e relaxação que deve ser resolvido,
a este rol dos acordes chama-se ``função de acorde'' \cite[pp. 155-156]{arbones2012armonia}.


Na música tonal, 
as notas musicais são organizadas seguindo a importância destas ao redor de uma nota  principal,
chamada ``tônica'', ou centro tonal \cite[pp. 27]{arbones2012armonia}.
Este ordem segue as regras de \hyperref[ref:paginadiatonicanumerica]{\textbf{consonância}} observadas pela escola pitagórica,
que provocam que a \hyperref[sec:pos:Diatonica]{\textbf{escala diatônica}} 
esteja formada usando frações em função de uma frequência fundamental, a tônica.

Mesmo que esta disposição de notas na música tonal, gere melodias muito consonantes,
problemas\footnote{Problemas na diminuição da consonância; é dizer o aumento da dissonância.} 
podem ser encontrados sim se desenvolvem melodias ao redor de uma nota diferente da tônica;
e dizer, longe da nota musical da qual foram calculadas as demais notas \cite[pp. 28]{arbones2012armonia},
de ali a importância da tônica na música tonal.




\subsection{Música modal}
\label{sec:MusicaAtonal}
\index{Música!Música modal}

No estilo modal o tempo tem se interpretado de dois jeitos diferentes:
\begin{itemize}
\item Como uma eternidade; como por exemplo no canto gregoriano; e dizer 
sem noção de passado, presente ou futuro.
\item Com o tempo como presente continuo, 
o acontecimento sonoro é completo em cada instante; é dizer,
 o sonido atual não esta condicionado ao passado ou ao futuro;
como por exemplo o ``bebop jazz''\cite[pp. 156]{arbones2012armonia}.
\end{itemize}



\subsection{Música atonal: dodecafonismo}
\label{sec:MusicaAtonal}
\index{Música!Música atonal}
O dodecafonismo é uma técnica de composição musical, 
criada na década de 1920 pelo compositor austríaco Arnold Schonberg \cite[pp. 121]{arbones2012armonia}\cite[pp. 263]{holst1998abc}.


A técnica dodecafônica, ou com igualitarismo tonal,
procura a ideia de uma música \textbf{atonal}; é dizer carente de centro tonal,
que se desliga de uma maior gravitação a uma nota, a tônica, sobre as demais;
O dodecafonismo desenvolve um método para evitar a preponderância de uma nota sobre outras,
de modo que todas tem o mesmo valor relativo,
e possam ser apreciadas da mesma forma numa composição musical 
\cite[pp. 122]{arbones2012armonia}.

Para evitar a centralização da peça musical numa nota especifica,
o compositor deve completar ciclos que usem as 12 notas 
(brancas e negras numa oitava nu piano). 
Apos uma nota musical ser usada, 
esta não podia volver a ser usada ate o seguinte ciclo de 12 notas \cite[pp. 123]{arbones2012armonia}.


