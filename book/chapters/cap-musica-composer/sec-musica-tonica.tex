\section{Tônica e dominante}
\index{Música!Tônica}


\subsection{Tônica}
\label{sec:Tonica}
Na \hyperref[sec:MusicaTonal]{\textbf{música tonal}} toda melodia tem uma ``tônica''; 
é dizer, uma nota mais importante \cite[pp. 19]{holst1998abc}, 
pois como é explicado na Seção \ref{sec:consonancia}, se temos uma escala diatônica,
onde as notas são afinadas buscando a \hyperref[ref:consonancia]{\textbf{consonância}},
como foi indicada pela escola pitagórica; então todas as notas são expressadas
como uma fração de uma nota principal, a tônica. 
Como uma consequência inesperada deste tipo de afinação, 
qualquer transposição de uma melodia, longe deste centro tonal, 
provocará a introdução de dissonâncias;
a este problema chama-se a ``coma pitagórica'' \cite[pp. 24]{arbones2012armonia}.
 

Do mesmo modo que o sujeito numa oração é identificável, a tônica na música tonal também é identificável.
No âmbito literário, o uso das palavras numa frase 
provoca que identifiquemos entre elas ao sujeito pelo papel que este cumpre na frase.
Acontece da mesma forma na linguagem musical; numa melodia,
a ordem de escolha (consecutiva) das notas musicais, 
provoca que a tônica seja perceptível, pois dá à melodia ou harmonia 
seu sentido musical como o centro onde se converge para dar um sentido de finalização à frase \cite[pp. 19]{holst1998abc}.

Assim a tônica é identificável, numa melodia ou harmonia, 
porque esta é geralmente  usada no final  da frase musical, 
quando se quer dar uma sensação de um final forte;
é dizer que expressa claramente a conclusão de uma ideia.
Nesse caso, a tônica comumente é usado apos uma nota que está separada um 
\hyperref[sec:intervalomelodico]{\textbf{intervalo}} de quinta;
chama-se a está nota de \hyperref[sec:dominante]{\textbf{dominante}}.

\subsection{Dominante}
\label{sec:dominante}

O intervalo entre a tônica e a quinta nota musical mais aguda 
(\hyperref[fig:abc-iquinta2]{\textbf{intervalo de quinta}} ascendente), 
na escala diatônica 
é o intervalo mais importante\footnote{Expeto se a tônica é um si, nesse caso é melhor contar 7 semitons},
pois é o intervalo \hyperref[tab:pitagorascromatica2]{\textbf{mais consonante}}, 
só superado pelo intervalo de oitava. 
Por conta disto o quinto grau dos modos da escala diatônica é chamado  de ``dominante'';
tendo esta uma forte atração com a tônica. 
Esta caraterística é um dos componentes que ajudam a manter interessante  uma melodia \cite[pp. 24]{holst1998abc}.
Na Seção \ref{sec:Cadencia}, onde é tratado o tema da \hyperref[sec:Cadencia]{\textbf{cadência}},
pode ser encontrada mais informação sobre a relação entre a tônica e a dominante.

\subsection{A tônica e as escalas modais}
Qualquer nota da escala diatônica pode ser escolhida como tônica;
assim, para cada uma destas escolhas existirá um distinto grupo de sete notas
ordenadas em relação a esta tônica. Estos grupos são chamados modos \cite[pp. 21]{holst1998abc};
a Tabela \ref{tab:modosdiatonica}, mostra o nome de todos os modos possíveis da escala diatônica
e indica com números romanos todos os intervalos possíveis,
sendo que ``I'' indica a tônica e ``V'' indica a dominante.

\begin{table}[h]
  \centering
  \begin{tabular}{|l||l|l|l|l|l|l|l|}
  \hline
  Modo      & I   & II  & III & IV  & V   & VI  & VII \\ \hline \hline
  Jônico    & dó  & ré  & mi  & fá  & sol & lá  & si  \\ \hline
  Dórico    & ré  & mi  & fá  & sol & lá  & si  & dó  \\ \hline
  Frígio    & mi  & fá  & sol & lá  & si  & dó  & ré  \\ \hline
  Lídio     & fá  & sol & lá  & si  & dó  & ré  & mi  \\ \hline
  Mixolídio & sol & lá  & si  & dó  & ré  & mi  & fá  \\ \hline
  Eólica    & lá  & si  & dó  & ré  & mi  & fá  & sol \\ \hline
  Lócrio    & si  & dó  & ré  & mi  & fá  & sol & lá  \\ \hline
  \end{tabular}
  \caption{Escalas modais e os graus.}
  \label{tab:modosdiatonica}
\end{table}

\subsection{Nome dos graus na escala diatônica}
A escala diatônica está formada por oito graus, 
sendo que cada um destes tem uma função especifica \cite[pp. 31]{cardoso1973curso};
A Tabela \ref{tab:relacaotonica} mostra todos os nomes para estes graus, 
o grau VIII não foi colocado pois é a repetição do grau I. 
\begin{table}[h]
  \centering
  \begin{tabular}{|l|l|p{7cm}|}
  \hline
  Nome           & Grau & Função\\ \hline \hline
  Tônica         & I    & Repouso natural da tonalidade.\\ \hline
  Supertônica    & II   & Grau acima da tônica.\\ \hline 
  Mediante       & III  & Grau intermediário entre a tônica e a dominante.\\ \hline 
  Subdominante   & IV   & Grau abaixo da dominante.\\ \hline 
  Dominante      & V    & Quinto grau a partir da tônica.\\ \hline 
  Superdominante & VI   & Grau acima da dominante.\\ \hline 
  Sensível       & VII  & Sétimo grau acima da tônica. \\ \hline
  \end{tabular}
  \caption{Nome dos graus nas notas musicais.}
  \label{tab:relacaotonica}
\end{table}


%%%%%%%%%%%%%%%%%%%%%%%%%%%%%%%%%%%%%%%%%%%%%%%%%%%%%%%%%%%%%%%%%%%%%%%%%%%%%%%%
%https://pt.wikipedia.org/wiki/T%C3%B4nica
