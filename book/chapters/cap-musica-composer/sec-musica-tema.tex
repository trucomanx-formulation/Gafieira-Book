\section{Tema}
\label{sec:tema}
\index{Música!Tema}

O tema ou também denominado sujeito (assunto) ou frase principal,
é o termo que se usa para designar aos passagens melódicos mais importantes de uma obra, 
que conservam um rasgo de continuidade \cite[pp. 411]{stainer2009dictionary} \cite[pp. 1496]{latham2008diccionario}.
Com diferença do termo \hyperref[sec:Motivo]{\textbf{motivo}},
o termo tema é usado para designar a \hyperref[sec:Frase]{\textbf{frases}} completas 
ou \hyperref[sec:Periodo]{\textbf{períodos}} \cite[pp. 1496]{latham2008diccionario},
por outro lado um motivo é muito mais curto e geralmente fragmentário \cite[pp. 545]{apel1969harvard},
para mais detalhes ir a Seção \ref{sec:Motivo}.

O tema é a frase principal de um movimento \cite[pp. 411]{stainer2009dictionary};
porem,  em movimentos em forma de sonata, 
devem existir  dois  temas principais sendo eles os primeiro e o segundo no movimento, e
tem o maior peso estrutural da mesma \cite[pp. 411]{stainer2009dictionary} \cite[pp. 1496]{latham2008diccionario}.

Existem composições politemáticas e monotemáticas;
é dizer com vários temas ou um tema, respectivamente; 
como já vimos, entre os exemplos de composições politemáticas,
temos as sonatas; e como exemplo para composições monotemáticas temos as fugas \cite[pp. 539]{apel1969harvard}.

\begin{example}
Na cultura moderna atual, os temas podem ser mais facilmente reconhecidos  na música cinematográfica.
Por exemplo, para muitos de nós, ficaram marcados os temas dos filmes: ``Superman'', ``Indiana Jones'',
``Star Wars'', ``O grande chefão (The Godfather)'', ``Harry Potter'', etc.
Em algumas ocasiões, os temas podem corresponder aos protagonistas, 
e em outros casos os temas podem fazer referencia a algum elemento na trama.
Porem em todos os casos estão bem gravados na nossa mente, 
de modo que ao escutá-los nos sentimos irreversivelmente obrigados a lembrar seu filme de origem.
\end{example}
