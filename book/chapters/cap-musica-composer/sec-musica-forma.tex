


\section{A forma musical}
\label{sec:FormaMusical}
\index{Música!Forma musical}

\begin{tcbinformation} 
\textbf{Seção:}
\index{Música!Seção}
\label{ref:Secao}
No sentido mais amplo, uma seção é uma divisão curta de uma composição;
sendo as seções limitadas rítmica e harmonicamente para ser entes diferenciados \cite[pp. 174]{baker1895dictionary}.
As seções geralmente tem forma de frases, períodos, sentenças, etc. 
\end{tcbinformation} 

A ``forma'' musical é o jeito em que os compositores organizam as \hyperref[ref:Secao]{\textbf{seções}} numa peça musical.
Cada seção se representa mediante uma letra maiúscula (ex: $A$, $B$, $C$, $D$, etc.) \cite[pp. 71]{bennett1993elementos};
assim, escrever $ABCBA$, indica que a peça musical tem as seções $A$, $B$ e $C$,
e que são executadas seguindo a ordem $A$, $B$, $C$, $B$ e finalmente $A$.


Os compositores utilizam sobre as seções os seguintes princípios estruturais:
\begin{description}
\item[Repetição:] Se refere a repetição das seções (ex: $AAA...A$) 
\cite[pp. 71]{bennett1993elementos} \cite[pp. 88]{howard1991aprendendo} \cite[pp. 53]{colluraimprovisacao} 
\cite[pp. 85]{holland2013music}.
A repetição é relativa à altura das notas, 
pois a letra que acompanha pode variar drasticamente entre seções.
 
\item[Variação :] Indica a repetição com variação das seções, 
 com o fim de manter o interesse na música
\cite[pp. 71]{bennett1993elementos} \cite[pp. 88]{howard1991aprendendo} \cite[pp. 53]{colluraimprovisacao}.
Geralmente se indica com $A'$ a variação de A (ex: $AA'A''$).

\item[Contraste:] O contraste é outro jeito de manter o interesse do 
público, onde as seções são  marcadamente diferenciadas e com ideias novas
\cite[pp. 71]{bennett1993elementos} \cite[pp. 88]{howard1991aprendendo} 
\cite[pp. 53]{colluraimprovisacao}  \cite[pp. 85]{holland2013music}.
Se $A$ é uma seção que contrasta com $B$, poderíamos escrever: $AB$, ou $ABA$.
\begin{example}
Se a estrofe é $A$ e o coro é $B$ então:
\begin{itemize} 
\item estrofe: caráter tranquilo, registro grave, um cantante, acordes menores, etc.
\item coro: caráter enérgico, registro agudo, vários cantantes, acordes maiores, etc.
\end{itemize}
\end{example}
\end{description} 

Entre as formas mais conhecidas de estruturar as seções, temos:
A \hyperref[subsec:formabinaria]{\textbf{forma binária}}, 
a \hyperref[subsec:formaternaria]{\textbf{forma ternária}}, 
a \hyperref[subsec:formarondo]{\textbf{forma rondó}}, 
a \hyperref[subsec:formacancao]{\textbf{forma canção}}, 
a \hyperref[subsec:formaabac]{\textbf{forma $\mathbf{ABAC}$}}, 
etc.
 



\subsection{Forma binária: $AB$}
\label{subsec:formabinaria}
\index{Música!Forma binária}
Uma peça musical em forma binária está constituída por duas seções 
com um peso equivalente; 
a estas seções as designaremos com as letras $A$ e $B$, 
onde $B$ pode ser maior ou igual que $A$ em duração; juntos
formam a estrutura $AB$;
ambas seções pode ter sinal de repetição; se é assim, então é possivel ver estruturas como 
\cite[pp. 71]{bennett1993elementos} \cite[pp. 93]{copland1974ouvir}:
\begin{itemize}
\item $AB$,
\item $AAB$ (é dizer $||:A:||B$) ou 
\item $AABB$ (é dizer $||:A:||:B:||$).
\end{itemize}
Depois das partes A e B, 
pode haver uma \hyperref[ref:Coda]{\textbf{coda}} 
ou uma seção final mais longa \cite[pp. 86-87]{holland2013music}.

\begin{example} ~
\begin{itemize}
\item ``Mamãe eu quero'' de Vicente Paiva e Jararaca,
interpretado por Carmen Miranda no filme ``Down Argentine Way'' de 1940. 
Tem seções de 8 compassos,
com o seguinte formato:
Introdução, $AB$, $||:A:||B$, $||:A:||^3$, final.
\end{itemize}
\end{example}




\subsection{Forma ternaria: $ABA$}
\label{subsec:formaternaria}
\index{Música!Forma ternária}
Uma peça musical em forma ternária está constituída por três seções 
estruturadas na ordem $ABA$; 
onde a seção $A$ se repete no final,
porem na segunda vez que se repete $A$,
modificações podem ser introduzidas \cite[pp. 71]{bennett1993elementos} \cite[pp. 88]{holland2013music}.


Seguindo  Aaron Copland, autor do livro ``Como ouvir e entender música'' \cite[pp. 89]{copland1974ouvir},
a forma ternaria se mantem, mesmo que seção $A$ seja repetida;
então é possivel ver estruturas como: 
\begin{itemize}
\item $ABA$ ou
\item $AABA$ (é dizer $||:A:||BA$).
\end{itemize}

\begin{example} ~
\begin{itemize}
\item ``Garota de Ipanema'' de Antônio Carlos Jobim e interpretado por Gal Costa,
com a seções de 8 compassos binários \cite{partituragarotaipanema1} \cite{partituragarotaipanema2}.
Tem uma estrutura: Introdução, $||:A:||:B:||A$, solo,  $||:A:||:B:||A$, final.
Com a introdução, e final de 8 compassos binários, 
e um solo com 5 seçoes de 8 compassos binários.
\end{itemize}
\end{example}

\subsubsection{Forma canção: $AABA$}
\label{subsec:formacancao}
\index{Música!Forma canção}
Uma peça musical em forma canção está constituída por duas seções, $A$ e $B$,
seguindo uma estrutura $AABA$; 
onde a seção $A$ se repete sempre variando um pouco suas carateristicas;
é comum ver seções de 8 compassos de duração, mas podem existir casos diferentes
\cite[pp. 53]{colluraimprovisacao} \cite[pp. 16]{adolfo1997composicao}.
\begin{example} ~
\begin{itemize}
\item ``Samba de uma nota só'' de Antônio Carlos Jobim \cite[pp. 53]{colluraimprovisacao} \cite[pp. 16]{adolfo1997composicao},
com seções de 8 compassos binários \cite{partiturasambadeumanotaso1}.
\end{itemize}
\end{example}

\subsection{Forma rondó: $ABACA$}
\label{subsec:formarondo}
\index{Música!Forma rondó}
Uma peça na forma rondó está composta por seções, 
chamadas aqui episódios, onde estas são rondadas pela seção inicial;
é dizer temos estruturas como $ABACA$, onde os episódios $B$ e $C$,
estão sendo rondados por $A$;
Cada vez que a seção $A$ se repete, 
pequenas modificações podem ser feitas \cite[pp. 72]{bennett1993elementos}.
Então é possivel ver estruturas como: 
\begin{itemize}
\item $ABACA$,
\item $ABACADA$ \cite[pp. 98]{copland1974ouvir},
\item $AABBACCA$ (é dizer $||:A:||:B:||A||:C:||A$).
\end{itemize}

\subsubsection{Forma $AABBACCA$: Choro}
\label{subsec:formachoro}
\index{Música!Forma do choro}
Uma peça musical na forma $AABBACCA$, é a forma típica do choro;
onde as seções $B$ e $C$ são contruidos nas regiões da dominante ou da subdominante 
\cite[pp. 83]{colluraimprovisacao}.
Esta forma é uma forma rondó escrita como  $||:A:||:B:||A||:C:||A$ \cite[pp. 53]{diniz2003almanaque}.
\begin{example} ~
\begin{itemize}
\item ``Espinha de bacalhau''  de Severino Araújo \cite[pp. 83]{colluraimprovisacao},
com seções de 16 compassos.
\item ``Cuidado, colega''  de Ernesto Nazareth \cite[pp. 83]{colluraimprovisacao},
com seções de 16 compassos.
\item ``Odeon'' de Ernesto Nazareth. Tem seções de 16 compassos binários.
\end{itemize}
\end{example}



\subsection{Forma $ABAC$}
\label{subsec:formaabac}
\index{Música!Forma $ABAC$}
Uma peça musical nesta forma está constituída por três seções, $A$, $B$ e $C$,
seguindo uma estrutura $ABAC$; 
é comum ver seções de 8 compassos de duração, mas podem existir casos diferentes
\cite[pp. 53]{colluraimprovisacao}.
\begin{example} ~
\begin{itemize}
\item ``Corcovado''  de Antônio Carlos Jobim \cite[pp. 53]{colluraimprovisacao},
com a seções de 8 compassos binários \cite{partituracorcovado1}.
\item ``Samba de verão'' de Marcos e Paulo Sérgio Valle  \cite[pp. 53]{colluraimprovisacao},
com a seções de 8 compassos quaternários.
\end{itemize}
\end{example}


%%%%%%%%%%%%%%%%%%%%%%%%%%%%%%%%%%%%%%%%%%%%%%%%%%%%%%%%%%%%%%%%%%%%%%
%\cite[pp. 289]{duckworth2007creative}
