%%%%%%%%%%%%%%%%%%%%%%%%%%%%%%%%%%%%%%%%%%%%%%%%%%%%%%%%%%%%%%%%%%%%%%%%%%%%%%%%
\section{Articulação}
\label{sub:Articulation}
\index{Música!Articulação}

Nas partituras podemos ver alguns símbolos, 
que o compositor coloca como indicação ao interprete,
para informar como as notas musicais devem ser executadas ou 
articuladas entre sim \cite[pp. 56]{alves2004teoria}.
%%%%%%%%%%%%%%%%%%%%%%%%%%%%%%%%%%%%%%%%%%%%%%%%%%%%%%%%%%%%%%%%%%%%%%%%%%%%%%%%
\subsection{Legato }
\label{subsec:Legato}
\index{Música!Legato}
O  ``legato'' é um símbolo  que indica uma ligadura de expressão entre as notas,
neste caso a informação que da ao compositor é que as
notas devem ser executadas sem interrupções e
criando uma mudança de tons gradual para passar de uma nota musical a outra \cite[pp. 56]{alves2004teoria} \cite[pp. 18]{holland2013music}.

\begin{example}
A Figura \ref{fig:legato1} mostra um exemplo de uso do legato. 
Alguns instrumentos podem facilmente articular um legato, por exemplo o violino.
\end{example}

\begin{figure}[h!]
\centering
\begin{abc}[name=abc-legato1,width=0.80\linewidth]
X: 1 % start of header
K: C % scale: C major
M: 2/4 %meter - compasso
 (G2 E2 | G1  A1  G1 E1 )|
\end{abc}
\caption{Melodia com notas que devem ser executadas de forma ligada.}
\label{fig:legato1}
\end{figure}

%%%%%%%%%%%%%%%%%%%%%%%%%%%%%%%%%%%%%%%%%%%%%%%%%%%%%%%%%%%%%%%%%%%%%%%%%%%%%%%%
\subsection{Staccato}
\label{subsec:Staccato}
\index{Música!Staccato}

O staccato é um símbolo, desenhado com um ponto (.), 
que indica a diminuição na \hyperref[sec:pos:Duracion]{\textbf{duração}} de uma nota (aproximadamente um 50\%), 
dando nela um efeito de separação ou destaque \cite[pp. 56]{alves2004teoria} \cite[pp. 16]{holland2013music}.

\begin{example}
A Figura \ref{fig:staccato1a} mostra um exemplo de uso do staccato. 
Na Figura \ref{fig:staccato1b} podemos ver uma escrita equivalente, sem o uso do símbolo de staccato.
\end{example}

\begin{figure}[h!]
\centering
\begin{subfigure}[c]{0.80\textwidth}
\begin{abc}[name=abc-staccato1a]
X: 1 % start of header
K: C % scale: C major
M: 2/4 %meter - compasso
 .G2 .E2 | .G1  .A1  .G1 .E1 | 
\end{abc}
\caption{Notação de notas musicais em staccato.}
\label{fig:staccato1a}
\end{subfigure}
~ %
\begin{subfigure}[c]{1.00\textwidth}
\begin{abc}[name=abc-staccato1b]
X: 1 % start of header
K: C % scale: C major
M: 2/4 %meter - compasso
 G1 z1 E1 z1 | G1/2 z1/2 A1/2 z1/2 G1/2 z1/2 E1/2 z1/2 | 
\end{abc}
\caption{Forma de execução de notas musicais em staccato.}
\label{fig:staccato1b}
\end{subfigure}
\caption{Melodia com notas que devem ser executadas em staccato.}
\label{fig:staccato1}
\end{figure}

\subsubsection{Staccatissimo}

O staccatissimo, staccato seco ou martelado é um símbolo, com uma função similar ao staccato;
porém indica uma diminuição maior na \hyperref[sec:pos:Duracion]{\textbf{duração}} 
de nota (aproximadamente ao 25\%) \cite[pp. 56]{alves2004teoria} \cite[pp. 16]{holland2013music}.

\begin{example}
A Figura \ref{fig:staccatissimo1a} mostra um exemplo de uso do staccatissimo. 
Na Figura \ref{fig:staccatissimo1b} podemos ver uma escrita equivalente, sem o uso do símbolo de staccatissimo.
\end{example}

\begin{figure}[h!]
\centering
\begin{subfigure}[c]{0.80\textwidth}
\begin{abc}[name=abc-staccatissimo1a]
X: 1 % start of header
K: C % scale: C major
M: 2/4 %meter - compasso
 !wedge!G2 !wedge!E2 | !wedge!G1  !wedge!A1  !wedge!G1 !wedge!E1 | 
\end{abc}
\caption{Notação de notas musicais em staccatissimo.}
\label{fig:staccatissimo1a}
\end{subfigure}
~ %
\begin{subfigure}[c]{1.00\textwidth}
\begin{abc}[name=abc-staccatissimo1b]
X: 1 % start of header
K: C % scale: C major
M: 2/4 %meter - compasso
 G1/2 z3/2 E1/2 z3/2 | G1/4 z3/4 A1/4 z3/4 G1/4 z3/4 E1/4 z3/4 | 
\end{abc}
\caption{Forma de execução de notas musicais em staccatissimo.}
\label{fig:staccatissimo1b}
\end{subfigure}
\caption{Melodia com notas que devem ser executadas em staccato.}
\label{fig:staccatissimo1}
\end{figure}

%%%%%%%%%%%%%%%%%%%%%%%%%%%%%%%%%%%%%%%%%%%%%%%%%%%%%%%%%%%%%%%%%%%%%%%%%%%%%%%%
\subsection{Tenuto}
\label{subsec:Tenuto}
\index{Música!Tenuto}

O tenuto (ou sostenuto) é um símbolo desenhado com uma linha reta (-)  
que indica que deve ser sustentada a 
\hyperref[sec:pos:Duracion]{\textbf{duração}} e a 
\hyperref[sec:pos:Intensidade]{\textbf{intensidade}} da nota ao máximo \cite[pp. 56]{alves2004teoria} \cite[pp. 17]{holland2013music}.

\begin{example}
A Figura \ref{fig:tenuto1} mostra um exemplo de uso do tenuto. 
\end{example}


\begin{figure}[h!]
\centering
\begin{abc}[name=abc-tenuto1,width=0.80\linewidth]
X: 1 % start of header
K: C % scale: C major
M: 2/4 %meter - compasso
 !tenuto!G2 !tenuto!E2 | !tenuto!G1  !tenuto!A1  !tenuto!G1 !tenuto!E1 |
\end{abc}
\caption{Melodia com notas que devem ser executadas de forma sustenido.}
\label{fig:tenuto1}
\end{figure}

%%%%%%%%%%%%%%%%%%%%%%%%%%%%%%%%%%%%%%%%%%%%%%%%%%%%%%%%%%%%%%%%%%%%%%%%%%%%%%%%
\subsection{Accénto}
\label{subsec:Accento}
\index{Música!Accénto}

O accénto é um símbolo desenhado com (>) 
que indica que a nota deve ser acentuada; 
é dizer, que a nota deve receber um aumento de \hyperref[sec:pos:Intensidade]{\textbf{intensidade}} \cite[pp. 56]{alves2004teoria}.

\begin{example}
A Figura \ref{fig:accento1} mostra um exemplo de uso do accénto.
Nela os tempos fracos tem um aumento de intensidade provocando \hyperref[sec:contratempo]{\textbf{contratempos}}. 
\end{example}


\begin{figure}[h!]
\centering
\begin{abc}[name=abc-accento1,width=0.80\linewidth]
X: 1 % start of header
K: C % scale: C major
M: 2/4 %meter - compasso
 G2 !>!E2 | G1  !>!A1  !>!G1 !>!E1 |
\end{abc}
\caption{Melodia com notas que devem ser executadas de forma sostenida.}
\label{fig:accento1}
\end{figure}
