\chapterimage{chapter_head_2.pdf} % Chapter heading image

\chapter{Normas gerais na dança}\index{Normas gerais}
%Normas tem um porque, regras sao impostas.
Neste capítulo veremos um conjunto de normas, e a explicação de como o cumprimento
ou não destas afetam ao desenvolvimento estético e técnico da dança. Serão usados
neste capítulo termos como "cavalheiro" e "dama", para designar ao "condutor" e ao "conduzido",
porem isto não implica nenhuma restrição de gênero para o uso de algum destes roles na dança,
só são um recurso literário para o melhor entendimento das explicações mostradas aqui.
\begin{lattention}
É importante aclarar
que as normas expostas neste capítulo não estão regulamentadas por nenhuma entidade; assim, estas
refletem, o meu aprendizado de distintos professores,
interpretações pessoais  e deduções. 
\end{lattention}

%%%%%%%%%%%%%%%%%%%%%%%%%%%%%%%%%%%%%%%%%%%%%%%%%%%%%%%%%%%%%%%%%%%%%%%%%%%%%%%%
\section{Normas gerais na dança de salão}
As normas que veremos aqui abrangem um conjunto amplo de estilos de 
dança, e não devem ser
tratadas como leis, e sim como diretrizes a serem usadas como padrão de inicialização, de modo que 
o dançarino ou dançarina analisará cada caso e se necessário criará uma exceção e atuará conforme ela.
A continuação são listadas as normas gerais na dança de salão.\\

\begin{itemize}
\item \textbf{Rodar o salão:} Que os dançarinos executem sua dança girando o salão, 
é importante para criar a ilusão de que o espaço de dança é maior, dado que mesmo
tendo uma pista de dança lotada, o espaço que um
casal deixa ao se movimentar é ocupado pelo casal que vem atrás deles, criando 
assim um fluxo de movimento circular que permite a todos os casais usar a pista de dança
na sua totalidade (por convenção o sentido de giro é sempre anti-horário); 
visto o anterior é importante ressaltar que né todas as danças tem
uma evolução circular na pista de dança, pois existem estilos de dança que são dançados em linha,
como por exemplo a "Salsa em linha" ou o "West coast swing", ou também existem estilos que
tem um comportamento hibrido entre circular e linha como o "Zouk". Neste sentido,
a "samba de gafieira"  tem um comportamento circular e se deveria dançar
rodando o salão para um boa etiqueta na pista de dança.
\item \textbf{Conduzir e ser conduzidos:} Trabalhar num paradigma da dança baseado
na condução é muito importante para as danças sociais, dado que isto implicará
que o cavalheiro  que sabe conduzir poderá dançar fluidamente com uma dama com quem nunca dançou
antes (se esta for conduzível). O mesmo acontecerá para as damas que desenvolvam
a sensibilidade necessária para serem conduzíveis, elas poderão dançar com qualquer
cavalheiro, inclusive poderão ter um desenvolvimento básico em estilos de dança pouco ou não conhecidos.
O caso contrario à condução, é ter um estilo de dança baseado em coreografias,
isto ate certo ponto pode ser visto como um vicio que geralmente aparece quando iniciamos
na dança. Isto acontece, no caso do cavalheiro, quando este assume que se realiza a parte do movimento 
que lhe corresponde (na parte visual) sem enviar nenhuma informação à dama, 
esta tem que reconhecer/adivinhar o movimento, e realizar a parte dela. Este enfoque
funciona bem quando ambos tem treinado antecipadamente os movimentos, e/ou conhecem a sequencia
em que estes movimentos serão executados, porem falha quando os dançarinos não se conhecem;
comprovar isto é fácil se imaginamos por exemplo o caso em que o cavalheiro executa um movimento
que tem a parte inicial muito parecida a outro movimento, neste caso, se a dama não tiver
um poder telepático confundirá um movimento com o outro e acontecerá um problema de comunicação. Assim, a coreografia
na dança deve estar reservada para apresentações, onde o casal volta 
a sua atenção para a encenação da peça, e não para detalhes mais mecânicos.

\item \textbf{Ter o peso do corpo sobre um pé só:} 
Na dança de salão é muito importante a leveza e o auto controle mostrado na 
execução dos movimentos; assim, para conseguir isto, é importante que os dançarinos
tenham o peso do corpo num pé só ao terminar a execução de cada movimento; o motivo
é facilmente percebido se fazemos um pequeno exercício. Ao ficar em pé separamos as
pernas uma distancia igual à de nosso quadril, e nesse momento levamos o peso do corpo
(concentrado no nosso centro de gravidade que nesse casso está perto do umbigo) a
apontar a um ponto médio entre nossos pés, nesse instante estamos dividindo o peso do corpo
entre nossos pontos de apoio, 50$\%$ no pé direito e 50$\%$ no pé esquerdo; agora, mantendo o peso
do corpo nesse lugar, tentaremos levantar qualquer de nossos pés, será evidente
que esse trabalho é muito difícil sem perder o equilíbrio, pois para mantê-lo
precisamos de ambos pontos de apoio; casos similares podem ser vistos com qualquer proporção de distribuição de peso,
por exemplo, 30$\%$ e 70$\%$ ou 20$\%$ e 80$\%$. Assim o único caso em que estamos
perfeitamente equilibrados e podemos executar nossos movimento e levantar um pé 
mantendo a postura e auto controle, é quando
temos o 100$\%$ do peso do corpo num pé só, o pé de apoio. 
Por outro lado se o cavalheiro é  o agente desequilibrante da dama, por exemplo no caso
de uma condução aplicada pelo cavalheiro;
esta norma tem um valor agregado, pois ao ter a dama o peso do corpo num pé, 
fica mais fácil para o condutor orientar
à dama a fazer o seguinte movimento, dado que o único pé que ela pode mover é o pé
que está livre, e que é o pé que o cavalheiro precisa que se movimente, 
além de que a força necessária pelo cavalheiro para tirar a dama do seu equilíbrio 
atual é muito menor ao caso quando ela tem dois pontos de apoio, a dama 
a sua vez tem um pé livre para se resguardar do desequilíbrio provocado pelo 
cavalheiro e adquirir um novo equilíbrio com esse pé.

\item \textbf{Ter os braços firmes:} Seguindo a ideia da condução, esta só pode
ser realizada se existe um médio de comunicação onde possa ser transmitida
o comando do cavalheiro e a resposta da dama.

sem fazer força (a informação chega por lá).
\end{itemize}

%%%%%%%%%%%%%%%%%%%%%%%%%%%%%%%%%%%%%%%%%%%%%%%%%%%%%%%%%%%%%%%%%%%%%%%%%%%%%%%%
\section{Normas gerais na samba de gafieira}


\begin{itemize}
\item \textbf{Quadril avança, ombros e pé acompanham}  O movimento inicia no quadril.
\item \textbf{Quando movimentar um pé chegar com 100$\%$ do peso do corpo} estética da samba de gafieira.
\item \textbf{Braços firmes da dama} estes transportam a informação para o quadril (ex: picadilho).
\item \textbf{Abraço uniforme de cavalheiro} este procura manter a mesma distancia (ex: gancho redondo).
\item \textbf{O tórax do cavalheiro conduz a dama} O tórax do cavalheiro conduz a dama (NÃO o braço).
\item \textbf{Procurar o paralelismo de ombros e linha de visão do casal} é responsabilidade da dama seguir ao cavalheiro, indução

\end{itemize}
