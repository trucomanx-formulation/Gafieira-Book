\chapterimage{chapter_head_2.pdf} % Chapter heading image

\chapter{\textcolor{green}{Regras na dança de salão}}

\index{Regra} 
\begin{definition}[Regra:]
Norma, lei, costume que dirige, orienta e regula procedimentos (regra gramatical; regras de etiqueta; regras na dança).
[http://www.aulete.com.br/regra]
\end{definition}
\index{Condução} 
\begin{definition}[Paradigma da condução (Dança):] Este é usado na dança de salão, 
e implica que na dança a dois existe uma transmissão de informação
entre as pessoas que conformam o casal, 
de modo que a informação/comando tem um fluxo unidirecional no médio de transmissão,
que vá desde o condutor ate o seguidor. 
\end{definition}
\index{Condutor} 
\begin{definition}[Condutor (Dança):] 
A pessoa que tem o papel de conduzir e propor o movimento. 
O objetivo técnico das pessoas que optam por este rol é chegar 
a ter o sensibilidade necessária para entender onde está localizado espacialmente, 
e como distribui o peso do corpo, o seguidor; 
de modo que seja possível para o condutor aplicar uma mistura de forças e torções sobre o seguidor para provocar o movimento desejado;
chama-se a isto saber conduzir.
Sinônimos de condutor: Líder.
\end{definition}
\index{Seguidor} 
\begin{definition}[Seguidor (Dança):] 
A pessoa que recebe a condução e proporciona uma resposta corporal. 
O objetivo técnico das pessoas que optam por este rol é chegar 
a ter a sensibilidade necessária para entender as conduções,
independentemente de quem seja a pessoa que aplique a condução;
chama-se a isto ser conduzível.
Sinônimos de seguidor: Conduzido.
\end{definition}
\index{Dança social} 
\begin{definition}[Dança social:]
É uma dança com fim recreativo de prática social, não cênica, nem competitiva, 
que não tem um interesse artístico, histórico, geográfico ou técnico; 
que se universaliza e consiste em movimentos dos corpos de um casal, 
onde existe a figura condutor e o seguidor (papeis intercambiáveis) \cite{Zamoner2012} ate onde o nível técnico do par o permita.
\end{definition}
\index{Dança de salão} 
\begin{definition}[Dança de salão:]
É uma arte que procura conservar suas características técnicas, 
sua origem histórica e geográfica, e se universaliza em práticas sociais. 
Esta arte consiste na interpretação improvisada da música por médio dos movimentos 
dos corpos de um casal, utilizando um paradigma da dança onde se tem um condutor e um seguidor (papeis intercambiáveis) \cite{Zamoner2012}.
\end{definition}

Neste capítulo veremos um conjunto de regras, e a explicação de como o cumprimento
ou não destas afetam ao desenvolvimento estético e técnico da dança de 
salão\footnote{Ou na dança social sim se interessa levar estas ideias a esse âmbito.}  no paradigma da condução.

Assim, serão usados neste capítulo termos como "condutor" e "seguidor", 
mas isto não implica nenhuma obrigatoriedade ou restrição, das pessoas, para o uso de algum destes papéis na dança;
porem é comum ver que o papel de condutor é escolhido tradicionalmente pelos "cavalheiros" e o papel de seguidor pelas "damas".
%só são um recurso literário para o melhor entendimento das explicações mostradas aqui.
\begin{lattention}
É importante aclarar
que as regras expostas neste capítulo não estão regulamentadas por nenhuma entidade ou intituição; assim, estas
refletem, o meu aprendizado de distintos professores,
interpretações pessoais  e deduções. 
\end{lattention}

%%%%%%%%%%%%%%%%%%%%%%%%%%%%%%%%%%%%%%%%%%%%%%%%%%%%%%%%%%%%%%%%%%%%%%%%%%%%%%%%
\section{\textcolor{green}{Regras gerais na dança de salão}}
As regras que veremos aqui abrangem um conjunto amplo de estilos de 
dança, e não devem ser
tratadas como leis, e sim como diretrizes a serem usadas como padrão de inicialização, de modo que 
o dançarino ou dançarina analisará cada caso e se necessário criará uma exceção e atuará conforme ela.
A continuação são listadas algumas regras gerais na dança de salão.\\

\begin{itemize}
\item \textbf{Rodar o salão:} Que os dançarinos executem sua dança girando o salão, 
é importante para criar a ilusão de que o espaço de dança é maior, dado que mesmo
tendo uma pista de dança lotada, o espaço que um
casal deixa ao se movimentar é ocupado pelo casal que vem atrás deles, criando 
assim um fluxo de movimento circular que permite a todos os casais usar a pista de dança
na sua totalidade (por convenção o sentido de giro é sempre anti-horário); 
visto o anterior é importante ressaltar que né todas as danças tem
uma evolução circular na pista de dança, pois existem estilos de dança que são dançados em linha,
como por exemplo a "Salsa em linha" ou o "West coast swing", ou também existem estilos que
tem um comportamento hibrido entre circular e linha como o "Zouk". Neste sentido,
a "samba de gafieira"  tem um comportamento circular e se deveria dançar
rodando o salão para um boa etiqueta na pista de dança.
\item \textbf{Conduzir e ser conduzidos:} Trabalhar num paradigma da dança baseado
na condução é muito importante para as danças de salão, dado que isto implicará
que um condutor habilidoso poderá dançar fluidamente com pessoas com quem nunca dançou
antes (se esta for conduzível). O mesmo acontecerá para as pessoas que desenvolvam
a sensibilidade necessária para serem conduzíveis, elas poderão dançar com qualquer
condutor, inclusive poderão ter um desenvolvimento básico em estilos de dança pouco ou não conhecidos.
O caso contrario ao paradigma da condução, é ter um estilo de dança baseado em coreografias;
este enfoque, dependendo da finalidade, pode ser visto como um vicio que geralmente aparece quando iniciamos
na dança. Isto acontece, no caso do condutor, quando este assume que se realiza a parte do movimento 
que lhe corresponde (na parte visual) sem enviar nenhuma informação ao seguidor, 
este tem que reconhecer/adivinhar o movimento, e realizar a parte que lhe corresponda. Este enfoque
funciona bem quando ambos tem treinado antecipadamente os movimentos, e/ou conhecem a sequencia
em que estes movimentos serão executados, porem falha quando os dançarinos não se conhecem;
comprovar isto é fácil se imaginamos por exemplo o caso em que o condutor executa um movimento
que tem a parte inicial muito parecida a outro movimento, neste caso, se o seguidor não tiver
um poder telepático confundirá um movimento com o outro e acontecerá um problema de comunicação. Assim, a coreografia
na dança deve estar reservada para apresentações, onde o casal volta 
a sua atenção para a encenação da peça, e não para detalhes mais mecânicos.

\item \textbf{Ter o peso do corpo sobre um pé só no final da proposta de movimento:} 
Na dança de salão é muito importante a leveza e o auto controle mostrado na 
execução dos movimentos; assim, para conseguir isto, é importante que os dançarinos
tenham o peso do corpo num pé só ao terminar a execução de cada movimento; o motivo
é facilmente percebido se fazemos um pequeno exercício. Ao ficar em pé separamos as
pernas uma distancia igual à de nosso quadril, e nesse momento levamos o peso do corpo
(concentrado no nosso centro de gravidade que nesse casso está perto do umbigo) a
apontar a um ponto médio entre nossos pés, nesse instante estamos dividindo o peso do corpo
entre nossos pontos de apoio, 50$\%$ no pé direito e 50$\%$ no pé esquerdo; agora, mantendo o peso
do corpo nesse lugar, tentaremos levantar qualquer de nossos pés, será evidente
que esse trabalho é muito difícil sem perder o equilíbrio, pois para mantê-lo
precisamos de ambos pontos de apoio; casos similares podem ser vistos com qualquer proporção de distribuição de peso,
por exemplo, 30$\%$ e 70$\%$ ou 20$\%$ e 80$\%$. Assim o único caso em que estamos
equilibrados e podemos executar nosso movimento e levantar um pé 
mantendo a postura e auto controle, é quando
temos o 100$\%$ do peso do corpo num pé só, o pé de apoio. 
Por outro lado, quando consideramos ao condutor como o agente desequilibrante do seguidor, por exemplo no caso
em que este aplica uma condução;
a regra de manter o peso do corpo num pé só, tem um valor agregado; 
pois fica mais fácil para o condutor orientar
ao seguidor a fazer o seguinte movimento, dado que o único pé que este pode mover é o pé
que está livre, e que é o pé que o condutor precisa que se movimente, 
além de que a força necessária pelo condutor para tirar ao seguidor do seu equilíbrio 
atual é muito menor ao caso quando o seguidor tem dois pontos de apoio.
Adicionalmente o seguidor tem um pé livre para se resguardar do desequilíbrio provocado pelo 
condutor e adquirir um novo equilíbrio com esse pé.

\item \textbf{Ter uma abraço de dança firme:} Seguindo a ideia da condução, esta só pode
ser realizada se existe um médio de comunicação, onde possa ser transmitido
o comando do condutor ao seguidor. Assim, um bom abraço garante este
fluxo de informação.
\end{itemize}

%%%%%%%%%%%%%%%%%%%%%%%%%%%%%%%%%%%%%%%%%%%%%%%%%%%%%%%%%%%%%%%%%%%%%%%%%%%%%%%%
\section{\textcolor{blue}{Regras gerais na samba de gafieira}}


\begin{itemize}
\item \textbf{Quadril avança, ombros e pé acompanham}  O movimento inicia no quadril.
\item \textbf{Quando movimentar um pé chegar com 100$\%$ do peso do corpo} estética da samba de gafieira.
\item \textbf{Braços firmes do seguidor} estes transportam a informação para o quadril (ex: picadilho).
\item \textbf{Abraço uniforme do condutor} este procura manter a mesma distancia (ex: gancho redondo).
\item \textbf{O tórax do condutor guia ao seguidor} O tórax do condutor guia ao seguidor (NÃO o braço).
\item \textbf{Procurar o paralelismo de ombros e linha de visão do casal} é responsabilidade do seguidor seguir ao condutor, indução.
\item \textbf{O pé de apoio debe estar apontando a um ponto médio entre os pês do par de dança} .
\end{itemize}
