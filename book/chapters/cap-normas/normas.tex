\chapterimage{chapter_head_2.pdf} % Chapter heading image

\chapter{\textcolor{green}{Regras na dança de salão}}

 
\begin{definition}[Regra:] 
\index{Regra}
\label{def:Regra}
Princípio que serve como padrão no estudo das artes e ciências \cite{priberamregra} \cite{dicioregra}.
Exemplo: regras gramaticais; regras de etiqueta; regras do jogo; regras na dança.
\end{definition}

\begin{definition}[Casal:] 
\index{Casal}
\label{def:Casal}
Par formado por um macho e uma fêmea. \cite{priberamcasal}.
Exemplo: casal de cavalos, casal de pombas, casal de humanos.
\end{definition}

\begin{definition}[Par:] 
\index{Par}
\label{def:Par}
Igual, semelhante, parceiro.
Cada uma das pessoas que constituem uma dupla na dança \cite{priberampar}.
\end{definition}

\begin{definition}[Paradigma da condução (Dança):] 
\index{Condução} 
\label{def:ParadigmaConducao} 
Este é um modelo de dança a dois usado na \hyperref[def:DancaSalao]{\textbf{dança de salão}},
\hyperref[def:DancaSocial]{\textbf{dança social}}, etc. 
E indica que entre as pessoas que conformam a dança, 
existe uma transmissão de informação relativa à movimentação do \hyperref[def:Par]{\textbf{par}}; 
de modo que, a informação tem um fluxo unidirecional no médio de transmissão,
que vá desde o \hyperref[def:Condutor]{\textbf{condutor}} ate o \hyperref[def:Seguidor]{\textbf{seguidor}}. 
\end{definition}
\index{Condutor} 

\begin{definition}[Condutor (Dança):] 
\index{Condutor} 
\label{def:Condutor} 
A pessoa que tem o papel de conduzir ou propor o movimento ao \hyperref[def:Seguidor]{\textbf{seguidor}}. 
O objetivo técnico das pessoas que optam por este rol na dança, é chegar 
a ter a sensibilidade necessária para entender onde está localizado espacialmente, 
e como distribui o peso do corpo, o \hyperref[def:Seguidor]{\textbf{seguidor}}; 
de modo que seja possível para o condutor aplicar sobre o \hyperref[def:Seguidor]{\textbf{seguidor}}, 
uma mistura de indicações, forças e torções,  
para provocar o movimento desejado;
chama-se a isto saber conduzir.
Sinônimos de condutor: Líder (Dança).
\end{definition}

\begin{definition}[Seguidor (Dança):] 
\index{Seguidor} 
\label{def:Seguidor} 
A pessoa que recebe a condução e proporciona uma resposta corporal. 
O objetivo técnico das pessoas que optam por este rol na dança, é chegar 
a ter a sensibilidade necessária para entender as conduções,
independentemente de quem seja a pessoa que aplique a condução;
chama-se a isto ser conduzível.
Sinônimos de seguidor: Conduzido (Dança).
\end{definition}

\begin{definition}[Dança social:]
\index{Dança social} 
\label{def:DancaSocial} 
É uma dança com fim recreativo de prática social, não cênica, nem competitiva, 
que não tem um interesse artístico, histórico, geográfico ou técnico; 
que se universaliza e consiste na movimentação dos corpos do \hyperref[def:Par]{\textbf{par}} de dança, 
onde existe rol do \hyperref[def:Condutor]{\textbf{condutor}} e do \hyperref[def:Seguidor]{\textbf{seguidor}} (papeis intercambiáveis no par) \cite{Zamoner2012}, ate onde o nível técnico do par o permita.
\end{definition}

\begin{definition}[Dança de salão:]
\index{Dança de salão}
\label{def:DancaSalao}  
É uma arte que procura conservar suas características técnicas, 
sua origem histórica e geográfica, e se universaliza em práticas sociais. 
Esta arte consiste na interpretação improvisada da música por médio dos movimentos 
dos corpos de um \hyperref[def:Par]{\textbf{par}} de dança, 
utilizando o \hyperref[def:ParadigmaConducao]{\textbf{paradigma da condução}} \cite{Zamoner2012}.
\end{definition}

Neste capítulo veremos um conjunto de \hyperref[def:Regra]{\textbf{regras}}, 
e a explicação de como o cumprimento ou não destas, 
afetam ao desenvolvimento estético e técnico da \hyperref[def:DancaSalao]{\textbf{dança de 
salão}}\footnote{Ou na \hyperref[def:DancaSocial]{\textbf{dança social}} sim se está interessado em levar estas ideias a esse âmbito.}  no \hyperref[def:ParadigmaConducao]{\textbf{paradigma da condução}}.

As \hyperref[def:Regra]{\textbf{regras}} que veremos neste capítulo não devem ser
tratadas como leis, e sim como diretrizes a serem usadas como padrão de inicialização, de modo que 
o dançarino ou dançarina analisará cada caso e se necessário criará uma exceção e atuará conforme ela.

Serão usados neste capítulo termos como \hyperref[def:Condutor]{\textbf{condutor}} e \hyperref[def:Seguidor]{\textbf{seguidor}}; 
mas, não existe nenhuma obrigatoriedade ou restrição para as pessoas, 
na escolha de algum destes papéis na dança.
Porem, é comum ver que o papel de condutor é escolhido tradicionalmente pelos homens e o papel de seguidor pelas mulheres.
%só são um recurso literário para o melhor entendimento das explicações mostradas aqui.
\begin{lattention}
É importante aclarar
que as regras expostas neste capítulo não estão regulamentadas por nenhuma entidade ou instituição; assim, estas
refletem, o meu aprendizado de distintos professores,
interpretações pessoais  e deduções. 
\end{lattention}

%%%%%%%%%%%%%%%%%%%%%%%%%%%%%%%%%%%%%%%%%%%%%%%%%%%%%%%%%%%%%%%%%%%%%%%%%%%%%%%%
\section{Regras gerais na dança de salão}


A continuação são listadas algumas regras gerais na dança de salão, 
que abrangem um conjunto amplo de estilos de dança.\\

\begin{description}

\item[Rodar o salão:] Que os dançarinos executem sua dança girando o salão, 
é importante para criar a ilusão de que o espaço de dança é maior, dado que mesmo
tendo uma pista de dança lotada, ao rodar o salão, o espaço que um
\hyperref[def:Par]{\textbf{par}} deixa ao se movimentar é ocupado pelo \hyperref[def:Par]{\textbf{par}} que vem atrás deles, criando 
assim um fluxo de movimento circular que permite a todos os casais usar a pista de dança
na sua totalidade (por convenção o sentido de giro é sempre anti-horário); 
visto o anterior é importante ressaltar que né todas as danças tem
uma evolução circular na pista de dança, pois existem estilos de dança que são dançados em linha,
como por exemplo a ``Salsa em linha'' ou o ``West coast swing'', ou também existem estilos que
tem um comportamento hibrido entre circular e linha como o "Zouk". Neste sentido,
a ``samba de gafieira'' tem um comportamento circular e se deveria dançar
rodando o salão para um boa etiqueta na pista de dança.


\item[Conduzir e ser conduzidos:] Trabalhar num \hyperref[def:ParadigmaConducao]{\textbf{paradigma da dança baseado
na condução}} é muito importante para na \hyperref[def:DancaSalao]{\textbf{dança de salão}}, dado que isto implicará
que um \hyperref[def:Condutor]{\textbf{condutor}} habilidoso poderá dançar fluidamente com pessoas com quem nunca dançou
antes (se esta for conduzível). De forma similar acontecerá para as pessoas que desenvolvam
a sensibilidade necessária para serem conduzíveis, elas poderão dançar com qualquer
\hyperref[def:Condutor]{\textbf{condutor}}, inclusive poderão ter um desenvolvimento básico em estilos de dança pouco ou não conhecidos.

O caso oposto ao \hyperref[def:ParadigmaConducao]{\textbf{paradigma da condução}}, é ter um estilo de dança baseado em coreografias;
este enfoque, dependendo da finalidade, pode ser visto como um vicio que geralmente aparece quando iniciamos
na dança. Isto acontece quando, a pessoa que deve conduzir, assume que se realiza a parte do movimento 
que lhe corresponde, sem enviar nenhuma informação ao \hyperref[def:Par]{\textbf{par}}, 
então o \hyperref[def:Seguidor]{\textbf{seguidor}} deve reconhecer/adivinhar o movimento, 
e realizar a parte que lhe corresponde. 
Este enfoque funciona bem quando ambos tem treinado antecipadamente os movimentos, 
e/ou conhecem a sequencia em que estes movimentos serão executados, 
porem falha quando os dançarinos não se conhecem;
comprovar isto é fácil se imaginamos por exemplo o caso em que o \hyperref[def:Condutor]{\textbf{condutor}} executa um movimento
que tem a parte inicial muito parecida a outro movimento, nesse caso, se o \hyperref[def:Seguidor]{\textbf{seguidor}} não tiver
um poder telepático confundirá um movimento com o outro e acontecerá um problema de comunicação. Assim, a coreografia
na dança deve estar reservada para apresentações, onde o \hyperref[def:Par]{\textbf{par}} volta 
a sua atenção para a encenação da peça, e não para detalhes mais mecânicos.

\item[Peso do corpo definido num pé:]
Ter o peso total do corpo bem definido sobre um pé, 
ao final de cada ação ou proposta no movimento,
quando sabemos que realizaremos mais movimentos imediatamente depois,
garante uma velocidade no tempo de reação para a seguinte ação ou movimento. 
Na dança de salão é muito importante a leveza e o auto controle mostrado na 
execução dos movimentos; assim, para conseguir isto, é importante que os dançarinos
tenham o peso do corpo num pé só ao terminar a execução de cada movimento; o motivo
é facilmente percebido se fazemos um pequeno exercício. 

Ao ficar em pé separamos as
pernas uma distancia igual à de nosso quadril, e nesse momento levamos o peso do corpo\footnote{
Em física podemos representar um corpo como um objeto com a masa concentrado no nosso centro de gravidade que no casso do ser humano está perto do umbigo.} a
apontar a um ponto médio entre nossos pés, nesse instante estamos dividindo o peso do corpo
entre nossos dois pontos de apoio, 50$\%$ no pé direito e 50$\%$ no pé esquerdo; agora, mantendo o peso
do corpo nesse lugar, tentaremos levantar qualquer de nossos pés, será evidente
que esse trabalho é muito difícil sem perder o equilíbrio, pois para mantê-lo
precisamos de ambos pontos de apoio; casos similares podem ser vistos com qualquer proporção de distribuição de peso,
por exemplo, 30$\%$ e 70$\%$ ou 20$\%$ e 80$\%$. Assim o único caso em que estamos
equilibrados e podemos executar nosso movimento e levantar um pé 
mantendo a postura e auto controle, é quando
temos o 100$\%$ do peso do corpo num pé só, o pé de apoio.
 
Por outro lado, quando consideramos ao \hyperref[def:Condutor]{\textbf{condutor}} como o agente desequilibrante do \hyperref[def:Seguidor]{\textbf{seguidor}}, por exemplo no caso
em que este aplica uma condução;
a regra de manter o peso do corpo num pé só, tem um valor agregado; 
pois fica mais fácil para o condutor orientar
ao seguidor a fazer o seguinte movimento, dado que o único pé que o seguidor pode mover é o pé
que está livre, e que é o pé que o condutor precisa que se movimente, 
além de que a força necessária pelo condutor para tirar ao seguidor do seu equilíbrio 
atual é muito menor ao caso quando o seguidor tem dois pontos de apoio.
Adicionalmente o seguidor tem um pé livre para se resguardar do desequilíbrio provocado pelo 
condutor e adquirir um novo equilíbrio com esse pé.

\textbf{Nunca podemos dividir o peso do corpo?} Esta caraterística é possível sim,
se nossa dança fosse um relato escrito, o peso de nosso corpo deve estar bem definido num pé,
ao final de cada palavra, em cada virgula e ponto e virgula; por outra lado, 
o peso do corpo pode estar dividido em cada ponto.

\item[Ter um bom abraço de dança:] Seguindo a ideia da condução, esta só pode
ser realizada se existe um médio de comunicação, onde possa ser transmitido
o comando do \hyperref[def:Condutor]{\textbf{condutor}} ao \hyperref[def:Seguidor]{\textbf{seguidor}}. 
Assim, um bom abraço garante este fluxo de informação entre o \hyperref[def:Par]{\textbf{par}} na dança. 
A forma exata do abraço varia ligeiramente entre os diferentes estilos de dança;
mas em todos os casos, 
o que se procura é ter a maior quantidade de pontos de contato no \hyperref[def:Par]{\textbf{par}},
pois quanto mais pontos de contato tenhamos, 
maior será a fidelidade com que a informação da condução chegue ao \hyperref[def:Seguidor]{\textbf{seguidor}}.
Outro ponto importante do abraço é a firmeza dos braços, 
pois é a traves deles que passa a maior parte da informação.
Ter os braços firmes não implica fazer força pra submeter ao par,
e sim ativar os músculos o mínimo e necessário para manter a posição e postura de braços.
Assim no caso do seguidor, qualquer informação que chegue pelos braços,
não afetará diretamente a posição e postura dos braços, 
se não que a informação passará quase sem afetar eles e se transmitirá maioritariamente ao corpo,
mudando assim este de estado ou posição. 

\end{description}

%%%%%%%%%%%%%%%%%%%%%%%%%%%%%%%%%%%%%%%%%%%%%%%%%%%%%%%%%%%%%%%%%%%%%%%%%%%%%%%%
\section{\textcolor{blue}{Regras gerais na samba de gafieira}}


\begin{description}
\item[Quadril avança, ombros e pé acompanham]  O movimento inicia no quadril.
\item[Quando movimentar um pé chegar com 100$\%$ do peso do corpo] estética da samba de gafieira.
\item[Braços firmes do seguidor] estes transportam a informação para o quadril, manter posição de articulações de cotovelo e ombro (ex: picadilho).
\item[Abraço uniforme do condutor] este procura manter a mesma distancia (ex: gancho redondo).
\item[O tórax] do \hyperref[def:Condutor]{\textbf{condutor}} guia ao \hyperref[def:Seguidor]{\textbf{seguidor}} O tórax do \hyperref[def:Condutor]{\textbf{condutor}} guia ao \hyperref[def:Seguidor]{\textbf{seguidor}} (NÃO o braço).
\item[Procurar o paralelismo de ombros e linha de visão] do \hyperref[def:Par]{\textbf{par}} é responsabilidade do \hyperref[def:Seguidor]{\textbf{seguidor}} seguir ao \hyperref[def:Condutor]{\textbf{condutor}}, indução.
\item[O pé de apoio] debe estar apontando a um ponto médio entre os pês do par de dança .
\end{description}
