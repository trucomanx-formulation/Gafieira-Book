%%%%%%%%%%%%%%%%%%%%%%%%%%%%%%%%%%%%%%%%%%%%%%%%%%%%%%%%%%%%%%%%%%%%%%%%%%%%%%%%
%% CAPITULO
%%%%%%%%%%%%%%%%%%%%%%%%%%%%%%%%%%%%%%%%%%%%%%%%%%%%%%%%%%%%%%%%%%%%%%%%%%%%%%%%
\chapterimage{chapter_head_2.pdf} % Chapter heading image

\chapter{Introdução}

Antes de iniciar as explicações da escrita e leitura da partitura coreográfica,
é necessário mencionar primeiro um paralelismo com outras formas de escritura,
como a forma escrita da música.

Quando um músico  cria una melodia ou música, este a expressa como um conjunto varições nas durações e as alturas dos sons;
 essa riqueza na interpretação nesses dois âmbitos, tempo e frequência,
pode ser escrita mediante uma partitura de  música; 
mas esta partitura sempre será uma representação inexata.  
Isto é assim, 
porque qualquer tentativa de digitalizar uma informação, 
que está intrinsecamente feita  para ser representada num formato analógico (sonido como onda mecânica),
implica necessariamente uma perda de informação.
Isto não é a priori ruim, 
pois se a versão digitalizada contem a sustância do mensagem que se deseja transmitir,
esta tem cumprido seu propósito.
Nesse sentido, uma partitura de música cumpre muito bem esse objetivo;
pois expressa de maneira eficiente as transições entre frequências e distribuições de tempos na música;
mas, como falamos anteriormente é uma representação inexata, com perda de informação.
Assim, a forma em que distintos músicos interpretam uma mesma partitura pode variar, 
pela nuanças que estes interpretes incluem, 
e o próprio erro que eles introduzem quando leem este tipo de formatos digitalizados.

Outra forma de abordar o problema de salvar de forma escrita (digitalizada) a música, 
é mediante o uso de arquivos em formato eletrônico,
A diferença dos formatos digitais como a partitura de música,
que é facilmente legível por um ser humano, as formas de escrita em formato eletrônico, 
como os arquivos mp3 o wav, são de difícil leitura por um ser humano;
mas tem a vantagem de preservar com uma maior fidelidade a informação da peça musical que se quer salvar.
Nesse sentido nos achamos frente a dois paradigmas diferentes:
\begin{itemize}
\item Um em que presentamos uma forma de escrita digital, que o ser humano pode ler facilmente 
e reproduzir a música com um aceitável nível de fidelidade, e 
\item Um formato eletrônico, em que o ser humano dificilmente pode ler e executar em tempo real a informação;
em contrapartida, a leitura nesse formato eletrônico permite que a informação reproduzida, 
tenha maior fidelidade com a informação original (analógica). 
\end{itemize}

Como é evidente para todas as pessoas que tem usado estes dois  formatos,
ambas aproximações ao problema são corretas,
a diferença radica no uso que damos  a ambos formatos escritos.
Um é mais apropriado para o estudo da musica, a leitura e a interpretação,  
e o outro para a reprodução.
