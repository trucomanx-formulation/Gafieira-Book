\section{Licenças Creative Commons}
% https://pt.wikipedia.org/wiki/Licen%C3%A7a_livre
O ``Creative Commons'' nos ajuda a compartilhar legalmente nosso conhecimento e 
criatividade, com o alvo de construir um mundo mais justo, acessível e inovador;
o ``Creative Commons'' oferece licenças de direitos autorais gratuitas e fáceis de usar;
 criando um jeito simples e padronizado, 
de dar ao público a permissão de compartilhar e usar nosso trabalho criativo \cite{creativecommonsabout}.


\begin{tcbinformation}{Licença livre}
\index{Licença livre}
\label{ref:licensalivre}
é qualquer licença que garanta ao receptor de uma obra protegida por direito autoral, 
as liberdades de utilizar  o trabalho de outro criador, 
dando-lhes quatro grandes liberdades.
\begin{itemize}
\item A liberdade de usar o trabalho e aproveitar os benefícios de usá-lo.
\item A liberdade de estudar o trabalho e aplicar o conhecimento adquirido.
\item A liberdade de fazer e redistribuir cópias, como um todo ou em parte, da informação ou expressão.
\item A liberdade de fazer mudanças e melhorias, e de distribuir as obras derivadas.
\end{itemize}
\end{tcbinformation} 

\subsection{Creative Commons como licenças livres}
\label{subsec:CCBYSA}
Algumas licenças ``Creative Commons'' estão categorizadas como \hyperref[ref:licensalivre]{\textbf{licenças livres}}
\cite{licensaculturalivre}.


\begin{table}[h]
\centering
\begin{tabular}{|p{1.0cm}||p{3.5cm}|p{9cm}|}
\hline
~ & Nome & Descrição  \\ \hline
\hline
\raisebox{-\totalheight}{\includegraphics[width=1cm]{copyright/Cc-by_new.eps}} & 
Atribuição (BY) & 
Os licenciados têm o direito de copiar, distribuir, 
exibir e executar a obra e fazer trabalhos derivados dela, 
conquanto que deem créditos devidos ao autor ou licenciador, 
na maneira especificada por estes \cite{creativecommons}. \\ \hline
\raisebox{-\totalheight}{\includegraphics[width=1cm]{copyright/Cc-sa.eps}} & 
CompartilhaIgual (SA)  & 
Os licenciados devem distribuir obras derivadas somente sob uma licença idêntica 
à que governa a obra original \cite{creativecommons}. \\ \hline
\end{tabular}
\caption{Símbolos de licenças livres}
\label{tab:licensa-livre}
\end{table}

\vspace{-20pt}
\subsection{Creative Commons como licenças não livres}

Algumas licenças ``Creative Commons'' não são categorizadas como \hyperref[ref:licensalivre]{\textbf{licenças livres}}, 
pois estas não garantem uma ou mais liberdades. 
Os seguintes atributos que podem ser incorporados na escolha de uma 
licença ``Creative Commons'' violam algumas liberdades e não podem ser incluídos em obras designadas como livres 
\cite{licensaculturalivre}.

\begin{table}[h]
\centering
\begin{tabular}{|p{1.0cm}||p{3.5cm}|p{9cm}|}%% *,3.5,8
\hline
~ & Nome & Descrição  \\ \hline
\hline
\raisebox{-\totalheight}{\includegraphics[width=1cm]{copyright/Cc-nc.eps}} & 
NãoComercial (NC)  & 
Os licenciados podem copiar, distribuir, 
exibir e executar a obra e fazer trabalhos derivados dela, 
desde que sejam para fins não-comerciais \cite{creativecommons}. \\ \hline

\raisebox{-\totalheight}{\includegraphics[width=1cm]{copyright/Cc-nd.eps}} & 
SemDerivações (ND)  & 
Os licenciados podem copiar, distribuir, 
exibir e executar apenas cópias exatas da obra, 
não podendo criar derivações da mesma \cite{creativecommons}.  \\ \hline

\end{tabular}
\caption{Símbolos de licenças não livres}
\label{tab:licensa-noa-livre}
\end{table}
