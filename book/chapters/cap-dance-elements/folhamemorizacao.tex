\newpage
\thispagestyle{plain}
\begin{center}\TITULOA{Folha de memorização de passos}\end{center}
\label{pos:page:folhamemorizacao}

\begin{center}\Large\textit{ Esta folha é uma ferramenta simples de memorização e registro dos passos aprendidos.}\end{center}

\TITULOB{Nome do passo aprendido}
\noindent Digite no espaço o nome do passo que você aprendeu.

\BOXDATA\;

\TITULOB{Passo a passo} 
\noindent Descreva abaixo (com riqueza de detalhes) todo o passo a passo da
figura que você aprendeu. Descreva cada ação feita para realizar o passo.
É importante que sua anotação seja possível de ser entendida por outra 
pessoa além de você, portanto não economize nos detalhes do passo.

\LINEDATA\;
\LINEDATA\;
\LINEDATA\;
\LINEDATA\;
\LINEDATA

\TITULOB{Suas conexões}	
\noindent Escreva aqui que conexões (imagens cotidianas, histórias etc.)
você pode fazer para te ajudar a lembrar do passo que você acabou de aprender.
O que é presente no seu cotidiano que se você fizesse uma associação ao passo
te ajudaria a lembrar dele?

\LINEDATA\;
\LINEDATA\;
\LINEDATA

\TITULOB{Ensine ao amigo imaginário}
\noindent Para fechar o processo de aprendizado, ensine o passo que você aprendeu.
O ato de ensinar não só reforça o seu aprendizado como também te ajuda a
compreensão do passo, haja vista que para ensinar é necessário domínio do
que está sendo ensinado. Portanto, ensine o passo aprendido para o amigo 
imaginário.\\

\FOOTPAGE\;
