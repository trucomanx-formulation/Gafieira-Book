
%%%%%%%%%%%%%%%%%%%%%%%%%%%%%%%%%%%%%%%%%%%%%%%%%%%%%%%%%%%%%%%%%%%%%%%%%%%%%%%%
%% SECTION
%%%%%%%%%%%%%%%%%%%%%%%%%%%%%%%%%%%%%%%%%%%%%%%%%%%%%%%%%%%%%%%%%%%%%%%%%%%%%%%%
\section{\textcolor{red}{Historia da samba de gafieira}}\index{Historia da samba de gafieira}


O aparecimento do samba, 
foi um grande impacto para as pessoas que frequentavam estes já existosos lugares de dança;
sendo considerado um ritmo novo e ligeiro,
que desagradou aos bailarinos de maior idade e menos ágeis \cite[pp. 3]{entrevistajuliojournalbrasil1}.
O senhor, Júlio Simões chegou a temer pelo futuro da ``Kananga do Japão'' e
do ``Elite Club''; porem, para sorte dele, 
o samba fez muito sucesso no Elite,
e passou a ser considerado matéria indispensável para qualquer pessoa que pretendesse ser bailarino, 
compositor ou instrumentista \cite[pp. 3]{entrevistajuliojournalbrasil1}.

\textcolor{red}{Historia
\begin{description}
\item [A] Perna, Marco Antonio (2001). Samba de Gafieira - a história da dança de salão brasileira. ISBN 85-901965-5-0
\end{description}
}
