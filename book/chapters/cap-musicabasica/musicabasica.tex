\chapterimage{chapter_head_2.pdf} % Chapter heading image

\chapter{Fundamentos de notação musical}
Nas seguintes sub seções abordaremos os conceptos de notação musical,
num sentido rítmico (distribuição de tempos) e não explicaremos o significado 
melódico (distribuição de frequências) das 
figuras musicais na partitura, devido a que as explicações mostradas aqui estão
orientadas para um público interessado na dança, que precisa numa primeira 
aproximação à música, conhecer rapidamente a parte rítmica dela. Para aprofundar mais na parte 
melódica recomendamos acudir a materiais, livros ou revistas especializadas \cite{medteoria}
\cite{azevedocompor} \cite{alves2004teoria} \cite{mascarenhascurso} \cite{adolfo2002musica} \cite{grabner2001teoria}.

%%%%%%%%%%%%%%%%%%%%%%%%%%%%%%%%%%%%%%%%%%%%%%%%%%%%%%%%%%%%%%%%%%%%%%%%%%%%%%%%
\section{Figuras musicais e pausas}
Tabela \ref{tab:abc-noteslengthbasic}

\begin{table}[h]
\centering
\begin{tabular}{|c|c|c|c|}
\hline
Figura       & Duração & Nome\\ \hline
\hline
\Ganz        & $S$   & Semibreve \\ \hline
\Halb        & $S/2$ & Mínima \\ \hline
\Vier        & $S/4$ & Semínima \\ \hline
\Acht        & $S/8$ & Colcheia \\ \hline
\Sech        & $S/16$ & Semicolcheia \\ \hline
\end{tabular}
\caption{Duração e símbolos de algumas figuras musicais}
\label{tab:abc-noteslengthbasic}
\end{table}

Tabela \ref{tab:silencelengthbasic}
\begin{table}[h]
\centering
\begin{tabular}{|c|c|c|c|}
\hline
Figura       & Duração & Nome\\ \hline
\hline
\GaPa        & $S$   & Semibreve \\ \hline
\HaPa        & $S/2$ & Mínima \\ \hline
\ViPa        & $S/4$ & Semínima \\ \hline
\AcPa        & $S/8$ & Colcheia \\ \hline
\SePa        & $S/16$ & Semicolcheia \\ \hline
\end{tabular}
\caption{Duração e símbolos de algumas silêncios musicais}
\label{tab:silencelengthbasic}
\end{table}

Figura \ref{fig:abc-figuraspausas}
\begin{figure}[H]
\centering
\begin{abc}[name=abc-figuraspausas]
% abcm2ps figuraspausas.abc  -O figuraspausas.ps
% ps2epsi figuraspausas.ps figuraspausas.eps
%
X: 1 % start of header
K: C % scale: C major
M:31/16
%T: Contratempo num compasso binário
V:1 clef=treble name="Figuras"   sname="Figuras"
V:2 clef=treble name="Silencios" sname="Silencios"
%
[V:1] "S"G8  "S/2"G4  "S/4"G2  "S/8"G1 "S/16"G/2           |
w:    Semibreve Mínima Semínima Colcheia Semicolcheia        
%
[V:2] "S"z8 "S/2"z4 "S/4"z2  "S/8"z1  "S/16"z/2   |
%       
\end{abc}
\caption{Figuras e pausas}
\label{fig:abc-figuraspausas}
\end{figure}

\begin{equation}
S+\frac{S}{2}+\frac{S}{4}+\frac{S}{8}+\frac{S}{16} =\frac{31S}{16}
\end{equation}



%%%%%%%%%%%%%%%%%%%%%%%%%%%%%%%%%%%%%%%%%%%%%%%%%%%%%%%%%%%%%%%%%%%%%%%%%%%%%%%%
\section{Pauta}

\subsection{Pauta ou Pentagrama}
\subsection{Pauta de percussão}


Figura \ref{fig:abc-monolinearperc}
\begin{figure}[H]
\centering
\begin{abc}[name=abc-monolinearperc]
% abcm2ps monolinearperc.abc  -O monolinearperc.ps
% ps2epsi monolinearperc.ps monolinearperc.eps
%
X: 1 % start of header
K: none stafflines=1 %K: C % scale: C major
M: 2/2 %meter - compasso
"Primeiro compasso"  G4 z2 G2 |"Segundo compasso" G4 A2 B1 B1  |
w:  Acentuado   N.A. Acentuado N.A. N.A. N.A.
\end{abc}
\caption{Figuras e pausas}
\label{fig:abc-monolinearperc}
\end{figure}





%%%%%%%%%%%%%%%%%%%%%%%%%%%%%%%%%%%%%%%%%%%%%%%%%%%%%%%%%%%%%%%%%%%%%%%%%%%%%%%%
\section{Compasso}
\label{sec:compaso}

O dicionário de Harvard de música \cite{randel2003harvard} define compasso (``measure'' em inglês)
como: ``uma unidade de tempo musical que consiste em um número fixo de valores de notas de um tempo dado, 
conforme determinada métrica, e uma notação musical delimitada por duas linhas de barra.'', 
também indica sobre a métrica  (``meter'' em inglês) que é
``O padrão de unidade temporal fixa, chamado batimentos, pelo qual o tempo de duração 
de um pedaço de música ou de uma seção dela é medida.'' e agrega que
``a métrica de uma obra ou uma porção dela é indicado geralmente por uma fração'', como por exemplo:
${2}/{2}$ , ${3}/{4}$ , ${4}/{4}$, etc; em português esta fração é chamada de formula do compasso. 

O numerador, da formula do compasso, indica o número de pulsações (tempos) que compõem cada compasso.
Por outro lado o denominador nos informa o valor temporal de cada um dos tempos do compasso.
A Tabela \ref{tab:abc-noteslength} exemplifica o significado do denominador da formula do compasso; 
\begin{table}[h]
\centering
\begin{tabular}{|c|c|c|c|}
\hline
denominador & Figura  & Duração & Nome\\ \hline
\hline
$1$   & \fullnote    & $S$   & Semibreve \\ \hline
$2$ & \halfnote    & $S/2$ & Mínima \\ \hline
$4$ & \quarternote & $S/4$ & Semínima \\ \hline
$8$ & \eighthnote  & $S/8$ & Colcheia \\ \hline
\end{tabular}
\caption{Duração e símbolos de algumas figuras musicais}
\label{tab:abc-noteslength}
\end{table}
onde a primeira coluna mostra o denominador da formula,
a segunda coluna mostra as figuras musicais que representam cada um dos tempos do compasso, e 
a terceira e quarta coluna, indicam a duração em segundos e o nome da figura musical.

podemos achar equivalências aos exemplos da formula do compasso dados
anteriormente; onde os compassos com formula $\mathbf{2}/2$ tem cada um, uma duração de $\mathbf{2}$\halfnote ~(duas mínimas),  
compassos com formula $\mathbf{3}/4$ tem uma duração de $\mathbf{3}$\quarternote ~(trés semínimas) 
e $\mathbf{4}/4$ uma duração de $\mathbf{4}$\quarternote ~(quatro semínimas). É importante
ressaltar que a duração em tempo das figuras musicais é relativa, como pode ser visto
na terceira coluna da Tabela \ref{tab:abc-noteslength}, onde as durações estão em função
da duração $S$ da semibreve. 


Se classificamos os compassos por sua métrica, os três tipos mais conhecidos 
são os compassos binários, ternários, quaternários \cite[pp. 27]{adolfo2002musica}.

\subsection{Um compasso binário} Ou compasso binário simples,
é uma estrutura rítmica que se carateriza por ter compassos com uma  duração de dois tempos,
sendo o primeiro tempo forte (acentuado), e o segundo de tempo fraco (não acentuado)
\cite[pp. 41]{grabner2001teoria} \cite[pp. 66]{adolfo2002musica}\cite[pp. 28]{alves2004teoria}. 
Os compassos binários (simples) tem uma formula de compasso na forma $2/B$,
onde $B$ pode ser $2$, $4$, $8$, etc. 
A Figura \ref{compasso:binario}, representa um exemplo de compasso binário simples, 
com formula de compasso $2/2 \equiv 2$\halfnote, 
e tempos com uma duração de $S/2$ (uma \halfnote), 
sendo que o primeiro compasso contem $2$\halfnote~e o segundo contem $4$\quarternote.
Na Figura \ref{compasso:binario} a sigla ``N.A.'' significa ``Não acentuado'', pelo que é fácil perceber
que em qualquer caso, só a nota que é executada no tempo 1 é acentuada.
\begin{figure}[H]
\centering
\begin{abc}[name=abc-compasso1]
X: 1 % start of header
K: C % scale: C major
M: 2/2 %meter - compasso
"Primeiro compasso" G4 F4 |"Segundo compasso" G2 D2 F2 D2  |
w: Acentuado N.A. Acentuado N.A. N.A. N.A.
\end{abc}
\caption{Exemplo de compasso binário (simples)}
\label{compasso:binario}
\end{figure}

Se falamos de forma mais geral, 
podemos ter dois tipos de compassos binários: os simples e os compostos.
Assim, 
para achar a formula de um compasso composto, correspondente a um compasso simples (usando quialteras de três)
usamos a seguinte operação \cite[pp. 74]{alves2004teoria}, 
\begin{equation}\label{eq:comcomposto}
Compasso~simples\times\frac{3}{2}=Compasso~composto.
\end{equation}
De modo que obtemos compassos binários compostos com as seguintes formulas de compasso: 
$6/4$, $6/8$, $6/16$, etc.
A diferencia do visto nos compassos binários simples, os compassos binários compostos tem 
um pulso forte (Acentuado) no tempo 1 e um pulso semiforte (Acentuado porem menor) no tempo 4, 
de modo que os tempos 2,3,5 e 6,
são classificados como tempos fracos (Não Acentuados)\cite[pp. 41]{grabner2001teoria}.
A Figura \ref{compasso:binariocomposto}, representa um exemplo de compasso binário composto, 
com formula de compasso $6/4 \equiv 6$\quarternote, 
e tempos com uma duração de $S/4$ (uma \quarternote), 
sendo que o primeiro compasso contem $6$\quarternote~e o segundo contem dois $2$\quarternote~e dois $2$\halfnote.
Da Figura \ref{compasso:binariocomposto} é fácil perceber
que em qualquer caso, só são acentuados as nota que são executadas no tempo 1 e 4; 
aclarando que as notas executadas no tempo 4 tem uma acentuação menor que as executadas no tempo 1.
\begin{figure}[H]
\centering
\begin{abc}[name=abc-compasso1c]
X: 1 % start of header
K: C % scale: C major
M: 6/4 %meter - compasso
"Primeiro compasso" G2 D2 D2 F2 D2 D2 |"Segundo compasso" G2 D4 F2 D4  |
w: Acentuado N.A. N.A. Acentuado N.A N.A. Acentuado N.A. Acentuado N.A. 
\end{abc}
\caption{Exemplo de compasso binário composto}
\label{compasso:binariocomposto}
\end{figure}

Alguns autores consideram aos compassos quaternários (ex: 4/4, 4/8) como um caso de compasso binário,
chamando eles de compasso binário duplo \cite[pp. 41]{grabner2001teoria}.




\subsection{Um compasso ternário} Ou compasso ternário simples,
é uma estrutura rítmica que se carateriza por ter compassos com trés tempos,
sendo o primeiro pulso forte (acentuado) e os outros dois fracos (não acentuados) 
\cite[pp. 67]{adolfo2002musica}\cite[pp. 30]{alves2004teoria}. 
Os compassos ternários (simples) tem uma formula de compasso da forma $3/B$, 
onde $B$ pode ser $2$, $4$, $8$, etc.
Por exemplo temos, as formulas de compassos ternários simples: $3/2$, $3/4$, $3/8$,  etc.

A Figura \ref{compasso:ternario}, representa um exemplo de compasso ternário (simples), com 
formula de compasso $3/4 \equiv 3$\quarternote, 
onde os tempos tem uma duração de $S/4$, o primeiro compasso contem $3$\quarternote~e
o segundo contem $6$\eighthnote.
Na Figura \ref{compasso:ternario}  é fácil perceber
que em ambos compassos, só a nota que é executada no tempo 1 é acentuada.
\begin{figure}[H]
\centering
\begin{abc}[name=abc-compasso2]
X: 1 % start of header
K: C % scale: C major
M: 3/4 %meter - compasso
"Primeiro compasso" G2 F2 F2 |"Segundo compasso" G1 F1 E1 D1 D1  D1  |
w: Acentuado N.A. N.A. Acentuado N.A N.A.  N.A. N.A. N.A. 
\end{abc}
\caption{Exemplo de compasso ternário}
\label{compasso:ternario}
\end{figure}


São chamados de compassos ternários compostos,  
quando estes tem uma formula de compasso como: $9/4$, $9/8$ e $9/16$.
Para gerar estes compassos compostos a partir de suas versões simples,
se segue a mesma operação descrita na Equação \ref{eq:comcomposto}.


\subsection{Um compasso quaternário} Ou compasso quaternário simples,
é uma estrutura rítmica que se carateriza por ter compassos com quatro tempos,
sendo o primeiro pulso forte (acentuado), o segundo fraco (não acentuado), 
o terceiro semiforte (acentuado porem menor) e o último fraco (não acentuado) 
\cite[pp. 67]{adolfo2002musica}\cite[pp. 32]{alves2004teoria}. 
Os compassos quaternários (simples) tem uma formula de compasso da forma $4/B$, 
onde $B$ pode ser $2$, $4$, $8$, etc.
Por exemplo temos, as formulas de compassos ternários simples: $4/2$, $4/4$, $4/8$,  etc.

A Figura \ref{compasso:quaternario}, representa um exemplo de compassos quaternário, com 
formula de compasso $4/4 \equiv 4$\quarternote, 
onde cada tempo tem uma duração de $S/4$, o primeiro compasso contem $4$\quarternote~e
o segundo contem $8$\eighthnote.
Na Figura \ref{compasso:ternario}  é fácil perceber
que em ambos compassos, só as notas que são executadas no tempo 1 e 3 são acentuadas.
\begin{figure}[H]
\centering
\begin{abc}[name=abc-compasso3]
X: 1 % start of header
K: C % scale: C major
M: 4/4 %meter - compasso
"Primeiro compasso" G2 D2 F2 D2|"Segundo compasso" G1 F1 D1 C1 F1 E1 D1 C1 |
w: Acentuado N.A. Acentuado N.A. Acentuado N.A N.A. N.A. Acentuado N.A. N.A. N.A. 
\end{abc}
\caption{Exemplo de compasso quaternário}
\label{compasso:quaternario}
\end{figure}

São chamados de compassos quaternários compostos,  
quando estes tem uma formula de compasso como: $12/4$, $12/8$ e $12/16$.
Para gerar estes compassos compostos a partir de suas versões simples,
se segue a mesma operação descrita na Equação \ref{eq:comcomposto}.
 


%%%%%%%%%%%%%%%%%%%%%%%%%%%%%%%%%%%%%%%%%%%%%%%%%%%%%%%%%%%%%%%%%%%%%%%%%%%%%%%%
\section{Tempo}

Como já foi sugerido na Seção \ref{sec:compaso}, é chamado de "tempo" 
à pulsação básica e unidade de medida dos compassos nas composições musicais;
assim, temos que compassos binários, ternários e quaternários; que tem uma duração de 2 tempos, 
3 tempos e 4 tempos, respetivamente. Por comodidade designaremos com a variável $T$ à duração em segundos de cada tempo,
sendo que o valor de $T$ variará dependendo da formula de compasso usada.
\subsection{Tempo forte}


\subsection{Tempo fraco}

\subsection{Formas de distribuir os tempos}
É importante ressaltar que os compassos que usem a mesma formula de compasso terão sempre a mesma duração em segundos;
por outro lado, podem ser achados casos onde compassos com diferente formula podem ter a mesma duração;
por exemplo: compassos binários com formula $\mathbf{2}/2$, comparados com compassos quaternários com 
formula $\mathbf{4}/4$; ambos compassos tem a mesma duração uma semibreve (\fullnote).

Na Figura \ref{fig:abc-tempo1} podemos ver compassos binários com formula $\mathbf{2}/2$, 
\begin{figure}[H]
\centering
\begin{abc}[name=abc-tempo1]
X: 1 % start of header
K: C % scale: C major
M: 2/2 %meter - compasso
G2 D2 F2 D2 | G4 F4 |
w: T/2 T/2 T/2 T/2  T T
\end{abc}
\caption{Exemplo de dois compassos com 2 tempos de duração T}
\label{fig:abc-tempo1}
\end{figure}
onde estos terão uma duração de dois tempos ($2T$) \cite[pp. 25]{azevedocompor} onde cada tempo ($T$) tem uma duração 
de uma mínima (\halfnote), ver Tabela \ref{tab:abc-noteslength};
por outro lado,
compassos quaternários com formula $\mathbf{4}/4$, como na Figura \ref{fig:abc-tempo2}, 
\begin{figure}[H]
\centering
\begin{abc}[name=abc-tempo2]
X: 1 % start of header
K: C % scale: C major
M: 4/4 %meter - compasso
G2 D2 F2 D2| G4 F4|
w: T T T T 2T 2T
\end{abc}
\caption{Exemplo de dois compassos com 4 tempos de duração T}
\label{fig:abc-tempo2}
\end{figure} 
terão uma duração de 4 tempos ($4T$) \cite[pp. 25]{azevedocompor} onde 
cada tempo ($T$) tem uma duração de uma semínima (\quarternote), ver Tabela \ref{tab:abc-noteslength}.
Assim, estas duas formulas ($2/2$ e $4/4$) representam compassos 
com a mesma duração em segundos, uma semibreve (\fullnote),
porem tem valores diferentes para a variável $T$ em segundos.

\begin{lattention}
Se interpretamos a música mostrada nas Figuras \ref{fig:abc-tempo1} e \ref{fig:abc-tempo2},
podemos perceber que ambas descrevem um sonido ligeiramente diferente; assim, não é fácil
distinguir se o sonido provem de um compasso com formula $2/2$ ou $4/4$.
Em estes casos a dica é perceber se existem dois tipos de tempos fortes, acentuados
com diferentes intensidades; se é assim, então estamos ante um compasso quaternário;
se não, então é um compasso binário.
\end{lattention}

%%%%%%%%%%%%%%%%%%%%%%%%%%%%%%%%%%%%%%%%%%%%%%%%%%%%%%%%%%%%%%%%%%%%%%%%%%%%%%%%
\section{Contratempo}
Um contratempo acontece quando as notas (representadas por figuras musicais na partitura) 
são executadas em tempos fracos do compasso
ou nas partes fracas dos tempos, sendo que estas estão intercaladas por pausas nos tempos
fortes ou partes fortes dos tempos \cite[pp. 16]{mascarenhascurso} 
\cite[pp. 36]{azevedocompor}, neste sentido o contratempo pode ser visto como a 
omissão de notas nos tempos fortes ou nas partes fortes dos tempos \cite[pp. 146]{medteoria}.
Ou ``num sentido mais amplo, o contratempo é a acentuação de um tempo fraco em vez de um tempo forte'' \cite[pp. 147]{medteoria}. 

Assim, a palavra ``contratempo'', referencia a como estão configuradas ou acentuadas 
as notas no compasso. Por exemplo:
A Figura \ref{fig:contratempoa} mostra 
quatro compassos (binários) com formula $2/4$, em cada compasso existem 
contratempos nos tempos fracos ou nas partes fracas dos tempos, sendo que cada tempo
tem uma duração de uma semínima (\quarternote) e cada compasso uma duração 
de uma mínima (\halfnote), ou seja duas semínimas (2\quarternote). 
\begin{itemize}
\item ``F''  indica que é o tempo é forte, 
\item ``f''  indica que é o tempo é fraco,
\item ``FF'' indica que é a parte forte de um tempo forte,
\item ``Ff'' indica que é a parte fraca de um tempo forte,
\item ``fF'' indica que é a parte forte de um tempo fraco,
\item ``ff'' indica que é a parte fraca de um tempo fraco, 
\end{itemize} 

finalmente
a figura musical \ViPa~ indica um silencio da mesma duração que uma semínima (\quarternote)
e a figura musical \AcPa~ indica um silencio da mesma duração que uma colcheia (\eighthnote).
\begin{figure}[H]
\centering
\begin{abc}[name=abc-contratempoa]
X: 1 % start of header
K: C % scale: C major
M:2/4
%T: Contratempo num compasso binário
V:1 clef=treble name="A" sname="A"
[V:1] "F"z2 "f"G2 | "FF"z1 "Ff"G1  "fF"z1 "ff"G1 | "FF"z1 "Ff"G1  "f"G2 |  "F"z2 "fF"z1 "ff"G1  |
w:          T          T/2            T/2             T/2     Tempo                 T/2
\end{abc}
\caption{Contratempos no tempos fracos ou nas partes fracas dos tempos}
\label{fig:abc-contratempoa}
\end{figure}
Na Figura \ref{fig:abc-contratempoa}, existem contratempos em todos os compassos porem estes estão
configurados de distintas formas;
no primeiro compasso acontece um contratempo dado que a única nota é executada 
no tempo fraco do compasso, no segundo compasso acontecem contratempos pois as 
notas são executadas nas partes fracas de cada tempo,
no terceiro compasso acontece um contratempo pela execução de uma nota na parte 
fraca do tempo forte, sendo o resto do tempo preenchido com um silencio, e 
finalmente no quarto compasso acontece um contratempo pela execução de uma nota
na parte fraca do tempo fraco, sendo o resto do compasso preenchido com silêncios.


Por outro lado, 
a Figura \ref{fig:abc-contratempob} mostra um caso similar ao da Figura \ref{fig:abc-contratempoa},
com contratempos expressados como a acentuação de um tempo fraco em vez de um silencio no tempo forte \cite[pp. 147]{medteoria}. 
É usado o símbolo $>$ para indicar esta acentuação na partitura.
\begin{figure}[H]
\centering
\begin{abc}[name=abc-contratempob]
X: 1 % start of header
K: C % scale: C major
M:2/4
%T: Contratempo num compasso binário
V:1 clef=treble name="A" sname="A"
[V:1] "F"G2 "f"+accent+G2 | "FF"G1 "Ff"+accent+G1  "fF"G1 "ff"+accent+G1 | "FF"G1 "Ff"+accent+G1  "f"G2  | "F"G2 "fF"G1  "ff"+accent+G1  | 
w:    T     T                T/2    T/2             T/2    T/2              T/2    T/2             T       T      T/2             T/2  
\end{abc}
\caption{Contratempos pela acentuação dos tempos fracos ou nas partes fracas dos tempos}
\label{fig:abc-contratempob}
\end{figure}

%%%%%%%%%%%%%%%%%%%%%%%%%%%%%%%%%%%%%%%%%%%%%%%%%%%%%%%%%%%%%%%%%%%%%%%%%%%%%%%%
\section{Sincopa}

%%%%%%%%%%%%%%%%%%%%%%%%%%%%%%%%%%%%%%%%%%%%%%%%%%%%%%%%%%%%%%%%%%%%%%%%%%%%%%%%
\section{Contagem dos tempos no compasso}



