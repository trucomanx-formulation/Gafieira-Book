
%%%%%%%%%%%%%%%%%%%%%%%%%%%%%%%%%%%%%%%%%%%%%%%%%%%%%%%%%%%%%%%%%%%%%%%%%%%%%%%%
\section{Nomenclatura e definições sobre passos de dança}



\begin{definition}[Duração do movimento] 
\index{Duração do movimento}
\label{def:DuracaoDoPasso}
Ou a \textbf{duração do passo}, 
é a longitude temporal de um passo de dança, contado em tempos da música ou da métrica coreográfica.
No caso de \hyperref[def:PassoCiclico]{\textbf{passos cíclicos}}, a duração do passo se refere a duração do ciclo.
\end{definition}

\subsection{Classificação dos passos seguindo sua função ou uso}

\begin{definition}[Passo a contratempo] 
\index{Passo a contratempo}
\label{def:PassoAContratempo}
É um passo de dança cuja execução promove que quando se esteja dançando com um pé especifico acompanhando o tempo forte da música,
ao finalizar o passo este pé esteja sendo marcado num tempo fraco.
Ou seja são movimentos os quais após de realizados, 
passamos de \hyperref[def:DancaNoTempo]{\textbf{dançar no tempo forte}} a 
\hyperref[def:DancaNoContratempo]{\textbf{dançar em contra do tempo forte}} e vice-versa. 
No samba de gafieira, 
passos acontratempo acontecem quando o número de tempos musicais que o passo usa é impar;
pois as músicas no samba tradicionalmente são escritas em compassos binários. 
%pelo que comumente acharemos um tempo forte intercalado por uno tempo fraco.
\end{definition}


\begin{definition}[Passo cíclico] 
\index{Passo cíclico}
\label{def:PassoCiclico}
É um passo de dança que dura um tempo indeterminado,
pois está composto por ciclos, cujas posturas de inicio e final são as mesmas.
\end{definition}

\begin{definition}[Passo simétrico] 
\index{Passo simétrico}
\label{def:PassoSimetrico}
É um passo de dança que tem a primeira metade do passo similar à segunda,
com a diferencia que se executa com os pés intercambiado (direita por esquerda)
ou com as posições dos corpos intercambiadas (seguidor por condutor).
\end{definition}

%\cite[pp. 79]{wright1945dance}
%\cite[pp. 93]{BallroomDancing1992}
\begin{definition}[Passo progressivo] 
%\index{Passo de deslocamento}
\index{Passo progressivo}
\label{def:PassoDeDeslocamento} 
É um passo de dança cujo propósito, consequência ou objetivo é o deslocamento no salão.
\end{definition}

\subsection{Classificação das posições seguindo seu uso na dança}

\begin{definition}[Posição de finalização] 
\index{Posição, tipos!Posição de finalização}
\label{def:PosturaFinaliza}
É uma postura a qual da a percepção, 
de que se tem chegado ao ponto final da ideia expressada com o movimento.
Se nossa dança fosse um relato escrito, esta postura seria equivalente a um ponto.
\end{definition}

\begin{definition}[Posição de transição] 
\index{Posição, tipos!Posição de transição}
\label{def:PosturaTransicao}
É uma postura a qual da a percepção, 
de que não se tem completado a ideia expressada com o movimento, e que possivelmente vem mais algum outro movimento.
Se nossa dança fosse um relato escrito, esta postura seria equivalente a uma virgula ou espaço em branco entre as palavras.
\end{definition}

