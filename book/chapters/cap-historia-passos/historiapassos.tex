
%%%%%%%%%%%%%%%%%%%%%%%%%%%%%%%%%%%%%%%%%%%%%%%%%%%%%%%%%%%%%%%%%%%%%%%%%%%%%%%%
%% CAPITULO
%%%%%%%%%%%%%%%%%%%%%%%%%%%%%%%%%%%%%%%%%%%%%%%%%%%%%%%%%%%%%%%%%%%%%%%%%%%%%%%%
\chapterimage{chapter_head_2.pdf} % Chapter heading image

\chapter{\textcolor{blue}{Passos do samba de gafieira}}
%\index{Passos}

\begin{definition}[Plano axial:] 
\index{Plano axial}
\label{def:PlanoAxial}
O Dicionário Priberam da Língua Portuguesa \cite{priberamplano} define plano axial como:
Plano transversal, plano horizontal que divide o corpo ou uma estrutura anatômica em parte superior e parte inferior.
Ver Figura \ref{fig:bodyhumanplane}.
\end{definition}

\begin{definition}[Plano frontal:] 
\index{Plano frontal}
\label{def:PlanoFrontal}
O Dicionário Priberam da Língua Portuguesa \cite{priberamplano} define plano frontal como:
Plano coronal,   plano vertical e paralelo à sutura coronal do crânio, que divide o corpo em parte anterior e parte posterior.
Ver Figura \ref{fig:bodyhumanplane}.
\end{definition}

\begin{definition}[Plano sagital:] 
\index{Plano sagital}
\label{def:PlanoSagital}
O Dicionário Priberam da Língua Portuguesa \cite{priberamplano} define plano sagital como:
Plano vertical e paralelo à sutura sagital do crânio, que divide o corpo em parte direita e parte esquerda.
Ver Figura \ref{fig:bodyhumanplane}.
\end{definition}

\begin{figure}[h!]
  \centering
    \includegraphics[width=0.50\textwidth]{body-plane/files/body-plane.png}
  \caption{ Planos e eixos no corpo humano.}
\label{fig:bodyhumanplane}
\end{figure}

\begin{definition}[Eixo axial:] 
\index{Eixo axial}
\label{def:EixoAxial}
É o eixo perpendicular ao plano axial.
Ver Figura \ref{fig:bodyhumanplane}.
\end{definition}

\begin{definition}[Eixo frontal:] 
\index{Eixo frontal}
\label{def:EixoFrontal}
É o eixo perpendicular ao plano frontal.
Ver Figura \ref{fig:bodyhumanplane}.
\end{definition}

\begin{definition}[Eixo sagital:] 
\index{Eixo sagital}
\label{def:EixoSagital}
É o eixo perpendicular ao plano sagital.
Ver Figura \ref{fig:bodyhumanplane}.
\end{definition}

\begin{definition}[Passo cíclico:] 
\index{Passo cíclico}
\label{def:PassoCiclico}
É um passo de dança que pode acontecer por um tempo indeterminado,
devido a que este está composto por ciclos, cuja postura de inicio e final é a mesma.
\end{definition}

\begin{definition}[Duração do passo de dança:] 
\index{Duração do passo de dança}
\label{def:DuracaoDoPasso}
É a longitude temporal de um passo de dança, contado em tempos da música.
No caso de \hyperref[def:PassoCiclico]{\textbf{passos cíclicos}}, a duração do passo se refere a duração do ciclo.
\end{definition}


\begin{definition}[Dançar no tempo forte:] 
\index{Dançar no tempo forte}
\label{def:DancaNoTempo}
Ou simplesmente \textbf{dançar no tempo}, indica que se está dançando com passos com o movimento principal ou inicial (dependendo do estilo de dança), 
se executando no tempo forte da música; ver Exemplo \ref{example:dancatempoforte}.
\end{definition}
\begin{example}
\label{example:dancatempoforte}
Se definimos um passo de dança como: Pisar, usando um pé cada vez, 
realizando um movimento com uma distribuição espacial, junto-junto-longo;
e definimos ao movimento ``longo'' como o movimento principal. 
Se executássemos o movimento ``longo'' no tempo forte, o passo junto-junto-longo,
estaria sendo dançado no tempo forte.
\end{example}

\begin{definition}[Dançar em contratempo:] 
\index{Dançar no contratempo}
\label{def:DancaNoContratempo}
Indica que se está dançando com passos com o movimento principal ou inicial (dependendo do estilo de dança), 
se executando em contra do tempo forte da música; é dizer, em contratempo.
Com diferencia de \hyperref[def:DancaNoTempo]{\textbf{dançar no tempo forte}}, 
que é único, pois só existe um tempo forte;
existem varias formas de dançar como o movimento principal em algum tempo fraco; ver Exemplo \ref{example:dancatempofraco}.
\end{definition}
\begin{example}
\label{example:dancatempofraco}
Se definimos um passo de dança como: Pisar, usando um pé cada vez, 
realizando um movimento com uma distribuição espacial, junto-junto-longo;
e definimos ao primeiro movimento ``junto'' como o movimento principal. 
Se executássemos o movimento ``longo'' no tempo forte, o passo junto-junto-longo,
estaria sendo dançado em contratempo.
\end{example}

\begin{definition}[Passo a contratempo:] 
\index{Passo a contratempo}
\label{def:PassoAContratempo}
É um passo de dança cuja execução promove que quando se esteja dançando com um pé especifico acompanhando o tempo forte da música,
ao finalizar o passo este pé esteja sendo marcado no tempo fraco.
É dizer, são movimento onde após de realizados, passamos de \hyperref[def:DancaNoTempo]{\textbf{dançar no tempo}} a \hyperref[def:DancaNoContratempo]{\textbf{dançar no contratempo}} e vice-versa. 
\end{definition}


%%%%%%%%%%%%%%%%%%%%%%%%%%%%%%%%%%%%%%%%%%%%%%%%%%%%%%%%%%%%%%%%%%%%%%%%%%%%%%%
%%%%%%%%%%%%%%%%%%%%%%%%%%%%%%%%%%%%%%%%%%%%%%%%%%%%%%%%%%%%%%%%%%%%%%%%%%%%%%%
\section{\textcolor{blue}{Que passos existem no samba de gafieira?}}

Nos presentes dias, \AnoLivro, exitem uma grande variedade de passos para o samba de gafieira,
tantos como a imaginação possa atingir, pois além dos movimentos mais conhecidos e  consagrados da dança,
podem existir variações  destes ou simplesmente estilos em que estos são realizados. 

Nas seguintes subseções, listaremos e descreveremos 
alguns dos passos que são possíveis de ver no samba de gafieira;
mas, estas descrições não pretendem ser uma guia de ensino,
e sim um instrumento para saciar a curiosidade do leitor em como os movimentos são realizados.

%%%%%%%%%%%%%%%%%%%%%%%%%%%%%%%%%%%%%%%%%%%%%%%%%%%%%%%%%%%%%%%%%%%%%%%%%%%%%%%
\PRLsep{Passos no samba de gafieira ate 1949}

\subsection{Balão} 
\label{def:PassoBalao}
\index{Passo!Balão}

Um movimento com este nome já existia desde as origens do samba nas gafieiras, 
sendo este movimento proveniente do maxixe \cite[pp. 142]{perna2002samba} 
\cite[pp. 93]{efege1974maxixe} \cite[pp. 465]{marcondes1977enciclopedia}.
%porem não tem se achado referencias sobre as caraterísticas do movimento no maxixe.



Seguindo o Prof. Gino Fornaciari, no ano 1947,  em São Paulo, se usava o nome balão e pião,
para representar a um mesmo movimento, um que agora chamaríamos de pião, 
porem o pião de 1947 era em sentido anti-horário \cite[pp. 68-72]{fornaciari1947aprender}.

Em \AnoLivro, o nome balão designa a um movimento que pode ser considerado aéreo, 
pois o \hyperref[def:Condutor]{\textbf{condutor}} tira do chão os pés do \hyperref[def:Seguidor]{\textbf{seguidor}}.
Um movimento com estas caraterísticas pode ser visto no filme ``Aviso aos navegantes'' (1950),
pelo que podemos especular que este era de uso comum desde muito antes \cite[min. 40:35]{AtlantidaDance};
porem, não pode-se souber sim lhe era atribuído ou não nessa época o nome de balão; 
mas pela semelhança com o movimento de quadril do passo chamado balão apagado,
é provável que sim, 
pois lembremos que o problema da homologação da nomenclatura dos passos existe ate em  nossos dias.



O movimento dura 3 tempos, o passo inicia com o seguidor ao lado direito do condutor, 
ligeiramente atrás dele, com um abraço de dança bem próximo e uma postura similar a postura de X só que com os pés mais juntos.
No primeiro tempo o condutor da um passo ao lado, e pisa com o pé direito,
de modo que o seguidor fique em pé atrás da perna direita do condutor, sem perder o abraço.
No segundo tempo, aproveitando a postura, 
o condutor faz um movimento circular anti-horário com seu quadril, no \hyperref[def:PlanoAxial]{\textbf{plano axial}},
de modo que sua perna direita, que está em contato com a perna direita do seguidor,
serva como alavanca para tirar ao seguidor do chão, 
e este gire ou voe ao redor \footnote{O giro do seguidor é com o corpo reto e pernas juntas, 
como se fosse uma fita solta de um lado e com o outro lado presso num ponto 
que provoca o giro da fita no \hyperref[def:PlanoAxial]{\textbf{plano axial}}, com giro ao redor do eixo axial.} 
do condutor em sentido anti-horário.
No terceiro tempo o seguidor senta-se, é dizer faz uma cadeirinha, sobre a perna esquerda do condutor,
que para receber ao seguidor  da um passo ao frente.

\begin{comment}
A Figura \ref{fig:balao1950} mostra um fotograma do filme ``Aviso aos navegantes'' (1950),
onde se observa o inicio do passo balão (\AnoLivro), quando a moça tira os pés do chão.
\begin{figure}[h!]
  \centering
    \includegraphics[width=0.7\textwidth]{chapters/cap-historia-passos/balao1950.png}
  \caption{Fotograma com o inicio do passo balão (\AnoLivro), do filme ``Aviso aos navegantes'' (1950) \cite[min. 40:35]{AtlantidaDance}.}
  \label{fig:balao1950}
\end{figure} 
\end{comment}

\subsection{Balão apagado}
\index{Passo!Balão apagado} 
\index{Passo cíclico!Balão apagado}
Um movimento com este nome já existia desde as origens do samba nas gafieiras, 
sendo este movimento proveniente do maxixe \cite[pp. 142]{perna2002samba} \cite[pp. 68]{efege1974maxixe}.
Um exemplo do movimento que agora designamos como balão apagado pode 
ser visto no filme ``Aviso aos navegantes'' (1950) \cite[min. 40:35]{AtlantidaDance}.
\begin{comment}
A Figura \ref{fig:balaoapagado1950} mostra um fotograma do filme onde se observa o movimento de quadril no balão apagado.
\begin{figure}[h!]
  \centering
    \includegraphics[width=0.7\textwidth]{chapters/cap-historia-passos/balaoapagado1950.png}
  \caption{Fotograma que mostra a execução do passo balão apagado, tirado do filme ``Aviso aos navegantes'' (1950) \cite[min. 40:35]{AtlantidaDance}.}
  \label{fig:balaoapagado1950}
\end{figure}
\end{comment}

Em \AnoLivro, este movimento tem um parecido ou lembrança com o \hyperref[def:PassoBalao]{\textbf{balão}} (\AnoLivro); 
porem, se realiza com o par num abraço de dança estando um frente ao outro, 
consequentemente o \hyperref[def:Seguidor]{\textbf{seguidor}} não voa ao redor do \hyperref[def:Condutor]{\textbf{condutor}}, 
se não que a intenção de voar se apaga e o seguidor nunca sai do chão; 
de modo que o casal fica dando giros, abraçados, num eixo comum e praticamente no lugar. 
Estes giros são promovidos por marcados movimentos circulares de quadril, que mudam
de velocidade e intenção num constante, abrupto, leve e leve,  na proporção de tempos \{1/2 tempo,1/2 tempo, tempo\}; 
semelhando assim ao movimento de um balão perdendo o ar.
Existem variantes deste movimento, onde o giro do par é realizado em sentido horário e anti-horário; porem, 
não saberia afirmar qual é a versão padrão; mas, pelas minhas observações a versão mais difundida,
é a que faz o giro em sentido anti-horário.

Este movimento é um \hyperref[def:PassoCiclico]{\textbf{passo cíclico}}, com ciclos que duram 4 tempos, 
sendo o primeiro par de tempos similar ao segundo, porem com os papeis intercambiados no par de dança.
No momento inicial, o casal está abraçado numa postura frente a frente, 
com o peso do corpo do lado da perna direita do condutor;
no tempo 1 o condutor da um passo e pisa com a perna esquerda pra traz, 
como se procura-se ocultar esta atrás da sua perna direita, 
este movimento de perna é promovido pelo movimento circular do 
quadril em sentido anti-horário no \hyperref[def:PlanoAxial]{\textbf{plano axial}};
por outro lado, 
o seguidor da um passo adiante com sua perna direita procurando manter a postura 
relativa com o condutor e acompanhando o movimento circular anti-horário do quadril, 
de modo que se seu pé direito tende a   rodear ao condutor.
Nos tempos 1.5 e 2 o par pisa no lugar, ajeitando suas posturas apagando o movimento do quadril, 
mas mantendo o giro do par, 
de modo que terminam abraçados  frente a frente com o peso do corpo no lado do pé esquerdo do condutor.
No próximo par de tempos, o movimento é similar, só que agora é o seguidor que inicia dando um passo com o pé esquerdo. 


\subsection{Puladinho }
\index{Passo!Puladinho}
\index{Passo cíclico!Puladinho}

\begin{comment} 
neste movimento não se pula; 
existem varias referencias não acadêmicas na internet, que datam desde o 2002,
onde não mencionam ao ``Puladinho'' e sim um passo chamado ``pruladinho'', 
pelo qual suspeito que este faz referencia ao mesmo passo;
pois indica corretamente que no movimento se vá ``pra o ladinho''.
\end{comment}

Na polca, trazida ao Brasil em 1845, 
existia um movimento chamado puladinho,
que era um movimento que se fazia sobre as pontas dos pés,
indo para adiante, iniciando com o pé esquerdo estacando obliquamente à esquerda,
num segundo momento o pé direito avança ate ficar junto ao outro, 
para logo deslizar o pé esquerdo para adiante, 
permitindo assim levantar o pé direito para ajeitar a postura 
e recomeçar o movimento com esquerdo \cite[pp. 58-59]{tinhorao1986pequena}.
É fácil perceber que este movimento tem alguns pontos semelhantes ao puladinho (\AnoLivro),
no fato do andar oblíquo e a troca de pesos, porem o puladinho da polca não era simétrico para ambos pés.

Por outro lado, no maxixe (dança)  que é descrito em 1920 na ``Revista do Brasil'',
se mencionam os termos maxixe ``puladinho'' e maxixe de ``esquentar a barriga'',
como descrições usadas pelos aficionados a esta dança, 
pelo que se entende que no maxixe também existiu um movimento chamado puladinho \cite[pp. 177]{1920revista}. 

Adicionalmente, no livro ``Oito décadas: memórias'', se menciona que na década de
1920 existia uma variante do samba que se chamava ``o puladinho'' 
que introduziu nos salões o carnaval do povo, 
e que provocava entre as jovens da época, em palavras da autora, 
``a externalização da sensualidade reprimida'' \cite[pp. 94-95]{nabuco2000oito}.

Conhecidas estas informações, é importante lembrar que a polca influenciou a criação do maxixe (dança), 
que a sua vez influenciou a criação do samba dançado nas gafieiras,
que se condensou no samba de gafieira (\AnoLivro), o qual tem em nossos dias um passo chamado puladinho. 
Pelo que pode-se considerar, pela existência do termo em todas as evoluções da dança; 
que um movimento chamado puladinho 
já estava também presente nos primeiros sambas dançados nas gafieiras. 

Reforçando esta hipótese, podemos achar uma referencia ao uso do termo puladinho, no título de uma música instrumental chamada 
``Puladinho na gafieira'' (1958)  de  Marisa com Moacyr Silva e seu conjunto: Convite à música \cite{puladinhogafieiramusic}.
Também, podemos ver uma menção a este movimento, junto a outros conhecidos no samba de gafieira,
em 1976 na revista ``Veja'' \cite[pp. 158]{1976veja},
em 1978 na letra da canção ``Baile no Elite'' \cite{BaileNoElite} e 
em 1979 na revista ``Isto é'' \cite[pp. 89]{revista1979isto}.



Finalmente, podemos ver uma referencia a esse passo de dança, no ``jornal dos sports''(RJ),
do dia 17 de julho de 1986 \cite[pp. 6]{gafieiraaredeout2}.


O puladinho, de \AnoLivro, é executado com o casal abraçado frente a frente, sendo este um \hyperref[def:PassoCiclico]{\textbf{passo cíclico}},
com uma duração de 4 tempos; onde o movimento dos dois primeiros tempos
é simétrico ao segundo par de tempos.
Desde o ponto de vista do  \hyperref[def:Condutor]{\textbf{condutor}}, 
este inicia o movimento levando o peso do corpo junto com seu pé direito para atrás, provocando que
o \hyperref[def:Seguidor]{\textbf{seguidor}} de um passo ao frente acompanhando-lhe;
porem o condutor realiza seu movimento predominantemente pela ação do quadril e com um ligeiro arco para a direita, 
com o fim de dar molejo ao movimento;
no tempo 1.5 sem deslocar os pés, 
se faz uma troca de peso do corpo para o pé esquerdo do condutor (direito do seguidor) usando o quadril,
e finalmente no tempo 2 o condutor volta a levar o peso 
do corpo para a sua perna direita (esquerdo do seguidor) novamente usando o quadril.
Neste ponto a metade do ciclo foi realizado e o movimento se repete simetricamente, 
de modo que no tempo 3 o condutor leva atrás o seu pé esquerdo (direito do seguidor) e continua em 3.5 e 4.

 

\subsection{Pião}
\index{Passo!Pião}
\index{Passo cíclico!Pião}
Seguindo o Prof. Gino Fornaciari, no ano 1947,  em São Paulo, os nomes pião e balão,
representavam ao mesmo movimento, no samba-batucada\footnote{Samba de gafieira primigeinio.}; 
porem o pião de 1947 era em sentido anti-horario \cite[pp. 68-72]{fornaciari1947aprender}.
Também, podemos ver uma menção a este movimento, junto a outros conhecidos no samba de gafieira, 
em 1979 na revista ``Isto é'' \cite[pp. 89]{revista1979isto}.
Todas estas afirmações são coerentes com as declarações de Jimmy de Oliveira 
que indica que o pião já existiam antes do 1990 \cite{sambafunkeadoJimmyDeOliveiraPart1}.

O pião, de \AnoLivro, é um \hyperref[def:PassoCiclico]{\textbf{passo cíclico}} que é executado com o casal abraçado, 
realizando giros sobre um eixo comum.
Cada giro dura 2 tempos, e é realizado tradicionalmente em sentido horário;
na primeira metade do giro (que dura 1 tempo) o eixo de giro do par é colocado sobre uma pessoa do par, 
de modo que a outra pessoa gira ao redor (em 1 tempo), ate chegar a uma postura similar à inicial, 
porem com os papeis intercambiadas no par, com respeito ao tempo anterior;
na outra metade do ciclo se repete o movimento, porem agora é a outra pessoa que terá o eixo do par.
Este é um movimento de deslocamento, de modo que se procura girar movimentando-se numa linha reta.


\subsection{Pica-pau} 
\index{Passo!Pica-pau}
\index{Passo cíclico!Pica-pau}
Nos primórdios do samba nas gafieiras existia um passo de dança com esse nome \cite[pp. 142]{perna2002samba}.

No Fandango\footnote{Para informação sobre o fandango, ir a Pag. \pageref{fig:fandango}.} 
rufado-bailado de São Paulo, em 1948, existiu uma dança chamada ``pica-pau'';
%com uma coreografia semelhante ao ``anucorrido'' (anu-corrido);
em Itanhaém (SP) esta dança é caraterizada por pares espalhados no salão,
onde o canto do violeiro é alternado com batidas de pé e palmas pelos 
cavalheiros \cite[pp. 607-608]{marcondes1977enciclopediav2} \cite[pp. 49]{fandangoSP},
esta caraterística do uso dos pés é o que da uma semelhança ao pica-pau do samba de gafieira (\AnoLivro),
que leva esse nome porque se bate o chão com a ponta (ou meia ponta) do pé simulando as bicadas de um pássaro pica-pau.
Porem, a existência do pica-pau no fandango não indica uma relação de vinculo parental do movimento,
e sim que no consciente coletivo, o estilo de imitar ao pica-pau, na dança,
já estava presente desde antes dessa época.

Um movimento com as caraterísticas do pica-pau (\AnoLivro) pode ser visto 
no filme ``Aviso aos navegantes'' (1950) \cite[min. 40:35]{AtlantidaDance}.

No \AnoLivro, o passo pica-pau, é um \hyperref[def:PassoCiclico]{\textbf{passo cíclico}} que dura 4 tempos, 
sendo os dois primeiros simétricos aos dois últimos, 
onde no primeiro par de tempos se usa só um pé,
e no segundo o outro.
Se iniciamos com o peso do corpo na perna esquerda, 
no tempo 1 marcamos com o pé direito, um pouco atrás do pé esquerdo, 
com a ponta ou meia ponta do pé e sem levar o peso,
no tempo 1.5 repetimos o mesmo movimento com o mesmo pé, e finalmente
no tempo 2 colocamos o pé direito ao lado e a direita do pé esquerdo, 
levando esta vez o peso do corpo. 
Nos tempo 3, 3.5 e 4 se repete o movimento explicado anteriormente; porem,
agora se usa o pé esquerdo.
  
\begin{figure}[t]
\begin{elaboracion}[title=Fandango]

Esta é a designação que se lhe dá a todas as danças de 
roda para adultos, em São Paulo, Paraná, Santa Catarina e Rio Grande do Sul;
este termo para eles significa baile rural ou popular \cite[pp. 261]{marcondes1977enciclopedia}.
No litoral paulista, em 1948, o Fandango é dividido principalmente em duas categorias: Fandango rufado, 
e Fandango bailado (ou valsado), porem existe a possibilidade de 
misturar e fazer um Fandango rufado-bailado \cite[pp. 48-49]{fandangoSP}.
O Fandango rufado é um conjunto de danças em que se usam batidas de pé e palmas, 
que exigem do dançarino muita energia; exemplos: O ``Chico'', ``Sapo'', 
``Farrabio'' ou ``Sarrabalho'', ``Vilão'', ``Querumana'', ``Anu-velho'', ``Recortado'' \cite[pp. 48-49]{fandangoSP}, 
etc.
O Fandango bailado é um conjunto de danças onde  não entram batidas de pé e palmas,
este é dançado dentro de casa quando os dançarinos estão cansados ou com menos energia;
exemplos: O ``Manjericão'', ``Faxineira'', ``Chamarrita'', ``Graciana'', ``Dandão'' \cite[pp. 49]{fandangoSP}, 
etc.
No Fandango rufado-bailado existem partes onde se dão batidas de pés e outras de deslisamentos e giros de valsa;
exemplos: O ``Pipoca'', ``Anucorrido'', ``Pica-pau'', ``Sinsará'' e ``Tonta'' ou ``Tontinha'' \cite[pp. 49]{fandangoSP}.


\end{elaboracion}
\label{fig:fandango}
\end{figure}



%%%%%%%%%%%%%%%%%%%%%%%%%%%%%%%%%%%%%%%%%%%%%%%%%%%%%%%%%%%%%%%%%%%%%%%%%%%%%%%
\PRLsep{Passos no samba de gafieira anteriores a 1986}

\subsection{Elevador}
\index{Passo!Elevador}
\label{def:PassoElevador}

Um movimento com as caraterísticas, do que no \AnoLivro~ conhecemos como elevador, 
pode ser visto no filme ``Aviso aos navegantes'' (1950) \cite[min. 40:35]{AtlantidaDance}.

No \AnoLivro, o elevador é um movimento que dura 2 tempos, e geralmente é
executado desde a postura de facão estando o par num abraço de dança.
Antes de iniciar o movimento o \hyperref[def:Condutor]{\textbf{condutor}} 
desce sua mão esquerda, em relação a linha dos ombros, 
para que o \hyperref[def:Seguidor]{\textbf{seguidor}}
a use de apoio para segurar o peso do corpo com sua mão direita no transcurso do movimento.
É importante ressaltar que o braço esquerdo do condutor não tenta elevar ao seguidor, 
e sim simplesmente tenta manter a postura,
o que misturado com o movimento de pernas do condutor,
provoca a saída do chão do seguidor.
No tempo 1 o condutor sai da postura de facão colocando seu pé esquerdo 
ao lado do outro, ficando em pé; dado que o condutor não deixa a postura do abraço de dança,
e este mantêm a postura da mão esquerda  como apoio ao seguidor, este é elevado, tirando os pés do chão;
no tempo 2 o seguidor desce tocando o chão, primeiro com o pé direito e logo com o esquerdo retoma a postura de facão.

É interessante ressaltar que o elevador de 1950, não inicia desde a postura do facão, e 
o seguidor  não se eleva por ação do braço apoio.
Analisando o video, é ``provável'' que o seguidor seja elevado principalmente pela ação da perna direita do condutor
e a curvatura para atrás que este faz com a parte baixa da coluna.
\begin{comment}
A Figura \ref{fig:elevador50} mostra um fotograma do filme onde se observa como o seguidor é elevado.
\begin{figure}[h!]
  \centering
    \includegraphics[width=0.7\textwidth]{chapters/cap-historia-passos/elevador1950.png}
  \caption{Fotograma do tempo 1 do passo elevador, tirado do filme ``Aviso aos navegantes'' (1950) \cite[min. 40:35]{AtlantidaDance}.}
  \label{fig:elevador50}
\end{figure}
\end{comment}

\subsection{Cadeirinha}
\index{Passo!Cadeirinha}

Um movimento com as caraterísticas do que no \AnoLivro~ conhecemos como cadeirinha, 
pode ser visto no filme ``Aviso aos navegantes'' (1950),
pelo que podemos especular que este passo existia desde muito antes \cite[min. 40:35]{AtlantidaDance}.
\begin{comment}
A Figura \ref{fig:cadeirinha1950} mostra um fotograma deste filme, onde se observa a cadeirinha.
\begin{figure}[h!]
  \centering
    \includegraphics[width=0.7\textwidth]{chapters/cap-historia-passos/cadeirinha1950.png}
  \caption{Fotograma da cadeirinha, tirado do filme ``Aviso aos navegantes'' (1950) \cite[min. 40:35]{AtlantidaDance}.}
  \label{fig:cadeirinha1950}
\end{figure}
\end{comment}

Posteriormente, podemos voltar a ver o passo numa fotografia com a posse final caraterística deste movimento, 
no ``jornal dos sports''(RJ),
do dia 17 de julho de 1986 \cite[pp. 6]{gafieiraaredeout2}. 

\begin{comment}
como pode ser visto na Figura \ref{fig:cadeirinha86}.
\begin{figure}[h!]
  \centering
    \includegraphics[width=0.45\textwidth]{chapters/cap-historia-passos/cadeirnha1986.jpg}
  \caption{Fotografia da pose da cadeirinha, publicada em 1986 no ``jornal dos sports''(RJ) \cite[pp. 6]{gafieiraaredeout2}.}
  \label{fig:cadeirinha86}
\end{figure}
\end{comment}

No \AnoLivro, o nome cadeirinha representa a qualquer movimento onde o \hyperref[def:Seguidor]{\textbf{seguidor}} 
sai do chão e se senta sobre a perna do \hyperref[def:Condutor]{\textbf{condutor}}  (comumente sobre a perna esquerda);
a saída do chão do seguidor pode-se originar desde vários movimentos ou passos,
como por exemplo do  \hyperref[def:PassoBalao]{\textbf{balão}}; porem,
todos conservam a mesma técnica, que ao igual que  no caso do passo  \hyperref[def:PassoElevador]{\textbf{elevador}},
a suspensão no ar do seguidor, é provocada pelo braço esquerdo do condutor,
que para este propósito  desce da linha dos ombros.
É importante lembrar que este braço não faz força para levantar ao seguidor, 
e sim simplesmente tenta manter a postura,
o que misturado com o movimento do corpo do condutor (esticar as pernas anteriormente flexionadas, principalmente)
provoca a saída do chão e suspensão do seguidor.

\subsection{\textcolor{blue}{Cruzado}}
\index{Passo!Cruzado}
\index{Passo cíclico!Cruzado}
Movimento sem data de criação conhecida \cite[pp. 143]{perna2002samba}.
Porem, podemos ver uma menção a este movimento, junto a outros conhecidos no samba de gafieira,
em 1976 na revista ``Veja'' \cite[pp. 158]{1976veja},
em 1978 na letra da canção ``Baile no Elite'' \cite{BaileNoElite}  e 
em 1979 na revista ``Isto é'' \cite[pp. 89]{revista1979isto}.

No \AnoLivro, este passo inicia desde a postura de X com um abraço de dança, 
sendo este um \hyperref[def:PassoCiclico]{\textbf{passo cíclico}} que dura 4 tempos.
Os dois primeiros tempos são simétricos aos dois últimos, porem com os pés intercambiados,
direita por esquerda.

\subsection{\textcolor{blue}{Bicicleta}}
\index{Passo!Bicicleta}
Movimento sem data de criação conhecida \cite[pp. 143,144]{perna2002samba}.

Também, podemos ver uma menção a este movimento, junto a outros conhecidos no samba de gafieira, 
em 1979 na revista ``Isto é'' \cite[pp. 89]{revista1979isto}.


%%%%%%%%%%%%%%%%%%%%%%%%%%%%%%%%%%%%%%%%%%%%%%%%%%%%%%%%%%%%%%%%%%%%%%%%%%%%%%%
\PRLsep{Passos no samba de gafieira anteriores a 1990}



\subsection{ Picadinho (Picadilho)}
\index{Passo!Picadilho}
\index{Passo!Picadinho}
\index{Passo cíclico!Picadinho}

Em palavras de Jimmy de Oliveira este movimento já existia antes de 1990 \cite{sambafunkeadoJimmyDeOliveiraPart1}.

Na década de 1920, podemos ver na revista ``A Cigarra'', 
referencias a um tipo de música e de dança denominado picadinho,
geralmente indicado com frases como \cite[pp. 13]{picadinho1}:
\begin{citando}
O chefe da turma é Zezé de Almeida, excellente pianista,
que nos entontece, quando toca os seus \textbf{picadinhos} apimentados;~\\
$[$...$]$~\\
é o Deco. Que rapasinho admiravel.. para dansar \textbf{picadinho}! 
Zezé tocando e Deco dansando, tenho eu divertimento 
para toda a vida e mais seis mezes.
\end{citando}
Em números posteriores da revista, 
podemos achar outras referencias ao picadinho como dança \cite[pp. 52]{picadinho2} \cite[pp. 49]{picadinho3}.


O picadilho ou picadinho, no \AnoLivro~ é um movimento de pouco deslocamento, 
se realiza com um abraço mais folgado, 
para dar espaço ao movimento do \hyperref[def:Seguidor]{\textbf{seguidor}}.
Cada ciclo do movimente dura 4 tempos, sendo o primeiro par de tempos do passo, simétrico ao segundo.
Para uma correta execução do movimento, o condutor envia informação de condução mediante o abraço de dança,
de modo que esta informação chegue quase sem degradação ate o quadril do seguidor,
é nesse ponto que a condução provoca o movimento das pernas (uma por vez), de modo que
o seguidor mantêm em todo momento as ``pernas fechadas''\footnote{
Ter as pernas fechadas, neste contexto, não indica literalmente ter as pernas juntas, 
e sim juntar as pernas como se em todo momento ao caminhar tentássemos segurar uma folha de papel na virilha.}.
Assim, na primeira metade do ciclo do passo, no tempo 1, o seguidor,
da inicialmente um passo ao frente ganhando o peso do corpo, 
logo no tempo 1.5 o pé livre que está atrás faz um movimento para fechar mais as pernas, 
ganhando este pé o peso do corpo; finalmente, o novo pé livre se movimenta ligeiramente ao frente no tempo 2, 
para ajeitar a postura e ganhar o peso do corpo, neste caso se descansa um tempo 2 completo.
A segunda metade do ciclo é similar à primeira e inicia no tempo 3, só que agora se começa com o outro pé, 
o pé livre do peso do corpo.



%%%%%%%%%%%%%%%%%%%%%%%%%%%%%%%%%%%%%%%%%%%%%%%%%%%%%%%%%%%%%%%%%%%%%%%%%%%%%%%
\PRLsep{Passos de samba de gafieira nas décadas 1980 e 1990}

\subsection{\textcolor{blue}{Caminhada}} 
\index{Passo!Caminhada a contratempo}
\index{Passo cíclico!Caminhada a contratempo}
\index{Passo a contratempo!Caminhada a contratempo}
 Caminhada ou caminhada a contratempo.
Este passo foi  criado entre o final da década de 1980 e inícios de 1990  \cite[pp. 143]{perna2002samba}.

\subsection{\textcolor{blue}{Chicote}} 
\index{Passo!Chicote}.
Este passo foi  criado entre o final da década de 1980 e inícios de 1990  \cite[pp. 143]{perna2002samba}.

\subsection{\textcolor{blue}{Esse}}
\index{Passo cíclico!Esse} 
\index{Passo!Esse}.
Este passo foi  criado entre o final da década de 1980 e inícios de 1990  \cite[pp. 143]{perna2002samba}.

\subsection{\textcolor{blue}{Facão}} 
\index{Passo!Facão}.
Este passo foi  criado entre o final da década de 1980 e inícios de 1990  \cite[pp. 143]{perna2002samba}.

\subsection{\textcolor{blue}{Faquinha}} 
\index{Passo!Faquinha}.
Este passo foi  criado entre o final da década de 1980 e inícios de 1990  \cite[pp. 143]{perna2002samba}.

\subsection{\textcolor{blue}{Gancho}} 
\index{Passo!Gancho}.
Este passo foi  criado entre o final da década de 1980 e inícios de 1990  \cite[pp. 143]{perna2002samba}.

\subsection{\textcolor{blue}{Gancho redondo}} 
\index{Passo!Gancho redondo}.
Este passo foi  criado entre o final da década de 1980 e inícios de 1990  \cite[pp. 143]{perna2002samba}.

\subsection{\textcolor{blue}{Letra}}
\index{Passo!Letra}.
Este passo foi  criado entre o final da década de 1980 e inícios de 1990  \cite[pp. 143]{perna2002samba}.

\subsection{\textcolor{blue}{Puladinho redondo}} 
\index{Passo!Puladinho redondo}.
Este passo foi  criado entre o final da década de 1980 e inícios de 1990  \cite[pp. 143]{perna2002samba}.

\subsection{\textcolor{blue}{Trança}}
\index{Passo cíclico!Trança}
\index{Passo!Trança}.
Este passo foi  criado entre o final da década de 1980 e inícios de 1990  \cite[pp. 143]{perna2002samba}.

\subsection{\textcolor{blue}{Tesoura}}
\index{Passo!Tesoura}.
Este passo foi  criado entre o final da década de 1980 e inícios de 1990  \cite[pp. 143]{perna2002samba}.


%%%%%%%%%%%%%%%%%%%%%%%%%%%%%%%%%%%%%%%%%%%%%%%%%%%%%%%%%%%%%%%%%%%%%%%%%%%%%%%
\PRLsep{Passos de samba de gafieira da décadas de 1990}

\subsection{\textcolor{blue}{Assalto}} 
\index{Passo!Assalto}
\index{Passo cíclico!Assalto}
Este passo foi criado por Jimmy de Oliveira apos o ano 1990 \cite{sambafunkeadoJimmyDeOliveiraPart1}.

\subsection{\textcolor{blue}{Boneca}} 
\index{Passo!Boneca}
Este passo foi criado por Jimmy de Oliveira apos o ano 1990 \cite{sambafunkeadoJimmyDeOliveiraPart1}.

\subsection{\textcolor{blue}{Elástico}} 
\index{Passo!Elastico}
Este passo foi criado por Jimmy de Oliveira apos o ano 1990 \cite{sambafunkeadoJimmyDeOliveiraPart1}.

\subsection{\textcolor{blue}{Escovinha}}
\index{Passo cíclico!Escovinha}
\index{Passo!Escovinha}
Este passo foi criado por Jimmy de Oliveira apos o ano 1990 \cite{sambafunkeadoJimmyDeOliveiraPart1}.

%\subsection{Homem na lua}
%Este passo foi criado por Jimmy de Oliveira apos o ano 1990 \cite{sambafunkeadoJimmyDeOliveiraPart1}.

\subsection{\textcolor{blue}{Pescaria}} 
\index{Passo!Pescaria}
Este passo foi criado por Jimmy de Oliveira apos o ano 1990 \cite{sambafunkeadoJimmyDeOliveiraPart1}.

\subsection{\textcolor{blue}{Romário}}
\index{Passo cíclico!Romário} 
\index{Passo!Romário}
Este passo foi criado por Jimmy de Oliveira apos o ano 1990 \cite{sambafunkeadoJimmyDeOliveiraPart1}.



%%%%%%%%%%%%%%%%%%%%%%%%%%%%%%%%%%%%%%%%%%%%%%%%%%%%%%%%%%%%%%%%%%%%%%%%%%%%%%%
\PRLsep{Passos de samba de gafieira sem data conhecida}




\subsection{\textcolor{blue}{Balanço}}
\index{Passo!Balanço}
\index{Passo cíclico!Balanço}
Movimento sem data de criação conhecida \cite[pp. 144]{perna2002samba}.

\subsection{\textcolor{blue}{Giro da dama}}
\index{Passo!Giro da dama}
Movimento sem data de criação conhecida \cite[pp. 144]{perna2002samba}.

\subsection{\textcolor{blue}{Saída lateral}}
\index{Passo!Saída lateral}
Movimento sem data de criação conhecida \cite[pp. 144]{perna2002samba}.

\subsection{\textcolor{blue}{Tirada ao lado}}
\index{Passo!Tirada ao lado}
Movimento sem data de criação conhecida \cite[pp. 144]{perna2002samba}.

\subsection{\textcolor{blue}{Mestre sala}}
\index{Passo!Mestre sala}
Movimento sem data de criação conhecida \cite[pp. 144]{perna2002samba}.

\subsection{\textcolor{blue}{Tirada de perna}}
\index{Passo!Tirada de perna}
Movimento sem data de criação conhecida \cite[pp. 144]{perna2002samba}.



\subsection{\textcolor{blue}{Enceradeira}}
\index{Passo!Enceradeira}
Movimento sem data de criação conhecida \cite[pp. 144]{perna2002samba}.


\begin{comment}

Passos acrobáticos ou para apresentações \cite[pp. 142-143]{perna2002samba}:
\begin{tasks}
\task \textbf{Cabide}, \index{Passo!Cabide} oriundo do rock.
\task \textbf{Baratinha} \index{Passo!Baratinha}
\task \textbf{Enceradeira}, \index{Passo!Enceradeira} criado em algum momento no final da década de 1980 e inícios da década de 1990.
\end{tasks}
\end{comment}

%%%%%%%%%%%%%%%%%%%%%%%%%%%%%%%%%%%%%%%%%%%%%%%%%%%%%%%%%%%%%%%%%%%%%%%%%%%%%%%
%%%%%%%%%%%%%%%%%%%%%%%%%%%%%%%%%%%%%%%%%%%%%%%%%%%%%%%%%%%%%%%%%%%%%%%%%%%%%%%
\section{Sobre o $syllabus$ do samba de gafieira}

O $syllabus$  do samba de gafieira, foi criado no ano 2001 em Rio de Janeiro,
este é um listado de passos ordenado em três níveis (básico, intermediário e avançado),
selecionados por votação,
onde são agrupados passos que se consideram essenciais para o ensino e competição;
neste listado não entram passos aéreos \cite[pp. 144]{perna2002samba}.


As personas que participaram da votação para a elaboração do $syllabus$ são \cite[pp. 144]{perna2002samba}:
\begin{inparaitem}[$*$]
\item Bob Cunha e Aurya
\item Bolacha
\item Bruno Barros
\item Carlinhos de Jesus
\item Dani Aguiar
\item Dani Escudero
\item Dani Galper
\item Egídio
\item Flávio Miguel
\item Gérson Reis
\item Kilve
\item Luis Florião/Adriana
\item Marcello Moragas
\item Marco Antonio Perna
\item Marquinhos Copacabana
\item Rogério Mendonça
\item Valdeci
\item Wanir Almeida
\end{inparaitem}.\\



Os passos de \textbf{nível básico} são:
\begin{tasks}(2)
\task Básico (frente-trás)
\task Balanço 
\task Caminhada (ou caminhada a contratempo)
\task Cruzado
\task Esse
\task Gancho
\task Giro da dama
\task Puladinho
\task Saída lateral
\task Tirada ao lado
\end{tasks}~\\


Os passos de \textbf{nível intermediário} são:
\begin{tasks}(2)
\task Assalto
\task Balão apagado
\task Escovinha
\task Facão
\task Gancho redondo
\task Mestre sala
\task Romário
\task Tesoura
\task Tirada de perna
\task Trança
\end{tasks}~\\

Os passos de \textbf{nível avançado} são:
\begin{tasks}(2)
\task Bicicleta
\task Enceradeira
\task Pião
\task Picadinho (ou ou picadilho)
\task Pica-pau
\end{tasks}

