%%\newpage
%%%%%%%%%%%%%%%%%%%%%%%%%%%%%%%%%%%%%%%%%%%%%%%%%%%%%%%%%%%%%%%%%%%%%%%%%%%%%%%
%%%%%%%%%%%%%%%%%%%%%%%%%%%%%%%%%%%%%%%%%%%%%%%%%%%%%%%%%%%%%%%%%%%%%%%%%%%%%%%
%%%%%%%%%%%%%%%%%%%%%%%%%%%%%%%%%%%%%%%%%%%%%%%%%%%%%%%%%%%%%%%%%%%%%%%%%%%%%%%
\section{Sobre o syllabus do samba de gafieira}

O syllabus do samba de gafieira foi criado no ano 2001 em Rio de Janeiro,
este é um listado de passos de samba de gafieira ordenados em três níveis: básico, intermediário e avançado.
Os passos foram selecionados e agrupados por votação, 
procurando evidenciar quais destes eram 
considerados essenciais para o ensino e a competição;
neste listado não foram contemplados passos aéreos \cite[pp. 144]{perna2002samba}.


As personas que participaram da votação para a elaboração do syllabus foram \cite[pp. 144]{perna2002samba}:
\begin{inparaitem}[$*$]
\item Bob Cunha e Aurya
\item Bolacha
\item Bruno Barros
\item Carlinhos de Jesus
\item Dani Aguiar
\item Dani Escudero
\item Dani Galper
\item Egídio
\item Flávio Miguel
\item Gérson Reis
\item Kilve
\item Luis Florião/Adriana
\item Marcello Moragas
\item Marco Antonio Perna
\item Marquinhos Copacabana
\item Rogério Mendonça
\item Valdeci
\item Wanir Almeida
\end{inparaitem}.\\



Os passos considerados de \textbf{nível básico} são:
\begin{tasks}(2)
\task Básico (frente-trás)
\task Balanço 
\task Caminhada (ou caminhada a contratempo)
\task Cruzado
\task Esse
\task Gancho
\task Giro da dama
\task Puladinho
\task Saída lateral
\task Tirada ao lado
\end{tasks}~\\


Os passos considerados de \textbf{nível intermediário} são:
\begin{tasks}(2)
\task Assalto
\task Balão apagado
\task Escovinha
\task Facão
\task Gancho redondo
\task Mestre sala
\task Romário
\task Tesoura
\task Tirada de perna
\task Trança
\end{tasks}~\\

Os passos considerados de \textbf{nível avançado} são:
\begin{tasks}(2)
\task Bicicleta
\task Enceradeira
\task Pião
\task Picadinho (ou ou picadilho)
\task Pica-pau
\end{tasks}



\index{Dicionario do samba de gafieira}
\begin{figure}[t]
\begin{elaboracion}{Alternativas ao syllabus}

Para o idioma espanhol existe\footnote{\url{https://www.rae.es/la-institucion}} a ``Real Academia Espanhola'' (RAE),
que ``é uma instituição com personalidade jurídica própria, cuja principal missão é garantir 
que as mudanças sofridas pela língua espanhola, 
em sua constante adaptação às necessidades do seus falantes, 
não quebrem a unidade essencial que ela mantém em toda a esfera hispânica".
Nesse sentido a RAE leva registro da língua mediante um \textbf{dicionario},
cuja função é recolher o uso atual da língua, dar um marco comum de comunicação aos falantes de espanhol,
e agregar em suas novas versões, novas palavras ou acepções;
tendo assim um papel de historiador ou cronista e não de regulador estrito do idioma.

Assim, minha opinião pessoal é que para evitar a desconfiança, que alguns profissionais da dança tem,
sobre a instituição de um syllabus, no sentido  que poderia diminuir a força criativa do samba de gafieira;
podemos em vez instituir, por exemplo, mediante um ``\textbf{Conselho nacional do samba de gafieira}'',
um ``\textbf{Dicionario do samba de gafieira}'', que recolha de forma similar ao syllabus,
um listado e descrição dos passos mais representativos para a data da liberação de cada edição do dicionario, 
ordenado eles em três níveis (básico, intermediário e avançado) 
com todas as acepções\footnote{Nomes diferentes para um mesmo movimento.}
e outros dados relativos ao samba de gafieira.
De modo que se leve um registro a traves do tempo de como evoluem os movimentos e nomes no samba de gafieira,
e no futuro se alguém o desejar, possa fazer mineração de dados e ver 
por exemplo, como era executado\footnote{Se alguém tem esta curiosidade,
porém para o ano de 1947, a duvida pode ser resolvida no livro
``Como aprender a dançar: novo método de danças modernas'' escrito por Gino Fornaciari \cite[pp. 72]{fornaciari1947aprender}.} 
o movimento que no \AnoLivro~era chamado de ``Pião''.
\end{elaboracion}
\label{fig:ImportanciaSyllabus}
\end{figure}


